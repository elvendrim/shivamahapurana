\chapter{Binduga’s salvation}

Śaunaka said:

1-2. O Sūta, the fortunate Sūta, thou art blessed with thy mind engrossed in
Śiva. The story that thou hast narrated to us is wonderful and conducive to the
increase of devotion. What did the woman Cañculā do after obtaining her
salvation? O intelligent one, please tell me in detail the story of her husband
too.

Sūta said:

3. Once she approached goddess Umā Pārvatī\footnote{Pārvatī is the daughter of
Himālaya and the wife of Śiva. In the cult of \tl{śakti} and \tl{tantras}, she
has been identified with \tl{prakṛti} itself. Almost all the \tl{purāṇas} speak
of her as \tl{prakṛti} and her three \tl{guṇas sattva, rajas} and \tl{tamas} are
the three Gods: Brahmā, Viṣṇu and Śiva}. She bowed and offered prayers to her
with palms joined in her flutter of delight.

Cañculā said:

4. O mother of Skanda, daughter of mountain, Thou art always served by men. O
beloved of Śiva, the bestower of all pleasures, having the form of Supreme
Brahman,

5. Thou art worthy of being served by Viṣṇu, Brahmā and others. Thou art both
endowed with and devoid of attributes. Thou art the subtle primordial Prakṛti,
with Existence, Knowledge and Bliss for thy forms.

6. Thou createst, maintainest and annihilatest. Thou hast the three Guṇas. Thou
art the refuge of the three types of divine beings. Thou sustainest Brahmā,
Viṣṇu and Maheśa.

Sūta said:

7. Offering thus her prayers to the Goddess, Cañculā who had attained salvation
ceased to talk with shoulders stooping and eyes brimming with tears of love.

8. Pārvatī, the beloved of Śiva, ever favouring her devotees, was greatly moved
by pity and said to Cañculā lovingly.

Pārvatī said:

9. O Cañculā, my friend, I am pleased to hear your prayer. O beautiful woman,
what is the boon you crave from me? Tell me. There is nothing that I cannot give
you.

Sūta said:

10. Thus urged by Girijā, Cañculā bowed to her. She asked her, bending her head
and joining her palms together with great devotion.

Cañculā said:

11. O Celestial Girijā, I do not know where my husband is at present, nor where
he is to go. O benignant favourite of the distressed, please make such
arrangements as would enable me to join him.

12. O great goddess Maheśānī, my husband had a \tl{śūdra} woman as his concubine.
He died before me. I do not know what befell that sinner.

Sūta said:

13. On hearing these words of Cañculā Pārvatī, the daughter of Himālaya, who is
fond of justice, replied lovingly.

Girijā said:

14. O daughter, your wicked sinful husband Binduga, the foolish wretch enamoured
of prostitutes has been to hell after his death.

15. He underwent the various tortures of hell for many years and has now become
a \tl{piśāca} due to the residue of sins, in the Vindhya mountains.

16. Even now that wicked fellow is undergoing various painful tortures. He, in
the form of a \tl{piśāca}, has only wind for his diet and is suffering from all
sorts of miseries.

Sūta said:

17. On hearing these words of Gaurī, Cañculā of auspicious rites was overwhelmed
by the pain at the news of her husband’s distress.

18. She somehow steadied her mind, bowed to Maheśvari and with a worried heart
asked the goddess.

Cañculā said:

19. O Maheśvarī, O great goddess, be kind to me. Please redeem my husband,
a wicked perpetrator of evil actions though he be.

20. What is the means by which my husband, the sinful wretch of crooked
intellect, can attain salvation. O goddess, obeisance to Thee. Please explain
to me.

Sūta said:

21. On hearing these words of the woman, Pārvatī, favourably disposed to her
devotees, replied to her chaperon Cañculā, delighted in her heart.

Pārvatī said:

22. If your husband were to hear the holy story of Śiva, he shall surmount
the misery entirely and attain salvation.

23. On hearing these words of Gaurī, little short of nectar, she bent her
shoulders, joined her palms and bowed repeatedly with great devotion.

24. She requested the goddess to provide an opportunity for her husband to hear
the story for quelling his sins and gaining redemption.

Sūta said:

25. Gaurī, the beloved of Śiva, on being frequently requested by the woman,
took pity on her, (making it clear thereby that) she was favourably disposed to
her devotees.

26. Lovingly she sent for the Gandharva king Tumburu who used to sing songs of
praise of Śiva. The daughter of Himalaya said thus to him.

Girijā said:

27. O Tumburu, the favourite of Śiva, ever ready to do as I wish, blessedness be
thine. Accompany this lady immediately to Vindhya mountain.

28. There is an awfully terrible \tl{piśāca} there. I shall tell you all his
antecedents. You will be interested to know the same.

29. This \tl{piśāca} had been a brahmin in his previous birth. Then he was the
husband of this woman who is my chaperon now. He was very wicked and had a
\tl{śudra} concubine.

30. He was impure, never caring for the daily performance of ablutions and
\tl{sandhyā} prayers. His mind was ever vitiated by anger. He ate all sorts of
foul things. He quarrelled with good men and whatever he undertook had been bad.

31. He was violent in his ways, bearing weapons and oppressing poor people
cruelly. He used to take food with his left hand. He used to commit arson in
other people’s house.

32. He was friendly with \tl{cāṇḍālas}. Every day he took delight in the company
of prostitutes forsaking his own wife. The roguish sinner took delight in
associating with the wicked.

33. In evil association with harlots he destroyed all his merits. Besides,
coveting more and more wealth, he made his own wife a fearless sharer of her
paramours’ beds.

34. His evil ways continued till the last moments of his life and when he died
he went to Yama’s city, the terrible place where sinners reap the fruits of
their misdeeds.

35. After undergoing the tortures of many hells, the wicked wretch is now
roaming in the Vindhya mountain as a roguish sinful \tl{piśāca}.

36. Narrate the holy sanctifying tale of sacred \tl{Śivapurāṇa}, that quells all
sins, in front of him.

37. Immediately after hearing the great story of \tl{Śivapurāṇa} his soul will
be cleared of sins and he will cast off his ghosthood.

38. I order you to set that Binduga free from the miserable plight of
\tl{piśāca} and bring him in the aerial chariot in the presence of lord Śiva.

Sūta said:

39. Commanded thus by Pārvatī, Tumburu, the lord of Gandharvas, was much
delighted and thought within himself how fortunate he was.

40-41. Tumburu, the comrade of Nārada, went to the Vindhya mountain seated in
the aerial chariot in the company of Cañculā, the sinless woman and saw the
\tl{piśāca} laughing, crying and loudly shouting by turns. His body was very
huge, his jaws were immensely large and his form was very crooked.

42. The powerful Tumburu, the singer of the excellent songs of praise of Śiva,
forcefully caught hold of the terrible \tl{piśāca} by means of nooses.

43. Thereafter, for the sake of the discourse on \tl{Śivapurāṇa}, Tumburu made
elaborate festive arrangements.

44-45. There was much talk and discussion among the people of all the worlds
“Oh, Tumburu has gone to the Vindhya\footnote{It is a range of mountains which
stretches across India and divides Madhyadeśa or Middle Land from the south. It
is one of the seven Kulaparvatas and is personified in the \tl{purāṇas}.}
mountain at the suggestion of Goddess, to narrate the story of \tl{Śivapurāṇa}
to redeem the \tl{piśāca}.” The divine sages too hastened to the place for
listening to the same.

46. The wonderful congregation of those who assembled there, reverently eager to
listen to \tl{Śivapurāṇa}, was very auspicious.

47. They bound the \tl{piśāca} with nooses and compelled him to sit there. With
the lute in his hands, Tumburu began to sing the story of Gaurī’s consort.

48. Starting with the first \tl{saṃhitā} (compendium) and ending with the
seventh one he clearly expounded the whole of \tl{Śivapurāṇa} along with its
\tl{Māhātmya} (greatness).

49. On hearing the \tl{Śivapurāṇa} consisting of seven compendiums with great
reverence all the listeners deemed themselves highly blessed.

50. The \tl{piśāca} too, on hearing the holy \tl{Śivapurāṇa}, cast-off all his
sins and discarded his ghostly body.

51. He assumed the divine form of the three-eyed moon-crested God (Śiva), white
in complexion, clad in white cloth, with the body illuminated and embellished by
all ornaments.

52. Taking up the divine body, the glorious Binduga accompanied by his wife sang
the story of Pārvatī’s consort.

53. On seeing his wife thus, all the divine sages had a welcome surprise and
were highly delighted in their minds.

54. Gratified on hearing the wonderful story of Śiva they returned to their
respective abodes delightedly glorifying Śiva.

55. Binduga in his divine form ascended the aerial chariot with great pleasure.
High up in the sky, with his wife at his side he shone brilliantly.

56. Singing the pleasing attributes of Śiva he hastened to Śiva’s region
accompanied by Tumburu and his own wife.

57. Binduga was welcomed by Śiva and Pārvatī and was lovingly made their
attendant. His wife became the chaperon of Girijā.

58. In that permanent abode of excellent bliss and sublime lustre he acquired an
unassailable residence and unobstructed pleasure.

59. Thus I have narrated this holy anecdote that removes sins, is highly
delightful to Śiva and Pārvatī in pure and heightening devotion.

60. He who listens to this account with devotion and recites this piously shall
enjoy immense pleasures and obtain liberation.
