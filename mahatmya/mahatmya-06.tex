\chapter{Rules for listening to Śivapurāṇa}

Śaunaka said:

1-2. O Sūta, O highly intelligent disciple of Vyāsa, obeisance to thee. Thou art
blessed and the foremost among Śiva’s devotees. Thy attributes are highly
praiseworthy. Please tell me about the rules for listening to \tl{Śivapurāṇa}
whereby the listener shall obtain all excellent fruits.

Sūta said:

3. O sage Śaunaka, I shall tell you the rules for listening to \tl{Śivapurāṇa}
so that the entire fruit may be derived by their observance.

4. The householder must invite an astrologer and propitiate him to fix an
auspicious day for the beginning, so that it may conclude without obstacles in
the middle.

5. News must be circulated in different localities that the auspicious discourse
is to take place and all who seek welfare must be present.

6. Women, \tl{śūdra} and others who are far removed from holy discourses and
stay away from singing glories of Śiva shall attend this discourse whence they
may have some enlightenment.

7. Wherever there are devotees of Śiva, eager to listen to the songs of praise
in the neighbourhood, they must also be invited with due reverence.

8. Thus there shall be a great festive gathering of saintly men at the discourse
of \tl{Śivapurāṇa}, a wonderful congregation.

9. With devotion, may all of you be pleased to join us for imbibing the sweet j
uice of \tl{Śivapurāṇa}, with due reverence.

10. If you do not have sufficient leisure, please grace the assembly at least
for a day. By all means, do come, even for a short stay or a while.

11. Thus all should be invited humbly. Those who come should be hospitably
received in all respects.

12. An excellent spot for the discourse on \tl{Śivapurāṇa} must be selected in
a temple of Śiva, or in a holy centre or in a park or in a private house.

13. The ground must be scrubbed, cleaned and smeared with cow-dung. It must be
decorated with metallic materials attended with all festivities. The whole
arrangement must be divinely exquisite and pleasing to diverse tastes.

14. All the rubbish must be removed and all unnecessary things must be hidden in
a corner away from the public view.

15. A high platform must be constructed, richly decorated with stumps of
plantain trees. The whole place should be covered with a canopy. Fruits and
flowers should be profusely used.

16. Flags and banners should be hoisted in the four quarters. They should be
neatly arranged to be pleasing to everyone.

17. A seat must be assigned to Śiva, the Supreme soul. A comfortable seat shall
be assigned to the orator.

18. Good places shall be reserved for the regular listeners as befitting their
position. O sage, for the other casual visitors, seats with ordinary comfort
shall be set apart.

19. People must be in as pleasant a mood as on marriage occasions: all worldly
worries and anxieties must be avoided.

20. The discourser faces the north and the listeners the east. There is no fear
of the criss-crossings of the feet.

21. Or the discourser faces the east as the worshipper or let the discourser
and the recipient face each other.

22. As long as he is seated in the seat of the discourser, the Purāṇist does not
bow to any one before the conclusion of the discourse.

23. Whether he is a boy or a youth, an old man, an indigent person, or a weakling,
the scholar well-versed in the \tl{purāṇa} is worthy of honour from all those
who seek merit.

24. Never shall anyone show demeaning disrespect towards a Purāṇa-scholar,
the speech from whose mouth is no less than the divine cow Kāmadhenu for all
persons.

25. Either as the cause of birth or of attributes there are many who may be
termed “Guru” (Elder, preceptor). Among them the Purāṇic scholar is the greatest
Guru.

26. Who can be a greater Guru than the person who bestows the highest salvation
on those who are disheartened due to the millions of births?

27. The person who undertakes to conduct a discourse on this sanctifying tale
shall be well-versed in \tl{purāṇas}, pure, skilful, quiet, free from malice,
saintly, sympathetic and eloquent.

28. The intelligent discourser shall start the narration of the story of
\tl{Śivapurāṇa} at sunrise and continue it for two and a half \tl{praharas}
(2½ x 3 = 7½ Hrs) earnestly.

29. This story shall not be narrated before rogues, wicked persons of crooked
professions and those bent on conquering others in disputes and arguments.

30. The discourse on this holy story shall not be conducted in a place infested
by wicked men, or surrounded by thieves or in the house of a rogue.

31. The orator shall have an interval of a \tl{muhūrta} (forty-eight minutes)
at midday for the sake of answering calls of nature.

32. The discourser must have his share on the day previous to the discourse so
that his vow be maintained. During the days of discourse he shall perform all
his daily routine (\tl{sandhyā} \etc) briefly.

33. Another scholar equally well-versed in \tl{purāṇas} should be sitting near
the discourser to help him. He must be competent to clear doubts and eager to
enlighten the people.

34. In order to ward off obstacles to the discourse, Gaṇanātha
\footnote{Gaṇanātha: It is an epithet of Śiva and also of Gaṇeśa. But as the
worship of Śiva is mentioned separately in the following line of this verse,
the term Gaṇanātha here signifies Gaṇeśa, the son of Śiva and Pārvatī (See v. 54
of this chapter). He is invariably propitiated at the beginning of any important
undertaking.} should be worshipped. The lord of the story Śiva and the book,
\tl{Śivapurāṇa}, too must be worshipped with piety.

35. The story of \tl{Śivapurāṇa} must be listened to with careful attention.
The recipient must be intelligent, pure in mind, delighted at the heart and a
follower of conventions.

36. If either the discourser or the recipient indulges in too many extraneous
activities, is a victim of any of the six base feelings of lust, anger \etc,
\footnote{\tl{Ṣadvikāras}: Six causes of perturbation are the following: lust
(\tl{kāma}), anger (\tl{krodha}), greed (\tl{lobha}), pride (\tl{mada}),
delusion (\tl{moha}), envy (\tl{matsara}).} is enamoured of women or is a
heretic he cannot gain any merit.

37. Casting off the worries of worldly affairs and those of wealth, house and
sons if any one of pure mind concentrates his attention on the discourse he will
secure the excellent fruit.

38. The recipients who are endowed with faith and piety, do not eagerly pursue
other activities and are unruffled, pure and restrained in speech derive great
merit.

39. Base men of impious nature who listen to this holy story do not have any
special merit derived out of it. They will have misery in every birth.

40. Those who do not honour this \tl{purāṇa} with presents according to their
capacity are fools. Even if they listen to the story they will not be sanctified.
They will become indigent.

41. Those who walk out of congregation in the middle of the discourse will have
the adverse effect: they will face the destruction of their wives and wealth in
the midst of enjoyment.

42. The sons and descendants of the people who attend the discourse with
turbaned head, become sinners defiling the whole race.

43. The attendants of Yama in hell force the people who chewed betel leaves
while attending the discourse, to eat their own faeces.

44. Those who listen to the story seated on a more elevated seat fall into hell
and after undergoing the tortures there are reborn as crows.

45. Those who listen to this auspicious story seated in the \tl{vīra}
pose\footnote{\tl{Vīrāsana} also called \tl{Paryaṅka bandha}. It is a particular
kind of posture practised by ascetics in meditation setting on the hams.} fall
into hell and after undergoing the tortures of hell are reborn as poisonous
plants.

46. Those who listen to the story without bowing to the discourser at first fall
into hell and after undergoing the tortures of hell are reborn as Arjuna trees.

47. Those who, not being sick, listen to the story lying down, fall into hell
and are reborn as pythons \etc

48. Those who listen to the story seated on the same level as the discourser
become as sinful as the defiler of the preceptor’s bed and fall into hell.

49. Those who speak ill of the discourser or of this sacred story are born as
dogs and lead miserable lives in hundred births.

50. Those who begin to argue and dispute while the discourse is being held fall
into hell and after undergoing the tortures there are reborn as donkeys.

51. Those who never listen to this sanctifying story fall into hell. After
experiencing the tortures there they are reborn as wild boars.

52. The rogues who create hindrances even as the discourse is being held fall
into hell. After undergoing the tortures there for millions of years they are
reborn as village-boars.

53. Realising all these, the listener shall always be pure, devoted to the
discourser and intelligent enough to listen to the story with devotion.

54. For warding off obstacles to the discourse Lord Gaṇeśa should be worshipped
at first. Every day at the end of the discourse he shall briefly perform
expiatory rites (for omissions and commissions).

55. He shall worship the nine planets\footnote{Sun, Moon, Mars, Mercury, Jupiter,
Venus, Saturn, Rahu and Ketu.} and the deities in the “Sarvatobhadra” array. He
shall worship the book according to the rites of Śiva’s adoration.

56. At the conclusion of the worship he shall offer prayer to the book
identified directly with Śiva, humbly and piously joining his palms in reverence.

57. (The Prayer) “Thou art the visible Maheśvara Śrīmat Śivapurāṇa. Thou hast
been accepted by me for listening purpose. Be thou pleased with me.

58. This wish of mine must be fulfilled by Thee. May this narration of the story
be concluded without obstacles.

59. I am immersed in the middle of the ocean of worldly existence. Please lift
me up from it, miserable wretch that I am, with my limbs caught in by the
crocodiles of Karman (Action): O Śaṅkara, I am Thy slave.”

60. The householder shall thus pray to \tl{Śivapurāṇa} identified directly with
Śiva, in words evoking pity. Then he shall begin the worship of the discourser.

61. He shall adore the discourser too in the same manner as in the rite of the
worship of Śiva and propitiate him with flowers, cloths, ornaments, incense,
lamps \etc

62. In the presence of the discourser he shall take vow and observe all
restraints with a pure mind and the same shall be maintained till the conclusion
to the extent of his capacity.

63. “O Thou, the foremost of discoursers, identified with Vyāsa, well-versed in
the sacred literature of Śiva, please remove my ignorance through the light of
this story.”

64. He shall invite five brahmins (if he can) or at least a brahmin for repeating
\tl{śiva pañcārṇa mantra}\footnote{\tl{Pañcārṇa mantra}: “\tl{namaḥ śivāya}.”
This mantra, consisting of five letters in Devanāgarī script, is dedicated to
Śiva.}.

65. Thus O sage, I have told you the rules of listening to the story with
devotion as well as those of governing the pious recipients. What else do you
wish to hear?
