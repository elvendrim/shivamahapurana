\chapter{Greatness of Śivapurāṇa}

Śaunaka\footnote{Śaunaka was the chief of the sages at the great sacrifice in
Naimiṣa forest to whom the \tl{Mahābhārata} and the \tl{purāṇas} were recited by
the Sūta in the reign of Adhisīmakṛṣṇa, the great-grandson of Janamejaya and
the sixth in generation from Arjuna in the Paurava line. \tl{Vāyu-purāṇa} 1.12;
99, 255-8; \tl{Padma} 1.1.19} said:

1. O Sūta of great intellect, O my lord, the knower of all philosophical
principles, please narrate to me the essence of the \tl{purāṇas} in detail.

2. How do good conduct, good devotion and power of discrimination flourish? How
are base feelings dispelled by good men?

3. In this terrible Kali age all living beings have almost become demoniac in
character. What is the effective mode of remedying the same?

4. Now tell me about the greatest means to achieve the most perfect weal,
the holiest of the holy modes.

5. What is that, the practice of which particularly purifies the soul? What is
that which enables a man of unsullied mind to attain Śiva?

Sūta\footnote{The Sūtas (\tl{Vāyu-purāṇa} 1.32-3; \tl{Padma} 1.1.27-28)
preserved the genealogies of Gods, sages, and glorious monarchs as well as
the traditions of great men. The Sūta here is not a caste that is described by
Manu (X.11.17) as the offspring of a \tl{kṣatriya} father and \tl{brahman
mother}. He is a venerable \tl{brāhmaṇa} who has preserved ballads, songs,
genealogies of Gods, sages and glorious Kings. See Pargiter: \emph{Ancient
Indian Historical Tradition} Ch. II; also Pusalkar: \emph{Studies in Epics and
Purāṇas of India}, Intro. p29. He is described as the disciple of Vyāsa.} said:

6. O foremost among sages, you are blessed indeed as you are desirous of hearing.
Hence I shall ponder over the greatest of the Sacred lore intelligently and
tell you.

7. O dear, listen to that divine panacea evolved out of all religious tenets,
heightening true devotion and conducive to the pleasure of Śiva.

8. It is destructive of the great fear of the Python of Kāla (Death). O sage,
it is the noble \tl{Śiva Purāṇa}\footnote{For the nomenclature and authenticity
of this \tl{purāṇa} see Introduction.} formerly narrated by Śiva Himself.

9. For the benefit of the people in the age of Kali, the sage Vyāsa
\footnote{According to the Pauranic tradition, Kṛṣṇa Dvaipāyana Vyāsa, the son
of Satyavatī, composed the eighteen \tl{purāṇas} or superintended their
compilation. \tl{Matsyapurāṇa} 53-70.} has abridged it out of great respect for
the sage Sanatkumāra\footnote{The \tl{purāṇas} were first compiled by Brahmā
(\tl{Vāyu-purāṇa} 1.60-61). Sanatkumāra, a son of Brahmā (SP 1.4.8-9; 1.5.17)
inherited them from his father and imparted them to Vyāsa who in turn abridged
them in 18 compendiums.} on being instructed by him.

10. O sage, there is nothing other than \tl{Śiva Purāṇa} for the purification of
the mind especially of the people of the Kali age\footnote{The beginning of
the Kali age has been discussed by Dr. Fleet (‘\emph{Journal of the Royal
Asiatic Society (Great Britain)}’, 1911, pp. 479, 675, 686) and he has pointed
out that it began on the day on which Lord Kṛṣṇa died, which the chronology of
the Mahābhārata places, as he shows, some twenty years after the great battle
and that it was then that Yudhiṣṭhira abdicated and Parikṣit began to reign.
Pargiter: \emph{Dynasties of the Kali Age}. Intro. p.X.}.

11. It is only the intelligent and the highly fortunate man who has accumulated
great merits in his previous birth who will be drawn towards it.

12. This \tl{Śivapurāṇa} is the greatest and the noblest of the sacred lore. It
is the form of Śiva and as such is to be served and realised in this world.

13. By reading this and listening to it the good man becomes very pious. By all
means he instantly attains Śiva’s region.

14. Hence every endeavour of men to read this is desirable. Loving care to
listen to it yields all desired results.

15. By listening to this \tl{purāṇa} of Śiva a man becomes sinless. After
enjoying all extensive worldly pleasures he will attain the region of Śiva.

16. Merely by listening to the story of Śiva a man secures that merit which
results from the performance of \tl{rājasūya}\footnote{\tl{Rajasūya} is a great
sacrifice performed by a universal monarch (in which the tributary princes also
take part) at the time of his coronation as a mark of his undisputed
sovereignty.} and a hundred \tl{agniṣṭomas}\footnote{\tl{Agniṣṭoma} is a
sacrificial rite extending over several days in spring and forming an essential
part of the \tl{jyotiṣṭoma}.}.

I 7. O sage, those who listen to \tl{Śivapurāṇa} the noblest of Sacred lore,
cease to be mere human beings. They must be undoubtedly considered as
manifestations of Rudra, a form of Śiva.

18. Sages consider the dust in the feet of those who habitually listen to that
\tl{purāṇa} and recite it, on a par with holy centres.

19. May those who wish to attain the seat of salvation, listen always to the holy
\tl{Śivapurāṇa} with great devotion.

20. O noblest among sages, if he is unable to listen to it always, let him hear
it for a short while every day with his mind fully controlled.

21. If any one is unable to listen to it every day, O sage, let him listen to
\tl{Śivapurāṇa} in the holy months.

22. Those who listen to that \tl{purāṇa} even for a \tl{muhūrta} (48 minutes),
half that period, one fourth of that period or even for a moment will not suffer
from mishaps.

23. O lord of sages, the man who listens to that \tl{purāṇa} crosses the ocean
of worldly existence after burning the great forest of \tl{karma} (binding actions).

24. O sage, the merit that accrues from all gifts and all Sacrifices becomes
stabilised after listening to \tl{Śivapurāṇa}.

25. Particularly in the age of Kali there is no greater virtue conducive to the
achievement of liberation by men, O sage, than listening to \tl{Śivapurāṇa}.

26. There is no doubt in this that, listening to the \tl{purāṇa} and reciting
the names of Śiva is as efficacious as the \tl{kalpa} tree
\footnote{\tl{Kalpadruma} is a tree supposed to grant all desires.} in yielding
one’s desires.

27. For the benefit of the evil-minded persons of the Kali age, bereft of
virtuous conduct, Lord Śiva has produced the nectar in the form of
\tl{Śivapurāṇa}.

28. A single man, the man who drinks nectar, becomes immortal and unageing. But
the nectar of the divine story of Śiva, if drunk, makes the whole family
immortal and unageing.

29. The sanctifying story of \tl{Śivapurāṇa} must always be resorted to,
definitely so.

30. Merely by listening to \tl{Śivapurāṇa} (if such good results) what am I to
say about the result when Śiva abides in the heart?

31. This work consists of twenty-four thousand verses divided into seven
\tl{saṃhitās} (compendiums). The three kinds of Devotion [(1) by meditation,
(2) recital of prayer and (3) acts of worship and service] are fully explained
in it. It must be listened to with great respect.

32. The first compendium is called \tl{Vidyeśvara saṃhitā}, the second is
\tl{Rudrasaṃhitā}, the third is \tl{Śata-Rudrā} and the fourth is
\tl{Koṭi-Rudrā}.

33. The fifth compendium is called \tl{Umāsaṃhitā}, the sixth is
\tl{Kailāsasaṃhitā} and the seventh is \tl{Vāyavīyā}. Thus, there are seven
\tl{saṃhitās} in this \tl{purāṇa}.

34. This divine \tl{purāṇa} of seven \tl{saṃhitās} and called after Śiva stands
on an equal footing with Brahman (\ie Vedic texts) and accords an achievement
that is superior to everything else.

35. He who reads the entire \tl{Śivapurāṇa} without omitting any of the seven
\tl{saṃhitās} can be called a \tl{jīvanmukta} (a living liberated soul).

36. O sage, the ignorant man is tossed about in the ocean of worldly existence
till the excellent \tl{Śivapurāṇa} reaches his ears.

37. Of what avail is listening to many sacred texts and other confounding
\tl{Purāṇas}? The \tl{Śivapurāṇa} alone loudly proclaims (its readiness) to
grant salvation.

38. The house where the discourse on this \tl{Śivapurāṇa} is held becomes a holy
centre. It destroys the sins of the inmates of the house.

39. Thousands of horse-sacrifces\footnote{In Vedic times the \tl{aśvamedha}
sacrifice was performed by kings desirous of offspring but subsequently it was
performed by them for the achievement of universal supremacy. A horse was turned
loose to wander at will for a year, attended by a guardian; when the horse
entered a foreign country, the ruler was bound either to submit or to fight. In
this way the horse returned at the end of a year, the guardian obtaining or
enforcing the submission of princes whom he brought in this train. After the
successful return of the horse, the horse was sacrificed amidst great rejoicings.
It is said that the horse was sometimes not immolated but kept bound during the
ceremony.} and hundreds of \tl{vājapeya}\footnote{\tl{Vājapeya} is one of the
seven forms of the \tl{soma}-sacrifice offered by kings or \tl{brāhmanas}
aspiring to the highest position, and preceding the \tl{rājasūya} and the
\tl{bṛhaspatisava}.} sacrifices do not merit even a sixteenth part of Śivapurāṇa.

40. O best of sages, a sinner is called a sinner till the moment he hears
\tl{Śivapurāṇa} with great devotion.

41. The holy rivers, Gaṅgā and others, the seven sacred cities\footnote{The seven
sacred cities of the Hindus are Ayodhyā, Mathurā, Māyā, Kāśī, Kāñcī, Āvantikā
and Dvārikā.} and Gayā can never be equal to \tl{Śivapurāṇa}.

42. If one wishes for the greatest of goals (Liberation) one shall recite at
least a stanza or even half of it from \tl{Śivapurāṇa}.

43. He who constantly listens to \tl{Śivapurāṇa} fully comprehending its meaning
or simply reads it with devotion is undoubtedly a meritorious soul.

44. Lord Maheśāna (Śiva) is extremely pleased with the sensible man who listens
to \tl{Śivapurāṇa} when death is imminent. Lord Śiva accords him a seat in his
own region.

45. He who adores this \tl{Śivapurāṇa} with great devotion enjoys in the world
all desired objects and attains Śivaloka.

46. Never slack in his devotion to the \tl{Śivapurāṇa} he who keeps this work
well wrapped in a silk cloth, will ever be happy.

47. The holy \tl{Śivapurāṇa}, the sole possession of a devotee of Śiva, should
assiduously be resorted to by a person who desires for happiness here and
hereafter.

48. The holy \tl{Śivapurāṇa} that accords the four aims of life (virtue, wealth,
love and salvation) must be heard and read with great devotion always.

49. The \tl{Śivapurāṇa}, the greatest harbinger of the perfect welfare among the
Vedas, Itihāsas and other sacred texts must be thoroughly understood by those
who seek salvation.

50. This \tl{Śivapurāṇa} is the greatest resort of the knowers of Ātman
(Spiritual Seekers) for ever; it is the noblest object worthy of adoration of
good men; it suppresses the three types of distresses (\ie physical illness,
extraneous attacks and divine calamities); it accords happiness always; and it
is very pleasing to all \tl{devas} led by Brahmā, Hari and Īśa.

51. With the mind extremely delighted I bow unto \tl{Śivapurāṇa} for ever. May
Śiva be pleased and bestow on me a devotion to His feet.
