\phantomsection
\addcontentsline{toc}{subsection}{Cañculā’s salvation}
\subsection*{Cañculā’s salvation}

The Brahmin said:

l-2. O Brahmin lady, fortunately you have realised at the proper time on hearing
the story of \tl{Śivapurāṇa} that is conducive to non-attachment. Do not be
afraid. Seek refuge in Śiva. All sins perish instantaneously by Śiva’s grace.

3. I shall explain to you that great object attached to the glorification of
Śiva whereby your course hereafter will be pleasant always.

4. It is by listening to the excellent story that your mind has now turned to
the pure path of repentance and detachment towards worldly pleasures.

5. Repentance is the only way of acquittance for all sinners. Saintly men have
extolled it as the only way of expiation for all sins.

6. Purity can be realised by repentence alone. If the sinner expiates in the
manner advised by saintly men it removes all sins.

7. After due expiation he becomes free from fear. By repentance he attains
salvation undoubtedly.

8. The mental purity that one derives on hearing the story of \tl{Śivapurāṇa}
cannot be gained by any other means.

9. As a mirror becomes free from dirt on being wiped with a cloth, so is the
mind undoubtedly purified by listening to this story.

10. Accompanied by Ambā, Śiva stays in the minds of pure men. The sanctified
soul thereupon attains the region of Śiva and Ambā.

11. Hence this story is the means of realising the fourfold aim of life. It is
for this that Mahādeva earnestly created this.

12. Listening to the story of Pārvatī’s consort (Śiva) brings about steady
contemplation. Contemplation leads to perfect knowledge which certainly brings
in salvation.

13. A person who listens to the story in this birth though he be unable to
meditate, realises the same in the next birth after which he reaches the goal
of Śiva.

14. Many repentant sinners have meditated upon Śiva after hearing this story and
have achieved salvation.

15. Listening to the excellent story is the cause of beatitude for all men.
Properly entertained, it dispels the ailment of worldly bondage.

16. Listening to the story of Śiva, constant meditations thereon and repeated
musings certainly purify the mind.

17. That (the purity of the mind) leads the meditator to a devotion of Maheśa
and his two sons (Gaṇeśa and Kārtikeya). With their blessings one undoubtedly
attains liberation.

18. A person devoid of that devotion with his mind entangled in the bondage of
ignorance is a brute. He can never be liberated from the worldly bondage.

19. Hence O Brahmin lady, you turn away from worldly pleasures. Listen to the
sanctifying story of Śiva with devotion.

20. Your mind, as you listen to the excellent story of Śiva, the Supreme Soul,
will become pure and thereafter you will realise liberation.

21. Liberation is assured in this very birth to a person who meditates on the
lotus-like feet of Śiva, with a pure mind. Truth, I am saying the truth.

Sūta said:

22. After saying this, that excellent brahmin with his mind melting with pity
ceased talking and turned his attention to the meditation on Śiva with the
purity of the Soul.

23. The wife of Binduga, called Cañculā, when thus addressed by the brahmin,
became delighted and her eyes brimmed with tears.

24. With great delight in her heart she fell at the brahmin’s feet. Cañculā with
her palms joined together said “I am blessed”.

25. Afterwards she rose up with great mental agitation. With her hands joined
together, her words faltering in excitement, the woman of good intellect in her
detached mood said to the brahmin, the great devotee of Siva.

Cañculā said:

26. O my lord, great brahmin devotee of Śiva, you are blessed. You are endowed
with the vision of Truth. You are devoted to rendering help to others. You are
to be described among great saintly men.

27—28. O saintly one, I am about to fall into the ocean of Hell. Save me. I am
now faithfully eager to listen to the \tl{purāṇa}. On hearing its excellent
story I became detached from worldly pleasures.

Sūta said:

29. So saying with reverence she got the blessings of the brahmin. Desirous of
hearing the \tl{purāṇa} she stayed there rendering service to him.

30. The intelligent brahmin devotee narrated the Purāṇic story to the woman on
the spot.

31. In this manner she listened to the excellent story of \tl{Śivapurāṇa} in
that holy centre from that excellent brahmin.

32. On hearing that excellent story that heightened devotion, knowledge and
detachment and yielded liberation, she became greatly blessed.

33. Favoured by the good preceptor she quickly gained purity of mind. By the
blessings of Śiva she could meditate on Śiva’s forms and features.

34. Thus, resorting to the good preceptor, her mind was drawn towards Śiva. She
constantly meditated on the sentient blissful body of Śiva.

35-36. She wore barks of trees and had her hair matted. She smeared ashes over
her body. She wore garlands of \tl{rudrākṣa} beads. Every day she took her
ablutions in the sacred water. She regularly repeated Śiva’s names. She
regulated her speech and diet. She propitiated Lord Śiva in the manner advised
by the preceptor.

37. O Śaunaka, thus for a long time Cañculā continued her meditation on Lord Siva.

38. When the stipulated period was over, Cañculā in her practice of the
three-fold\footnote{The three kinds of devotion are: (1) the devotion of hearing
(\tl{śravaṇa}), (2) of glorifying (\tl{kīrtana}) and (3) of deliberating
(\tl{manana}) the attributes of God. \tl{Śiva-purāṇa} (\tl{Vidyeśvara-saṃhitā})
3.21-25} devotion cast-off her body without any difficulty.

39. The divine aerial chariot shining in brilliant colours, sent by Tripurāri
\footnote{Śiva is called Tripurāri, the slayer of Tripura, for he killed the
demon Tripura who presided over three cities of gold, silver and iron in the sky,
air and earth built for demons by Maya.} (Śiva) Himself, accompanied by His
attendants, arrived there quickly.

40. With her dirt and sin removed she mounted the aerial chariot and was
immediately taken to Śiva’s city by the lord’s noble attendants.

41. She assumed a divine form. Her limbs were divine in their features. She
assumed the form of Gaurī with the crescent moon as her coronet and divine
ornaments shining brilliantly.

42. She saw the three-eyed Mahādeva, the eternal, being served devotedly by
Viṣṇu, Brahmā and other gods.

43. He had the brilliance of ten million suns and was reverently served by
Gaṇeśa, Bhṛṅgi, Nandīśa, Vīrabhadreśvara and others.

44. His neck had a blue hue; he had five faces, three eyes, the crescent moon as
crest-ornament and his left side was apportioned to Gaurī who had the brilliance
of lightning.

45. He was white in complexion like camphor and wore all ornaments. Besmeared
with white ashes all over the body and clad in white cloth he shone brilliantly.

46. The woman Cañculā became highly delighted on seeing Śaṅkara. In her flutter
of delight she bowed again and again to Him.

47. She joined her palms in reverence with great pleasure, love and humility. In
her great delight she shed tears of joy and had feelings of horripilation.

48. With sympathy she was allowed to approach Pārvatī and Śaṅkara who gracefully
looked at her.

49. Cañculā, the beloved wife of Binduga, thus attained a divine form and was
blessed with divine pleasures and made a chaperon by Pārvatī.

50. In that permanent abode of excellent bliss and sublime lustre she acquired
a permanent residence and unobstructed pleasure.
