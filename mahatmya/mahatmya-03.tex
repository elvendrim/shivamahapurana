\chapter{Cañculā’s disillusion and detachment}

Śaunaka said:

1. O Sūta of great intellect, thou art extremely blessed and omniscient. By thy
favour I am gratified to satiety again and again.

2. My mind rejoices much on hearing this old anecdote. Please narrate another
story equally increasing devotion to Śiva.

3. Nowhere in the world are those who drink nectar honoured with liberation. But
in regard to the nectar of the story of Śiva it is different. When drunk, it
straightway accords salvation.

4. Thou art blessed, blessed indeed. Blessed, blessed is the story of Śiva on
hearing which a man attains Śivaloka.

Sūta said:

5. O Śaunaka, please listen I shall tell you, though it is a great secret, since
you are the foremost among Vedic scholars and a leading devotee of Śiva.

6. There is a seaside village Bāṣkala\footnote{Bāṣkala-grāma: Cf.
\tl{Skanda-purāṇa} 3.111.32.50. It has not been possible to identify and
locate this village.} where sinful people bereft of Vedic virtue reside.

7. They are wicked debauchees with deceptive means of livelihood, atheists,
farmers bearing weapons and adulterous rogues.

8. They know not anything about true knowledge, detachment or true virtue. They
are brutish in their mental make-up and take a great deal of interest in
listening to evil gossips and slander.

9. People of different castes are equally roguish never paying attention to
their duties. Always drawn to worldly pleasures they are ever engrossed in one
evil action or another.

10. All the women too are equally crooked, whorish and sinful. Evil-tempered,
loose in morals they are devoid of good behaviour and disciplined life.

11. In the village Bāṣkala peopled by wicked people, there was a base brahmin
called Binduga.

12. He was a wicked sinner traversing evil paths. Although he had a beautiful
wife he was enamoured of a prostitute. His passion for her completely upset his
mind.

13. He forsook his devoted wife Cañculā and indulged in sexual dalliance with
the prostitute overwhelmed by Cupid’s arrows.

14. Many years thus elapsed without any abatement in his evil action. Afraid of
violating her chastity Cañculā, though smitten by Cupid bore her distress
(calmly for a short while).

15. But later on as her youthful health and boisterous virility increased,
Cupid’s onslaught became extremely unbearable for her and she ceased from
strictly adhering to her virtuous conduct.

16. Unknown to her husband she began to indulge in sexual intercourse with her
sinful paramour at night. Fallen thus from \tl{sāttvic} virtues she went ahead
along her evil ways.

17. O sage, once he saw his wife amorously indulging in sexual intercourse with
her paramour at night.

18. Seeing his wife thus defiled by the paramour at night he furiously rushed at
them.

19. When the roguish deceitful paramour knew that the wicked Binduga had
returned to the house he fled from the scene immediately.

20. The wicked Binduga caught hold of his wife and with threats and abuses
fisted her again and again.

21. The whorish wicked woman Cañculā thus beaten by her husband became
infuriated and spoke to her wicked husband.

Cañculā said:

22. Foul-minded that you are, you indulge in sexual intercourse with the harlot
every day. You have discarded me your wife, ever ready to serve you with my
youthful body.

23. I am a youthful maiden endowed with beauty and mentally agitated by lust.
Tell me what other course can I take when I am denied the amorous sport with my
husband.

24. I am very beautiful and agitated with flush of fresh youth. Deprived of
sexual intercourse with you I am extremely distressed. How can I bear the pangs
of passion?

Sūta said:

25. That base brahmin Binduga, when addressed thus by his wife, foolish and
averse to his own duties said to her.

Binduga said:

26. True indeed is what you have said with your mind agitated by passion. Please
listen, my dear wife, I shall tell you something that will be of benefit to you.
You need not be afraid.

27. You go ahead with your sexual sports with any number of paramours. No fear
need enter your mind. Extract as much of wealth as you can from them and give
them enough sexual pleasure.

28. You must hand over all the amount to me. You know that I am enamoured of my
concubine. Thus our mutual interests will be assured.

Sūta said:

29. His wife Cañculā on hearing these words of her husband became extremely
delighted and assented to his vicious proposal.

30. Having thus entered into their nefarious mutual contract the two wicked
persons—the husband and the wife—fearlessly went ahead with their evil actions.

31. A great deal of time was thus wasted by the foolish couple indulging in
their vicious activities.

32. The wicked Binduga, the brahmin with a \tl{śūdra} woman for his concubine,
died after some years and fell into Hell.

33. The foolish fellow endured distress and torture in Hell for many days. He
then became a ghost in the Vindhya mountain range continuing to be terribly
sinful.

34-35. After the death of her husband the wicked Binduga, the woman Cañculā
continued to stay in her house with her sons. The woman foolishly continued her
amorous dalliance with her paramours till she no longer retained her youthful
charms.

36. Due to divine intercession it chanced that on an auspicious occasion she
happened to go to the Gokarṇa\footnote{Gokarṇa: lit. ‘cow’s ear’. It is a place
of pilgrimage sacred to Śiva, on the west coast, near Mangalore. It has the
temple of Mahādeva, supposed to have been established by Rāvaṇa.} temple in the
company of her kinsmen.

37. Casually moving about here and there with her kinsmen she happened to take
her bath in a holy pond as a normal routine affair.

38. In a certain temple a scholar of divine wisdom was conducting a discourse on
the holy \tl{Śivapurāṇa} story some of which she happened to hear.

39-40. The portion that fell on her ears was the context in which it was said
that the servants of Yama would introduce a red hot iron into the vaginal
passage of women who indulge in sexual intercourse with their paramours. This
narrative made by the Paurāṇika to increase detachment, made the woman tremble
with fear.

41. At the end of the discourse when all the people dispersed, the terrified
woman approached the scholarly brahmin and spoke to him in confidence.

Cañculā said:

42. O noble sir, please listen to the ignoble activities which I performed
without knowing my real duties. O lord, on hearing the same you will please take
pity on me and lift me up.

43. O lord, with a mind utterly deluded I have committed very great sin. Blinded
by lust I spent the whole of my youth in incontinent prostitution.

44. Today on hearing your learned discourse abounding in the sentiments of
non-attachment I have become extremely terrified and I tremble much.

45. Fie upon me, the foolish sinner of a woman deluded by lust, censurable,
clinging to worldly pleasures and averse to my own duties.

46. Unknowingly a great sin that produces excessive distress has been committed
by me for a fleeting glimpse of an evanescent pleasure, a criminal action.

47. Alas, I do not know which terrible goal this will lead me to. My mind has
always been turned to evil ways. Which wise man will come to my succour there?

48. At the time of death how shall I face the terrible messengers of Yama? How
shall I feel when they tie nooses forcibly round my neck?

49. How shall I endure in Hell the mincing of my body to pieces? How shall I
endure the special torture that is excessively painful?

50. I bewail my lot. How can I peacefully proceed with the activity of my
sense-organs during the day? Agitated with misery how shall I get peaceful sleep
during the night?

51. Alas! I am undone! I am burnt down! My heart is torn to pieces! I am doomed
in every respect. I am a sinner of all sorts.

52. O adverse Fate! it was you who directed my mind along evil lines. With a
hateful stubbornness you made me commit great sins. I was led astray from the
path of my duty that would have bestowed all happiness.

53. O Brahmin, my present pain is millions of times more than that of a man
stuck to the stake or hurled from a high mountain-top.

54. My sin is so great that it cannot be washed away even if I take ablutions in
the Gaṅgā for a hundred years or even if I perform a hundred sacrifices.

55. What shall I do? Where shall I go? Whom shall I resort to? I am falling into
the ocean of Hell. Who can save me in this world?

56. O noble sir, thou art my preceptor. Thou art my mother. Thou art my father.
I seek refuge in Thee. I am in a pitiable plight. Lift me; lift me.

Sūta said:

The intelligent brahmin mercifully lifted up Cañculā who had become disgusted
(with worldly affairs) and had fallen at his feet. That brahmin then spoke
(as follows).
