\phantomsection
\addcontentsline{toc}{subsection}{The liberation of Devarāja}
\subsection*{The liberation of Devarāja}

Śaunaka said:

1. O Sūta, thou art the most blessed and the most fortunate knower of the
greatest Truth. Thou hast narrated to us, out of great compassion, this divine
wonderful tale.

2. This wonderful narrative that destroys hosts of sins, purifies the mind,
and propitiates Lord Śiva has been heard by us.

3. Thanks to thy compassion we have decisively realised that there is nothing so
fine and nice as this tale.

4. Who are those among sinners in the Kali age who get sanctified by this story?
Please enlighten us. Make the whole world gratified.

Sūta said:

5. Men who habitually commit sins, wicked persons indulging in vicious
activities and persons of lecherous disposition become pure hereby.

6. This is a great \tl{jñānayajña} (sacrificial rite of wisdom); it yields
worldly enjoyment as well as salvation; it dispels all sins and delights Śiva.

7. Men overwhelmed by the thirst of covetousness, those devoid of truthfulness,
those who decry even their parents, haughty vain fellows and persons prone to
violent activities become sanctified by this.

8. Those who never practise the duties of their \tl{varṇas} and \tl{āśramas} and
those of malicious temperament become sanctified thanks to the \tl{jñānayajña}
even in the Kali age.

9. Those who habitually practise deception and those who are ruthless and of
cruel disposition are sanctified by this \tl{jñānayajña} even in the Kali age.

10. Those who misappropriate the wealth of brahmins and thereby nourish
themselves and those who indulge in heinous crimes of adultery become sanctified
by this \tl{jñānayajña} even in the Kali age.

11. Those who always indulge in sinful actions and those who are roguish persons
of wicked mind become sanctified by this \tl{jñānayajña} even in the Kali age.

12. Men of unclean habits and wicked minds, men who know no peace and men who
swallow temple and trust properties become sanctified by this \tl{jñānayajña}
even in the Kali age.

13. The merit accruing from this \tl{purāṇa} destroys great sins, yields worldly
enjoyments and salvation and delights Lord Śiva.

14. In this context an ancient anecdote is cited as an example, the mere hearing
of which, removes all sins utterly.

15. In the city of Kirātas there lived a brahmin extremely poor and deficient in
(brahmanical) knowledge. He used to sell various kinds of beverage and was
averse to the worship of gods or to virtuous activities.

16. He never practised the daily \tl{sandhyā} prayers or ablutions. His practice
resembled a \tl{vaiśya’s} mode of living. He never hesitated to deceive
credulous persons. His name was Devarāja.

17. Either by killing or by using various deceitful means he used to rob
brahmins, kṣatriyas, vaiśyas, śūdras and others.

18. Thus by foul means much wealth was later accumulated by him. But the sinner
that he was, not even the slightest part of his wealth was utilised in virtuous
acts.

19. Once that brahmin went to a lake to take his bath. There he saw a harlot
called Śobhāvatī and was much agitated at her sight.

20. The beautiful woman was extremely delighted on coming to know that a rich
brahmin had become her willing slave. The brahmin’s heart was filled with love
due to her pleasant talk.

21. He decided to make her his wife and she consented to have him as her
husband. Thus in mutual love they sported for a long time.

22. Sitting, lying, eating, drinking and playing together they were not at all
different from any other wedded couple.

23. Dissuaded again and again by his mother, father, first wife and others
though he was, he never paid heed to their words but continued his sinful
activities.

24. Once he became so enraged as to kill his mother, father and wedded wife at
dead of night while they were asleep and took possession of their wealth.

25. Enamoured of the courtesan he handed over to her his own wealth and also
the wealth that he looted from his father, mother and first wife.

26. In the company of this harlot he used to eat all sorts of forbidden food,
became an addict to wine and spirituous liquors and partook of his food from
the same plate as his concubine.

27. Once, by chance, he came to the city of Pratiṣṭhāna.He saw a Śiva temple
where saintly men had congregated.

28. During his stay there, he was afflicted by an acute fever. He heard
the discourse on Śiva conducted by a brahmin.

29. The brahmin Devarāja suffering from fever died at the end of a month. He was
bound with nooses by Yama’s attendants and forcibly taken to Yama’s city.

30-33. In the mean while Śiva’s attendants dressed in white, smeared with ashes
all over the body, wearing garlands of Rudrākṣa and wielding tridents in their
hands started furiously from Śivaloka and reached Yama’s city. They threatened
the attendants of Yama (the God of death) and thrashed them. Releasing Devarāja
from their clutches they seated in a wonderful aerial chariot. When they were
about to start to Kailāsa a great tumult arose in the middle of Yama’s city on
hearing which Dharmarāja (the God of Death) himself came out of his palace.

34. On seeing the four messengers who appeared like replicas of Rudra Himself,
Dharmarāja the knower of virtues honoured them in accordance with the custom.

35. Yama came to know of everything through his vision of wisdom. Out of fear he
did not question the noble attendants of Śiva.

36. Being duly honoured and adored by Yama, they went to Kailāsa and handed over
the brahmin to Śiva, the very ocean of mercy and to the divine mother Pārvatī.

37. Blessed indeed is the story of Śivapurāṇa, the holiest of holy stories,
a mere hearing of which qualifies even the greatest sinner for salvation.

38. The great seat of Sadāśiva is the greatest abode and the noblest of
positions which Vedic scholars have extolled as stationed above all \tl{lokas}
(worlds).

39-40. Devarāja the base brahmin, addicted to wine, enamoured of a vile harlot,
slayer of his own father, mother and wife and who out of greed for money had
killed many brahmins, kṣatriyas, vaiśyas and śūdras and others became a
liberated soul instantaneously on reaching that supreme \tl{Loka}.
