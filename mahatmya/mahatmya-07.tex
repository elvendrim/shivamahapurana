\phantomsection
\addcontentsline{toc}{subsection}{The rite of listening to Śivapurāṇa}
\subsection*{The rite of listening to Śivapurāṇa: Injunctions and prohibitions}

Śaunaka said:

1-2. O Sūta, Sūta of great intellect, thou art foremost among devotees of Śiva
and the most blessed. Thou hast narrated this wonderfully auspicious story, O
sage, please tell me the rules governing those who perform the rite of listening
to \tl{Śivapurāṇa}, for the benefit of the whole world.

Sūta said:

3. O Śaunaka, listen with devotion to the rules governing those persons. If you
hear the excellent story with due observance of the rules, the fruit is
excellent and there is no obstacle in the achievement of the fruit.

4. Persons devoid of initiation are not entitled to listen to the story. Hence
those who wish to listen must take initiation, O sage, from the discourser.

5. The devotee who takes up this rite shall take his daily meal only at the end
of the daily discourse. He must observe \tl{brahmacarya} (celibacy) during those
days. He must lie on the ground and take food only in the \tl{patrāvallī} (a
number of leaves stitched together to serve the purpose of a plate).

6. The man who has the strength in abundance shall observe fast till the
conclusion of the whole \tl{purāṇa} and listen to the excellent \tl{Śivapurāṇa}
with great devotion and purity.

7. He may drink only milk or ghee throughout and listen to the story with
pleasure. He may live on fruit diet or take a single meal or even eschew that
and proceed with the listening rite.

8. Or he may take \tl{haviṣyānna} (cooked rice soaked in ghee and sacrificially
offered) once a day and maintain the rite. The diet part is according to
convenience and comfort but the listening shall be strictly maintained.

9. If there is more facility in hearing let the devotee take food. If observing
fast causes hindrance to listening to the story it is not to be recommended.

10-12. The householder taking the rite shall avoid heavy indigestible pulses
like Niṣpāva, Masūrikā \etc, stale food, defiled food, brinjals, gourds, radish,
pumpkins, cocoa-nuts, garlic, onion, asafoetida, intoxicating beverages and all
kinds of meat.

13. He shall avoid the six base feelings of lust, anger \etc, he shall not
despise brahmins and bear ill will towards chaste ladies and good men.

14. He shall not look at women in their menstrual period. He shall not converse
with fallen people, nor talk to haters of brahmins or unbelievers in the Vedas.

15. The house-holder shall practise and strictly adhere to truthfulness, purity,
mercy, restraint in speech, straightforwardness, humility, liberalmindedness and
other virtues.

16. The householder may listen to the story with any specific desire cherished
in his mind or absolutely free from any desire. If he has any desire it will be
fulfilled; if he is free from desire he shall attain salvation.

17. An indigent person, a consumptive, a sinner, an unfortunate person and a
person having no child shall hear this excellent story.

18. The seven types of wicked women like \tl{kākavandhyā} (a woman having a
single child) and those suffering from miscarriages shall hear this story.

19. Whether women or men, all must hear the story of \tl{Śivapurāṇa}, O sage,
in the manner prescribed.

20. The days of discourse on \tl{Śivapurāṇa} must be considered very excellent,
even on a par with millions of sacrifices.

21. Gifts duly bestowed on these excellent days, even though they may not be
much in quantity, yield everlasting benefit.

22. Observing the rites thus, and listening to the great story the flourishing
house-holder shall delightedly perform the \tl{udyāpana} rite (at the end of
completion).

23. This \tl{udyāpana} rite is on a par with the \tl{caturdaśī} rite (observed
on the fourteenth day of the lunar month). Rich men who wish to secure the
fruits thereof must perform it likewise.

24. Indigent devotees usually do not and need not perform the \tl{udyāpana} rite.
They are sanctified by the listening alone. Pious devotees of Śiva are free
from desires.

25. After the festive celebration of the sacrifice of the discourse on
\tl{Śivapurāṇa} is thus concluded, the listeners shall perform the worship.

26. O sage, due worship must be performed in front of the book in the manner of
the worship of Śiva.

27. A fine new cloth to cover the book and a strong silken cord to tie it up
must be given.

28. Those who give silken cord and new cloth for the book of \tl{purāṇa} become
\tl{yogins} endowed with knowledge in every birth they take.

29. Many kinds of valuable objects, cloth, ornaments, vessels and much wealth in
particular should be given to the discourser.

30-31. Those who give carpets, deer skins, cloth, elevated couches and planks to
keep the volume of \tl{purāṇa} on, attain heaven, enjoy all desirable pleasures,
stay in Brahmā’s region for the duration of a \tl{kalpa} and finally attain
Śiva’s region.

32-33. After performing the worship of the book as stipulated, O foremost among
sages, and also that of the dis-courser with great eclat, the scholar who had
been appointed assistant should be duly honoured in the same manner but with a
smaller sum of money.

34. Food and monetary gifts and other things must be given to the brahmin
visitors. A great festival must be celebrated with vocal and instrumental music
and performance of dances.

35. The listener shall gradually become detached and especially on the next day,
O sage, the holy \tl{gitā} narrated by Śiva to Rāmacandra must be read.

36. If the listener is a householder he must perform \tl{homa} with pure
\tl{havis} (holy ghee) for tranquilising the rite.

37. The \tl{homa} must be performed with \tl{Rudrasaṃhitā} or with each verse of
\tl{Gāyatrī}, for in fact, this \tl{purāṇa} is identical with it,

38. or with the \tl{mūlamantra} of Śiva of five syllables. If he is incompetent
to perform \tl{homa} let him give the ghee offering to a brahmin.

39. In order to suppress the defects of deficiency and excess he shall either
read or listen with devotion to the thousand names of Śiva.

40. Undoubtedly, thus, every thing shall be fruitful and the fruit too shall be
excellent since there is no greater in the three worlds than this.

41. He shall feed eleven brahmins with honey and milk puddings. He must give
them \tl{dakṣiṇā} also to complete the rite.

42-44. If he is competent, O sage, he must make an image of a lion with three
\tl{palas} of gold and either engrave the name of this \tl{purāṇa} on it or
affix a label with the name written on it. He must worship his preceptor of
great restraint with the gifts of cloth, ornaments, scents, \etc, and hand them
over to him for propitiating Śiva.

45. O Śaunaka, by the power of this gift and of the \tl{purāṇa} he shall secure
the blessings of Śiva and be freed from the bondage of worldly existence.

46. If these rites are performed, the \tl{Śivapurāṇa} shall yield entire fruit,
enjoyment of worldly pleasures and salvation.

47. Thus I have narrated to you the greatness of \tl{Śivapurāṇa} that bestows
every cherished desire. What else do you wish to hear?

48. The \tl{Śivapurāṇa} holds the mark of distinction among all \tl{purāṇas}.
It is highly pleasing to Śiva. It wards off the ailment of worldly existence.

49. Those who are always engaged in the meditation of Śiva, those whose tongue
adores the attributes of Śiva, and those whose ears listen to the story of Śiva,
cross the ocean of worldly existence.

50. I seek refuge in Śiva the great, of infinite thickset bliss, Śiva whose form
is unaffected by all the three \tl{guṇas}, Śiva who manifests Himself within and
without this world, within and without the mind, Śiva whose form is variously
evolved by mental ideas and verbal expressions.
\clearpage
