\chapter{The glorification of the \emph{Rudrākṣa} and of the names of Śiva}

The sages said:

1-2. O Sūta, Sūta the fortunate disciple of Vyāsa, obeisance to thee. Please
explain again the glorification of the holy ashes, of the \tl{rudrākṣa} and of
Śiva’s names. Lovingly explaining the three, please delight our minds.

Sūta said:

3-4. It is good that you have referred to this matter that is highly beneficent
to the world. You are blessed, holy and ornaments to your families since you
own Śiva as your sole great favourite deity. The anecdotes of Śiva are dear to
you all for ever.

5. Those who adore Śiva are blessed and content. Their birth is fruitful and
their family is elevated.

6. Sins never touch those from whose mouth the names Sadāśiva, Śiva \etc come
out for ever, as they do not touch the burning charcoal of the khadira wood.

7. When a mouth utters “Obeisance to Thee, holy Śiva” that mouth (face) is on
a par with holy centres destroying all sins.

8. It is certain that the benefit of making pilgrimages to holy centres accrues
to one who lovingly looks at His holy face.

9. O \tl{brāhmaṇas}, the place where these three are found is the most
auspicious one. A mere contact of the place accords the benefit of taking a holy
dip in the sacred Triveṇī.

10. Śiva’s name, the ashes and the \tl{rudrākṣa} beads—the three are very holy
and are on a par with Triveṇī\footnote{The place of confluence (Prayāga, now
Allahabad) of the Ganges with the Yamunā and the subterranean Sarasvatī.}
(the confluence of the three holy rivers).

11. The sight of the persons who have these three in their bodies is a rare
occurrence. But when obtained it removes all sins.

12. There is no difference at all between these two—a sight of the holy man and
a bath in the Triveṇī. He who does not realise this is undoubtedly a sinner.

13. The man who has no ashes on his forehead, has not worn \tl{rudrākṣa} on his
body and does not utter names of Śiva shall be shunned as one does a base man.

14. As said by Brahmā, Śiva’s name is on a par with Gaṅgā, the ash is equal to
Yamunā and \tl{rudrākṣa} destroys all sins (and is equal to Sarasvatī).

15-16. Brahmā wishing to bestow beneficence weighed one against the other. He
put on one side the benefit achieved by a person in whose body the three things
were present. On the other side he put the blessedness achieved by those who
took their bath in the holy Triveṇī. Both were found equal. Hence scholars shall
wear these always.

17. From that time onwards Brahmā, Viṣṇu and other \tl{devas} wear these three.
Their very sight dispels sins.

The sages said:

18. O righteous one, you have explained the benefit of the three things: Śiva’s
name \etc Please explain it vividly.

Sūta said:

19. O brahmanical sages, you are all good devotees of Śiva, gifted with
knowledge and great intellect. You are the foremost among the wise. Please
listen with reverence to their greatness.

20. O \tl{brāhmaṇas}, it is mysteriously hidden in sacred texts, Vedas and
Purāṇas. Out of love for you I reveal the same to you now.

21. O foremost among the \tl{brāhmaṇas}! Who ever does know the real greatness
of the three except Śiva who is beyond all in the whole universe?

22. Briefly I shall explain the greatness of the names as prompted by my
devotion. O \tl{brāhmaṇas}, do you lovingly listen to his greatness:
the destroyer of all sins.

23. Mountainous heaps of great sins are destroyed as in a blazing forest fire
when the names of Śiva are repeated. They are reduced to ashes without any
difficulty. It is true, undoubtedly true.

24. O Śaunaka, different sorts of miseries with sins as their roots can be
quelled only by muttering Śiva’s names, and not by anything else entirely.

25. The man who is devotedly attached to the \tl{japas} of Śiva’s names in
the world, is really a follower of the Vedas, a meritorious soul and a blessed
scholar.

26. O sage, instantaneously fruitful are the different sacred rites of those who
have full faith in the efficacy of the \tl{japas} of Śiva’s names.

27. O sage, so many sins are not committed by men in the world as are and can be
destroyed by Śiva’s names.

28. O sage, Śiva’s names repeated by men, immediately destroy the countless
heaps of sins such as the slaughter of a \tl{brāhmaṇa}.

29. Those who cross the ocean of worldly existence by resorting to the raft of
the names of Śiva do definitely destroy those sins that are the root-cause of
worldly existence.

30. O great sage, the destruction of sins that are the roots of worldly
existence is certainly effected by the axe of Śiva’s names.

31. The nectar of Śiva’s names shall be drunk by those who are distressed and
scorched by the conflagration of sins. Without it, the people who are scorched
by the conflagration cannot have any peace.

32. Those who are drenched by the nectarine downpour of Śiva’s names never feel
ill at ease even in the middle of the conflagration of worldly existence.

33. The noble souls who have acquired great devotion to the names of Śiva, and
those like them, attain perfect liberation instantaneously.

34. O lord of sages, devotion to the names of Śiva, that destroys all sins can
be acquired only by him who has performed penances in the course of many births.

35. Salvation is easy of access only to him who has extraordinary and unbroken
devotion for the names of Śiva. I believe in this.

36. Even if he has committed many sins, a person who has reverence for
the \tl{japa} of Śiva’s names, becomes certainly free from all sins.

37. Just as the trees in a forest are burnt and reduced to ashes by the forest
fire, so also are the sins destroyed by Śiva’s names.

38. O Śaunaka, he who regularly sanctifies his body by the holy ashes and who
performs the \tl{japa} of Śiva’s names crosses even the terrible ocean of
worldly existence.

39. A person who undertakes the \tl{japa} of Śiva’s names is not sullied by sins
even after misappropriating a \tl{brāhmaṇa’s} wealth and killing many
\tl{brāhmaṇas}.

40. After going through all the Vedas it has been decided by our ancestors that
the noblest means of crossing the ocean of worldly existence is the performance
of the \tl{japa} of Śiva’s names.

41. O excellent sages, why should I say much? By means of a single verse I shall
mention the greatness and efficacy of the names of Śiva or the destruction of
all sins.

42. The power of the names of Śiva in destroying sins is more than the ability
of men to commit them.

43. O sage, formerly the king Indradyumna who was a great sinner, attained
the excellent goal of the good through the influence of Śiva’s names.

44. O sage, similarly a \tl{brāhmaṇa} woman too of very sinful activities
attained the excellent goal of the good through the influence of Śiva’s names.

45. O excellent \tl{brāhmaṇas}, thus I have told you about the surpassing
excellence of the names. Now please listen to the greatness of holy ashes,
the most sacred of all.
