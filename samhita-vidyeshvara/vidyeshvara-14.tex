\chapter{Description of the Fire-sacrifice (\emph{agniyajña})}

The sages said:

1. O lord, please tell us in order in detail all these rites \viz the fire
sacrifice, the sacrifice to gods, \tl{brahmayajña}, the worship of the preceptor
and the gratification of \tl{brāhmaṇas}.

Sūta said:

2-3. The offering made into the fire is called fire-sacrifice (\tl{agniyajña}).
In the case of persons in the \tl{brahmacarya āśrama} (\ie religious students)
it is called \tl{samidādhāna} (collection of sacrificial twigs). O \tl{brāhmaṇas},
until the rite of \tl{aupāsana} (fire sacrifice of the householder) all
the persons in the first \tl{āśrama} perform their \tl{vratas} and special
sacrifices in the fire from sacrificial twigs.

4. O \tl{brāhmaṇas}, in the case of ascetics and forest-dwellers who have
consigned the sacred fire to the Ātman, taking a restricted quantity of
wholesome food is itself the sacrificial offering.

5. Householders who have started their \tl{aupāsana} rite shall maintain
the rite in the sacrificial fire kept in a vessel or pit always.

6. The sacrificial fire shall be maintained either in the Ātman or in
the \tl{araṇī} (the sacrificial churning twig from which fire is kindled) lest
the fire should be extinguished by royal or divine intercession.

7. O \tl{brāhmaṇas}, the offering in the fire in the evening for the fire-god is
the bestower of prosperity. The offering in the morning for the sun-god is
conducive to longevity.

8-9. This is called \tl{agniyajña} in as much as it enters the sun during
the day. The different sacrifices \tl{sthālīpāka} \etc for the propitiation of
Indra and other gods by offerings in the fire are called \tl{devayajña}.
The rites of \tl{caula} (ceremony of tonsure) \etc are performed in the ordinary
fire.

10. The regular study of the Vedas is called \tl{brahmayajña}. A \tl{brāhmaṇa}
shall perform this constantly for the propitiation of gods.

11. This is to be practised by all and hence no special rules are prescribed
here. Now attend to the explanation of certain \tl{devayajñas} without fire.

12. At the beginning of the first creation, the omniscient, merciful lord
Mahādeva created the different week days for the benefit of the entire world.

13. Lord Mahādeva, the global physician, the omniscient, the panacea of all
panaceas, made the first day his own day that bestows good health.

14-17. Next he created the day of his Māyā (Illusion) the bestower of
prosperity. Afterwards when the birth of Kumāra was attended with some mishaps
he created the day for the sake of surmounting mishaps and idleness. With
a desire to bless the worlds and for their nurture and protection he created
the next day dedicated to Viṣṇu, the protector of the worlds. The next day
created by the lord is for the sake of the longevity of the worlds dedicated to
the creator of the three worlds, Brahmā, called also Parameṣṭhin, who is
the bestower of longevity too. Hence this day too bestows longevity.

18. The last two days of the week created by the lord are those of Indra and
Yama. In the beginning when the lord created Puṇya and Pāpa (Virtue and Sin) for
making the three worlds flourish, these deities who preside over them were
assigned these two days.

19-22. The last two days are the bestowers of worldly enjoyments and removers of
premature death respectively. The lord made the sun \etc who are His own
manifestations and are firmly established in the solar cycle. It is said that
the respective merits of the different days are secured through
the gratification of the gods. Śiva is the ultimate bestower of the fruits
accruing from the worship of other gods as well.

23-24. The worship for the propitiation of the deities is fivefold:
(1) the repeated recitation of the respective mantras, (2) sacrifice, (3)
charitable gift, (4) austerities and (5) propitiation on the altar, idol, fire
or a \tl{brāhmaṇa}. The sixteen forms of service and homage shall be duly
observed.

25-26. Of the fivefold forms of worship the latter are more efficacious than
the former. In the absence of the earlier ones the latter ones can be observed.
In the ailments of the eyes or head or for quelling leprosy, the sun shall be
worshipped and the \tl{brāhmaṇas} fed for a day, a month, a year or three years.

27-28. If the action meritorious or otherwise that has begun to fructify is
sufficiently strong, the ailment, old age \etc are alleviated. The repetition of
the mantras of the favourite deity accords the respective benefits of the day of
the week. The first day of the week dedicated to the sun has the special merit
of the removal of sin, especially for \tl{brāhmaṇas}.

29. For the sake of riches, the intelligent devotee shall worship Lakṣmī \etc
on Monday with cooked rice soaked in ghee and shall feed \tl{brāhmaṇa} couples.

30. For alleviating ailments the devotee shall worship Kāli and others on
Tuesday. He shall feed \tl{brāhmaṇas} with an \tl{āḍhaka} (a measure) of cooked
rice, the pulse, black gram and green gram.

31. The scholarly devotee shall worship Viṣṇu with curd-rice on Wednesday. Sons,
friends, womenfolk \etc will always be well-nourished for ever.

32. A person who seeks longevity shall worship the deities for their
gratification, with sacred thread, cloth, milk and ghee on Thursday.

33. On Friday, for the sake of enjoyment of worldly pleasures, the devotee shall
worship \tl{devas} with concentration. \tl{Brāhmaṇas} should be propitiated with
the cooked food consisting of six flavours\footnote{Six flavours are: (1)
pungent, (2) sour, (3) sweet, (4) salt, (5) bitter and (6) astringent.}.

34-35. Good cloth should be presented to women to gladden them. The wise devotee
shall worship Rudra and others on Saturday that wards off premature death, by
performing \tl{homa} with gingelly seeds. He shall make gifts to
the \tl{brāhmaṇas} and feed them with cooked rice and gingelly seeds. Thus
worshipping the deities he shall derive the fruit of good health \etc

36-38. In the daily or special sacrifices of the deities, ceremonial ablutions,
charitable gifts, repeated muttering of mantras, sacrifices, propitiation of
the \tl{brāhmaṇas}, in the worship of the different \tl{devas} in view of
special dates or special conjunction of the planets, or in the different days of
the week it is the omniscient lord of the universe who bestows health and other
benefits by assuming the different forms. He bestows the same according to
the time, place and the deserts of the recipient.

39. The articles for worship shall be in accordance with one’s faith or local
conventions. The lord bestows health \etc in accordance with the comparative
quality of the same.

40. In the beginning of the period of auspiciousness, the end of the period of
inauspiciousness, on birth days (according to the stars) \etc the householder
shall worship the planets, Sun \etc in his own house for his good health \etc

41. Hence the worship of gods bestows all desired fruits. The worship conducted
by \tl{brāhmaṇas} must be along with mantras and by means of gesticulations in
the case of others.

42. The worship shall be carried out by men seeking good benefits in all
the seven days in accordance with their capacity.

43. Indigent men shall worship \tl{devas} with austerities and rich men by
spending money. Again and again they shall do virtuous actions with sufficient
faith.

44-46. After enjoying the pleasures in heaven they are reborn again in
the world. For better enjoyment the rich shall always plant trees for shade,
dig tanks \etc install deities, and carry on virtuous activities. After
the lapse of some time, when the virtue becomes ripe he shall achieve perfect
knowledge. O \tl{brāhmaṇas}, he who hears this chapter, or reads it or he who
facilitates the hearing of the same shall derive the fruit of \tl{devayajña}.
