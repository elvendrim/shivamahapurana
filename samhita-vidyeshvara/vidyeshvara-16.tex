\chapter{Different modes of worship of clay idols and their results}

The sages said:

1. O excellent one, please explain the rules of the worship of clay idols by
following which all desired results will be achieved.

Sūta said:

2. You have requested for a very good thing. It bestows all wealth always. It
suppresses misery instantaneously. I shall explain it. Please listen.

3-4. It wards off premature and foul death. Even a timely death it prevents.
O \tl{brāhmaṇas}, it bestows womenfolk, sons, wealth, grains \etc The worship of
idols made of clay \etc is conducive to the attainment of all cherished desires
in the world. From it the devotee derives food and other edible things,
cloth \etc

5. Both men and women are authorized in this. The clay should be brought from
the beds of rivers, lakes or wells.

6. It should be washed well and pasted with scented powder and milk. The idol
should be made with the hands on a raised platform.

7. All the limbs, joints \etc should be perfectly shaped with the respective
weapons of the deity concerned. It should be seated on \tl{padma āsana}
(the lotus pose) and worshipped respectfully.

8. The five deities Gaṇeśa, Sun, Viṣṇu, Pārvatī and Śiva shall be usually
worshipped in their images. But a \tl{brāhmaṇa} shall always worship the phallic
emblem of Śiva.

9. In order to derive the full benefit of worship, the sixteen forms of service
shall be observed. The sprinkling of water over the idol shall be performed with
flowers. The pouring of water shall be performed with mantras.

10-11. The food offering shall consist of cooked rice of \tl{śāli} variety. In
the worship conducted in the house, 12 handfuls of rice (= \tl{kudava}) shall be
used. In the worship in a temple constructed by men, a \tl{prastha}
(a particular measure) of cooked rice shall be used. In a divine temple three
\tl{prasthas} of cooked rice shall be used. In the worship of self-risen image
five \tl{prasthas} of cooked rice shall be used. If thus used it gives complete
benefit. By using twice or thrice this quantity the benefit shall be greater.

12-15. By performing this worship a thousand times, a \tl{brāhmaṇa} shall attain
Satyaloka. A vessel made of wood or iron twelve \tl{aṅgulas} in width, 24
\tl{aṅgulas} in length and sixteen \tl{aṅgulas} in height is called Śiva. An
eighth part of it is called a \tl{prastha} and it is equal to four \tl{kuḍavas}.
If ten, hundred or thousand \tl{prasthas} of water, oil, incense \etc are used
in temples of human construction, of saintly worship or of self-risen idol,
the worship is called \tl{mahāpūjā}.

16. The ceremonial bath is conducive to the purity of the soul; the application
of scented paste yields virtue. The food offering is conducive to longevity and
gratification and the incense yields wealth.

17. The lighting of the lamp is conducive to knowledge and the betel leaves are
conducive to enjoyment. Hence in all worships these six items are scrupulously
observed.

18. Obeisance to the deity and repeated recitation of mantras accord all
cherished desires. They must be observed at the end of the worship by men who
seek both worldly enjoyment and salvation.

19. At first all items shall be gone through mentally and then item by item
every rite shall be performed. By the worship of deities, the devotee attains
the different regions.

20. In the subsidiary worlds also there is an ample scope for enjoyment.
O \tl{brāhmaṇas}, I shall narrate the special types of worship to which please
listen with faith.

21-22. By the worship of Gaṇeśa the devotee shall attain his wish in this world
itself. The days of special worship of Gaṇeśa are Fridays, the fourth day of
the bright half of the lunar months of Śrāvaṇa and Bhādrapada, and
the Śatabhiṣak star of the month of Dhanus. He shall be worshipped duly on
these days. Or the devotee shall worship continuously for hundred or thousand
days.

23. As a result of the faith in the deity and in the fire, the worship yields
sons or the different wishes to the devotees. It quells all sins and the various
hardships.

24. The worship of Śiva and others on their respective days of the week is
conducive to the purity of soul. In regard to \tl{kāmya} rites, the basis is
either the Tithi or the star or the particular combinations of planetary
positions.

25. The day of the week is the basis for the worship of Brahman and others.
There is no increase or decrease with respect to the days of the week as in
regard to the Tithi, star \etc A day is calculated from sunrise to sunrise.

26-28. The worship of the deities on the respective Tithis \etc is conducive to
full enjoyment for the devotees. In regard to rites of the manes, the earlier
part must be in contact with the night previous. In the worship of deities
the latter part must be in conjunction with the day. If the Tithi extends to
mid-day, that part of it which falls at sunrise shall be taken for the worship
of the deities, so also in regard to the stars. Hence a devotee shall consider
all these aspects and proceed with the worship, repeated recitation of
the mantras \etc

29-30. The word \tl{pūjā} is thus derived: \tl{Pūḥ} means “the achievement of
the fruits of enjoyment”. By the rite one achieves the fruits. \tl{Jāyate} means
“is born”. Good ideas, knowledge \etc also are included in this. The word
\tl{pūjā} is used in this sense amongst the people as well as in the sacred
texts.

31-32. The daily and occasional rites yield their benefits in due course but
the fruits of \tl{kāmya} rites are instantaneous. The necessary rites are
performed everyday. The occasional rites are performed in particular months,
fortnights, years or on special occasions. In the \tl{kāmya} rites one derives
the fruits after the sin has been duly quelled. Mahāgaṇapati \tl{pūjā} shall be
performed on the \tl{caturthī} day of the dark half of the lunar month.

33. That rite wipes off the sin of the whole fortnight and yields enjoyment for
full fortnight. The worship performed on the \tl{caturthī} day of the lunar
month of Cakra accords benefit for a month.

34-36. The worship performed in the months of Siṃha and Bhādrapada accords
enjoyment of worldly pleasures for a year. The worship of the sun shall be
performed on Sundays, or \tl{saptamī} (seventh) day or in the star Hasta of
the month of Śrāvaṇa or on the \tl{saptamī} in the bright half of the month
of Māgha. The worship of Viṣṇu is conducive to the attainment of all desires
and wealth if performed on Wednesdays, \tl{dvādaśī} (12\th) day or in the star
of Śravaṇa in the months of Jyeṣṭha and Bhādrapada. The same worship in
the month of Śrāvaṇa yields all desired wishes and good health.

37. Propitiation of Viṣṇu on the \tl{dvādaśī} day yields the same benefit as is
derived from the gift of the twelve things with ancillary rites.

38. The devotee shall worship twelve \tl{brāhmaṇas} on the \tl{dvādaśī} day
assigning them the twelve names of Viṣṇu with all the sixteen forms of service.
He shall gratify the deity thereby.

39. Similarly twelve \tl{brāhmaṇas} shall he worshipped after assigning them
the twelve names of any deity to gratify that deity.

40. A person who seeks prosperity shall worship Pārvatī who bestows all worldly
pleasures on Mondays, \tl{navamī} (ninth) day, and in the star of Mṛgaśiras in
the month of Karkaṭaka.

41-42. The \tl{navamī} in the bright half of the month of Āśvayuj accords all
desired benefits. The worship of Śiva shall be performed on Sundays,
\tl{caturdaśī} (fourteenth) day of the dark half of the month of Māgha on
the Ārdrā star and on the Mahārdrā day. It accords all cherished desires.

43-45. The worship is conducive to longevity, prevents premature death and
accords the achievement of everything. The worship of the different
manifestations of Śiva with all sixteen forms of service and homage on
the Mahārdrā day in the month of Jyeṣṭha, on \tl{caturdaśī} day or on the Ārdrā
day in the month of Mārgaśīrṣa is on a par with Śiva’s worship and yields
worldly enjoyment and salvation. The worship of the first deity of the week days
in the month of Kārtika is specially recommended.

46-47. When the month of Kārtika has arrived, the sensible man shall worship all
the deities by giving gifts and observing austerities, \tl{homas}, \tl{japas},
restraints and the sixteen forms of service. The idol shall be worshipped with
mantras. \tl{Brāhmaṇas} shall be fed. The devotee shall be freed of desires and
distresses.

48. The worship of deities in the month of Kārtika yields all worldly pleasures,
dispels all ailments and removes the adverse effects of spirits and evil planets.

49. The worship of the sun on Sundays in the month of Kārtika together with
the gifts of gingelly seeds and cotton alleviates leprosy \etc

50. By making gifts of \tl{harītakī} (one of the myrobalans), chillies, cloth,
milk \etc and by installing Brahman, the alleviation of consumption is brought
about.

51-53. By making gifts of lamps and mustard seeds epileptic fits are alleviated.
The worship of Śiva on Mondays in the month of Kārtika suppresses excessive
poverty and increases prosperity. The worship of Skanda on Tuesdays in the month
of Kārtika, and making gifts of houses, fields, domestic articles and utensils,
lamps, bells \etc the devotee gains eloquence without delay.

54. The worship of Viṣṇu on Wednesdays in the month of Kārtika together with
the gift of cooked rice with curds yields good progeny.

55. The worship of Brahman on Thursdays in the month of Kārtika and the gift of
honey, gold and ghee affords the increase of worldly pleasures.

56. The worship of the elephant-faced Gaṇeśa\footnote{Gajakomeda is
the elephant-shaped God Gaṇeśa, the son of Śiva and Pārvatī. There is a variety
of legends accounting for his elephant head. See J. Dowson: \tl{Hindu Mythology}
p. 207.} together with the gifts of scented flowers affords the enjoyment of
worldly pleasures.

57-59. Even a barren woman gets a good son making gifts of gold, silver \etc
The worship of the guardians of the quarters, the elephants of the quarters,
the serpents, the guardians of dams, the three-eyed\footnote{One of the eleven
names of Rudras (\tl{Matsya-purāṇa} (‘A Study’ by V. S. Agrawal) 5.29-30) which
has been variously interpreted. It represents the various triads on which
the entire cosmos is based. It is both the deity of the three eyes or
the conscious principles of Jagrat, Svapna and Suṣupti or Sūrya, Candra and Agni
and also the son of three Mothers, Ambā, Ambikā and Ambālikā. These three
sisters represent the three fires of the cosmic \tl{yajña} or the three Mothers
who create the three great principles of mind, life and matter.
\tl{Matsya-purāṇa} (‘A Study’ by V. S. Agrawal) pp.66-67.} Rudra and Viṣṇu,
the remover of sins, bestows perfect knowledge. The worship of Brahman,
Dhanvantari\footnote{Dhanvantari, said to be the physician of the Gods was
produced at the churning of the ocean with a cup of \tl{amṛta} in his hands.
He is the supposed author of the Āyurveda, the Indian medical science.} and of
the twin deities—Aśvins\footnote{Aśvins, two Vedic deities, are represented as
the physicians who ride in a golden car drawn by horses. Professor Goldstucker
(cp Muir’s Texts, Vol. V) thinks that the Aśvins represented two distinct
elements, the cosmical and the human blended into one. The human element is
represented by those legends which refer to the wonderful cures effected by them.
The cosmic element relates to their luminous nature. It is more likely that
there were some horsemen or warriors of great renown who inspired their
contemporaries with awe by their wonderful deeds and more especially by their
medical skill.} alleviates ailments, prevents foul death and suppresses all
sickness instantaneously.

60-62. Gifts of salt, iron, oil, pulses, \tl{trikaṭuka}, fruits, scents,
drinking water \etc, liquids in \tl{prastha} measures and solids in \tl{pala}
weights enable the devotee to attain heaven. The worship of Śiva and others
early in the morning in the month of Dhanus enables the devotees to achieve
everything gradually. The offering of eatables shall preferably be ghee-soaked
rice of the \tl{śālī} variety and well-cooked.

63. The offering of various kinds of cooked rice is specially recommended in
the month of Dhanus. The person who gives cooked food in the month of Mārgaśīrṣa
shall attain all desired benefits.

64-65. The giver of cooked food in the month of Mārgaśīrṣa shall attain
destruction of sins, achievement of the desired objects, good health, virtue,
good comprehension of the Vedic passages, good practices, great enjoyment here
and hereafter, the permanent unification with the Godhead and the realisation of
the perfect knowledge of the Vedanta.

66. A person who desires enjoyment of worldly pleasures shall worship
the deities early in the morning throughout the month of Mārgaśīrṣa or at least
for three days. No one shall be without sacred rites in the month of Dhanus.

67-70. Rites in Dhanurmāsa (month of Dhanus) prescribed for the morning can be
performed upto the \tl{sangava} time (3 \tl{muhūrtas} from sunrise).
A \tl{brāhmaṇa} shall observe fast in the month of Dhanus and restrain all his
senses. Till midday he shall repeat the \tl{Gāyatrī} mantra. Till the time of
going to bed, he shall repeat the mantras such as the five-syllabled one \etc
After acquiring perfect knowledge he shall attain salvation after death. Other
men and women shall repeat the five-syllabled mantra alone throughout and take
three baths every day. They will attain perfect knowledge. They shall secure
the annihilation of the great sins by repeating their favourite mantras.

71-75. The great offering of eatables shall be made to Śiva especially in
the month of Dhanus. The constituent parts of the great offering are as follows:

Rice of the \tl{śālī} variety a \tl{bhāra} by weight; pepper measuring
a \tl{prastha}; countable articles twelve in number; honey and ghee
a \tl{kuḍava} each; a \tl{droṇa} measure of green gram; twelve varieties of
side dishes; cake fried in ghee, sweets made of \tl{śālika} rice; curd and milk
twelve \tl{prasthas} each; twelve coconuts; twelve betel nuts, thirty-six clove
leaves; camphor powder; five \tl{saugandika}\footnote{A collection of five kinds
of aromatic vegetable substances, \viz cloves, nutmeg, camphor, aloe wood and
kakkola.} flowers; betal leaves.

76. This great offering of eatables made to the deities shall be distributed
among devotees in the order of their castes.

77. A devotee who makes the offering of cooked rice becomes the Lord of
a kingdom in the world. But by making gift of great offering of eatables,
a man attains heaven.

78. O excellent \tl{brāhmaṇas}, by offering this a thousand times the devotee
attains Satyaloka and lives the full span of life therein.

79. By offering this twenty-thousand times, he attains still higher world and
is not born again.

80-81. Twenty-six thousand great offerings constitute life-time offering. Making
gift of this is called great accomplishment. A devotee who makes this is not
born again.

82-83. In the month of Kārttika, on an auspicious day, life-time offering shall
be made. It shall be done at the time of the transit of the sun, on birthdays
(based on star), on full-moon days, annual birthdays \etc In other months when
the natal star comes in conjunction with the planets, this can be performed.

84. Even if the conjunction is only partial the offering shall be made. One gets
the benefit of dedicating oneself by that.

85. Śiva is delighted by the dedication of selves and bestows the salvation of
complete identity. This life-time offering shall be made only to Śiva.

86. Śiva exemplifies birth in as much as He has the form of both \tl{yoni}
(vaginal passage) and \tl{liṅga} (penis). Hence in order to ward off births
the \tl{janmapūjā} is of Śiva alone.

87. The entire universe consisting of the movable and the immovable is of
the nature of \tl{bindu} (dot) and \tl{nāda} (sound). \tl{Bindu} is \tl{śakti}
(power) and Śiva is \tl{nāda}. Hence the universe is pervaded by Śiva and Śakti.

88. \tl{Bindu}\footnote{\tl{Bindu} is a dot over a letter representing
the \tl{anusvāra}. It is supposed to be connected with Śiva and is of great
mystical importance.} is the support of \tl{nāda}\footnote{\tl{Nāda} is a nasal
sound represented by a semicircle and used as an abbreviation in mystical
words.}. The universe has the support of \tl{bindu}. Both \tl{bindu} and
\tl{nāda} together support the entire universe.

89. The unification of the \tl{bindu} and the \tl{nāda} is called
\tl{sakalīkaraṇa} and the universe takes its birth as a result of this
\tl{sakalīkaraṇa}.

90. The phallic emblem is the fusion of \tl{bindu} and \tl{nāda} and is
the cause of the universe. \tl{Bindu} is the goddess and Śiva is the \tl{nāda}
and the fusion of the two is the phallic emblem of Śiva.

91. Hence to ward off future births, the devotee shall worship the phallic
emblem of Śiva. Goddess of the form of \tl{bindu} is the mother and Śiva of
the form of \tl{nāda} is the father.

92. Great bliss is the result of the worship of the parents. The devotee shall
worship the phallic emblem for the acquisition of the Great Bliss.

93. That goddess is the mother of the universe and that Śiva is the father of
the universe. Sympathy towards the son who renders service naturally increases
in the minds of the parents.

94-95. O foremost among sages, ordinary parents bestow hidden treasures to
the son who renders special service. Hence a devotee shall worship the phallic
emblem in the manner of mother and father for the acquisition of the hidden
great bliss. Bharga is Puruṣa (Cosmic man or Being) and Bhargā is Prakṛti
(Cosmic Nature).

96. Puruṣa is of hidden latent conception and Prakṛti is of manifest inner
conception.

97. Since it is the father who conceives first, the Puruṣa has the primordial
conception. The unification of Puruṣa and Prakṛti is the first birth.

98. Its manifestation in the Prakṛti is called the second birth. The creature,
dead even as it is born, takes up its birth from the Puruṣa.

99. Certainly the birth is induced by the Māyā as an extraneous source. The word
\tl{jīva} (the individual soul) means that which gets decayed even from
the time of birth.

100. Another meaning of the word \tl{jīva} is that which is born enmeshed and
entwined. Hence the devotee shall worship the primordial phallic image for
unravelling the knots and nooses of the birth.

101-102. The world \tl{bhaga} means the primordial nature because it increases
and flourishes. The Śabdamātrā \etc (the cosmic sound principle \ie all objects
of enjoyment) evolved out of Prakṛti, being enjoyed by the sense organs;
the word \tl{bhoga} comes to mean that which gives \tl{bhaga}. The principal
\tl{bhaga} is of course the Prakṛti and Bhagavān is Lord Śiva Himself.

103. The lord alone is the bestower of enjoyment (\tl{bhoga}) and not anyone
else. The Lord who is the master of \tl{bhaga} is called Bharga by wise men.

104-105. The phallus is united with vagina and vagina is united with phallus.
For the sake of perpetual enjoyment here and hereafter the devotee shall worship
the phallic emblem which is lord Śiva Himself. He is the sun giving birth and
sustenance to the worlds. His symbol is justified in the coming into existence
of things.

106-107. Persons should worship Śiva, the cause of birth, in his phallic form.
That which makes the Puruṣa known, is called \tl{liṅga}, the symbol.
The unification and fusion of the symbols of Śiva and Śakti is thus called
\tl{liṅga}.

108. The lord delighted at the worship of His symbol wards off the function of
the symbol. That function being birth \etc cease.

109. Hence the devotee shall worship the phallic emblem with the sixteen forms
of service and homage to acquire the benefit from Prakṛti and Puruṣa through
means inherent or extraneous.

110. The worship thus performed on Sundays wards off births. The devotee shall
worship the great phallic emblem on Sundays with the syllable \tl{Oṃ}.

111-112. The ceremonial ablution of the phallic emblem with \tl{pañcagavya} on
Sundays is specially recommended. \tl{Pañcagavya} is the compound of cow’s urine,
dung, milk, curd and ghee. Milk, curd and ghee can severally be used with honey
and molasses. The offering of rice cooked in cow’s milk must be made with
the syllable \tl{Oṃ}.

113-114. The syllable \tl{Oṃ} (a + u + ṃ) is \tl{dhvani liṅga}.
The \tl{svayambhū liṅga} is \tl{nāda liṅga}; the \tl{yantra} (diagrammatic
contrivance) is \tl{bindu liṅga}. “M” syllable is the installed
(\tl{pratiṣṭhita}) \tl{liṅga}. “U” syllable is mobile (\tl{cara}) \tl{liṅga} and
the “A” syllable is a \tl{liṅga} of huge form (\tl{guruvigraha}). A person who
worships the \tl{liṅgas} perpetually becomes liberated soul undoubtedly.

115-116. A devout worship of Śiva liberates man from the bondage of births.
A fourth benefit is achieved by wearing \tl{rudrākṣa} beads sacred to Śiva and
a moiety is achieved by smearing the holy ashes over the forehead. Three-fourths
can be achieved by the recital of mantras and a man becomes full-fledged devotee
by means of worship. A man who worships both the phallic emblem of Śiva and
the devotees of Śiva attains salvation.

117. O \tl{brāhmaṇas}, stable devotion can be found firmly established and
flourishing only in that person who reads this chapter or listens to it
attentively.
