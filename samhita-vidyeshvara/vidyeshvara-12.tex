\chapter{The narrative of Śiva’s holy centres and temples}

Sūta said:

1. O wise sages, please listen to the narrative of holy centres with Śiva’s
temples all of which accord salvation. Thereafter I shall tell you their
traditions for the welfare of the people.

2. The Earth, fifty \tl{crores} of \tl{yojanas} in extent, abounding in
mountains and forests, supports the people at the bidding of Śiva.

3. The lord has Himself raised up these temples and holy centres in different
places for the liberation of the residents of these localities.

4. These temples whether self-risen or not, in view of their being accepted (as
their frequent resort) by the sages and \tl{devas} are intended for
the redemption of the people.

5. In these holy centres and temples, ablutions, charitable gifts, \tl{japas}
\etc must be regularly performed. Otherwise men are sure to be affected by
ailments, penury, dumbness \etc

6. If a man dies anywhere in the Bhāratavarṣa\footnote{Bhārata-varṣa is one of
the nine divisions of the earth as separated off by certain mountain ranges,
the other eight divisions being Kuru, Hiraṇmaya, Ramyaka, Ilāvṛta, Hari,
Ketumāla, Bhadrāśva and Kinnara. It is surrounded by oceans in the south west
and east and by the Himālaya in the North. \tl{Sk}. 7.1.11.13. Bharata who gave his
name to this country was the descendant of Svāyambhuva Manu. He was a king of
Agnīdhra’s family.} he shall be reborn again as a man if he has resided in
a holy centre where there is a self-risen phallic emblem of Śiva.

7. O brahmins, committing sins in a holy centre is of ineffable character. When
a man stays in a holy centre he must not commit even the smallest sin.

8. Somehow men must strive to find a residence in a holy centre. On the shores
of the ocean in the confluence of hundreds of rivers there are many such holy
centres and temples.

9. The holy river Sarasvatī is said to have sixty mouths or holy centres on its
banks. Hence an intelligent man must stay on its banks. He shall attain Brahma’s
region gradually.

10-11. The river Gaṅgā flowing from the Himālaya mountains is very holy with its
hundred mouths. There are many holy centres on its banks such as Kāśī \etc Its
banks are highly sacred in the month of Mārgaśīrṣa or when Bṛhaspati (Jupiter)
is in the zodiac Capricornus. The river Śoṇabhadra\footnote{The river Śoṇa (also
called Sone, Sonā) rises in Gondwana, in Madhya Pradeśa, on the table-land of
Amarakaṇṭaka, four or five miles east of the source of Narmadā river and running
first northerly and then easterly for 500 miles falls into the Ganges above
Pāṭaliputra or Patna. It is called Māgadhī nadī, since it forms the Western
boundary of Magadha. \tl{Sk}. 1.3 u 2.7 (ii).} of ten mouths is holy and yields
all cherished desires.

12-13. By ablutions therein and observing fast the devotee shall attain
the region of the god Gaṇeśa. The holy Narmadā\footnote{It rises in the Vindhya
mountain and falls into the gulf of Cambay. It flows in a wide flood-plain and
is fairly deep. It forms a suitable boundary between the political units north
and south of it.} is a great river of twenty-four mouths. By a dip therein and
residing on its banks the devotee shall attain the region of Viṣṇu. The river
Tamasā\footnote{It is identified with Tons which issues from the Ṛkṣapāda
mountain, appears in the Bundelkhand region and flows into the Ganges below
Allahabad.} is of twelve mouths and Revā\footnote{Revā and Narmadā are the two
small branches of one and the same river in the upper course which are later
united into one.} has ten mouths.

14. Godāvarī\footnote{This river known as Godā or Godāvarī forms an important
unit in the historical geography of South India. It drains a large area mainly
composed of Deccan lavas and flows through a wide fertile valley towards
the east. Its catchment area is bounded in the north by the Sahya mountain,
the Nirmala and Satmala ranges and the hills of Bastar and Orissa known to
the Purāṇas as Mahendra Parvata.} is very holy and it quells the sins of
murdering a brahmin or slaughtering a cow. It is said to have twentyone mouths
and accords Rudraloka.

15. Kṛṣṇāveṇī\footnote{It rises from the Sahya mountain. It is the united stream
of Kṛṣṇā and Veṇī. It flows into the bay of Bengal \tl{Cf Sk}. 2.1.29.44.} is
a sacred river destroying all sins. It is said to have eighteen mouths and it
accords Viṣṇuloka.

16. Tuṅgabhadrā\footnote{It rises from the Sahya mountain and joins the Kṛṣṇā
river.} has ten mouths and it accords Brahmaloka. The holy
Suvarṇamukharī\footnote{It is one of the most sacred rivers of Southern India.
After issuing from the Mahendra mountain, it falls into the southern sea,
passing through beautiful hills and dales along with its tributary streams.} is
said to have nine mouths.

17-19. Those who fall from Brahmaloka are born there. By residing on the banks
of the auspicious rivers Sarasvatī, Pampā\footnote{It is a tributary of
Tuṅgabhadrā river.}, Kanyā\footnote{Not identified. The country situated on
the bank of this river is sacred to Śiva. \tl{Cf Sk}. 1.3 and 2.7-19.} and
Śvetanadī\footnote{Not identified.} one shall attain Indraloka. The great river
Kāverī\footnote{It is one of the most sacred rivers which takes its rise from
the Sahya mountain. It is said to have many \tl{tīrthas}, particularly \tl{śiva-
kṣetras}, on its bank. \tl{Sk}. 1.3.6.98.} flowing from the mountain Sahya is
very holy and is said to have twenty-seven mouths. It accords all cherished
desires. Its banks are the bestowers of heaven and the regions of Brahmā and
Viṣṇu.

20-28. The devotees of Śiva are the bestowers of Śivaloka and accord cherished
desires. When the Jupiter and the sun are in the zodiac of Meṣa, the devotee
shall take the holy bath in Naimiṣa\footnote{Naimiṣa, modern Nimsar, is a sacred
region of Uttarapradeśa in the district of Sitapur, on the bank of Gomatī.
Naimiṣa was sacred in the Kṛta age, as Puṣkara in the Tretā, Kurukṣetra in
the Dvāpara, the Ganges in the Kali age.} and Badara\footnote{Name of
the hermitage of Nara and Nārāyaṇa in the neighbourhood of Gaṅgodbheda,
the source of the Ganges.}. Worship \etc thereafter accords Brahmaloka. When
the sun is in Karkaṭaka or Siṃha one shall take bath in the Sindhu\footnote{This
sacred river of Ancient India, takes its rise from the Himalayas, flows in
the Western Pakistan and falls into the Western Sea.}. On that occasion
the drinking of the sacred water of Kedāra\footnote{It refers to Kedāra Gaṅgā or
Mandākinī in Garhwal.} and ablution therein accords perfect knowledge. Śiva
Himself has mentioned before that the bath in the Godāvarī in the month of Siṃha
when Jupiter is also in the zodiac of Siṃha accords Śiva region. When Jupiter
and the sun are in the zodiac of Kanya, ablution shall be performed in
the rivers—Yamunā\footnote{The river rises in the Himalaya mountains among
the Jumnotri peaks, flows for 860 miles on the plains before it joins the Ganges
at Allahabad.} and Śoṇa, the fruit of which is great enjoyment in the worlds of
Dharma and Dantin (Gaṇeśa). When the sun and the Jupiter are in Tulā,
the devotee shall take bath in the Kāverī the fruit whereof is the attainment of
all cherished desires as stated by Viṣṇu Himself. The devotee who takes bath in
the river Narmadā in the month of Vṛścika, when the Jupiter is in the zodiac of
Vṛścika, attains Viṣṇuloka. Brahmā has stated that the bath in
the Suvarṇamukharī when the sun and the Jupiter are in the zodiac of Dhanus
accords Śivaloka. The devotee shall take bath in the Jāhnavī (Ganges) in
the month of Mārgaśīrṣa when Jupiter is in the zodiac of Capricornus. After
enjoying pleasures in the regions of Brahmā and Viṣṇu he will gain perfect
knowledge in the end.

29-30. In the month of Māgha when the sun is in the zodiac of Kumbha,
\tl{śrāddha}, offerings of \tl{piṇḍa} and water libations with gingelly seeds
raise the \tl{crores} of manes on both the sides (Paternal and maternal) of
the family. When the sun and the Jupiter are in the zodiac of Mīna, ablution
shall be performed in Kṛṣṇāveṇī.

31-32. The ceremonial ablutions taken in the different sacred waters in
the respective months accord the region of Indra. An intelligent man shall
resort to Gaṅgā or the Kāverī river. Certainly his sin will be quelled thereby.
There are many holy centres yielding Rudraloka.

33. The rivers Tāmraparnī\footnote{It issues from the Malaya mountain called
the Travancore hills in the southern parts of the Western Ghats.} and
Vegavatī\footnote{It is the modern Baiga or Bijari in the district of Madura.
G.D. p.38.} accord Brahmaloka. There are holy centres on their banks bestowing
heaven on the worshipper.

34. In between these rivers there are meritorious holy centres. Intelligent men
residing there will reap the respective fruits thereof.

35. Only by good conduct, good predilections and good concepts as well as by
being sympathetic can the devotee derive the benefit, not otherwise.

36. Meritorious actions performed in a holy centre flourish in many ways. Sinful
acts committed in a holy centre, though slight, become manifold.

37-38. If the sin committed in a holy centre is only for livelihood, the merit
will destroy that sin. Merit accords prosperity and quells physical, verbal and
mental sins. O brahmins, the mental sin is adamantine in sticking to the sinner
and it continues for many \tl{kalpas}.

39-40. The mental sin can be wiped off only by meditation and not otherwise.
The verbal sin is wiped off by \tl{japas} and the physical sin by forcefully
causing the emaciation of the body. Sins committed by means of wealth can be
wiped off by making charitable gifts and not otherwise, though \tl{crores} of
\tl{kalpas} (aeons) may elapse. In some places the increasing sin destroys
the merit.

41-43. Both Merit and Demerit have three aspects: the seed stage, flourishing
stage and the enjoyment stage. If they are in the seed stage they can be quelled
by perfect knowledge. If they are in the flourishing stage they can be quelled
in the manner described before. If they are in the enjoyment stage they get
destroyed only by enjoying and experiencing their fruits and not otherwise
though one might have performed \tl{crores} of meritorious deeds. If the seed or
the flourishing seedlings are destroyed what remains must be experienced and
wiped off. If one regularly performs worship of gods, makes gifts to brahmins
and performs sufficient penance, the enjoyment becomes bearable. Hence those who
wish for happiness must refrain from committing sins.
