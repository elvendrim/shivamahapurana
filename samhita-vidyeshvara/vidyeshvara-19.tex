\chapter{The worship of Śiva’s Earthen phallic image (\emph{pārthiva-liṅga})}

The sages said:

1-2. O Sūta, Sūta, be long-lived. Thou art a blessed devotee of Śiva.
The greatness of Siva’s phallic image in according excellent benefit has been
well explained by you. Now speak about the greatness of Earthen phallic image of
Śiva which is far superior to all others.

Sūta said:

3. O sages, please listen all of you with great devotion and respect. Now I am
going to speak on the greatness of earthly phallic image of Siva.

4. The Earthly phallic image of Śiva is the most excellent of all such images of
Śiva. Many \tl{brāhmaṇas} have achieved great things by worshipping it.

5. O \tl{brāhmaṇas}, Hari, Brahmā, Prajāpati and other sages have attained all
they desired by worshipping this Earthly phallic image.

6. \tl{Devas, asuras}, men, \tl{gandharvas}, serpents, \tl{rākṣasas} and many
others have attained greatness after worshipping it.

7. The phallic emblem of Śiva made of precious gems was considered the best in
the Kṛta age; of pure gold in the Dvāpara; of mercury in the Tretā and of earth
in the Kali age.

8. Among the eight\footnote{\tl{Śatapatha-brāhmaṇa} (6.1.3.1-18) gives
the following version of the eight forms of Śiva: “When the life-principle
became manifest it had no name, so it cried. Prajāpati asked the reason and
being informed that the child wanted a name, first gave him the name Rudra,
then Śarva, Paśupati, Ugra, Aśani, Bhava, Mahādeva and Īśāna. This was
the conception from which the \tl{purāṇa} writers developed the \tl{aṣṭamūrti}
conception of Śiva. The fact is that the eight forms of Śiva symbolise the five
gross material elements (ether, air, fire, water, and earth), two opposite
principles of \tl{prāṇa} and \tl{apāna} (heat and cold represented by the sun
and the moon) and the principle of mind (\tl{manas}) which is the eighth.}
cosmic bodies of Śiva, the Earthen body is the best. Since it is not worshipped
by any one else O \tl{brāhmaṇas}! it yields great benefit.

9. Just as Śiva is the oldest and the most excellent of all deities, so also his
earthly phallic image is the most excellent of all.

10. Just as the celestial river Gaṅgā is the oldest and the most excellent of
all rivers, so also is the earthen phallic image of Śiva the most excellent of
all.

11. Just as the \tl{Praṇava} is considered the greatest of all mantras, so also
the earthen phallic image of Śiva that is worthy to be worshipped, is the most
excellent of all.

12. Just as the \tl{brāhmaṇa} is spoken of as the most excellent of all
\tl{varṇas} so also is the earthen phallic image of Śiva the most excellent of
all other phallic images.

13. Just as Kāśī is considered the most excellent of all holy cities, so also
the earthly phallic image of Śiva is spoken of as the most excellent of all
other phallic images.

14. Just as the rite of Śivarātri is the greatest of all sacred rites so also
the earthly phallic image of Śiva is the most excellent of all other phallic
images.

15. Just as Śiva’s energy is considered the greatest of all goddesses so also
the earthen phallic emblem of Śiva is spoken of as the most excellent of all.

16. Discarding the worship of the earthen phallic image if any one were to
worship another deity, that worship becomes fruitless. Ceremonial ablutions,
charitable gifts \etc are of no avail.

17. The propitiation of the earthen phallic image is sanctifying, bestower of
bliss, longevity, satiety, nourishment and fortune. It must be observed by all
good aspirants.

18. A devotee endowed with unflinching faith shall worship the earthen phallic
image with such modes of service as are easily available. It accords
the achievement of all desired objects.

19. He who worships the earthen phallic image after constructing an auspicious
altar becomes affluent and glorious here itself and becomes Rudra in the end.

20. He who worships the earthen phallic image in the three junctures of
the threefold division of the day every day gains the bliss for twenty-one
future births.

21. He is honoured in Rudraloka with this body alone. His body dispels the sins
of every man by mere sight or touch.

22. He is a living liberated soul, he is wise, he is Śiva, there is no doubt.
A mere sight of him accords enjoyment of worldly pleasures and salvation.

23-24. He who worships the earthen phallic emblem of Śiva every day stays in
Śivaloka for so many years of Śiva, as he had been visiting Śiva’s temple in his
life. If he had any wish he would be reborn in the land of Bharata as
a sovereign monarch.

25. If a man without any desire worships every day the excellent earthen phallic
image, he shall stay in Śiva’s region for ever. He shall attain the Sāyujya type
of salvation.

26. If a \tl{brāhmaṇa} does not worship the earthen phallic image he shall fall
in the terrible hell with a terrible trident pierced through his body.

27. By any means the phallic image shall be made beautiful. The \tl{pañcasatra}
rite shall be performed with the earthen phallic image.

28. The earthen phallic image shall be made as a single whole. Making it
piecemeal \ie if the image is made joining two or more pieces, he will never
derive the merit of worship.

29. Whether it is made of gems, gold, mercury, crystals or \tl{pusparāga} it
shall be a single whole.

30. All mobile phallic emblems shall be a single whole. Stationary phallic
images shall be made of two pieces. This is the rule about broken and unbroken
phallic images both immobile or mobile.

31. The pedestal is the great Māyā; the phallic image is lord Śiva. Hence in
immobile image two-piece construction is recommended.

32. This has been mentioned by those who know the principles of Śaiva cult that
a stationary phallic image shall be made of two pieces.

33. Only those who are deluded by ignorance make the mobile phallic image of
two pieces. The sages who know the Śaiva cult; and are well versed in Śaiva
Sacred texts do not enjoin that.

34. Those who make a stationary phallic image as a single whole and a mobile one
pieced are fools. They never reap the benefit of worship.

35. Hence, one shall make with very great pleasure the mobile one as a single
whole and the stationary one as two-pieced according to rules laid down in
the sacred texts.

36. The worship of an unbroken mobile image yields full benefit while
the worship of two-pieced mobile image brings about great harm.

37. This has been stated by those who know the lore that the worship of
a stationary image of a single piece not only withholds the cherished desire but
is also full of hazards.
