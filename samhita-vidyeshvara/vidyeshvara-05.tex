\chapter{The greatness of the \tl{liṅga} of Śiva}

Sūta said:

1. A person incompetent to perform the three rites of \tl{śravaṇa} \etc shall
fix the \tl{liṅga} or the image of Śiva and worship them every day. He can thus
cross the ocean of worldly existence.

2. As far as he can afford, the devotee shall make gifts of wealth too without
deceiving others. He shall offer them to the phallic emblem or the image of Śiva.
He must worship them constantly.

3-7. The worship must be performed elaborately. Construction of platforms,
ornamental portals, monasteries, temples, holy centres, \etc, offerings of cloth,
scents, garlands, incense, lamps, with due piety; oblations of various cooked
rice, pancakes, pies \etc with side dishes; umbrellas, fans, chowries with all
paraphernalia—everything shall be maintained in the worship of Śiva. In fact,
all royal homage shall be paid. Circumambulation and obeisance with \tl{japas}
according to capacity shall be performed. All the different usual rites in
worships like invocation shall be maintained with due devotion. A person who
worships the phallic emblem or the image in this manner will attain salvation
even without \tl{śravaṇa} \etc Many noble men of yore have been uberated solely
by this simple worship.

8. Everywhere the deities are worshipped only in their image. How is it that
Śiva is worshipped both in the image and the phallus?

Sūta said:

9. O sages, this question is holy and wondrous. Here the speaker is Śiva Himself
and not any ordinary person.

10. I shall tell you what Śiva Himself had said and what I heard from my own
preceptor. Śiva alone is glorified as \tl{niṣkala} (nameless and formless) since
He is identical with supreme Brahman.

11. He is also \tl{sakala} as He has an embodied form. He is both \tl{sakala}
and \tl{niṣkala}. It is in his \tl{niṣkala} aspect that the \tl{liṅga}
is appropriate.

12-13. In the \tl{sakala} aspect the worship of his embodied form is appropriate.
Since He has the \tl{sakala} and \tl{niṣkala} aspects He is worshipped both in
the phallic and in the embodied form by the people and is called the highest
Brahman. Other deities, not being Brahman, have no \tl{niṣkala} aspect anywhere.

14. Hence the deities are not worshipped in the formless phallic symbol.
The other deities are both non-Brahman and individual souls.

15. In view of their being embodied alone they are worshipped solely in the
bodily form. Śaṅkara has \tl{brahmatva} and the others \tl{jīvatva}.

16. This has been explained in the meaning of the \tl{praṇava} (Om), the essence
of Vedānta, by Nandikeśvara when asked by Sanatkumāra, the intelligent son of
Brahmā, at the mountain Mandara.

Sanatkumāra said:

17-18. The embodied form alone is often observed in the worship of the deities
other than Śiva. But both the phallic and the embodied forms are seen only in
the worship of Śiva. Hence O benevolent one, please tell me precisely making me
understand the truth.

Nandikeśvara said:

19. It is impossible to answer this question without revealing the secret of
Brahman.

20-24. O sinless one, since you are pious I shall tell you what Śiva Himself has
said. Since Śiva has the bodiless aspect in virtue of His being the supreme
Brahman, the \tl{niṣkala liṅga}, in conformity with the Vedic implication, is
used only in His worship. Since He has an embodied form as well, His embodied
form is also worshipped and accepted by all people. According to the decision in
the Vedas, the embodied form alone is to be used in the worship of other deities
who are only individual souls embodied. \tl{Devas} have only the embodied aspect
in their manifestation. In sacred literature both the phallic and the embodied
forms are mentioned for Śiva.

Sanatkumāra said:

25. O Fortunate one, you have explained the worship of phallus and image
distinctly for Śiva and the other deities. Hence, O lord of \tl{yogins}, I wish
to hear the feature of the manifestation of the phallic aspect of Śiva.

Nandikeśvara said:

26-27. O dear one, out of love for you I shall tell you the truth. Long long ago,
in the famous first \tl{kalpa}\footnote{The term \tl{kalpa} in a precise sense
means a vast cosmic period but this seems to have been a later application of it,
when the scheme of cosmological time was developed. It is often used in
a simpler and unspecialized way to mean ‘a period of time’, ‘an age.’ This seems
to have been its earlier signification, as where it is said ‘\tl{Purā Kalpe,
mahākāle}’ in old time, long, long ago. In such texts \tl{purākalpa} is often
used loosely and has the general sense of ‘Old time’.}, the noble souls Brahmā
and Viṣṇu fought each other.

28. In order to eradicate their arrogance lord Parameśvara showed his unembodied
\tl{niṣkala} form in the form of a column in their midst.

29. He showed his phallus emblem separate, evolved out of the column, with a
desire to bless the worlds.

30. From that time onwards the divine phallus and the embodied image, both, were
assigned to Śiva alone.

31. The embodied form alone was assigned to deities other than Śiva.
The different types of the embodied forms of the different \tl{devas} yield only
enjoyments. In regard to Śiva the phallic emblem and the embodied form together
bestow auspicious enjoyment and salvation.
