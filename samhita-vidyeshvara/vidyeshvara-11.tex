\chapter{Mode of worshipping the phallic form of Śiva and making gifts}

The sages said:

1. How is the phallic form of Śiva to be installed? What are the characteristic
features of the form? How is it to be worshipped? What is the appropriate time
and place for worship. What sort of performer he must be?

Sūta said:

2-3. I shall tell you everything for your sake, please listen attentively.
The time must be convenient and auspicious. The place must be a holy centre. It
can be on the bank of a river or anywhere facilitating a daily worship. It can
be of \tl{pārthiva} (earth), \tl{āpya} (watery) or \tl{taijasa} (fiery) type.

4. If it has all the characteristics mentioned in the sacred texts, the devotee
derives the fruit of worship. If it has all characteristics, it accords
the fruit of worship instantaneously.

5. A subtle one is recommended if it be mobile and a gross one if it is
stationary. The phallic emblem of good characteristics shall be set up in
the seat of the same sort.

6. The seat can be circular, square or triangular in shape. The one shaped like
a cot in the middle is of middle efficacy.

7. At first, the emblem was made of earth or rock; then it used to be made with
the metals. If it is stationary, the emblem and the \tl{pīṭha} should be of
the one and the same material.

8. Save the one which the \tl{asura} Bāṇa worshipped, both the emblem and
the seat shall be unitary, if emblem be mobile. The length of the emblem shall
be of the measure of twelve fingers of the devotee.

9. If it is shorter it is less efficacious; if it is longer there is no harm.
A shortage by the breadth of the finger of the devotee in regard to the mobile
one is similarly harmless.

10-12. A \tl{vimāna} (chariot-like structure) of artistic beauty shall be made
at first wherein the divine attendants shall be represented. In its firm and
beautiful \emph{sanctum sanctorum} shining like a mirror studded with the nine
precious gems—sapphire, lapis lazuli, emerald, pearl, coral, \tl{gomedaka},
diamonds and rubies, the emblem shall be installed on the altar.

13-17. The emblems hall be worshipped with the mantras beginning with
“\tl{sadyo}”\footnote{vs. 29. 36.} in five different places in order.
Sacrificial offerings shall be made in the fire. Śiva and the gods of His family
shall be adored. The preceptor is given monetary gifts. Kinsmen are propitiated
with whatever they desire. Money is distributed among the mendicants. All
objects sentient or otherwise, and all living beings movable or immovable are
duly gratified. The cavity is filled with gems. Mantras “\tl{sadyo}” \etc are
recited. The auspicious supreme lord is meditated upon. The great mantra
\tl{Oṃkāra} resonant with its mystical sound is repeated. The \tl{liṅga} is then
united with the \tl{pīṭha} (pedestal). The two are then welded together.

18. Similarly the embodied image shall also be fixed there auspiciously. For
the sake of festivals the embodied image shall be installed outside with the
five-syllabled mantra.

19. The embodied image shall be taken from the preceptors or it must be one that
has been worshipped by saintly men. Such an adoration of the embodied image and
the phallic emblem accords the region of Śiva.

20. The phallic emblem is of two varieties: the stationary and the mobile. Trees,
hedges \etc represent the stationary.

21. Worms, insects \etc represent the mobile. For the stationary one, tending
and similar service is recommended. For the mobile one \tl{tarpaṇa}
(propitiation) is recommended.

22. With a love for the happiness of different beings \tl{śiva pūjā} shall be
performed—so say the wise men. The pedestal represents Śiva’s consort—Pārvatī
and his phallic emblem represents the sentient being.

23. Just as lord Śiva remains ever in close embrace of the Goddess Pārvatī, so
also the phallic emblem holds on to the pedestal, for ever.

24. Such is the installation of Śiva’s great phallic emblem which shall be
worshipped with due homage. The daily worship shall be made in accorance with
one’s capacity; so also the fixation of banner \etc

25-29. The devotee shall install the phallic emblem and it will accord directly
the region of Śiva. Or the devotee shall worship the mobile emblem with
the sixteen types of homage and services as prescribed. It accords the region of
Śiva gradually. The sixteen types of service are\footnote{The sixteen acts of
homage to a deity are slightly different in other texts; Compare “\tl{āsanaṃ
svāgataṃ pādyamarghyamācamanīyakam | madhuparkācamanasnānaṃ vasanābharaṇāni ca |
gandhapuṣpe dhūpadīpau naivedyaṃ vandanaṃ tathā}”. \tl{Tantrasāra} mentions 64
\tl{upacāras}.}: invocation (\tl{āvāhana}); offering the seat (\tl{āsana});
water offering (\tl{arghya}); washing of the feet (\tl{pādya}); water for
rinsing the mouth as a mystical rite (\tl{ācamana}); oil bath (\tl{abhyaṅga
snāna}); offering of cloth (\tl{vastra}); scents (\tl{gandha}); flowers
(\tl{puṣpa}); incense (\tl{dhūpa}); lamps (\tl{dīpa}); food offering
(\tl{nivedana}); waving of lights (\tl{nīrājana}); betel leaves (\tl{tāmbūla});
obeisance (\tl{namaskāra}); and mystical discharge and conclusion
(\tl{visarjana}).

Or the devotee need perform the rites from water-offering to food offering alone
duly. Or the devotee shall daily perform, as he can, ablution (\tl{abhiṣeka});
food offering (\tl{naivedya}); and obeisance (\tl{namaskāra}) and propitiation
(\tl{tarpaṇa}),—all these in order. It will accord him the region of Śiva.

30. Or he shall perform all the sixteen rites in the phallic emblem of human,
saintly or godly origin, or in one naturally risen up (Svayambhū) or in one of
very extraordinary nature installed duly.

31. If the devotee makes gifts of articles of worship he will get some benefit
or other. By circumambulation and obeisance he will attain Śiva’s region
gradually.

32-33. Regular vision of the phallic emblem accords benefit. Or the devotee can
make a phallic emblem out of clay, cow-dung, flowers, \tl{karavīra} fruit,
jaggery, butter, ashes or cooked rice as he likes and worship it according to
the prescribed rules.

34. Some authorities have recommended the worship of the phallic emblem on
the thumb \etc In these rites of phallic worship, there is no sort of
prohibition whatsoever.

35. Everywhere Śiva accords benefit as befitting the endeavour put in. Or he
shall make gifts of the phallic emblem or the value of its construction.

36. Whatever is given to a devotee of Śiva with sincere faith accords Śiva’s
region. Or the devotee can repeat the \tl{praṇava} mantra ten thousand times
every day.

37. Repetition of \tl{Oṃ} (\tl{praṇava} mantra) a thousand times at dawn and at
dusk is known to be according Śiva’s region. At the time of the repeated
utterance (\tl{japa}) of the mantra, ending with “Ṃ” purifies the mind.

38. At the time of \tl{samādhi} (meditation) the repetition of \tl{Oṃkāra} must
be mental. Muttering of it in low voice can be practised at all times. The same
with \tl{bindu} (dot) and \tl{nāda} (sound) is also of the same efficacy.

39. Or the devotee can with due reverence repeat the five-syllabled mantra ten
thousand times every day or a thousand times at dawn and at dusk. It accords
the region of Śiva.

40. Repetition of the five-syllabled mantra (\tl{namaśśivāya}) by a brahmin is
specially efficacious with the \tl{oṃ} (\tl{praṇava}) prefixed. A mantra must be
received from a preceptor with proper initiation for the acquisition of
the desired fruit.

41. The ceremonial ablution when the sun is in transit to the Zodiac Kumbha,
initiation for mantras, the \tl{nyāsa} of Mātṛkās\footnote{\tl{Nyāsas} are
particular diagrams which are closely associated with the divine mothers and are
written in characters to which a magical power is ascribed. These are
the personified energies of the principal deities connected with the worship of
Śiva. They are reckoned sometimes 7, sometimes 8, 9 or 16 in number.} a brahmin,
a person with soul purified by truth; a preceptor of perfect knowledge—all these
are splendid.

42. Brahmins shall begin with \tl{namaḥ} and the others shall end with
\tl{namaḥ}. With regard to some women the mantra shall end with \tl{namaḥ} duly.

43. Some say that brahmin women begin with \tl{namaḥ}. Repetition of this for
five crores of times will render a person equal to Sadāśiva.

44. By repeating it one, two, three or four crores of times, the devotee shall
attain the region of Brahmā and others. One can repeat any of the syllables
a hundred thousand times or all of the syllables separately a hundred thousand
times.

45-47. Or a hundred thousand times all the syllables together, if repeated,
accord Śiva’s region. Or if the devotee repeats it a thousand times every day
and completes a million times in a thousand days, he can achieve whatever he
desires. He shall feed brahmins every day. A brahmin shall repeat
the \tl{Gāyatrī} a thousand and eight times every day in the morning. He shall
attain Śiva’s region gradually. He shall repeat Vedic verses and hymns with
the observance of restraints.

48. The \tl{daśārṇa} mantra shall be repeated either 99 times or nine hundred
times or nine thousand nine hundred times.

49. The regular study of the Vedas accords Śiva’s region. All the other sorts of
mantras shall be repeated a hundred thousand times.

50. If the mantra consists of only one syllable it shall first be repeated
a crore times and thereafter a thousand times with great devotion.

51. Doing thus according to one’s capacity one shall gradually attain Śiva’s
region. It is the duty of every one to repeat a mantra pleasing to him every day
till his death.

52-53. If a man repeats “Om” a thousand times he shall get all his desires
fulfilled at the bidding of Śiva. If he plants a flower-garden for the sake of
Śiva or even renders service by sweeping and cleaning Śiva’s temple and
precincts he shall attain Śiva’s region. The devotee shall reside for ever in
Śiva’s temple with great devotion.

54. It yields worldly enjoyment and salvation to every one sentient or
insentient. Hence an intelligent man shall reside in a temple of Śiva till
death.

55. In a temple built by ordinary man, the space upto a hundred \tl{hastas} (1
hasta = 30 cms) from the phallic image is holy. In a temple dedicated to sages,
the space upto a thousand \tl{aratnis} (1 \tl{aratni} = 45 cms) from the phallic
image is holy. In a temple dedicated to sages, the space upto a thousand
\tl{aratnis} from the emblem is holy.

56. If the phallic emblem had been installed by gods the space upto a thousand
\tl{aratnis} is holy. In a temple where phallic emblem is self-risen, the space
upto a thousand \tl{dhanuḥ pramāṇas} (a \tl{dhanuḥ pramāṇa} = 4 hastas) is holy.

57. The tank, well, pond \etc in a holy centre shall be considered Śiva-Gaṅgā in
accordance with Śiva’s statement.

58. By taking bath or making gifts or muttering mantras in that centre one will
attain Śiva. One shall seek shelter in a temple of Śiva and stay there till
death.

59-61. The rites of obsequies of the second day or the tenth day, the offerings
of monthly \tl{piṇḍas}, the rite of \tl{sapiṇḍīkaraṇa} or the annual \tl{śrāddha}
shall be performed in a holy centre. He will instantly attain Śiva’s region. By
staying there for seven, five or three nights or a single night he will attain
Śiva’s region gradually. He will obtain results according to his conduct and
befitting his caste.

62. By the uplift in the caste and devotion the fruit gains more efficacy.
Anything done with a desire in view yields results immediately.

63-64. Anything done with no specific desire in view yields the region of Śiva
directly. Of the three periods of time, ordained rites shall be performed in
the morning, rites for the fulfilment of desires in the midday and rites for
the suppression of the evil in the evening. The same thing holds good for
nights too.

65. The two middle \tl{yāmas} (1 \tl{yāma} is equal to 3 hours) at night are
called \tl{niśītha}. The worship of Śiva at that time accords desired results.

66. If a man performs rites after realising this, he shall achieve the due
results. Especially in the Kali age the achievement of fruit is only due to
the precise performance of actions.

67. If the man is well behaved, afraid of sins and the observer of good actions
at other man’s suggestion or at his own he shall attain due results.

The sages said:

68-69. O Sūta, foremost among excellent \tl{yogins}, please tell us briefly
about the various holy centres by resorting to which women and men shall attain
the region (of Śiva). Please tell us about the traditions of Śiva temples also.

Sūta said:

All of you listen faithfully to the account of all holy centres and their
traditions.
