\chapter{The greatness of the holy ashes}

Sūta said:

1. The ashes (\tl{bhasma}) of auspicious nature are of two types. I shall
explain their characteristics. Please listen attentively.

2. One is known as \tl{mahābhasma} (great ashes) and the second is known as
\tl{svalpa} (the little). The \tl{mahābhasma} is of various types.

3. It is of three types: \tl{śrauta} (Vedic), \tl{smārta} (resulting from
\tl{smṛti} rites) and \tl{laukika} (prepared from ordinary fire).
The \tl{svalpa} is the ordinary ash which is of various forms.

4. The \tl{śrauta} and the \tl{smārta} ashes are to be used only by
the twice-born. The \tl{laukika} can be used by every one.

5. Sages have said that the twice-born should apply the holy ashes repeating
mantras. The others can simply apply without any mantra.

6. When dry cow-dung is reduced to ashes it is called \tl{agneya} (fiery).
O great sage, for the sake of \tl{tripuṇḍra} this ash can be used.

7. The ashes resulting from \tl{agnihotra} and other sacrificial rites shall be
used for the \tl{tripuṇḍra} by men seeking intellect.

8. When the ashes are put on the forehead or smeared with water, the seven
mantras “\tl{agni}”\footnote{Compare \tl{Bhasmajābālopaniṣad}. The mantras
referred to are: (1) “\tl{agniriti bhasma}”, (2) “\tl{vāyuriti bhasma}”,
(3) “\tl{jalamiti bhasma}”, (4) “\tl{sthalamiti bhasma}”,
(5) “\tl{vyomiti bhasma}”, (6) “\tl{devā bhasma}”, (7) “\tl{ṛṣyo bhasma}”} \etc
mentioned in the \tl{Jābālopaniṣad}, shall be recited.

9. People of all \tl{varṇas} and \tl{āśramas} shall put on \tl{tripuṇḍra} on
the forehead or dust their bodies with the mantras mentioned in the \tl{Jābāla-
Upaniṣad} or if no mantra is used they shall do the same with reverence.

10. Dusting with the holy ashes and smearing the \tl{tripuṇḍra} in horizontal
parallel lines shall not be abandoned by those who seek salvation. \tl{Śruti}
lays down that they shall not get negligent.

11-12. Śiva, Viṣṇu, Umā, Lakṣmī, goddess of speech and other gods and goddesses,
\tl{brāhmaṇas, kṣatriyas, vaiśyas} and persons of mixed castes and hill tribes
have observed \tl{tripuṇḍra} and dusting always.

13. Those who do not observe \tl{tripuṇḍra} and \tl{uddhūlana} cannot practise
well the various rites of the different \tl{varṇas} and \tl{āśramas}.

14. Those who do not observe with faith \tl{tripuṇḍra} and \tl{uddhūlana} cannot
be liberated from the world even if they take ten million births.

15. Even after hundreds of crores of \tl{kalpas}, Śiva-knowledge will not dawn
upon those who do not observe with faith \tl{tripuṇḍra} and \tl{uddhūlana}.

16. This is the final conclusion of all sacred texts that those who do not
observe with faith \tl{tripuṇḍra} and \tl{uddhūlana} are tarnished by great sins.

17. Any action performed by those who do not observe \tl{tripuṇḍra} and
\tl{uddhūlana} with faith will give adverse results.

18. O sage, the hatred towards \tl{tripuṇḍra} and \tl{uddhūlana} is kindled in
the hearts of only those great sinners who hate everyone.

19. After performing the sacred rites of Śiva in the fire, the devotee who has
realised the Self shall smear the forehead with the ashes repeating the mantra
beginning with “\tl{tryāyuṣā}”\footnote{\tl{Vājasaneyisaṃhitā} (of
\tl{Śuklayajurveda}) 3.62.}. The moment the ashes (\tl{bhasma}) come in contact
with his body he will be freed from sins of his impious acts.

20. He who observes \tl{tripuṇḍra} with white ashes during the three
\tl{sandhyās} every day becomes free from all sins and rejoices with Śiva.

21. He who makes the \tl{tripuṇḍra} on the forehead with white ashes shall
attain, on death, the primordial worlds.

22. No one shall repeat the six-syllabled mantra without applying ashes on
the body. After making the \tl{tripuṇḍra} with the ashes he shall perform
the \tl{japa}.

23-24. All holy centres and all sacrifices will be present for ever in the place
where a man after having put ashes on his body stays permanently, no matter
whether he is ruthless, base, sinful or commits morning sins, or is a fool or
a fallen man.

25. Even a sinful person is worthy of being honoured by \tl{devas} and
\tl{asuras} if he has \tl{tripuṇḍra} on his forehead. What then of a faithful
man endowed with a pure soul?

26. All the holy centres and sacred rivers go ever to the place which a person
who is endowed with \tl{śivajñāna} (knowledge of Śiva) and has put on ashes
casually visits.

27. Why should I say more? The sensible person shall always apply the ash, shall
always worship the phallic image and shall always repeat the six-syllabled
mantra of Śiva.

28. Neither Brahmā, nor Viṣṇu, nor Rudra, nor sages, nor the \tl{devas} can
explain adequately the greatness of the application of the ashes.

29. Even if a person has eschewed the duties of the different \tl{varṇas} and
\tl{āśramas}, even if a person has omitted the holy rites of the \tl{varṇas},
he shall be freed from the sin if he wears \tl{tripuṇḍra} once.

30. Those men who exclude a man wearing \tl{tripuṇḍra} and perform holy rites
are not liberated from worldly bondage even after crores of births.

31. If a \tl{brāhmaṇa} wears the \tl{tripuṇḍra} with the ash on his forehead he
must be considered as having learnt everything from the preceptor and as having
performed every sacred rite.

32. Those who begin to strike on seeing a person who has applied the ash are
reborn of \tl{cāṇḍāla} parents. O holy one, this can be guessed by the wise.

33. With great devotion \tl{brāhmaṇas} and \tl{kṣatriyas} shall apply the holy
ashes over such parts of the body as are prescribed by the rule repeating
the mantra “\tl{mā nastoke}”\footnote{Ibid. 16.16.} \etc

34. A \tl{vaiśya} shall apply the ashes repeating the \tl{tryambaka}
\footnote{Ibid. 3.60.} mantra and a \tl{śūdra} with the five-syllabled mantra
\footnote{\tl{namaś śivāya}.}. Widows and other women shall do like
the \tl{śūdras}.

35. A house-holder shall repeat the \tl{pañcabrahma}
\footnote{\tl{Vājasaneyisaṃhitā} (of \tl{Śuklayajurveda}) 29.11.} mantra \etc
and a \tl{brahmacārin} shall repeat the \tl{tryambaka-mantra}
\footnote{Ibid. 3.60.} at the time.

36. The \tl{vānaprastha} shall repeat the \tl{aghora} mantra\footnote{Ibid. 162.}
and an ascetic shall observe with the \tl{praṇava} alone.

37. A \tl{śivayogin} being outside the pale of \tl{varṇa} and \tl{āśrama} rites
because of his conception “I am Śiva” shall wear ashes with the \tl{īśāna} mantra.

38. Śiva has ordained that the rite of wearing ashes shall not be eschewed by
the people of any caste and outside the bounds of caste by other living beings.

39. A person who has applied ashes on his body actually wears as many \tl{liṅgas}
as there are particles of the ash that remain on his body.

40-41. \tl{Brāhmaṇas, kṣatriyas, vaiśyas, śūdras}, people of mixed castes, women,
widows, girls, heretics, a \tl{brahmacārin}, a householder, a forest-dweller,
an ascetic, performer of sacred rites and women who have \tl{tripuṇḍra} marks
are undoubtedly liberated souls.

42. Just as the fire when touched with or without knowledge burns the body so
does the ash worn consciously or unconsciously sanctify the man.

43. No man shall drink or eat even a bit without applying \tl{bhasma} or wearing
\tl{rudrākṣa}. If he eats or drinks, whether he is a householder or
\tl{vānaprastha} or an ascetic, a man of the four castes or of mixed caste, he
becomes a sinner and goes to hell. If a man of the four castes repeats
\tl{Gāyatrī}\footnote{Ibid. 16.} or if an ascetic repeats the \tl{Praṇava} he
shall be liberated.

44. Those who censure \tl{tripuṇḍra} actually censure Śiva. Those who wear it
with devotion actually wear Śiva.

45. Fie upon the forehead that is devoid of ash. Fie upon the village that has
no Śiva temple. Fie upon that life that does not worship Śiva. Fie upon the lore
that does not refer to Śiva.

46. Great indeed is the sin accruing even from the sight of those who censure
Śiva who is the support of three worlds and those who censure the man wearing
\tl{tripuṇḍra} on his forehead. They are on a par with pigs of rubbish heap,
demons, donkeys, dogs, jackals and worms. Such sinful persons are hellish fiends
even from their very birth.

47.\footnote{The text is corrupt and the English rendering is not certain.} They
may not see the sun during the day and the moon during the night. They may not
see them even during sleep. They may be freed by repeating the Vedic
\tl{Rudra Sūkta}. Those who censure a person wearing the \tl{tripuṇḍra} are
fools. A mere talk with them may cause the fall into hell. There is no way of
saving then.

48. O sage, \tl{tāntrika} is not authorised in a \tl{śivayajña} nor a person
having \tl{ūrdhvapuṇḍra} (worn on the forehead by vertical mark by
a \tl{vaiṣṇava}). A person marked with a heated wheel (a mark of
a \tl{vaiṣṇava}) is excluded from \tl{śivayajña}.

49. There are many worlds to be attained as explained in \tl{Bṛhajjābāla
Upaniṣad}; taking that into consideration a man shall be devoted to the ashes
(\tl{bhasma}).

50. Just as sandal paste alone can be applied over sandal paste, so also only
the ash shall be applied over the sacred mark on the forehead. A sensible person
will not apply anything over the forehead that wears the ornamental mark of
ashes on it.

51. The \tl{tripuṇḍra} shall be applied upto the forelocks by women. \tl{Brāhmaṇas}
and widows shall apply the ash also. Similarly it shall be applied by persons of
all \tl{āramas}. Thus it bestows salvation and destroys all sins.

52. He who makes \tl{tripuṇḍra} duly with the ash is freed from groups of great
as well as small sins.

53-54. A \tl{brahmacārin}, a householder, a forest-dweller or an ascetic,
\tl{brāhmaṇas, kṣatriyas, vaiśyas, śūdras}, the low and the base people become
pure by means of \tl{tripuṇḍra} and \tl{uddhūlana} applied according to
the prescribed manner and get their heaps of sins destroyed.

55. A person regularly applying the ash (\tl{bhasma}) is freed of the sins of
slaughter of women and cows and that of heroes and horses. There is no doubt
about it.

56-60. By means of \tl{tripuṇḍra}, the following and similar others of
innumerable sorts are destroyed immediately:—Theft of others’ wealth, outraging
the modesty of other men’s wives, censuring others, usurping and forcibly
occupying others’ fields, harassing others, theft of plants, parks \etc,
incendiarism, acceptance from base people of the gifts of cow, gold, buffalo,
gingelly seeds, blankets, cloths, cooked rice, food-grains, water \etc; sexual
intercourse with prostitutes, women of the tribal castes, fisher women, slave
women, actresses, widows, virgins and women in their menstrual periods, selling
of flesh, hides, gravy \etc and salt, calumny perjury, deceitful arguments and
utterance of falsehood.

61. The theft of Śiva’s property, censure of Śiva in certain places and
the censure of the devotees of Śiva can be dispelled by the rites of expiation.

62. Even a \tl{cāṇḍāl}a who wears \tl{rudrākṣa} over his body and
the \tl{tripuṇḍra} on his forehead, is worthy of respect. He is the most
excellent of all castes.

63. He who wears the \tl{tripuṇḍra} on his forehead gains the same merit as one
who takes his bath in the sacred rivers like Gaṅgā and whatever other sacred
ponds, lakes and holy centres there are in the world.

64. The five-syllabled mantra which grants \tl{śiva kaivalya} is on a par with
seven crores of great and many crores of other mantras.

65. O sage, mantras of other deities bestowing all blessedness are easily
accessible to the devotee who wears the \tl{tripuṇḍra}.

66. He who wears \tl{tripuṇḍra} raises a thousand predecessors and a thousand
successors in his family.

67. In this life he will enjoy all worldly pleasures and live long without any
disease. At the end of the span of his life he will have a peaceful death.

68-70. He will assume then a divine auspicious body endowed with eight
accomplishments. He will travel by a divine aerial chariot attended by celestial
gods. He will enjoy the pleasures of \tl{vidyādharas}, powerful \tl{gandharvas},
in the worlds of Indra and other guardians of the quarters and those of
Prajāpatis and finally reach Brahmā’s region where he will sport with a hundred
virgins.

71. He will enjoy different kinds of pleasures there for the full period of
the span of life of Brahmā. He will then enjoy the pleasures in the Viṣṇuloka
till hundred Brahmas die.

72. Thereafter he will attain Śivaloka and enjoy everlasting bliss there.
Finally he will attain Śivasāyujya. No suspicion need be entertained in this
matter.

73. After going through the essence of all \tl{Upaniṣads} again and again, this
is what has been arrived at that the \tl{tripuṇḍra} is conducive to great
excellence.

74. A \tl{brāhmaṇa} who censures the ash is no longer a \tl{brāhmaṇa} but of
another low caste. He will undergo the tortures of terrible hell for the period
of the span of life of the four-faced Brahmā.

75. A man who wears the \tl{tripuṇḍra} while performing \tl{śrāddha, yajña, japa,
homa, vaiśvadeva} and the worship of the deities is a purified soul and he
conquers even death.

76. When impurities are evacuated, a bath with water shall be performed; a bath
with the ash (\tl{bhasma}) is always purificatory; a bath with mantras removes
sin and if a bath with knowledge is taken, the greatest goal will be reached.

77. A man who takes the bath of ashes derives that benefit which all holy
centres accord. He gets the merit thereof.

78. Bath with the ash is a holy centre where Gaṅgā \tl{snāna} is possible every
day. Śiva is represented by the ash which directly sanctifies the three worlds.

79. Infructuous is the knowledge, meditation, gift and \tl{japa} if these are
performed by a \tl{brāhmaṇa} without wearing \tl{tripuṇḍraka}.

80. A forest-dweller, virgins and men without initiation shall apply the ash
pasted in water upto the midday and thereafter without water.

81. He who wears \tl{tripuṇḍra} like this regularly with a pure controlled mind
must be considered a true devotee of Śiva. He derives worldly pleasures and
salvation.

82. If a person does not wear a bead of \tl{rudrākṣa} which accords many merits,
if he is devoid of \tl{tripuṇḍra} as well, his life becomes futile.

83. Thus I have briefly told you the greatness of \tl{tripuṇḍra}. This is a
secret to be safely guarded by you from all living beings.

84. O leading sages, in the different parts of the body as the forehead \etc
three lines constitute the \tl{tripuṇḍra}.

85. The \tl{tripuṇḍra} on the forehead extends from the middle of the eyebrows
to the tips of the brows on either side.

86. With the middle and the ring fingers a line drawn in the opposite direction
is called \tl{tripuṇḍra}.

87. With the three middle fingers, take the ashes and apply the \tl{tripuṇḍra}
on the forehead. It would give worldly pleasures and salvation.

88. For each of the three lines there are nine deities everywhere in the body.
I shall mention them. Listen attentively.

89-90. The nine deities of the first line are: The syllable “A”, Gārhapatya fire
(sacrificial fire), Earth, Dharma, the attribute Rajas, Ṛgveda, Kriyāśakti
(the power to do), Prātaḥsavana (morning rituals) and Mahādeva. O foremost among
sages, this shall be carefully understood by those who are initiated in the cult
of Śiva.

91-92. The nine deities of the second line are: The syllable “U”, Dakṣiṇā fire
(sacrificial fire), the principle of Ether, Attribute Sattva, Yajurveda,
Mādhyandina Savana (midday rituals), Icchāśakti (the will-power), the Antarātman
(the immanent soul) and Maheśvara. O foremost among sages, this must be
carefully understood by those who are initiated in the cult of Śiva.

93-94. The nine deities of the third line are: The syllable “M”, Āhavanīya
(sacrificial) fire, the supreme soul, the attribute Tamas, heaven, Jñāna Śakti,
Sāmaveda, the third Savana (evening rituals) and Śiva. O foremost among sages,
this must be carefully understood by those initiated in the cult of Śiva.

95. Thus making obeisance to the deities of the different parts with devotion,
one shall apply the \tl{tripuṇḍra}. One will become pure and derive worldly
pleasures and salvation.

96. Thus I have mentioned, O lordly sages, the deities of the different parts of
the body. Now please listen to the different parts connected with them.

97. These lines are to be made either in thirty-two places, or half of that—in
sixteen places, or in eight places or in five places.

98-102. The thirty-two places are: head, forehead, two ears, two eyes, two noses,
mouth, neck, two arms, two elbows, two wrists, chest, two sides, navel, two
testicles, two thighs, two knees, two calves, two heels and two feet. The names
of the following shall be uttered when the \tl{tripuṇḍra} is applied: Fire,
Water, Earth, Wind, the quarters, the guardians of the quarters, the eight
Vasus\footnote{The eight Vasus mentioned in this verse differ in certain names
from those in the \tl{Śatapatha Brāhmaṇa}: \tl{pṛthivī ca vāyuścāntarikṣaṃ
cādityaśca dyauśca candramāśca nakṣatrāṇi caite vasava} (\tl{Śatapatha-brāhmaṇa}
11.6.3.6.)}. The eight Vasus are Dharā, Dhruva, Soma, Āpa, Anila, Anala,
Pratyūṣa and Prabhasa.

Or the devotee shall apply the \tl{tripuṇḍra} in sixteen parts of the body.

103-109. The sixteen parts mentioned before are: head, forehead, neck, two
shoulders, two arms, two elbows, two wrists, chest, navel, two sides and back.
The names of the deities presiding over them and which are to be worshipped are:
two Aśvins, Dasra and Nāsatya, Śiva, Śakti, Rudra, Īśa, Nārada, and nine Śaktis
—Vāmā \etc, or the sixteen parts are: Head, hair, two eyes, the mouth, two arms,
chest, navel, two thighs, knees, two feet and the back. The deities are: Śiva,
Candra, Rudra, Brahmā, Vighneśvara, Viṣṇu, Śrī in the heart, Śambhu, Prajāpati
in the navel, Nāga, Nāgakanyās, Ṛṣikanyās in the feet and the ocean of vast
expansion in the back. Now the eight parts are mentioned.

110. The private parts, forehead, the excellent pair of ears, two shoulders,
chest and navel—these are the eight parts of the body.

111. The presiding deities are Brahmā and the seven sages\footnote{Seven sages,
\viz Marīci, Atri, Aṅgiras, Pulastya, Pulaha, Kratu and Vasiṣṭha are represented
by a group of seven stars called Ursa Major.}. O lordly sages, this is what has
been mentioned by those who know about the efficacy of the ashes (\tl{bhasma}).

112. Or these five parts are to be used for applying ashes as mentioned by those
who know more about the efficacy of the ashes. They are: forehead, two arms,
chest and navel.

113-114. Considering the place and time whatever possible shall be done by
the devotee. If incapable of dusting the whole body with the holy ashes he shall
have the \tl{tripuṇḍra} on the forehead alone, remembering lord Śiva,
the three-eyed, the support of the three \tl{guṇas} and the progenitor of
the three \tl{devas} by repeating \tl{namaḥ śivāya} (obeisance to Śiva).

115. He shall have \tl{tripuṇḍra} in the sides saying \tl{īśābhyāṃ namaḥ}
(obeisance to Śiva and the goddess) and in the forearms by saying \tl{bījābhyām
namaḥ} (obeisance to the generating seeds).

116. He shall apply the ashes beneath by saying \tl{namaḥ pitṛbhyām} (obeisance
to the manes) and above by saying \tl{namaḥ umesābhyām} (obeisance to Umā and
Īśa), on the back and the back of the head by saying \tl{namaḥ bhīmāya}
(obeisance to Bhīma).
