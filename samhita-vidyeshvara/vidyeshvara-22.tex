\chapter{On the partaking of the \emph{Naivedya} of Śiva and the greatness of Bilva}

The sages said:

1. O good sage, we have heard before, that the offering of eatables
(\tl{naivedya}) made to Śiva should not be taken by others. Please tell us
decisively about this and also about the greatness of Bilva\footnote{\emph{Aegle
marmelos L.} Its leaves and fruits are sacred to Śiva.}.

Sūta said:

2. O sages, all of you please hear now attentively. With pleasure I shall
explain everything. All of you who take up Śiva’s sacred rites are really blessed.

3. A devotee of Śiva who is pure and clean, devoutly performing good rites and
of fixed resolve shall partake of Śiva’s \tl{naivedya}. He shall abandon all
thoughts which are not worthy of being entertained.

4. Even at the sight of the \tl{naivedya} of Śiva, all sins disappear. When it
is taken in, crores of merits flock in, in no moment.

5. A thousand sacrifices are of no avail. Hundred millions of sacrifices are
useless. When Śiva’s \tl{naivedya} is eaten one will attain identity with Siva.

6. If in a family Śiva’s \tl{naivedya} becomes popular with the members, that
house becomes sacred and it can make others also sacred.

7. When Śiva’s \tl{naivedya} is offered it shall be accepted with pleasure and
humility. It shall be eaten eagerly while remembering Siva.

8. If anyone who is offered Śiva’s \tl{naivedya} delays taking it immediately,
thinking that it can be taken afterwards, he will incur sins.

9. If anyone has no inclination to take Śiva’s \tl{naivedya} he becomes a sinner
of sinners and is sure to fall into hell.

10. After initiation in shaiva cult, the devotee shall partake of the offerings
of eatables made to the phallic image whether conceived in the heart or made of
moon-slab, silver, gold \etc

11. The \tl{naivedya} of all phallic icons is called a great favour and is
auspicious. A devotee after initiation into shaiva cult shall eat it.

12. Please listen to the decision with pleasure on partaking of Śiva’s
\tl{naivedya} by persons who take initiation in other cults but maintain their
devotion to Śiva.

13-15. With regard to the following phallic images \viz that which is obtained
from \tl{śālagrāma} stone, \tl{rasaliṅga}, \tl{liṅgas} made of rock, silver,
gold, crystals and gems, \tl{liṅgas} installed by \tl{devas} and \tl{siddhas},
Kāśmīra \tl{liṅgas} and \tl{Jyotirliṅgas}\footnote{Jyotirliṅgas are twelve in
number: (1) Somanātha (at Somanath Pattan, Gujarat), (2) Mallikārjuna or Śrīśaila
(on a mountain near the river Kṛṣṇā), (3) Mahākāla, Mahākāleśvara (at Ujjain),
(4) Oṃkāra Māndhātā on the Narmadā, (5) Amareśvara (at Ujjain),(6) Vaidyanātha
also called Nāganātha (at Deogarh Bengal), (7) Rāmeśa or Rāmeśvara (on the island
of Rameśvara), (8) Bhīma Śaṅkara (in the Rājamundry district), (9) Viśveśvara at
Benares, (10) on the banks of the Gomatī, (11) Gautameśa, also called Vāmeśvara
(not located), (12) Kedārnatha in the Himalayas.}, the partaking of
the \tl{naivedya} of Śiva is on a par with the rite of \tl{cāndrāyaṇa}
\footnote{\tl{Cāndrāyana} is a religious observance, an expiatory penance,
regulated by the period of the moon’s waxing and waning. In this rite, the daily
quantity of food which consists of fifteen mouthfuls at the full moon is
diminished by one mouthful every day during the dark fortnight till it is
reduced to zero at the new moon and is increased in like manner during the bright
fortnight.}. Even the slayer of a \tl{brāhmaṇa} if he partakes of the remains of
the food offered to the God quells all his sins immediately.

16-17. In regard to Bāṇaliṅga, metallic \tl{liṅga}, Siddha-liṅga and
Svayambhū-liṅga and in all other idols, Caṇḍa, one of the attendants of Śiva, is
not authorised. Where Caṇḍa is not authorised, the food-offering can be partaken
of by men with devotion. But no man shall partake of the food-offering where
Caṇḍa is authorised.

18. After performing the ceremonial ablution duly if any one drinks the water
three times, all the three types of sins committed by him are quickly destroyed.

19-20. If at all anything from \tl{śivanaivedya} is not to be taken it is that
article which is actually put on the \tl{liṅga}. O great sages, that what is not
in contact with the \tl{liṅga} is pure and as such, it can be partaken of. When
it is in contact with \tl{śālagrāma śilā}, it is pure and can be taken whether
it is food-offering, leaf, flower, fruit or water.

21. O great sages, thus I have told you the decision about food-offering, Now,
hear me attentively, with devotion. I shall explain the greatness of Bilva.

22. This Bilva is the symbol of Śiva. It is adored even by the Gods. It is
difficult to understand its greatness. It can only be known to a certain extent.

23. Whatever holy centre there is in the world finds a place under the root of
Bilva.

24. He who worships Mahādeva in the form of \tl{liṅga} at the root of Bilva
becomes a purified soul; he shall certainly attain Śiva.

25. He who pours water over his head at the root of a Bilva can be considered to
have taken his bath in all sacred waters in the earth. Verily he is holy.

26. Seeing the water basin round the foot of the Bilva tree full of water, Śiva
becomes greatly pleased.

27. The man who worships the root of a Bilva tree offering scents and flowers
attains the region of Śiva. His happiness increases; his family flourishes.

28. He who places a row of lighted lamps at the root of Bilva tree with
reverence becomes endowed with the knowledge of truth and merges into Śiva.

29. He who worships the Bilva tree abounding in fresh tender sprouts becomes
free from sins.

30. If a man piously feeds a devotee of Śiva at the root of a Bilva tree he
reaps the fruit thereof, ten million times more than in the usual course.

31. He who makes a gift of rice cooked in milk and ghee to a devotee of Śiva,
at the root of a Bilva tree will never become poor.

32. O \tl{brāhmaṇas}, thus I have explained to you the mode of worship of Śiva’s
phallic image with all its divisions and sub-divisions. It is of two types: one
is enjoined for those who are actively engaged in worldly pursuits and the other
is meant for those who have actually renounced them.

33. The worship of the pedestal yields all cherished desires to those who are
engaged in worldly pursuits. They shall perform the complete worship in a vessel.

34. At the end of consecration, he shall offer cooked rice \tl{śāli} as
food-offering. At the conclusion of worship, the phallic image shall be kept in
a pure casket separately in the house.

35. He who has renounced the world (the \tl{nivṛtta}) shall perform \tl{karapūjā}
(worship in the palm of the hand). He shall offer that food to the deity which
he is accustomed to take himself. The subtle phallic image is specially
recommended for the \tl{nivṛttas}.

36. He shall offer holy ashes both for worship and food offering. At the end of
worship he shall always keep the phallic image on his head.
