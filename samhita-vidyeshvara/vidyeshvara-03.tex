\chapter{The deliberation on the achievable and the means of achievement}

Vyāsa said:

1. On hearing the words of Sūta, the great sages said, “Please narrate
the wonderful \tl{purāṇa} that fully treats of the essence of Vedānta”.

2. Very delighted at the request of the sages Sūta meditated on Śiva and spoke
to them.

Sūta said:

3. Contemplating on Śiva free from ailments may ye all hear this \tl{Śivapurāṇa},
the foremost among \tl{purāṇas}, that amplifies the essence of the Vedas.

4-5. Where the trio, \tl{bhakti} (Piety) \tl{jñāna} (Wisdom) and \tl{vairāgya}
(non-attachment) has been proclaimed and the object which is knowable only
through Vedānta, has been particularly described.

Sūta said:

6-8. May ye all hear the \tl{purāṇa} that embibes the essence of the Vedas.
Formerly, when many \tl{kalpas} (Aeons) elapsed and this \tl{kalpa} started with
the process of creation, a great dispute arose among the sages of six clans who
held divergent views as to which is great and which is not. They approached
Brahmā the Creator, to ask him about the imperishable.

9-12. All of them with palms joined in reverence addressed him with words
couched in humility—“Thou art the creator of the entire universe, the cause of
all causes. Who is that Being older than all Principles, the greatest of
the great?“

Brahmā said:

“That from whom words recede, not approaching him even with the mind; that from
whom this entire universe beginning with Brahmā, Viṣṇu, Rudra and Indra, along
with all elements and all sense-organs, is evolved at first; he is the lord
Mahādeva the omniscient, the lord of the universe. He can be realised by supreme
devotion and not by other means.

13. Rudra, Hari, Hara and other lords of \tl{devas} are ever desirous of seeing
Him, moved by great devotion.

14. Of what avail is a verbose statement? One is liberated by devotion unto Śiva.
Devotion to the deity is due to His Grace; and His grace is due to devotion just
as the seed gives rise to the sprout and the sprout produces the seed.

15. Hence, O Brahmins, all of you descend to the earth, to propitiate the Lord.
You have to perform a sacrifice of long duration for a thousand years.

16. It is by the grace of Śiva alone who will be the presiding deity of this
sacrifice that the means of achievement of the Achievable can be realised and
that is the essence of the \tl{vidyā} (mystic learning) mentioned in the Vedas.

The sages said:

17. What is that great Achievable? What is that great means of achievement? Of
what sort is the performer of the rite? Please mention these precisely.

Brahmā said:

18. The attainment of Śiva’s region is the Achievable. Means of achievement is
the service rendered unto Him. \tl{Sādhaka} (the performer of the rite) is
the person who is free from desire even for permanence which attitude is
the result of His grace.

19. Rites mentioned in the Vedas should be performed with the fruits thereof
dedicated to Him. Thence, through \tl{sālokya}\footnote{The devotee attains
exemption from further transmigration and his identification with the deity,
gradually through four stages; \viz \tl{sālokya} (being in the same world with
the deity), (\tl{sāmīpya} (nearness to the deity), \tl{sāyujya} (intimate union
with the deity) and \tl{sārūpya} (assimilation to the deity). SP. adds
\tl{sārṣṭi} (9.26) (equality in rank, condition or power) as one of the grades
of \tl{mukti}.} he attains the feet of the great Lord.

20. All attain the great fruit according to the standard in devotion achieved.
The ways of achieving these standards are manifold as expounded by Isa Himself.

21-22. I shall condense the same and tell you the essential means. Listening to
the glory of Śiva, glorifying him by means of words, and deliberation in
the mind, these constitute the greatest of the means. Maheśvara is to be heard,
glorified and meditated upon.

23. Thus \tl{śruti}\footnote{The word \tl{śruti} in the \tl{purāṇas} does not mean
‘sacred tradition’ but simply ‘tradition’. ‘\emph{Ancient Indian
Historical Tradition}’ (Pargiter) Ch.2.} is our authority. Resorting solely to
this great means, all of you attain the Achievable.

24. Regarding visible things people see with their eyes and begin their activity.
Concerning the invisible everywhere, they know through the ears and activise
themselves.

25. Hence \tl{śravaṇa} (listening) is the first rite. The intelligent scholar
must listen to the oral explanation of the preceptor and then practise the
other rites—\tl{kīrtana} (glorifying) and \tl{manana} (deliberation).

26-27. When all the means upto \tl{manana} are well exercised, \tl{śivayoga}
(unification with Śiva) results gradually through \tl{sālokya} \etc All
the ailments of the body are nullified and supreme bliss is realised. Painful
indeed is the process but later on everything becomes auspicious from beginning
to end.
