\chapter{The greatness of Rudrākṣa}

Sūta said:

1. O sage Śaunaka, highly intelligent, of the form of Śiva, noble-minded, please
listen to the greatness of \tl{rudrākṣa}. I shall explain it briefly.

2. \tl{Rudrākṣa} is a favourite bead of Śiva. It is highly sanctifying. It
removes all sins by sight, contact and \tl{japas}.

3. O sage, formerly the greatness of \tl{rudrākṣa} was declared to the Goddess
by Śiva, the supreme soul, for rendering help to the worlds.

Śiva said:

4. O Śivā, Maheśāni, be pleased to hear the greatness of \tl{rudrākṣa}. I speak
out love for you from a desire for the benefit of the devotees of Śiva.

5-7. O Mahesāni, formerly I had been performing penance for thousands of divine
years. Although I had controlled it rigorously, my mind was in flutter. Out of
sport, I being self-possessed just opened my eyes, O Goddess, from a desire of
helping the worlds. Drops of tears fell from my beautiful half-closed eyes. From
those tear-drops there cropped up the \tl{rudrākṣa} plants.

8. They became immobile. In order to bless the devotees they were given to
the four \tl{varṇas} devoted to the worship of Viṣṇu.

9-10. \tl{Rudrākṣas} grown in Gauḍa\footnote{Gauḍa-deśa, according to
\tl{Skandapurāṇa}, was the central part of Bengal extending from Vaṅga to
the borders of Orissa: \tl{vaṅgadeśāṃ samārabhya bhūvaneśāntagaḥ śive |
gauḍadeśaḥ samākhyātaḥ sarvavidyāviśāradaḥ}} land became great favourites of Śiva.
They were grown in Mathurā, Laṅkā, Ayodhyā, Malaya\footnote{Malaya: a mountain
range on the west of Malabar, the western ghats, abounding in sandal trees.},
Sahya\footnote{Sahya: It is one of the seven principal ranges, the other six
being Mahendra, Malaya, Sūktimat, Rikṣa, Vindhya and Pāripātra or Pāriyātra.}
mountain, Kāśī and other places. They are competent to break asunder
the clustered sins unbearable to the others, as the sacred texts have declared.

11. At my bidding they were classified into \tl{brāhmaṇas, kṣatriyas, vaiśyas
and śūdras}. These \tl{rudrākṣas} are of auspicious nature.

12. The colours of the four types of \tl{rudrākṣas} are respectively white, red,
yellow and black. All people shall wear the \tl{rudrākṣa} of their own
\tl{varna}.

13. If they desire their benefit, namely worldly pleasures and salvation and if
the devotees of Śiva wish to gratify Śiva they must wear the \tl{rudrākṣa}.

14. A \tl{rudrākṣa} of the size of an \emph{Emblic myrobalan} (or Dhātrīphala)
is mentioned as the most excellent; one of the size of the fruit of the jujube
tree (or Badarīphala) is spoken of as the middling.

15. O Pārvatī, lovingly listen to this from a desire for the benefit of
the devotees. The meanest of \tl{rudrākṣas} is of the size of a gram according
to this excellent classification.

16. O Maheśvarī, even the \tl{rudrākṣa} which is only of the size of the fruit
of the jujube accords the benefit and heightens happiness and good fortune.

17. That which is of the size of the \emph{Emblic myrobalan} is conducive to
the destruction of all distresses. That which is of the size of a Guñjā (the
berry) is conducive to the achievement of the fruit of all desires.

18. The lighter the \tl{rudrākṣa}, the more fruitful it is. Each of these is
fruitful and that of a weight of one tenth is considered by scholars as the most
fruitful.

19. The wearing of \tl{rudrākṣa} is recommended for the sake of destroying sins.
Hence that which is conducive to the achievement of every object has to be worn
certainly.

20. O Parameśvarī, no other necklace or garland is observed in the world to be
so auspicious and fruitful as the \tl{rudrākṣa}.

21. O Goddess, \tl{rudrākṣas} of even size, glossy, firm, thick and having many
thornlike protrusions yield desires and bestow worldly pleasures and salvation
for ever.

22. Six types of \tl{rudrākṣas} shall be discarded: that which is defiled by
worms, is cut and broken, has no thornlike protrusions, has cracks and is not
circular.

23. That which has a natural hole from end to end is the most excellent; that
which is bored through by human effort is the middling one.

24. The wearing of \tl{rudrākṣa} is spoken of as conducive to the destruction of
great sins. If eleven hundred \tl{rudrākṣas} are worn on the person, the man
assumes the form of Rudra.

25. Even in hundreds of years it is impossible to describe adequately
the benefit derived by wearing eleven hundred and fifty \tl{rudrākṣas}.

26. A devout man shall make a coronet consisting of five hundred and fifty
\tl{rudrākṣa}s.

27. A person of pious nature shall make three circular strings in the manner of
the sacred thread, each having three hundred and sixty beads.

28. O Maheśvarī, three \tl{rudrākṣas} must be worn on the tuft and six in each
of the ears right and left.

29-30. Hundred and one \tl{rudrākṣas} shall be worn round the neck; eleven
\tl{rudrākṣas} shall be worn round each of the arms, elbows and wrists. Devotees
of Śiva shall have three \tl{rudrākṣas} in the sacred thread and round the hips
five \tl{rudrākṣas} shall be tied.

31. O Parameśvarī, the person by whom so many \tl{rudrākṣas} are worn is worthy
of being bowed to and adored by all like Maheśa.

32. Such a person while in contemplation shall be duly seated and addressed
“O Śiva”. Seeing him, every one is freed from sins.

33. This is the rule regarding eleven hundred \tl{rudrākṣas}s If so many are not
available, another auspicious procedure I mention to you.

34-36. One \tl{rudrākṣa} shall be worn on the tuft, thirty on the head, fifty
round the neck; sixteen in each of the arms; twelve round each of the wrists;
five hundred on the shoulders, and three strings each having hundred and eight
in the manner of the sacred thread. He who wears in all a thousand
\tl{rudrākṣas} and is of firm resolve in performing rites is bowed to by all
\tl{devas} like Rudra himself.

37-39. One \tl{rudrākṣa} shall be worn on the tuft, forty on the forehead,
thirty-two round the neck; hundred and eight over the chest; six in each of
the ears; sixteen round each of the arms; O lord of sages, according to
the measurement of the forearms, twelve or twice that number shall be worn there.
A person who wears so many, out of love, is a great devotee of Śiva. He shall
be worshipped like Śiva. He is worthy of being always honoured by all.

40. It shall be worn on the head repeating \tl{īśāna mantra}\footnote{
\tl{Vājasaneyisaṃhitā} (of \tl{Śuklayajurveda}) 39.8.}; on the ears with
\tl{tripuruṣa mantra}\footnote{Ibid. 17.11.}; round the neck with \tl{aghora
mantra}\footnote{Ibid. 16.2.} and on the chest also likewise.

41. The wise devotee shall wear the \tl{rudrākṣa} round the forearms with
\tl{aghora bīja} mantra. A string of fifteen beads shall be worn on the stomach
with \tl{vāmadeva mantra}\footnote{TA. 10.44.1; Mahā 4.17.2.}.

42. With five mantras—\tl{sadyojāta} \etc three, five or seven garlands shall
be worn. Or all beads shall be worn with the \tl{mūla mantra}\footnote{The
five-syllabled mantra “\tl{namaḥ śivāya}” is the basic \tl{mūla} mantra of Śiva.}.

43. A devotee of Śiva shall refrain from eating meat, garlic, onion, red garlic,
potherb, \tl{śleṣmātaka}, pig of rubbish and liquors.

44. O Umā, daughter of the mountain, the white \tl{rudrākṣa} shall be worn by
the \tl{brāhmaṇa}, the red by the \tl{kṣatriya}, the yellow by the \tl{vaiśya},
the black by the \tl{śūdra}. This is the path indicated by the Vedas.

45. Whether he is a householder, forest-dweller, ascetic or of any order, none
shall go out of this secret advice. Only by great merits can the opportunity to
wear the \tl{rudrākṣa} be obtained. If he misses it he will go to hell.

46. The \tl{rudrākṣas} of the size of an \emph{Emblic myrobalan} and those of
lighter weight but depressed with thorns, those eaten by worms or without holes
and those characterized by other defects shall not be worn by those wishing for
auspicious results. They shall avoid small ones of the size of gram. O Umā,
\tl{rudrākṣa} is an auspicious complement to my phallic image. The small one is
always praiseworthy.

47. People of all \tl{varṇas} and \tl{āśramas} even women and \tl{śūdras} can
wear \tl{rudrākṣa} at the bidding of Śiva. The ascetics shall wear it with
the \tl{Praṇava}.

48. If any one wears it during the day he is freed from sins committed during
the night; if he wears it during the night he is freed from the sins committed
during the day. Similar is the result with its wearing during morning, midday or
evening.

49. Those who wear \tl{tripuṇḍra}, the matted hair and the \tl{rudrākṣa} do not
go to Yama’s abode.

50-52. [Yama’s directive to his attendants:—] “Those who wear at least one
\tl{rudrākṣa} on their heads, \tl{tripuṇḍra} on the forehead and repeat the
five-syllabled mantra shall be honoured by you all. They are indeed saintly men.
You can bring the man here who has no \tl{rudrākṣa} on his person, and no
\tl{tripuṇḍra} on his forehead and who does not utter the five-syllabled mantra.
All those who have the ash and \tl{rudrākṣa} shall be honoured always by us
after knowing their power. They shall never be brought here”.

53. Yama commanded his attendants like this. They too remained quiet agreeing
to it. In fact they were surprised.

54. Hence Mahādevī, the \tl{rudrākṣa} as well as the person who wears it is my
favourite. O Pārvatī, even if he has committed sins he becomes pure.

55. He who wears \tl{rudrākṣa} round the hands and arms and over the head cannot
be killed by any living being. He shall roam in the world in the form of Rudra.

56. He shall be respected by the Gods and Asuras always. He shall be honoured
like Śiva. He removes the sin of any one seen by him.

57. If a person is not liberated after meditation and acquisition of knowledge
he shall wear \tl{rudrākṣa}. He shall be freed from all sins and attain
the highest goal.

58. A mantra repeated with \tl{rudrākṣa} is a crore times more efficacious.
A man wearing \tl{rudrākṣa} derives a hundred million times more merit.

59. O Goddess, as long as the \tl{rudrākṣa} is on the person of a living soul he
is least affected by premature death.

60. One shall attain Rudra on seeing a person with \tl{tripuṇḍra}, his limbs
covered with \tl{rudrākṣa} and repeating the \tl{mṛtyuñjaya mantra}
\footnote{\tl{Vājasaneyisaṃhitā} (of \tl{Śuklayajurveda}) 30.60.}.

61. He is a favourite of the five deities\footnote{The five deities referred to
here are: the sun, Gaṇeśa, Goddess Durgā, Rudra and Viṣṇu. Cp. \tl{ādityaṃ
gaṇanāthaṃ ca devīṃ rudraṃ ca keśavam | pañcadaivatamityuktaṃ sarvakarmasu
pūjayet}} and a favourite of all gods. O beloved, a devotee shall repeat all
mantras wearing a garland of \tl{rudrākṣa}s (or counting on the beads).

62. Even the devotees of Viṣṇu and other deities shall unhesitatingly wear
the \tl{rudrākṣa}. Especially the devotee of Rudra shall wear \tl{rudrākṣas}
always.

63. \tl{Rudrākṣas} are of various types. I shall explain their different
classifications. O Pārvatī, hear with great devotion. These \tl{rudrākṣas}
bestow worldly pleasures and salvation.

64. A \tl{rudrākṣa} of a single face is Śiva Himself. It bestows worldly
pleasures and salvation. The sin of \tl{brāhmaṇa-slaughter} is washed off at
its mere sight.

65. Where it is adored, Fortune cannot be far off. Harms and harassments perish.
All desires are fulfilled.

66. A \tl{rudrākṣa} with two faces is Īśa, the lord of \tl{devas}. It bestows
the fulfilment of all desires. Especially, that \tl{rudrākṣa} quickly quells
the sin of cow-slaughter.

67. A \tl{rudrākṣa} with three faces always bestows means of enjoyment. As a
result of its power all lores become firmly established.

68. A \tl{rudrākṣa} of four faces is Brahmā Himself. It quells the sin of
man-slaughter. Its vision and its contact instantaneously bestow the achievement
of the four aims of life.

69. A \tl{rudrākṣa} with five faces is Rudra Himself. Its name is Kālāgni. It is
lordly. It bestows all sorts of salvation and achievement of all desired objects.

70. A five-faced \tl{rudrākṣa} dispels all sorts of sins such as accrue from
sexual intercourse with a forbidden woman and from eating forbidden food.

71. A \tl{rudrākṣa} with six faces is Kārtikeya. A man who wears it on the right
arm is certainly absolved of the sins of \tl{brāhmaṇa-slaughter} and the like.

72. A \tl{rudrākṣa} with seven faces, O Maheśāni, is called Anaṅga. O Deveśī, by
wearing it even a poor man becomes a great lord.

73. A \tl{rudrākṣa} with eight faces is called Vasumūrti and Bhairava. By
wearing it a man lives the full span of life. After death he becomes
the trident-bearing lord (Śiva).

74. A \tl{rudrākṣa} with nine faces is also Bhairava. Its sage is Kapila. Its
presiding goddess is Durgā of nine forms, Maheśvarī Herself.

75. That \tl{rudrākṣa} shall be worn on the left hand with great devotion. He
shall certainly become Sarveśvara like me.

76. O Maheśānī, a \tl{rudrākṣa} with ten faces is Lord Janārdana Himself.
O Deveśī, by wearing it, the devotee shall achieve the fulfilment of all desires.

77. O Parameśvarī, a \tl{rudrākṣa} with eleven faces is Rudra. By wearing it one
becomes victorious everywhere.

78. One shall wear the twelve-faced \tl{rudrākṣa} on the hair of the head. All
the twelve Ādityas (suns) are present therein.

79. A \tl{rudrākṣa} with thirteen faces is Viśvedeva. By wearing it, a man will
attain the realisation of all desires. He will derive good fortune and
auspiciousness.

80. A \tl{rudrākṣa} with fourteen faces is the highest Śiva. It shall be worn on
the head with great devotion. It quells all sins.

81. O daughter of the king of mountains, thus I have explained to you
the different types of \tl{rudrākṣas} based on the number of faces. Please listen
to the mantras with devotion:

\begin{itemize}
  \item\tl{oṃ hrīṃ namaḥ} (single-faced)
  \item\tl{oṃ namaḥ} (2 faces)
  \item\tl{oṃ klīṃ namaḥ} (3 faces)
  \item\tl{oṃ hrīṃ namaḥ} (4 faces)
  \item\tl{oṃ hrīṃ namaḥ} (5 faces)
  \item\tl{oṃ hrīṃ huṃ namaḥ} (6 faces)
  \item\tl{oṃ huṃnamaḥ} (7 faces)
  \item\tl{oṃ huṃ namaḥ} (8 faces)
  \item\tl{oṃ hrīṃ huṃ namaḥ} (9 faces)
  \item\tl{oṃ hrīṃ namaḥ namaḥ} (10 faces)
  \item\tl{oṃ hrīṃ huṃ namaḥ} (11 faces)
  \item\tl{oṃ krauṃ kṣauṃ rauṃ namaḥ} (12 faces)
  \item\tl{oṃ hrīṃ namaḥ} (13 faces)
  \item\tl{oṃ nama} (14 faces)
\end{itemize}

82. For the achievement of all desired objects, the devotee shall wear
the \tl{rudrākṣa} with mantras. He shall have great devotion and faith. He shall
be free from lethargy.

83. The man who wears the \tl{rudrākṣa} without mantra falls into a terrible
hell and stays there during the tenure of fourteen Indras.

84-85. On seeing a man with the garland of \tl{rudrākṣas}, all evil spirits,
ghosts, \tl{piśācas}, witches like \tl{dākinī} and \tl{śākinī}, other malignant
spirits, evil charms and spells \etc fly away suspecting a quarrel.

86. Seeing a devotee with the garland of \tl{rudrākṣas}, O Pārvatī, Śiva, Viṣṇu,
Devī, Gaṇapati, the sun and all the Gods are pleased.

87. Thus realising its greatness the \tl{rudrākṣa} must be worn well,
O Maheśvarī, repeating the mantras with devotion to make virtues flourish.

88. Thus, the greatness of ash and \tl{rudrākṣa} that bestow worldly pleasures
and salvation, was explained to Girijā by Śiva, the supreme soul.

89. The persons who apply ash and wear \tl{rudrākṣa} are great favourites of
Śiva. Enjoyment of worldly pleasures and salvation are certainly due to their
influence.

90. He who applies ash and wears \tl{rudrākṣa} is called a devotee of Śiva.
A person devoted to the \tl{japa} of the five-syllabled mantra is a perfect and
noble being.

91. If Mahādeva is worshipped without the \tl{tripuṇḍra} of ash and without
the garland of \tl{rudrākṣa}, he does not bestow the fruit of cherished desire.

92. Thus, O lord of sages, whatever has been asked has now been explained.
The greatness of ash and \tl{rudrākṣa} bestows the luxuriant fulfilment of
all desires.

93. He who regularly listens to the highly auspicious greatness of ash and
\tl{rudrākṣa} with devotion shall attain the fulfilment of all desires.

94. He will enjoy all happiness here. He will be blessed with sons and grandsons.
In the next world he will attain salvation. He will be a great favourite of Śiva.

95. O lordly sages, thus the compendium of \tl{Vidyeśvara-saṃhitā} has been
narrated to you all. As ordered by Śiva it bestows achievement of everything
and salvation.
