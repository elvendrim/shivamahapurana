\chapter{Answers Clarifying the Doubts of the Sages}

Sūta said:

1. O saintly men, the question that you put me is very pertinent. Prompted by my
love towards you all I shall, remembering my preceptor, the benefactor of the
three worlds, tell you everything. All of you listen attentively.

2. The entire essence of Vedānta is contained in the excellent \tl{Śivapurāṇa}.
It dispells all sins. It affords the attainment of the highest truth hereafter.

3. O brahmins, the great glory of Śiva, that destroys the sin of the Kali age,
unfolds itself in the \tl{purāṇa} and yields the fruits of the four varieties
(\tl{dharma, artha, kāma} and \tl{mokṣa}).

4. By the single-minded study of that most excellent \tl{Śivapurāṇa} excellent
brahmins will attain salvation.

5. It is only as long as the \tl{Śivapurāṇa} has not risen high in the world,
that \tl{brahma-hatyā} (the sin of slaying a brahmin) and other sins display
themselves.

6. It is only as long as the \tl{Śivapurāṇa} has not risen high in the world,
that the evil portents of Kali fearlessly roam about.

7. It is only as long as the \tl{Śivapurāṇa} has not risen high in the world,
that the different sacred texts clash together in disputation.

8. It is difficult even to great men to comprehend Śiva’s features as long as
the \tl{Śivapurāṇa} has not risen high in the world.

9. The cruel attendants of Yama roam about fearlessly as long as the
\tl{Śivapurāṇa} has not risen high in the world.

10. All the other \tl{purāṇas} roar loudly on the earth as long as the
\tl{Śivapurāṇa} has not risen high in the world.

11. All the holy centres enter into mutual wrangles and disputes on the earth as
long as the \tl{Śivapurāṇa} has not risen high in the world.

12. All the mantras rejoice in mutual disputes as long as the \tl{Śivapurāṇa}
has not risen high in the world.

13. All the sectors of pilgrimage engage themselves in mutual disputes as long
as the \tl{Śivapurāṇa} has not risen high in the world.

14. All the altars and pedestals engage themselves in mutual disputes as long as
the \tl{Śivapurāṇa} has not risen high in the world.

15. All the gifts engage themselves in disputes as long as the \tl{Śivapurāṇa}
has not risen high in the world.

16. All those gods engage themselves in mutual disputes as long as the
\tl{Śivapurāṇa} has not risen high in the world.

17. All the philosophical tenets engage themselves in mutual disputes as long as
the \tl{Śivapurāṇa} has not risen high in the world.

18. O foremost among brahmanical sages, I cannot adequately describe the fruit
accruing from reciting and listening to this \tl{Śivapurāṇa}.

19. Even then, O sinless ones, I shall succinctly describe its greatness as
narrated to me by Vyāsa. Please listen attentively.

20. He who reads a single stanza or even half of it piously becomes free from
sin instantaneously.

21. He who reads every day as much of \tl{Śivapurāṇa} as he can with devotion
and alertness is called \tl{jīvanmukta} (a living liberated soul).

22. He who continues to worship this \tl{Śivapurāṇa} daily derives the fruit of
horse-sacrifice undoubtedly.

23. He who with a craving for an ordinary position in life listens to
\tl{Śivapurāṇa} even from a person other than me is freed from sin.

24. He who bows near this \tl{Śivapurāṇa} derives undoubtedly the fruit of
adoration of all the gods.

25. Please listen to the meritorious benefit that accrues to the man who copies
\tl{Śivapurāṇa} and gives the manuscript to the devotees of Śiva.

26. He will have that benefit—very difficult to attain in the world—as that of
the study of \tl{śāstras} (sacred lore) and of commenting on the Vedas.

27. He who observes fast on the \tl{caturdaśī} (fourteenth day in the lunar
fortnight) and conducts discourses and comments on the \tl{Śivapurāṇa} in the
assembly of the devotees of Śiva is the most excellent of all.

28. He shall derive the benefit of the repetition of Gāyatrī\footnote{Gāyatrī:
a most sacred verse of the \tl{Ṛgveda} which is the duty of every \tl{brāhmaṇa}
to repeat in his every day prayers. It is addressed to the Sun, Savitṛ and is
called Sāvitrī also.} syllable by syllable. He will enjoy all worldly pleasures
here and attain salvation hereafter.

29. I shall tell you the benefit derived by him who reads or listens to this
after observing fast on the \tl{caturdaśī} day by keeping awake in the night.

30-31. This is the truth, undoubtedly the truth that he will get the benefit
derived by the man who makes gifts of wealth equal in weight to himself to
brahmins with Vyāsas at their head at the complete eclipse of the sun, many
a time, in all holy centres, Kurukṣetra, \etc

32. Indra and other \tl{devas} wait eagerly for the directives of the man who
chants day and night the verses of the \tl{Śiva-purāṇa}.

33. The sacred rites performed by the man who regularly reads or listens to
the \tl{Śivapurāṇa} are effective millions of times more than usual.

34. He who reads the \tl{Rudrasaṃhitā} portion of \tl{Śiva-purāṇa} with pure and
concentrated mind becomes a purified soul within three days even though he might
have killed a brahmin.

35. He who reads the \tl{Rudrasaṃhitā} three times a day near the image of
Bhairava, refraining from useless talk, shall get all cherished desires
fulfilled.

36. If a slayer of brahmin circumambulates the trees of \tl{vaṭa} and \tl{bilva}
reciting the verses from \tl{Rudrasaṃhitā} he will become purified of the sin of
Brahmin-slaughter.

37. The \tl{Kailāsa saṃhitā} is even greater than that. It is of Vedic status
and stature. The meaning of \tl{praṇava} (the sacred syllable Om) is amplified
in it.

38. O Brahmins, Lord Śiva knows the greatness of \tl{Kailāśasaṃhitā} in its
entirety. Vyāsa knows half of it and I a moiety of the same.

39. A part of it, I shall tell you, since it is impossible to say everything.
On comprehending it people attain purity of their minds instantaneously.

40. O Brahmins, seeking for it ever and anon, I do not see a sin that cannot be
quelled by \tl{Rudrasaṃhitā}.

41. Drinking that nectar prepared by Lord Śiva after churning the ocean of
the \tl{upaniṣads} and handed over to Kumāra (Lord Kārtikeya) the devotee shall
become immortal.

42. The person intending to perform expiatory rites for the sins of Brahma-hatyā
\etc should read that \tl{saṃhitā} for a month. He shall be freed of that sin.

43. By a single recital, that \tl{saṃhitā} destroys the sin originating from
the acceptance of monetary gifts from defiled persons, partaking of defiled food
and indulging in foul talks.

44. The benefit derived by a person who reads that \tl{saṃhitā} in the grove of
\tl{bilva} trees in a temple of Śiva is beyond description in words.

45. If a person reads that \tl{saṃhitā} with devotion at the time of performing
\tl{śrāddha} and feeding the brahmins, all his \tl{pitṛs} (manes) attain
the great region of Śiva.

46. The devotee who observes fast on the \tl{caturdaśī} day and reads that
\tl{saṃhitā} under the \tl{bilva} tree is directly identified with Śiva and is
worshipped by the gods.

47. The other \tl{saṃhitās} are no doubt the bestowers of the benefit of
fulfilling all cherished desires. These two \tl{saṃhitās} are particularly
excellent as they are full of divine sports and divine knowledge.

48. Such is the \tl{Śivapurāṇa}, extolled on a par with the Vedas, created by
Lord Śiva Himself at first and commensurate with the supreme Brahman.

49-51. Originally the \tl{Śivapurāṇa} was of very enormous size consisting of
twelve sacred \tl{saṃhitās}: (1) \tl{Vidyeśvara} (2) \tl{Rudra},
(3) \tl{Vaināyaka}, (4) \tl{Aumika}, (5) \tl{Mātṛ} (6) \tl{Rudraikādaśa},
(7) \tl{Kailāsa}, (8) \tl{Śatarudraka}, (9) \tl{Sahasrakoṭirudra},
(10) \tl{Koṭirudra}, (11) \tl{Vāyavīya} and (12) \tl{Dharmasaṃjña}. O brahmins,
I shall mention the number of verses in those \tl{saṃhitās}. Please listen with
due attention.

52. The first \tl{saṃhitā} of \tl{Vidyeśvara}, consisted of ten thousand verses.
The \tl{Raudra, Vaināyaka, Aumika} and \tl{Mātṛ saṃhitās} consisted of eight
thousand verses each.

53. O brahmins, the \tl{Rudraikādasa saṃhitā} consisted of thirteen thousand
verses; the \tl{Kailāsa saṃhitā} of six thousand verses and the \tl{Śatarudra}
of three thousand verses.

54. The \tl{Koṭirudra saṃhitā} consisted of nine thousand verses;
the \tl{Sahasrakoṭi-Rudra saṃhitā} of eleven thousand verses.

55. The \tl{Vāyavīya saṃhitā} consisted of 4000 verses and the \tl{Dharma
saṃhitā} of twelve thousand verses. Thus the whole \tl{Śivapurāṇa} contained
a hundred thousand verses.

56. That has been condensed by Vyāsa to twenty-four thousand verses; that is to
about a fourth of the original \tl{purāṇa} and he retained seven \tl{saṃhitās}.

57. The Purāṇic lore at the time of the first creation as conceived by Śiva
contained a thousand million (hundred \tl{crores}) verses.

58. In the Kṛta age\footnote{\tl{Yugas}: According to tradition, historical time
is divided into four ages, \viz the Kṛta (or Satya), Tretā, Dvāpara and Kali.
This system is the peculiarity of India alone. Kṛta age ended with the
destruction of the Haihayas by Rāma Jāmadagnya; Tretā began with Sagara and
ended with Rāma Dāśarathi’s consecration at Ayodhyā and closed with the Bhārata
war; the Kali began immediately after the passing away of the great heroes of
the Bharata war, Kṛṣṇa and the Pāṇḍavas and with the changes in the political
condition of Northern India that ensued.} Dvaipāyana and others condensed it
into four hundred thousand verses which in the beginning of Dvāpara age was
separated into eighteen different \tl{purāṇas}.

59. Of these the \tl{Śivapurāṇa} contains twenty-four thousand verses with seven
\tl{saṃhitās} and the \tl{purāṇa} is on a par with the Vedas (in excellence).

60. The first \tl{saṃhitā} is called \tl{Vidyeśvara}, the second \tl{Rudra},
the third \tl{Śatarudra} and the fourth \tl{Koṭirudra}.

61. The fifth is \tl{Aumi} (of Umā), the sixth \tl{Kailāsa} and the seventh
\tl{Vāyavīya}; these are the seven \tl{saṃhitās}.

62. Thus the divine \tl{Śivapurāṇa} with its seven \tl{saṃhitās} stands on
a par with the Vedas, according salvation more than anything else.

63. He who reads this \tl{Śivapurāṇa} complete with the seven \tl{saṃhitās}
devotedly is a living liberated soul.

64. Hundreds of other sacred texts as the Vedas, \tl{smṛtis, purāṇas, itihāsas},
and \tl{āgamas} do not merit even a sixteenth of this \tl{Śivapurāṇa}.

65. \tl{Śivapurāṇa} is first expounded by Śiva and then condensed by Vyāsa,
a devotee of Śiva. It is pure and brief and as such it renders help to all
living beings. As a queller of the threefold calamities (physical, extraneous
and divine) it is unrivalled. It bestows welfare upon the good.

66-67. Undeceptive virtue is extolled herein; it is, in the main, of the nature
of Vedantic wisdom. It contains mantras, and three aims of life and the thing
knowable by wise men of unprejudiced mind. The \tl{Śivapurāṇa} is the best among
the \tl{purāṇas}, extolling the great Being that glows in Vedānta and the Vedas.
He who reads and listens to it with devotion becomes a favourite of Śiva and
attains the supreme position (here and hereafter).
