\chapter{Worshipping an earthen phallic image by chanting Vedic mantras}

Sūta said:

1. Now, the mode of worshipping an earthen phallic image according to the Vedic
rites is being explained. It yields worldly pleasures and salvation to the Vedic
worshippers.

2. The devotee shall take bath in accordance with the rules prescribed in
the sacred code. He shall duly perform his \tl{sandhyā} prayers. After
performing the \tl{Brahma yajña}, one of the five daily sacrifices, he shall
perform \tl{tarpaṇa} (a rite of offering water oblation to the manes).

3-4. After finishing the daily rites he shall apply ashes and wear \tl{rudrākṣa},
all along remembering Lord Śiva. With great devotion he shall then worship
the excellent earthen phallic image according to Vedic injunctions in order to
realise the full benefit.

5. The worship of the earthen phallic image shall be performed on the bank of
a river or a tank or on the top of a mountain or in a forest, or in a Śiva
temple. It must be in a clean place.

6. O \tl{brāhmaṇas}, he shall bring clay from a clean place and carefully make
the phallic image.

7. White clay is to be used by a \tl{brāhmaṇa}; red clay by a \tl{kṣatriya};
yellow clay by a \tl{vaiśya} and black clay by a \tl{sūdra}. Anything available
shall be used if the specified clay is not found.

8. After taking the clay he shall place it in an auspicious place for making
the image.

9. After washing the clay clean with water and kneading it slowly he shall
prepare a good earthen phallic image according to the Vedic direction.

10. Then he shall worship it with devotion for the sake of enjoying worldly
pleasures here and salvation hereafter.

11. The material of worship shall be sprinkled with water, chanting the formula
“namaḥ śivāya”\footnote{\tl{Vājasaneyisaṃhitā} (of \tl{Śuklayajurveda}) 16.41.}.
With the mantra “bhūrasi”\footnote{Ibid. 13.18.} \etc the achievement of
the sanctity of a holy centre (\tl{kṣetra siddhi}) shall be effected.

12. Water shall be sanctified with the mantra “\tl{āposmān}”\footnote{Ibid. 4.2.}
\etc The rite of “\tl{phāṭikābandha}” shall be performed with
“\tl{namaste rudra}”\footnote{Ibid. 16.1.} mantra.

13. The purity of the place of worship shall be heightened with the mantra
“\tl{śambhavāya}”\footnote{Ibid. 16.4 1.} \etc The sprinkling of water over
\tl{pañcāmṛta}\footnote{Five kinds of food \viz milk, curd, butter, honey and
sugar are called \tl{pañcāmṛṭa}.} shall be performed with the word
“\tl{namaḥ}” prefixed.

14. The excellent installation of the phallic image of Śiva shall be made
devoutly with the mantra “\tl{namaḥ nīla-grīvāya}”
\footnote{\tl{Vājasaneyisaṃhitā} (of \tl{Śuklayajurveda}) 16.28.}
(obeisance to the blue-necked).

15. The worshipper following the Vedic path shall make devoutly the offer of
a beautiful seat with the mantra “\tl{etatte rudrāya}”\footnote{Ibid. 3.61.} \etc

16. The invocation (\tl{āvāhana}) shall be performed with the mantra
“\tl{mā no mahāntam}”\footnote{Ibid. 16.15.} \etc The seating (\tl{upaveśana})
shall be performed with the mantra “\tl{yā te rudreṇa}”\footnote{Ibid. 16.2.}.

17. With the mantra “\tl{yāmiṣum}”\footnote{Ibid. 16.3.} \etc the \tl{nyāsa}
(ritualistic touching of the body in various parts) shall be performed.
The offering of fragrance shall be performed endearingly with the mantra
“\tl{adhyavocat}”\footnote{Ibid. 16.5.} \etc

18. The \tl{nyāsa} of the deity shall be performed with the mantra “\tl{asau
jīva}”\footnote{Not traceable.} \etc The rite of approaching the deity
(\tl{upasarpaṇa}) shall be performed with the mantra “\tl{asau yovasarpati}”
\footnote{\tl{Vājasaneyisaṃhitā} (of \tl{Śuklayajurveda}) 16.17.} \etc

19. The water used for washing the feet (\tl{pādya}) shall be offered with
the mantra “\tl{namostu nīlagrīvāya}”\footnote{Ibid. 16.8.} (obeisance to
the blue-necked). The water for the respectful reception (\tl{arghya}) shall be
offered with the \tl{Rudragāyatrī} mantra\footnote{\tl{Kāṭhaka-saṃhitā} 17.11.}
and the sipping water (\tl{ācamana}) with the \tl{Tryambaka}
\footnote{\tl{Vājasaneyisaṃhitā} (of \tl{Śuklayajurveda}) 3.60.} mantra.

20. The ceremonial ablution with milk shall be performed with the mantra
“\tl{payaḥ pṛthivyām}”\footnote{Ibid. 18.36.} \etc The ceremonial ablution with
curd shall be performed with the mantra “\tl{dadhi krāvṇaḥ}”
\footnote{Ibid. 23.32.} \etc

21-22. The ceremonial ablution with ghee shall be performed with the mantra
“\tl{ghṛtam ghṛtayāvā}”\footnote{\tl{Atharvaveda} 13.1.24.} \etc The ceremonial
ablution with honey and sugar candy shall be performed with three hymns
beginning with “\tl{madhuvātā}”\footnote{\tl{Vājasaneyisaṃhitā}
(of \tl{Śuklayajurveda}) 13.27.}, “\tl{madhu naktam}”\footnote{Ibid. 13.28.} and
“\tl{madhumānnaḥ}”\footnote{Ibid. 13.29.}. Thus the \tl{pañcāmṛta} ablution is
explained. Or the ablution with \tl{pañcāmṛta} can be performed with
the \tl{pādya mantra} “\tl{namostu nīlagrīvāya}”\footnote{Ibid. 16.8.}.

23. The tying of the waistband (\tl{kaṭibandhana}) shall be performed devoutly
with the mantra “\tl{mā nastoke}”\footnote{Ibid. 16.16.} \etc The piece of cloth
to be worn on the upper part of the body shall be offered with the mantra
“\tl{namo dhṛṣṇave}”\footnote{Ibid. 16.36.} \etc

24. The pious follower of Vedic rites shall make an offering of cloth
(\tl{vastrasamarpaṇa}) duly to Śiva with the four hymns beginning with
“\tl{yā te heti}”\footnote{Ibid. 16.11-14.} \etc

25. The intelligent devotee shall offer scents devoutly with the mantra
“\tl{namaḥ śvabhyaḥ}”\footnote{Ibid. 16.28.} \etc He shall offer \tl{akṣatas}
(raw rice grains) with the mantra “\tl{namastakṣabhyaḥ}”\footnote{Ibid. 16.27.} \etc

26. Flower offerings shall be made with the mantra “\tl{namaḥ pāryāya}”
\footnote{Ibid. 16.42.} \etc Bilva leaves shall be offered with the mantra
“namaḥ parṇāya”\footnote{Ibid. 16.46.} \etc

27. The incense shall be offered with the mantra “\tl{namaḥ kapardine ca}”
\footnote{Ibid. 16.29} \etc in accordance with the rules. The lamp shall be
offered in the prescribed manner with the mantra “\tl{namaḥ āśave}”
\footnote{Ibid. 16.31} \etc

28. The excellent \tl{naivedya} shall be offered with the mantra “\tl{namo
jyeṣṭhāya}”\footnote{Ibid. 16.32} \etc \tl{Ācamana} shall be offered again with
the mantra “\tl{tryambakam}”\footnote{Ibid. 3.60.} \etc

29. Fruit shall be offered with the mantra “\tl{imā rudrāya}”\footnote{Ibid.
16.48.}. Everything shall be dedicated to Śiva with the mantra “\tl{namo
vrajyāya}”\footnote{Ibid. 16.44.} \etc

30. We shall make an offering of eleven raw rice grains to the eleven Rudras
\footnote{The names of eleven Rudras are variously mentioned in the \tl{purāṇas}.
According to \tl{Matsya-purāṇa} (‘\emph{A Study}’ by V. S. Agrawal) they are:
Ajaikapād, Ahirbudhnya, Hara, Virūpākṣa, Raivata, Bahurūpa, Tryambaka, Savitā,
Jayanta, Pināki, Aparājita. In the \emph{VP} the first three are the same;
the rest are substituted by Nirṛta, Īśvara, Bhuvana, Aṅgāraka, Ardhaketu, Mṛtyu,
Sarpa, Kapālin.} with the two mantras ”\tl{mā no mahāntam}”
\footnote{\tl{Vājasaneyisaṃhitā} (of \tl{Śuklayajurveda}) 16.15.} \etc and
“mā nastoke”\footnote{Ibid. 16.16.} \etc

31. The scholarly devotee shall offer sacrificial fee (\tl{dakṣiṇā})
\footnote{Ibid. 13.4.} with the three mantras beginning with “\tl{hiraṇya-
garbha}” \etc and shall perform ablution (\tl{abhiṣeka}) with the mantra
“\tl{devasya tvā}”\footnote{Ibid. 11.28.} \etc

32. The rite of waving lights \tl{nīrājana} for Śiva shall be performed with
the mantra for the lamp (\tl{namaḥ āśave})\footnote{Ibid. 16.31.}.
\tl{Puṣpāñjali} (offering of handful of flowers) shall be performed with
devotion with the hymn “\tl{imā rudrāya}”\footnote{Ibid. 16.48-50.} \etc

33. The wise devotee shall then perform the \tl{pradakṣiṇā} (circumambulation)
with the mantra “\tl{mā no mahāntam}”\footnote{Ibid. 16.15.} and the intelligent
one shall perform \tl{sāṣṭāṅga} (eight limbs touching the ground) prostration
with the mantra “mā nastoke”\footnote{Ibid. 16.16.} \etc

34. He shall show the “\tl{śiva mudrā}” with the mantra “\tl{eṣa te}”
\footnote{Ibid. 9.35.}; the \tl{abhayamudrā} with the mantra “\tl{yato yataḥ}”
\footnote{Ibid. 36.22.} \etc and the \tl{jñāna mudrā} with the \tl{tryambaka}
\footnote{Ibid. 3.60.} mantra.

35. The \tl{mahāmudrā} shall be shown with the mantra “\tl{namaḥ senā}”
\footnote{Ibid. 16.26.} \etc He shall then show the \tl{dhenumudrā} with
the mantra “\tl{namo gobhyaḥ}” \etc

36. After showing all these five \tl{mudrās} he shall perform
the “\tl{śiva mantra japa}”\footnote{\tl{namah śivāya}.}. The devotee well
versed in the Vedas shall recite the “\tl{śatarudriya}” mantra.

37. \tl{Pañcāṅgapāṭha} shall then be performed by the Vedic scholar. Then
\tl{visarjana} (ritualistic farewell) shall be performed with the mantra
“\tl{devā gātu}”\footnote{\tl{Taittirīya-brāhmaṇa} 3.7.4.1.} \etc

38. Thus the Vedic rite of the worship of Śiva has been explained in detail. Now
listen to the excellent Vedic rite in brief.

39. The clay shall be brought with the mantra “\tl{sadyo jātam}”
\footnote{\tl{Vājasaneyisaṃhitā} (of \tl{Śuklayajurveda}) 29.36}. The sprinkling
of water shall be performed with the mantra “\tl{vāmadevāya}”\footnote{TA. 10.44.1.}.

40. The phallic image shall be prepared with the \tl{aghora mantra}\footnote{
\tl{Vājasaneyisaṃhitā} (of \tl{Śuklayajurveda}) 16.2.}. The \tl{āhvāna}
(invocation) shall be performed with the mantra “\tl{tatpuruṣāya}”
\footnote{\tl{Kāṭhaka-saṃhitā} 17.11; \tl{Maitrāyaṇī-saṃhitā} 2-9.1: 119.7.}.

41. The phallic image of Hara shall be fixed to the pedestal with the \tl{īśāna
mantra}\footnote{\tl{Vājasaneyisaṃhitā} (of \tl{Śuklayajurveda}) 27.35.}.
The intelligent devotee shall perform all other rites in brief.

42. With the five-syllabled mantra or any other mantra taught by the preceptor
the intelligent devotee shall perform, as prescribed by the rule, the adoration
with due observance of the sixteen \tl{upacāras} (and the following prayer):

\begin{shloka}\itshape
  bhavāya bhavanāśāya mahādevāya dhīmahi \\
  ugrāya ugranāśāya śarvāya śaśimauline
\end{shloka}

43. “We meditate upon Bhava, the destroyer of worldly existence, on the great
lord, on Ugra, the annihilator of terrible sins, on Śarva the moon-crested”.

44. The intelligent devotee shall perform the worship of Śiva with this mantra
or with the Vedic mantra with great devotion and abandoning errors. Śiva accords
benefits when with devotion he is propitiated.

45. Notwithstanding the Vedic mode of worship as stated above, O \tl{brāhmaṇas},
we now proceed to explain the common procedure of Śiva’s worship.

46. This mode of worship of Śiva’s earthen phallic image is the muttering of
the names of Śiva. O excellent sages, it yields all cherished desires. Please
listen to me. I shall explain it.

47-48. The eight names of Śiva \viz Hara, Maheśvara, Śambhu, Śūlapāṇi,
Pinākadhṛk, Śiva, Paśupati and Mahādeva shall be used respectively for the rites
of bringing the clay, kneading, installation, invocation, ceremonial ablution,
worship, craving the forbearance and ritualistic farewell.

49. Each of the names shall be prefixed with \tl{oṃkāra}. The name shall be used
in the dative case and \tl{namaḥ} shall be added to them. The rites shall be
performed respectively with great devotion and joy.

50. The \tl{nyāsa} rite shall be duly performed and the \tl{aṅganyāsa} of
the two hands shall also be performed. The devotee shall perform meditation with
the six-syllabled mantra—\tl{oṃ namaśśivāya}.

51. The devotee shall meditate on Śiva seated in the middle of his seat on
the pedestal in Kailāsa, worshipped by Sananda\footnote{Sananda is one of the 4,
7 or 10 mind-born sons of Brahmā.} and others. Śiva is a forest fire, as it were,
for the dry wood in the form of the distress of the devotees. He is immeasurable.
He is the Ornament of the universe being closely embraced by his consort, Umā.

52. He shall meditate on Śiva always in the following way:—He is like a silver
mountain. He wears the beautiful moon, on his forehead. His limbs are
resplendent with ornaments of gems. He holds the axe, the deer, the \tl{mudrā}
of boon and the \tl{mudrā} of freedom from fear in His four hands. He is joyful.
He is seated in the lotus pose. The assembled \tl{devas} stand around Him and
offer prayers. He wears the hide of the tiger. He is the primordial Being,
the seed of the universe. He dispels all fears. He is the three-eyed
\footnote{Three-eyed Śiva, so called because a third eye burst from his forehead
with a great flame when his wife playfully placed her hands over his eyes after
he had been engaged in austerities in the Himalayas. This eye has been very
destructive. It reduced Kāma, the God of Love, to ashes. \emph{Dowson, H.M.}
\emph{See} under Trilocana.} lord with five faces.

53. After the meditation and worship of the excellent earthly image he shall
duly perform the \tl{japa} of the five-syllabled mantra taught by the preceptor.

54. O foremost among \tl{brāhmaṇas}, the intelligent devotee shall adore
the lord of \tl{devas} with different sorts of hymns and recite
the \tl{śatarudriya mantra}.

55. He shall take raw rice grains and flowers by means of palms joined together
in the form of a bowl and pray to Śiva by means of the following mantras:

\begin{shloka}\itshape
  tāvakastvadguṇaprāṇastvaccittohaṃ sadā mṛḍa \\
  kṛpānidha iti jñātvā bhūtanātha prasīda me \\

  ajñānādyadi vā jñānājjapa pūjādikaṃ mayā \\
  kṛtaṃ tadastu saphalaṃ kṛpayā tava śaṃkara \\

  ahaṃ pāpī mahānadya pāvanaśca bhavānmahān \\
  iti vijñāya gaurīśa yadicchasi tathā kuru \\

  vedaiḥ purāṇaiḥ siddhāntairṛṣibhirvividhairapi \\
  na jñātosi mahādeva kutohaṃ tvaṃ mahāśiva \\

  yathā tathā tvadīyosmi sarvabhāvairmaheśvara \\
  rakṣaṇīyastvayāhaṃ vai prasīda parameśvara
\end{shloka}

56-60. “O Śiva, the merciful, I am Thine. Thy attributes are my vital breath.
My mind is ever fixed in Thee. Knowing this, O lord of goblins, be pleased with
me. Consciously or unconsciously, whatever I have performed by way of \tl{japa}
or adoration may that O Śiva, with Thy favour, be fruitful. I am the greatest
sinner and Thou art the greatest sanctifier. O Lord of Gaurī, knowing this, do
thou whatever thou dost wish. O great lord, Thou art not known by Vedas,
\tl{Purāṇas}, systems of philosophy or the different sages. O Sadāśiva, how can
I know Thee? In whatever manner, I belong to Thee, O Śiva, by all my
thoughtforms. I am to be saved by Thee. Be pleased with me O Siva”.

61. After repeating the hymn, the devotee shall place the flowers and the rice-
grains over the phallic image of Śiva. O sages, he shall then prostrate before
Śiva with devotion (his eight limbs touching the ground).

62. The intelligent devotee shall perform circumambulation in the manner
prescribed. He shall pray to the lord of \tl{devas} with great faith.

63. Then he shall make a full-throated sound\footnote{It is said that Dakṣa’s
sacrifice being destroyed by the Gaṇas of Śiva assumed the form of a goat while
Dakṣa became a deer and escaped. A devotee who imitates the sound of
a terror-struck goat in the presence of the phallic image of Śiva pleases
the God.}. He shall humbly bow down his head. He shall then make a formal
request and perform the rite of ritualistic farewell.

64. O foremost among sages, thus have I explained to you the procedure for
the worship of the phallic image that accords worldly pleasures, salvation and
increases devotion to Śiva.

65-66. Whoever reads or listens to this chapter with a pure mind shall be
purified of all sins and shall attain all wishes. This excellent narration
bestows longevity, health, fame, heaven and happiness by way of sons and
grandsons.
