\chapter{Description of good conduct (\tl{sadācāra})}

The sages said:

1. Kindly tell us the mode of good conduct (\tl{sadācāra}) whereby the sensible
man quickly attains higher worlds. Please tell us about virtue and evil that
cause attainment of heaven or hell.

Sūta said:

2. A \tl{brāhmaṇa} endowed with strict adherence to good conduct is perfectly
wise. A \tl{brāhmaṇa} learned in Vedas and of good conduct is called
a \tl{vipra}. A \tl{brāhmaṇa} endowed with only one of these two is a mere
\tl{dvija}.

3. A \tl{brāhmaṇa} following some of the prescribed rules of conduct and with
a smattering of the Vedas is a \tl{kṣatriya brāhmaṇa}, at best a royal servant.
Very careless in following the rules of conduct the \tl{brāhmaṇa} is really
a \tl{vaiśya brāhmaṇa}. One engaged in agriculture and trading activities is
also likewise.

4. A \tl{brāhmaṇa} ploughing the field himself is a \tl{śūdra brāhmaṇa}. One of
envious and spiteful temperament is a degraded \tl{dvija}.

5. A \tl{kṣatriya} who rules over a kingdom is a “King”; others are mere
\tl{kṣatriyas}. A merchant dealing in grains \etc is a \tl{vaiśya} and others of
his caste are mere \tl{vaṇiks}.

6. A person rendering service to \tl{brāhmaṇas}, \tl{kṣatriyas} and \tl{vaiśyas}
is called a \tl{śūdra}. A working agriculturist is a \tl{vṛṣala} and the others
are \tl{dasyus}.

7. It is the duty of everyone of the four castes to get up early in the morning
and sit facing the east and meditate on gods. He shall then think about
the various acts of virtue, of matters regarding monetary dealings, the problems
connected with them, the sources of income and the items of expenditure.

8. The direction in which one casts one’s first glance on waking up indicates
the good or bad that is likely to attend one on that day—the eight effects in
order are—longevity, hatred, death, sin, fortune, sickness, nourishment and
strength.

9. The last \tl{yāma} (3 hours) of the night is called Uṣā and the latter half
of it is \tl{sandhi} (period of conjunction). A \tl{brāhmaṇa} shall get up at
that hour and answer the calls of nature.

10. It must be in a place far off from the house. It must be a covered place. He
shall sit facing the north. If it is not possible due to any obstacle he can sit
facing other directions.

11. He must never sit in front of water, fire, a \tl{brāhmaṇa} or the idol of
any god. He must screen the penis with the left hand and the mouth with
the right.

12. After evacuating the bowels, the faeces should not be looked at. Water drawn
out in a vessel should be used for cleaning (\ie no one should sit inside
the tank or river-water for cleaning purpose).

13. Any way no one shall enter the holy tanks and rivers dedicated to deities,
manes \etc and frequented by the sages. The rectum must be cleaned with mud
seven, five or three times.

14. The penis must be cleaned with mud as large as a cucumber fruit and
the quantity of mud for the purification of the rectum shall be \tl{prasṛti}
(half a handful). After the purification of the excretory organs, hands and feet
must be washed and gargling shall be done for eight times.

15. For gargling, the water can be taken in any vessel or a wooden cup; but
water shall be spit outside (not in the river or tank). Washing of the teeth
with any leaf or twig must be without using the index finger and outside
the water.

16. After making obeisance to the gods of water, the twice-born shall perform
the ablution with mantras. Sick or weak persons shall take bath upto the neck
or hips.

17. Sprinkling water upto the knees he shall perform the \tl{mantrasnāna}. He
shall propitiate deities \etc sensibly with the water from the holy tank or
river.

18. A washed dry cloth should be taken and worn in the form of \tl{paūcakaccha}
(wearing of the lower garment in a special way). In all sacred rites the upper
cloth should also be used.

19-20. While taking bath in the holy river or tank, the cloth worn shall not be
rinsed or beaten. The sensible man shall take it to a separate tank or well or
to the house itself and beat it on a rock or on a plank to the gratification of
the manes, O \tl{brāhmaṇas}.

21-23. The \tl{tripuṇḍraka}\footnote{Three lines horizontally drawn over
the forehead with the ash slightly pasted with water.} shall be drawn on
the forehead with the \tl{jābālaka mantra}. If anyone enters water otherwise,
he will surely go to hell. According to scholarly authorities
the \tl{mantrasnāna} is as follows: Repeating the mantra “\tl{āpo hi ṣṭhā}”
\footnote{\tl{Vājasaneyisaṃhitā} (of \tl{Śuklayajurveda}) 11.50.} \etc water
shall be sprinkled over the head for suppressing sins. Repeating the mantra
“\tl{yasya kṣayāya}”\footnote{\tl{Vājasaneyisaṃhitā} (of \tl{Śuklayajurveda}) 11.52.}
\etc water shall be sprinkled over the joints in the legs. The order is as
follows: feet, head, chest; head, chest, feet and chest, feet, head for
sprinkling with water thrice.

24. It is enough if one performs \tl{mantrasnāna} when one is slightly
indisposed, or when there is danger from the king or when there is civil
commotion, or when there is no other way or when one is about to undertake
a journey.

25. He shall drink by way of \tl{ācamana} reciting the mantras from
\tl{Sūryānuvāka} in the morning or from \tl{Agni-Anuvāka} in the evening and
perform the ceremonial sprinkling in the middle.

26. O \tl{brāhmaṇas}, at the end of the \tl{japa} of \tl{Gāyatrī-mantra}
\footnote{Three-footed sacred mantra of the \tl{Ṛgveda} well-known after its
metre \tl{Gāyatrī}. It is addressed to the sun (\tl{savitar}) and is therefore
called \tl{Savitrī}. It runs

\begin{shloka}\itshape
tatsaviturvareṇyaṃ bhargo devasya dhīmahi dhiyo yo naḥ pracodayat
\end{shloka}

— “We meditate on that excellent light of the sun. May he illuminate our minds.”}
\tl{arghya} shall be offered thrice to the sun towards east and once also
thereafter.

27. The offering of \tl{ārghya} in the morning is by lifting both the hands high
up; that in the midday by letting off the water through the fingers and that in
the evening by letting the water over the ground facing the west.

28. In the midday the sun is to be viewed through the fingers reciting
the mantra prescribed for that. The circumambulation of oneself is performed (in
the prescribed manner) and the pure \tl{ācamana} (without mantras) is performed.

29-30. \tl{Sandhyā} prayer performed before the prescribed time is ineffective.
Hence \tl{sandhyā} shall be performed at the prescribed time. The expiatory rite
for the omission of \tl{sandhyā} prayer for a day is the repetition of
\tl{Gāyatrī} a hundred times more than the usual number of times for ten days.
If the omission is for ten days or more, \tl{Gāyatrī} must be repeated for
a hundred thousand times as atonement.

31-32. If one omits \tl{sandhyā} for a month one has to be re-invested with
the sacred thread\footnote{It is one of the purificatory rites prescribed in
the \tl{Dharma-sūtras} and explained in the \tl{Gṛhyasūtras} in which the boy
is invested with the sacred thread and thus endowed with second or spiritual
birth and qualified to learn the Veda by heart. A \tl{brāhmaṇa} is initiated in
the eighth year, a \tl{kṣatriya} in the eleventh, a \tl{vaiśya} in the twelfth;
but the term could be delayed. \tl{Cf. Maitrāyaṇī-saṃhitā} 2.36-38.}. For
the sake of prosperity deities shall be propitiated such as Īśa, Gaurī,
Guha\footnote{Guha, literally the mysterious one, is Kārttikeya, so called
because of his mysterious birth. According to a legend he was the son of Śiva
produced without the intervention of a woman. Śiva cost his seed into fire which
was afterwards received by the Ganges: Kārttikeya was the result. He is
therefore called as the son of Agni and Gaṅgā. When born he was fostered by
the six Kṛttikas and these offering their six breasts to the child he became
six-headed.}, Viṣṇu, Brahmā, Candra (the moon) and Yama. Thereafter the entire
rite shall be dedicated to the supreme Brahman and pure \tl{ācamana} shall be
performed.

33-34. Towards the right of the holy water, in a splendid prayer hall, temple or
a common \tl{maṭha}, or in a stipulated place in one’s own house, one shall sit
firmly with the mind in concentration and perform the \tl{Gāyatrī japa} after
due obeisance to all gods. He shall not omit the practice of the \tl{praṇava
mantra}.

35-37. While practising the \tl{Praṇava} he shall realise fully the identity of
\tl{Jīva} (the individual soul) with the supreme Brahman. The full implication
of the \tl{Gāyatrī} must be borne in the mind when the \tl{japa} is performed.
“We pray to Brahmā, the creator of the three worlds, to Acyuta the sustainer and
Rudra the Annihilator\footnote{\tl{Devi Bhāgavata-purāṇa} 1.3.3-4:

\begin{shloka}\itshape
brahmā viṣṇuśca rudraśca trayo devāḥ sanātanāḥ\\
nātaḥ parataraṃ kiñcid brahmāṇḍe'sminmahāmate\\
brahmā sṛjati lokānvai viṣṇuḥ pātyakhilaṃ jagat\\
rudraḥ saṃharate kāletraya pate'tra kāraṇam
\end{shloka}

also \tl{Śiva-purāṇa} (\tl{Vidyeśvara-saṃhitā}) 10:

\begin{shloka}\itshape
tridhā vibhajya cātmānaṃ trailokyaṃ sampravartate
sṛjate grasate caiva vīkṣate ca tribhissvayam
\end{shloka}

The idea is often repeated in the Purāṇas.}. We meditate on the Self-luminary
that prompts us in the activities of virtue and wisdom bestowing enjoyment and
salvation, the Self-luminary that is the driving force behind the sense-organs,
mind, intellect and acts of volition.” The devotee who dwells thus on
the meaning constantly attains the Brahman.

38. Or if incompetent to dwell on the meaning let him at least continue
the recitation of the mere mantra to keep his \tl{brāhmaṇahood} in tact.
An excellent \tl{brāhmaṇa} must repeat the mantra a thousand times in
the morning every day.

39. Others shall repeat as many times as they can. In the midday \tl{Gāyatrī}
shall be repeated a hundred times; in the evening at least twenty times along
with \tl{śikhāṣṭaka} [A set of eight as the tuft \ie eight times more than
stipulated.]

40-41. He shall meditate on Vidyeśa, Brahmā, Viṣṇu, Īśa, Jīvātman and
Parameśvara stationed in the twelve esoteric centre of the body from Mūlādhāra
(basic support) to the Brahmarandhra (the mystical aperture at the crown of
the head), as identical with Brahman with the conception of \tl{soham} (I am He)
and continue the \tl{japa}. He shall then meditate on them as stationed outside
the body as well.

42-43. From \tl{mahat tattva} (the cosmic principle) there are a thousand
extraneous bodies each of which is to be passed by each mantra slowly and
the Jīva must be unified with the Supreme. This is the principle on which
the \tl{japa} is based. This \tl{japa} for the sake of the extraneous bodies is
for two thousand times with the \tl{śikhāṣṭaka}.

44. This is the tradition regarding the \tl{japas}. Repetition for a thousand
times accords Brahmahood and that for a hundred times accords the region
of Indra.

45. Repetition for less number of times may guard the soul to a certain extent
and bring about rebirth in the family of a \tl{brāhmaṇa}. After the worship of
the sun, the \tl{brāhmaṇa} shall practise thus every day.

46. A \tl{brāhmaṇa} who has completed one million two hundred thousand
repetitions becomes a full-fledged \tl{brāhmaṇa}. A \tl{brāhmaṇa} who has not
completed at least a hundred thousand repetitions of \tl{Gāyatrī} is not
authorised in Vedic rites.

47. Till he completes his seventieth year he shall follow these rules.
Afterwards he can take to renunciation. After renunciation he shall repeat
the \tl{Praṇava} twelve thousand times in the morning every day.

48. Omissions and deficiencies of one day must be made good the next day. If
the omission is continued for a month, the atonement is repetition for one
hundred and fifty thousand times.

49. If the omission extends beyond this, he shall take the order of \tl{sanyāsa}
afresh. Then only can the defect be fully effaced. Otherwise he is sure to fall
into Raurava, the terrible hell.

50. Only the person who has a cherished desire shall endeavour for virtue and
wealth and not others. A \tl{brāhmaṇa} shall seek salvation and practise
the ways of realising Brahman for ever.

51. From virtue wealth is derived and from wealth enjoyment. \tl{Vairāgya} (non-
attachment) is the outcome of enjoyment. That is to say, when one fully enjoys
the pleasures by means of wealth acquired by virtuous means one comes to
the stage of \tl{vairāgya} (detached state).

52-53. If the enjoyment is through the wealth acquired by other means,
the result is the increase of passion alone. \tl{Dharma} is twofold: one through
the sacrificial offering and the other through the body by performing ceremonial
ablutions in a sacred river \etc One can earn wealth through virtue and divine
form through penance.

54. A person freed from desire gains purity and by purity he acquires knowledge.
There is no doubt about it. In the ages of Kṛtā, Tretā and Dvāpara penance was
recommended for attaining \tl{dharma}; but in the age of Kali it is
the sacrificial offering that secures \tl{dharma} for us.

55. In the Kṛta age knowledge was acquired through meditation; in the Tretā
through penance; in the Dvāpara through sacrifice and now in the Kali age it is
through the worship of idols.

56. The fruit is in accordance with the nature of merit and sin. Deficiency,
increase, decrease \etc are due to the difference in the articles employed and
the part of the body and items of rites.

57. Evil is of violent character and virtue is of pleasant nature. A person
becomes miserable due to evil and secures happiness on account of virtue.

58. It must be known that bad conduct leads to misery and good conduct to
happiness. Hence it is the duty of everyone to acquire virtue for the sake of
worldly enjoyment as well as salvation.

59. If any one regularly offers sufficient material means to a \tl{brāhmaṇa}
with four members in his family, for a hundred years he will remain in
Brahmaloka.

60. The rite of \tl{cāndrāyaṇa} performed a thousand times yields Brahmaloka. It
is the duty of a \tl{kṣatriya} to establish and sustain a thousand families.

61-63. It yields Indraloka to him. If he maintains ten thousand families he
attains Brahmaloka. According to scholars in the Vedas, a man attains the region
of that deity in meditation of whom he makes charitable gifts. A man devoid of
wealth shall endeavour to accumulate penance and austerities. Everlasting
happiness is achieved by pilgrimages to holy centres and penances. Now I shall
expound the mode of acquiring wealth through pure and lawful means.

64. A \tl{brāhmaṇa} shall earn wealth without cringing or exerting himself too
much. He can accept monetary gifts and fees for presiding over sacrifices duly
performed.

65. A \tl{kṣatriya} shall earn wealth by valorous exploits and a \tl{vaiśya} by
means of agriculture and cattle-breading. The charitable gifts of wealth
acquired by lawful means alone, are attended with good results.

66. Salvation is achieved by the acquisition of perfect knowledge by every one
with the blessings of the preceptor. Salvation is realisation of one’s own real
form and the perfect bliss.

67. O \tl{brāhmaṇas}, men realise all these things only if they cultivate
the association of good people. A householder shall make charitable gifts of
everything like money, grain \etc

68. A person who desires permanent welfare for himself shall give to
\tl{brāhmaṇas} fruit, grain or other articles especially when the need for
the same arises.

69. Water shall always be given to the thirsty. Food shall be given to
the hungry and the sick. Gift of food is of four types—field, unhusked grain (or
seed), uncooked food and cooked food.

70. A giver of food receives half the merit of the receiver which he accumulates
till the time that food is digested or as long as the glory of lord Śiva reaches
his ears.

71. The receiver of a gift must expiate for his sin by means of austerities or
by making gifts to others. Otherwise he will fall into the Raurava hell.

72-73. Everyone shall set apart a third of his wealth for \tl{dharma}, another
third for \tl{vṛddhi} (flourishing) and the rest for his \tl{bhoga} (enjoyment).
With the part intended for \tl{dharma} he shall perform the three rites of
virtue \viz \tl{nitya} (daily prayers \etc), \tl{naimittika} (casual acts of
piety) and \tl{kāmya} (specific rites for the fulfilment of desires). By means
of the second part he shall increase his wealth. By utilising the third part he
shall enjoy with restraint in pure and wholesome ways.

74. One tenth of the wealth acquired by agricultural operations must first be
given in charity (before making the three-fold divisions) in order to wipe off
the sin. He can utilise the rest as mentioned before. Otherwise he shall fall
into Raurava.

75. Or he is sure to be evil-minded hastening towards his own certain ruin.
Sensible persons acquiring much wealth by way of usury or trading activities
must likewise give away a sixth of that wealth in charity (before making
the threefold divisions).

76. Excellent \tl{brāhmaṇas}, accepting monetary gifts from decent people, shall
give away a fourth of that wealth in charity. They shall likewise give away half
in charity in case of an unexpected windfall.

77. If a \tl{brāhmaṇa} accepts a monetary gift from an indecent fellow he shall
give away the entire amount in charity. A defiled gift shall be thrown into
the sea. It is more creditable if one invites persons and makes gifts to them.
One’s own enjoyment gains by it.

78. A man must give others what they beg of him according to his ability. If
a thing requested for is not given he will be indebted to that extent even in
his next birth.

79. A sensible person shall not proclaim others’ faults. O \tl{brāhmaṇas},
whatever is seen or heard should not be spitefully repeated.

80. An intelligent man shall not speak words wounding the hearts of others. For
achieving prosperity he shall perform sacred rites in the fire at dawn and
at dusk.

81-82. Persons unable to perform the same, both the times, shall do so once,
worshipping the sun and the fire duly. Raw rice, other food grains, ghee,
fruits, bulbous roots, cooked food soaked in ghee for sacrificial rites—all
these things shall be duly used as prescribed in the sacred texts.
\tl{Sthālīpāka} (offerings of cooked food in the vessel itself) shall be
performed at the stipulated time in the manner laid down. If there is no
\tl{havya} (cooked rice offering) the main sacrifice alone shall be performed.

83. Thus the daily rites have been narrated. These shall be performed always; or
repeated muttering of mantra alone or the worship of the sun shall be performed.

84-85. Those who seek welfare of the soul shall do like this. A person who seeks
wealth also shall do likewise. All persons devoted to \tl{brahmayajña}, worship
of gods, worship of fire, reverence to preceptors and gratification of
\tl{brāhmaṇas} deserve to attain heaven.
