\chapter{Number of phallic images of Śiva used in worship}

The sages said:

1-2. O Sūta, O Sūta the fortunate, disciple of Vyāsa, obeisance be to Thee. Thou
hast clearly explained the procedure of the worship of the earthen phallic
images. Now kindly explain the number of phallic images as based on the wishes
one may have. Thou art favourably disposed to the distressed and the miserable.

Sūta said:

3. O sages, you listen to the rules of procedure in the worship of earthen
phallic image, by following which a man reaps full satisfaction.

4. If anyone worships another deity without making the earthen phallic image,
his worship shall be fruitless. His restraint and charitable gifts go in vain.

5. The number of earthen phallic images in regard to different desires is being
stipulated which will, O foremost among sages, certainly yield the benefit.

6. The first invocation, installation and worship are all separate. Only
the shape of the phallic image is the same. Everything else is different.

7. A person who seeks learning shall with pleasure make a thousand earthen
phallic images and offer worship. Certainly he will get that benefit.

8. A person desirous of wealth shall make five hundred earthen phallic images;
wishing for a son—a thousand five hundred; wishing for garments—five hundred.

9. A person desirous of salvation—a crore; desirous of lands—a thousand; craving
for mercy—three thousand; desirous of a holy centre—two thousand.

10. A person desirous of friends—three thousand; desirous of the power of
controlling—eight hundred; desirous of bringing about the death of a person—seven
hundred; desirous of enchanting—eight hundred.

11. A person desirous of sweeping off his foes—a thousand; desirous of
numbifying—a thousand; desirous of kindling hatred—five hundred.

12. A person desirous of freeing himself from fetters—a thousand five hundred.
If there is fear from a great king—five hundred.

13. If there is danger from thieves, robbers \etc—two hundred; if there is
the evil influence of Ḍākinī\footnote{A female imp or fiend attendant upon Kāli
and feeding on human flesh. The Ḍākinīs are also called Asrapās, ‘blood-drinkers’
.} and other foul spirits—five hundred.

14. In poverty—five thousand. If ten thousand such are made, all wishes will be
fulfilled. O great sages, I shall now mention the daily procedure. Please listen.

15. One such is said to remove sins. Two confer wealth. Three are mentioned as
the cause for the fulfilment of all desires.

16. Above this, more and more benefits accrue until the stipulated number is
reached. I shall now mention another opinion coming from a different sage.

17. An intelligent person can certainly remain fearless after making such ten
thousand images. It removes the fear from great kings.

18. A sensible man shall cause ten thousand such to be made for freedom from
imprisonment. When there is the fear of the evil influence of Ḍākinī and other
evil spirits he shall cause seven thousand such to be made.

19. A person having no sons shall cause fifty-five thousand such to be made. One
shall get daughters by causing ten thousand such to be made.

20. A devotee shall achieve the prosperity and splendour of Viṣṇu and others by
making ten thousand images. He shall derive unrivalled glory and wealth by
making one million images.

21. Surely if a man makes a crore he shall become Śiva Himself.

22. The worship of earthen phallic images accords the the benefit of a crore
sacrifices. It gives all worldly pleasures and salvation to those who desire
them.

23. He who spends his time in vain without worship of such images will incur
great loss. He is no better than a wicked, evil-souled man.

24. If the worship of such images is weighed against all the charitable gifts,
sacred rites, holy centres, restraints and sacrifices, both will be found equal.

25. In the age of Kali the worship of the phallic image is excellent as is
evident from what we see in the world. There is nothing else. This is
the conclusion of all sacred texts and religious cults.

26. The phallic image yields worldly pleasures and salvation. It wards off
different sorts of mishaps. By worshipping it, man attains identity with Śiva.

27. Since the phallic image is enjoined to be worshipped even by the sages, it
shall be worshipped by everyone in the manner stipulated.

28. Based on sizes the images are of three types—Excellent (\tl{uttama}), normal
(\tl{madhyama}) and inferior (\tl{nīca}). O foremost of sages, I shall explain
them, please listen.

29. A phallic image, four \tl{angulas} (inches) in height, with a splendid
pedestal is mentioned as the most excellent by sages who are well-versed in
sacred lore.

30. Half of that is middling. Half of this latter is inferior. Thus I have
mentioned three types of phallic images.

31. He who worships many such images every day with great devotion and faith can
achieve the fulfilment of any desire conceived in his heart.

32. In the four Vedas, nothing else is mentioned so holy as the worship of
the phallic image. This is the conclusion arrived at in all sacred lores.

33. All other rites can entirely be abandoned. A really learned man shall
worship only the phallic image with great devotion.

34. If the phallic image is worshipped, it means that the entire universe
consisting of the mobile and the immobile has been worshipped. There is no other
means to save persons submerged in the ocean of worldly existence.

35. Men of the world are blind due to ignorance. Their minds are sullied by
worldly desires. Except for the worship of the phallic image there is no other
raft to save them from destruction.

36-38. Hari, Brahmā and other \tl{devas}, sages, \tl{yakṣas}, \tl{rākṣasas},
\tl{gandharvas}, \tl{cāraṇas}, \tl{siddhas}, \tl{daityas}, \tl{dānavas}, Śeṣa
and other serpents, Garuḍa and other birds, all the Manus, Prajāpati, Kinnaras,
men \etc have worshipped the wealth-yielding phallic image with great devotion
and have achieved their desires surging in their heart of hearts.

39. \tl{Brāhmaṇas, kṣatriyas, vaiśyas, śūdras}, persons born of inter-caste
marriages and others shall worship the phallic icon with great devotion with
the respective mantras.

40. O \tl{brāhmaṇa} sages, why shall I tell much? Even women and others are
authorized in the worship of the phallic image.

41. The twice-born can very well worship according to the Vedic rites but not so
the others who are not authorized.

42. Lord Śiva Himself has enjoined that the twice-born shall perform the worship
according to the Vedic rites and not by any other means.

43. But those \tl{dvijas} who have been cursed by Dadhīci, Gautama and others do
not follow the Vedic rites faithfully.

44. The man who rejects the Vedic rites and follows those laid down in Smṛtis or
any other rite will not derive the conceived fruit.

45. The true devotee after performing worship in the prescribed manner shall
worship the eight cosmic bodies (of Śiva) consisting of the three worlds.

46. The Earth, the waters, the fire, the wind, the ether, the sun, the moon and
the sacrificer—these are the eight cosmic bodies.

47. Śarva, Bhava, Rudra, Ugra, Bhīma, Īśvara, Mahādeva and Paśupati are
the manifestations of Śiva who shall be worshipped with these cosmic bodies
respectively.

48. Then he shall worship retinue of Śiva with great devotion with sandal paste,
raw rice and holy leaves in the quarters beginning with North-East.

49. They are Īśāna, Nandī, Caṇḍa, Mahākāla, Bhṛṅgin, Vṛṣa, Skanda, Kapardīśa,
Soma and Śukra.

50. Vīrabhadra in front and Kīrtimukha at the back. Then he shall worship eleven
Rudras.

51-52. Then he shall repeat the five-syllabled mantra, \tl{Śatarudriya}, many
\tl{śaiva} hymns and read \tl{pañcāṅga} and perform circumambulation. After
obeisance he shall bid farewell to the phallic image. Thus have I mentioned
the worship of Śiva with due devotion.

53-54. Divine rites shall always be performed facing the north in the night.
Similarly Śiva’s worship shall always be performed facing the north, not
the east. \tl{Śaktisaṃhitā} shall not be recited facing the north or the west
since it is the back.

55-56. Śiva shall not be worshipped without \tl{tripuṇḍra}, \tl{rudrākṣa} and
\tl{bilvapatra}. O best of sages! when the worship is on, if the ash is not
available, \tl{tripuṇḍra}, (three lines on the forehead) shall be drawn with
the white clay.
