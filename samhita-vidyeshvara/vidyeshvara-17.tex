\chapter{The glorification of the syllable Om and the five-syllabled mantra}

The sages said:

1. O lord, tell us the greatness of the syllable \tl{oṃ} and that of the six
\tl{liṅgas}, O great sage. Also please tell us the worship of the devotees of
Śiva in order.

Sūta said:

2. All of you, sages, have now requested for a good thing. Only Śiva can explain
this properly. No one else.

3. Still I shall explain the same with Śiva’s grace. May Śiva increasingly guard
us, you and every one else.

4. The syllable \tl{oṃ} means an excellent boat to cross the ocean of worldly
existence.

5. Or \tl{Praṇava} may mean: “There is no world for you” or it may mean “That
which leads to salvation”.

6-8. Or it may mean “That which leads to new knowledge.” After annihilating all
actions it gives the persons who repeat the mantra or worship, a fresh knowledge
of the pure soul. This \tl{Praṇava} is two-fold (1) the subtle (2) the gross.

9. The subtle one is of a single syllable where the constituent five syllables
are not differentiated clearly. The gross one is of five syllables where all
the constituent syllables are manifest.

10. The subtle one is for the liberated living soul (\tl{jīvanmukta}). The need
for the contemplation of the meaning through the mantra is only upto
the destruction of the physical body.

11. When the body is destroyed he completely merges in Śiva undoubtedly.
The mere repeater of the mantra attains the yogic communion with Śiva certainly.

12. A person who repeats the mantra thirty-six \tl{crores} of times certainly
attains the yogic communion. The subtle \tl{Praṇava} is again two-fold—the short,
and the long.

13-15. The long one is present in the heart of the Yogins alone—separately in
the form of “A” syllable, “U” syllable, “M” syllable, Bindu and Nāda. It is
endowed with all the digits of the time sound. Śiva, Śakti and their union are
indicated by “M” syllable ramified into three and this is called the short
subtle \tl{Praṇava}. The short \tl{Praṇava} shall be recited and repeated by
those who desire their all sins annihilated.

16-18. The five elements ether, air, fire, water and earth and their five subtle
causes sound, touch, form, taste, and smell together activised in relation to
achievement of desires are called \tl{pravṛttas}. The short subtle \tl{Praṇava}
is for those who desire the continuation of mundane existence and the long one
is for those who are averse to the same\footnote{The words \tl{pravṛtta} and
\tl{nivṛtta} designate respectively the persons who desire continuation of
mundane existence and those who are averse to the same.}. The \tl{Praṇava} is to
be used in the beginning of the \tl{vyāhṛtis}\footnote{\tl{Vyāhṛtīs} are
the mystical utterances, seven in number, \viz

\begin{shloka}\itshape
bhūḥ, bhuvaḥ, svaḥ, mahaḥ, janaḥ, tapaḥ, satyam
\end{shloka}

Each of the \tl{vyāhṛtis} are preceded by the \tl{Oṃ}.}, mantras, in
the beginning of the Vedas, and during the prayer at dawn and at dusk along with
Bindu and Nāda. If the devotee repeats it nine \tl{crores} of times he becomes
pure.

19. A further repetition for nine \tl{crores} of times enables him to win over
the Earth element. A further repetition for nine \tl{crores} of times enables
him to win over the water element.

20. Similarly for each repetition of nine \tl{crores} of times he is able to win
over the elements of fire, wind and the ether.

21. The attributes of “smell” \etc are to be similarly won over by successive
repetitions of nine \tl{crores} of times. The egotism is to be won over by
another repetition of nine \tl{crores} of times.

22. By repeating it daily for a thousand times the devotee becomes perpetually
pure. O \tl{brāhmaṇas}, thereafter the repetition of the mantra is conducive to
the achievement of desires.

23. A devotee who thus completes one hundered and eight \tl{crores} of
\tl{japas} of \tl{Praṇava} (Oṃ) and is thus fully enlightened shall master
\tl{śuddhayoga}.

24-25. A person who has thus mastered \tl{śuddhayoga} becomes certainly
a liberated living soul. A \tl{mahāyogin} who performs \tl{japas} and
meditations perpetually of Śiva in the form of \tl{Praṇava} and maintains mystic
trance, certainly becomes Śiva Himself. He must perform \tl{japas} after duly
performing the \tl{aṅganyāsa} (ritualistic placing of the finger over
the different parts of the body as prescribed) and invoke the sages concerned,
the deities presiding over and the name of the metre in which the verse is
composed.

26. The devotee who practises the \tl{japa} of \tl{Praṇava} (Oṃ) with due
ritualistic placings of fingers on the parts of his body becomes a sage. He
shall attain all the benefits of the ritualistic \tl{nyāsa} such as
the blessings of ten mothers and the (attainment of) six pathways.

27-30. As for those who are devoted to activities and those who both refrain
from and indulge in activities, the gross \tl{Praṇava} is recommended.
Śivayogins are of three types being devoted to rites, austerities and \tl{japas}.
The Kriyāyogin is the one who engages himself in sacred rites and worship
spending money, using limbs of the body and uttering words \tl{namaḥ}
(obeisance) \etc Tapoyogin is the one who desists from injuring others, restains
all external sense organs, takes limited quantities of food and performs
worships. Japayogin is the one who is quiet, performs \tl{japa} always, is free
from all sorts of desires and maintains all these observances mentioned before.

31. A pure man shall obtain liberation only step by step, beginning with Sālokya
as a result of being purified by the worship of Śivayogins with sixteen services
and homage.

32. O \tl{brāhmaṇas}, I shall now explain \tl{japayoga}, please listen. Even
the person practising austerities shall perform \tl{japas} to purify himself.

33. O \tl{brāhmaṇas}, the five-syllabled mantra of Śiva is the gross \tl{Praṇava}.
The name Śiva is used in the dative case with \tl{namaḥ} prefixed. (\tl{namaḥ
śivāya}—Homage to Śiva) It implies the five principles.

34. The \tl{japa} of the five-syllabled mantra shall always be performed along
with \tl{praṇava}. A man can achieve everything by means of the \tl{japa} of
the five-syllabled mantra.

35. O \tl{brāhmaṇas}, the devotee shall take instruction from his preceptor,
sit comfortably on the ground cleaned well, and start the \tl{japa}.
The practice shall start on the \tl{caturdaśī} day of the bright half and
concluded on the \tl{caturdaśī} day of the dark half.

36-37. The months of Māgha and Bhādrapada are the most auspicious of all
occasions. During the days of \tl{japa} be shall take only a single meal during
the day in limited quantities. He shall abstain from useless talk and curb all
his sense organs. He shall uninterruptedly render service to his parents and
the king, or any master whom he serves. By performing the \tl{japa} a thousand
times, he shall be free from indebtedness, otherwise not.

38-42. The five-syllabled mantra shall be repeated five hundred thousand times,
all the time remembering the various aspects of Lord Śiva who is seated in
the lotus pose. He is the bestower of all auspiciousness. He has the crescent
moon for his coronet. He has given shelter to Gaṅgā in His matted hair. With
Śakti seated on His left thigh, He shines with His great concourse of attendants
around Him. He bears the moon (on his forehead). He shows the gestures of
bestowing boons and offering freedom from fear. He is the cause of perpetual
blessing. He is Sadāśiva. He shall be mentally worshipped at first or as
stationed in the heart or in the solar zone. While performing the \tl{japa} of
the five-syllabled mantra he shall sit facing the east. All his actions shall
be pure. In the morning of the \tl{caturdaśī} day of the dark half, after
finishing the daily rites he shall sit in a clean beautiful place. He shall
control his mind and senses. He shall repeat the five-syllabled mantra twelve
thousand times in this way.

43-44. For the sake of the worship he shall invite five great devotees of Śiva
along with their wives. One of those shall be an excellent preceptor who shall
be assigned the Sāmba form, another will represent Īśāna, the third will
represent the Aghora aspect of Śiva, the fourth will represent the Vāma aspect
of Śiva and the fifth will represent Sadyojāta aspect of Śiva.

45-47. All the articles for the worship shall be ready and the worship shall
start. When it is performed duly, the sacrifice shall follow. All the rites from
the beginning to the end shall be performed according to the rules laid down in
the scriptural code which the devotee follows. The ghee used shall be the one
prepared from the milk of a tawny cow. He shall make ten, hundred or a thousand
offerings or he shall bid the devotees of Śiva make the offerings. In that case
the offerings are one hundred and eight in number.

48-49. At the end of the sacrifice monetary gifts shall be given: the preceptor
shall be given two cows (or a cow and a bull) as extra. The five devotees shall
be duly worshipped; the householder shall take bath with the water wherewith
the feet of the devotees shall be washed. He shall thereby reap the benefit of
taking bath in 36 \tl{crores} of holy rivers and tanks.

50-52. He shall make gifts of cooked rice and ten ancillary constituents with
great piety. The preceptor’s wife must be considered as the great goddess
(Parā.) The wives of the other devotees Īśāna and the rest shall be duly
worshipped and honoured. They shall be presented with the beads sacred to Śiva,
garments, and sumptuously fed with milk pudding, pulse, pies, sweet pies \etc
after the oblations are duly given. The \tl{japa} is then concluded with due
prayers to the lord of gods.

53. After the performance of \tl{puraścaraṇa} (repetition of the mantra followed
by sacrifice), the householder becomes endowed with the efficacy of the mantra.
If he completes another five hundred thousand \tl{japas}, all the sins will be
wiped off.

54. For every set of five hundred thousand \tl{japas} the householder shall be
blessed with the riches and prosperity of the different \tl{lokas} beginning
with Atala and ending with Satyaloka in order.

55. If the householder dies in the middle, he shall be reborn in the world after
due enjoyment of pleasures in the other worlds. He shall then continue
the \tl{japa} and derive the benefit of being near to Brahman.

56. After a repetition of five hundred thousand further \tl{japas} he derives
the benefit of assimilation to Brahman. If ten million \tl{japas} are completed
in all he shall become identical with Brahman.

57. Thus attaining the absorption into Kāryabrahman (the action Brahman) he
gains all such enjoyments as can be wished for till the time of final
dissolution.

58. In the next \tl{kalpa} he will be born as Brahma’s son. Becoming illuminated
with the penance he shall be ultimately liberated.

59. Fourteen worlds beginning with Pātāla and ending with Satya are evolved out
of the five elements, such as the earth \etc These are called Brahmā’s worlds.

60-61. There are fourteen Viṣṇu worlds beyond Satya world and ending with Kṣamā.
In the Kṣamā world the action—Viṣṇu is stationed in the excellent city of
Vaikuṇṭha in the company of action-Lakṣmī protecting the great recipients of
enjoyment. Beyond that and ending with Śuciloka there are twenty-eight worlds.

62. In the pure world of Kailāsa, Rudra, the annihilator of the living beings,
is stationed. Beyond that are the fifty-six worlds ending with Ahiṃsā region.

63. The action-lord who has screened everything is stationed in the city of
Jñānakailāsa in the Ahiṃsā region.

64-67. At the end of the same is the wheel of Time and beyond the ken of Time
there is the space called Kālātīta. There Kāla (God of death and Time) backed by
Śiva and in the name of Cakreśvara, unites every one with Time. In his activity
he occupies \tl{dharma} in the form of a buffalo whose four legs are untruth,
untidiness, violence and ruthlessness. He can assume any form he wishes. He
assumes the form of a great buffalo, is rich in Atheism, has evil association
and utters sounds other than those of the Vedas. He has an active association
with Anger. He is black in colour. He is called great lord (Maheśvara) to that
extent. The ability to vanish is up to that extent.

68. Beneath that is the \tl{karmabhoga} enjoyment as a result of activity.
Beyond that point is \tl{jñānabhoga} (enjoyment due to knowledge). Beneath that
point is \tl{karmamāyā} and beyond that point is \tl{jñānamāyā}.

69. Explanation of \tl{karmamāyā}—Mā means Lakṣmī \ie \tl{karmabhoga}. Attainment
of the same is Māyā. The word \tl{Mā} is then interpreted as \tl{jñānabhoga}.
Attainment of the same is Māyā.

70. Beyond that point is \tl{nityabhoga} (perpetual enjoyment). Beneath that
point is \tl{naśvarabhoga} (evanescent enjoyment). Beneath that is evanescence
and beyond that there is freedom.

71. The bondage of nooses is only beneath that point. There is no bondage beyond
that. Those who perform actions with desire alone, hover beneath that point.

72. The enjoyment of rites performed with no desire is said to be beyond that
point. Those who are devoted to the worship of womb, hover beneath that.

73. The worshippers of the phallic emblem who are unaffected by desire can go
beyond that. Worshippers of deities other than Śiva, hover beneath that.

74. Those who are devoted to Śiva alone can go beyond that. \tl{Crores} of
\tl{jīvas} live beneath that point. There is a great fort-wall as it were above
the same.

75. Persons bound by worldly existence remain beneath that point and those who
are liberated go beyond that. Those who worship the natural substances hover
beneath that.

76. Those who worship the entity of Puruṣa go beyond that point. Śaktiliṅga is
beneath that point but Śivaliṅga is beyond.

77. The unmanifest \tl{liṅga} is beneath that point but the manifest one is
beyond. The conceived \tl{liṅga} is beneath and the unconceived one is beyond.

78. The external \tl{liṅga} is beneath that point and the internal one is beyond.
The \tl{śaktilokas} numbering hundred and twelve are beneath that point.

79. The \tl{bindurūpa} is beneath that point and \tl{nādarūpa} is beyond.
The Karmaloka is beneath that point and Jñānaloka is beyond that.

80-81. Obeisance which is beyond that point quells pride and egotism. The word
\tl{jan} means evanescence, \tl{na} is a negative particle. The word \tl{jñāna},
therefore, means that which wards off evanescence. Those who worship elements
hover beneath that point.

82-83. And those who worship spiritual things go beyond that point.
The \tl{vedibhāga} (the portion of the Altar) in that great world of Ātmaliṅga
is only up to that point. The eight fixation of Prakṛti \etc. is also at
the extremity of the Vedi. Such is the customary and the scriptural procedure.

84. Those who are endowed with the virtue of truthfulness \etc and those who are
devoted to the worship of Śiva cross Kālacakra who is seated on Adharmamahiṣa
(The buffalo of evils).

85. Beyond that stands, ahead of Śivaloka, the bull of Virtue in the form of
celibacy. It has the legs of Truthfulness \etc

86. The bull of Dharma has forbearance for its horns, restraint for its ears,
faith for its eyes, sighs for its intellect and mind. It is embellished by
the sound of Vedic chants.

87. The bulls of sacred rites \etc are to be understood as stationed in
the causes. Kālātīta (\ie Maheśvara) presides over the bull of sacred rites.

88. The span of life of Brahmā, Viṣṇu and Maheśa is a day. Beyond that, there is
neither day nor night, neither birth nor death.

89-90. The worlds ending with Kāraṇasatya, of the Kāraṇabrahmā (Brahmā the cause)
evolved out of the subtle elements, smell \etc are stationed beyond it. In all
these fourteen worlds, the subtle smell \etc give the due form. The fourteen
worlds of Kāraṇaviṣṇu are stationed there.

91. The \tl{lokas} of Kāraṇarudra are twenty-eight in number. The \tl{lokas} of
Kāraṇa-īśa numbering fifty-six are beyond that.

92-93. The Brahmacaryaloka accepted by Śiva is beyond that. There in
the Jñānakailāsa that has five coverings, the primary phallic from of Śiva is
stationed in the company of primary energy of Śiva. It has five zones and five
Brahmakalās.

94. This is called the abode of Śiva, Śivālaya, the supreme Ātman. There alone
stays Parameśvara in the company of Parāśakti.

95. He is skilled in the performance of the five functions of creation,
maintenance evanescence and blessing. His body is Existence, Knowledge and Bliss.

96. He is always in meditation. He is ever bent on blessing. He is seated in
the pose of trance. He shines resting in his own self.

97-98. His vision is possible gradually through sacred rites, meditation \etc
By performing the daily rites and worships, the mind is diverted towards
the sacred rites of Śiva the performance whereof gives the sight of Śiva. Those
who come within His vision are certainly liberated.

99. The liberation is in the form of realising the nature of Ātman. It is
a relaxation and resting in one’s own soul. It is based on sacred rites, penance,
\tl{japa}, knowledge, meditation and virtue.

100-101. Relaxation is assured at the vision of Śiva. Śiva, the merciful,
removes ignorance even as the sun removes all impurities and darkness by means
of its rays. When ignorance is dispelled, the knowledge of Śiva begins to
function.

102. On acquiring the knowledge of Śiva a person achieves relaxation. He becomes
gratified at the acquisition of relaxation.

103-104. Again by means of ten million \tl{japas} he acquires Brahmā’s region.
A further ten million \tl{japas} enable him to achieve Viṣṇu’s region. By
a further ten million \tl{japas} he attains Rudra’s region and by a further ten
million \tl{japas} Īśvara’s region is attained.

105. Again by a similar \tl{japa} performed with concentration he attains
Kālacakra, the first in the Śivaloka.

106-107. The Kālacakra consists of five wheels, one being over the other. Sight
and delusion (\tl{drṣṭi} and \tl{moha}) constitute the Brahmacakra; Enjoyment
and delusion (\tl{bhoga} and \tl{moha}) constitute the Viṣṇu Cakra. Anger and
delusion (\tl{kopa} and \tl{moha}) constitute the Raudra Cakra, Revolution
(Bhramaṇa) is Īśvaracakra. Knowledge and illusion (\tl{jñāna} and \tl{moha})
constitute the Śivacakra. Thus scholars have explained the five \tl{cakras}.

108. Then by ten \tl{crores} of \tl{japas} he achieves the region of Kāraṇa
Brahman. Again by ten \tl{crores} he attains the prosperity of that region.

109-110. Thus, gradually, attaining the region of Viṣṇu and those of other Gods
as well as the prosperities of those regions, completing assiduously
the repetitions to the tune of hundred and five \tl{crores} of times, he attains
Śivaloka outside the fifth sheath.

111. There is a silver platform there, an excellent river bed, and a bull in
the form of penance.

112. The fifth sheath is the excellent station of Sadyo-Jāta (a form of Śiva).
The fourth is the station of Vāmadeva.

113. The third is the abode of Aghora. The second is the abode of Sāmba Puruṣa.

114. The first is the abode of Īśāna. The fifth is the place of Dhyāna Dharma
(virtue of meditation).

115. The abode of Balinātha is the bestower of the full \tl{amṛta} (deathlessness,
nectar). Thereafter is the fourth \tl{maṇḍapa} with the idol of Candraśekhara
(a form of Śiva).

116. The abode of Somaskanda is the third \tl{maṇḍapa}. The faithfuls say that
the second \tl{maṇḍapa} is the Nṛtya-Maṇḍapa.

117. The first \tl{maṇḍapa} is the abode of Mūlamāyā (primary delusion) and is
very auspicious and stationed there itself. Beyond that is the sanctum sanctorum,
the auspicious place of the phallic form of Śiva.

118. No one can realise the flourishing power of Śiva stationed at the back of
Nandi. Nandīśvara sits outside and repeats the five-syllabled mantra.

119. This knowledge has come down from the preceptors. I got it from Nandīśa.
Beyond this, it must be inferred from it and it is actually experienced only by
Śiva.

120. The full grandeur and greatness of Śivaloka can be known by any one only
out of the grace of Śiva and not otherwise, so say the faithfuls.

121. It is thus that \tl{brāhmaṇas} of controlled sense-organs become liberated
gradually. I shall tell you the process in some other cases. Please listen
attentively.

122-123. \tl{Brāhmaṇa} women must take instruction from a preceptor and perform
the \tl{japa} with \tl{namaḥ} at the end. They shall repeat the five-syllabled
mantra five hundred thousand times for their longevity. That is the rule. Again
they must repeat it five hundred thousand times to wipe off womanhood. Becoming
a man first, the liberation will be acquired gradually.

124. A \tl{kṣatriya} must repeat the mantra five hundred thousand times to
remove \tl{kṣatratva}. A further repetition of five hundred thousand times
enables him to become a \tl{brāhmaṇa}.

125. After the \tl{mantrasiddhi} he shall gradually become liberated.
A \tl{vaiśya} dispels the \tl{vaiśyatva} by five hundred thousand \tl{japas}.

126. Then he becomes a \tl{mantra-kṣatriya} by repeating it five hundred
thousand times. He then dispels the \tl{kṣatratva} by five hundred thousand
\tl{japas}.

127-129. He then becomes a \tl{mantrabrāhmaṇa} by repeating the mantra five
hundred thousand times. A \tl{śūdra}, repeating the mantra with \tl{namaḥ} at
the end, for two million five hundred thousand times becomes
a \tl{mantrabrāhmaṇa} and so pure enough for liberation. If one is sick, whether
man or woman, of \tl{brāhmaṇa} caste or otherwise, one must repeat it always
with \tl{namaḥ} in the beginning or at the end. As for the women, the preceptor
shall instruct them in proper order.

130. At the end of every five hundred thousand \tl{japas}, the aspirant shall
perform \tl{mahābhiṣeka} and \tl{naivedya}. He shall worship devotees of Śiva
for gratifying Śiva.

131. Śiva becomes delighted at the worship of the devotee. There is no
difference between Śiva and the devotee of Śiva. He is Śiva Himself.

132. The mantra is of the nature of Śiva. By holding the mantra the physical
body of the devotee becomes identified with Śiva.

133-134. Devotees of Śiva know all the rites, nay all the Vedic rites. The more
an aspirant repeats the mantra of Śiva, the greater is the presence of Śiva in
his body. For the woman devotee of Śiva, the symbol of the goddess shall be
the form for concentration.

135. The presence of the goddess continues to be felt as long as the mantra
continues to be repeated. An intelligent man who continues to worship Śiva
becomes worthy of the name and form.

136. Even when the aspirant has become Śiva he shall worship the Parā. He shall
worship Śakti, the embodied and the phallic form of Śiva after faultlessly
making images of the same.

137-142. He shall consider the phallic form as Śiva and himself as Śakti or he
shall consider \tl{śaktiliṅga} as the goddess and himself as Śiva or he shall
consider \tl{śivaliṅga} in the form of \tl{nāda} and \tl{śakti} in the form of
\tl{bindu} and give the primary or secondary character to either or consider
both united together. Whatever be the form of Upāsti, he shall worship both Śiva
and Śakti. He becomes Śiva in virtue of his basic realisation. With the sixteen
forms of service and homage, he shall worship devotees of Śiva who are verily
the mantra of Śiva personified or identical with Śiva. He will thereby achieve
whatever he desires. Śiva being highly pleased with him yields to his
gratification. Without being undeceptive in regard to money, body, mantra or
the conception he shall gratify five, ten or hundred couples of Śiva’s devotees
by feeding them and rendering them other services, in the company of his wife.

143-146. He will assume the form of Śiva and Śakti and will not be born again.
Just below the umbilicus is the part of Brahmā, till the armpit is the part of
Viṣṇu and the face is the phallus in the body of a devotee of Śiva. If any one
dies, the householder shall worship the primordial father Śiva, the primordial
mother Śivā and the devotees of Śiva. Thereby, whether the dead body is properly
cremated or not, the dead man shall go to the world of the manes and gradually
attain salvation. A person endowed with \tl{tapas} is far better than ten
persons endowed with rites.

147-153. A person endowed with \tl{japa} is superior to a hundred persons
endowed with \tl{tapas}. A person endowed with the knowledge of Śiva is superior
to a thousand persons endowed with \tl{japas}. A person endowed with meditation
is superior to a hundred thousand persons who have the knowledge of Śiva.
A person endowed with the power of trance is superior to a crore of meditating
persons. Since the latter are superior to the former they shall be selected for
worship. Even sensible persons cannot fully comprehend the excellence of
benefit. An ordinary man cannot understand the greatness of the devotee of Śiva.
The worship of the devotee of Śiva is on a par with the worship of Śiva and
Śakti. He who worships any of these piously becomes Śiva and attains Śiva. He
who reads this significant chapter, that agrees with the Vedic injunctions,
becomes a \tl{brāhmaṇa} endowed with the knowledge of Śiva and rejoices in
the company of Śiva. O scholarly lords of sages, a person who knows special
things must narrate them to the devotees of Śiva. By Śiva’s grace he will be
blessed.
