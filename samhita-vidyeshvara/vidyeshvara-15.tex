\chapter{Qualification, time and place for \tl{devayajña}, etc.}

The sages said:

1. O Sūta, foremost among those who know everything, please expound to us
the place \etc

Sūta said:

The pure house accords normal benefit in the rites of \tl{devayajña} \etc

2. The cowshed is of ten times more benefit than that. The bank of a tank is of
ten times more benefit than that and the root of Tulasi plant or of Bilva or
Aśvattha trees is again of ten times more benefit than that.

3-5. Similarly a temple, the bank of a holy tank, the bank of an ordinary river,
the bank of a holy river and the banks of the seven holy Gaṅgās are each of ten
times more benefit than the previous. The seven holy Gaṅgās are Gaṅgā, Godāvarī,
Kāverī, Tāmraparṇikā, Sindhu, Sarayū\footnote{It is a well known river,
mentioned in the RV. (5.53.9) along with the rivers Sarasvatī, Sindhu, Gaṅgā,
Yamunā, and Śutudrī. Gharghara (Ghāgrā) and Tamasā (Tons) are its tributaries.
It is a sacred river of Northern Kosa, with Ayodhyā, the sacred city of great
antiquity, lying along its bank.} and Revā. The shores of the sea are of ten
times more benefit than the previous. The summit of a mountain is of ten times
more benefit than the shores of the sea.

6-7. The place where the mind is quite at home is the most excellent of all
places. \tl{Yajña, dāna} \etc accord full benefit in the Kṛta age. In the Tretā
age they yield three-fourths of the benefit. In the age of Dvāpara the benefit
derived is half. In the age of Kali only one fourth of the benefit is obtained.
When half of the Kali age passes on, the benefit is only three-fourths of this
one-fourth.

8. A holy day accords a normal benefit to a pure-souled devotee. O Scholars,
the period of transit of the sun from one Zodiac to another yields ten times
more benefit than that.

9. The period of equinoxes, the period of tropical transit, the period of
transit to the Capricornus, and the time of lunar eclipse are each of ten times
more benefit than the previous one.

10. The auspicious hour of complete solar eclipse is of still more benefit, than
the previous. Since the sun of cosmic form is infested with poison then, there
is the likelihood of ailments spreading.

11. Hence for the alleviation of the serious effects of poison, the devotee
shall observe ceremonial ablutions, offer gifts and mutter prayers. That period
is specially holy inasmuch as it is intended for the alleviation of the after-
effects of poison.

12. The birth-star, and the concluding period of holy rites are of the same
efficacy as the period of Solar eclipse. The time spent in the company of noble
holy men is of the efficacy of \tl{crores} of solar eclipses.

13. Persons of unflinching devotion to austerities and perfect knowledge,
\tl{yogins} and ascetics deserve holy worship since they quell others’ sins.

14. A \tl{brāhmaṇa} who has repeated the \tl{Gāyatrī} mantra two million four
hundred thousand times also deserves the same and accords full benefit and
wordly enjoyments.

15. The word \tl{pātra} (one who deserves) means one who protects the giver
from downfall.

16-17. The word \tl{Gāyatrī} means that which saves the reciter from downfall.
Only a person of purified soul can save others, just as only a rich man can
donate anything to others. A man of no means cannot give anything to others in
this world.

18-19. Only he who has purified himself by means of \tl{Gāyatrī japa} can be
called a pure \tl{brāhmaṇa}. He alone deserves the position of presiding over
all holy rites, \tl{dāna, japa, homa, pūjā} \etc He alone can save others. Any
hungry man or woman deserves charitable gifts of cooked food.

20-21. An excellent \tl{brāhmaṇa} must be invited on an auspicious occasion and
given sufficient sums of money with piety and pleasing words. They accord all
desired results. A charitable gift given to a needy person yields the utmost
benefit. If it is given after entreaties it yields only half the benefit.

22-23. Monetary gifts to servants accord only one-fourth benefit. O excellent
\tl{brāhmaṇas}, charitable gifts to an indigent person, only because he is born
a \tl{brāhmaṇa}, accord worldly enjoyment for ten years. Gifts to
a \tl{brāhmaṇa} Vedic scholar accord heavenly enjoyment for ten years.

24. Gifts to a \tl{brāhmaṇa} who regularly repeats \tl{Gāyatrī mantra}, accord
Satyaloka for ten years. Gifts to a \tl{brāhmaṇa} devotee of Viṣṇu accord
Vaikuṇṭha Loka.

25. Gifts to a \tl{brāhmaṇa} devotee of Śiva accord Kailāsa. All kinds of gifts
accord enjoyments in the different \tl{lokas}.

26-28. A person who gives cooked food attended with the ten ancillary services,
on a Sunday, attains good health for ten years even in the next birth. The ten
ancillary services are honouring, inviting, providing oil bath, washing and
serving the feet, bestowing cloth, scents \etc, serving side dishes of six
tastes, pancakes prepared in ghee and sweet juices, betel leaves, monetary gifts,
formal farewell and following a few steps — This is called \tl{daśāṅga annadāna}.

29-30. A man who renders ten sorts of ancillary services to ten \tl{brāhmaṇas}
on Sunday attains good health for a hundred years. If he gives the same on
Monday or any other day, he attains the benefit as stipulated for that day.
The benefit of food-gifts is secured in this world itself either in this birth
or in the next.

31. If in this manner he gives food on all the seven days to ten \tl{brāhmaṇas}
he secures good health and all other benefits for a hundred years.

32. Similarly he who gives cooked rice in this manner to hundred \tl{brāhmaṇas}
on Sunday secures good health in Śivaloka for a thousand years.

33. If he gives the same for a thousand \tl{brāhmaṇas} he secures the benefit
for ten thousand years. Similarly the benefit accrued for gifts on Monday and
other days can be understood by a thoughtful man.

34. By giving food to a thousand \tl{brāhmaṇas} whose minds have been purified
by \tl{Gāyatrī}, on Sunday, the devotee attains good health and other benefits
in Satyaloka.

35. By giving food to ten thousand persons he secures the benefits in Viṣṇuloka.
By giving it to a hundred thousand persons he derives benefits in Rudraloka.

36. Those who seek learning must make gifts to children considering them on
a par with Brahmā. Those who seek sons and other ends must make gifts to young
men considering them on a par with Viṣṇu.

37. Those who seek knowledge must make gifts to old men considering them on
a par with Rudra. Those who seek intellect must make gifts to young maidens
considering them on a par with Bhāratī (Goddess of Speech).

38. Excellent men seeking enjoyments must make gifts to youthful maidens
considering them on a par with Lakṣmī (Goddess of Wealth). Those who seek purity
of Ātman must make gifts to old women considering them on a par with Pārvatī.

39. That which is acquired by gleaning more than one ear of corn at a time or
gleaning corns one by one, by fees received from disciple is called
\tl{śuddhadravya} (clean wealth). This wealth yields complete benefit.

40. Wealth acquired by acceptance of monetary gifts is called middlesome wealth.
Wealth acquired by agricultural or trading activities is called lowliest wealth.

41. Wealth acquired by \tl{kṣatriyas} using their valour or \tl{vaiśyas} by
trading activities is called excellent. So also the wealth acquired by
the \tl{śūdras} by salaries for service.

42-45. Patrimony or sum received from husbands forms the wealth of virtuous
women. There are twelve things to be given in the twelve months beginning with
Caitra or all together on an auspicious occasion for the flourishing of what is
cherished. They are: (1) cow, (2) plots of land, (3) gingelly seeds, (4) gold,
(5) ghee, (6) cloth, (7) food-grains, (8) jaggery, (9) silver, (10) salt, (11)
ash gourd and (12) a virgin. Gift of cows, milk-products, cow-dung (in the form
of manure \etc) ward off the sins accruing from wealth and grain while sins
connected with water, oil \etc are warded off by cow’s urine.

46. The three kinds of sins—physical \etc are warded off by milk, curd and ghee.
Their nourishment can be understood by scholars.

47. Gift of plots of land is conducive to stability here and hereafter, O
\tl{brāhmaṇas}. Gift of gingelly seeds is conducive to strength and to
the conquest of premature death.

48. Gift of gold increases the power of the gastric fire and is conducive to
virility. Gift of ghee is nourishing and that of cloth is conducive to long life.

49. Gift of food-grains is conducive to the increase of food production. Gift of
jaggery yields sweet food. Gift of silver is conducive to the increase in
the quantity of semen and that of salt is conducive to the happy admixture of
the six tastes.

50. The gift of pumpkin gourd is conducive to nourishment. All kinds of gifts
increase everything and secure all kinds of enjoyment here and hereafter,
O \tl{brāhmaṇas}.

51-53. Gift of a virgin is conducive to worldly enjoyment throughout life.
Sensible persons shall make gifts of fruits according to the season such as
the fruits of jack, mango, wood apple trees, plantains, fruits from hedges,
pulses of black gram, green gram, vegetables, chillies, mustards, their plants
\etc

54. Sensible men shall gratify the sense-organs of hearing \etc of other people
for the gratification through sound \etc It gratifies the quarters too.

55. Theism is that feeling in which one fully realises that all actions are
fruitful. It is necessary that Vedas and sacred texts should be learnt direct
from preceptors.

56. Devotion to God out of fear for kinsmen or royal punishment is of inferior
sort. An indigent person bereft of all means of livelihoods shall worship
verbally or by means of physical activities.

57. Verbal worship means recital of mantras, hymns and \tl{japas}. Worship of
physical activities means pilgrimages, observance of fast and other rites.

58. Whatever one does, whether it is great or small, whatever be the means
employed,—if that is dedicated to deities it becomes conducive to enjoyment.

59-61. The two—practice of austerities and making charitable gifts—must be
carried out always. Asylum should be given according to the caste of the person
concerned. It is conducive to the satisfaction of the \tl{devas} and worldly
enjoyments as well. Such a devotee shall always attain noble birth and
enjoyments here and hereafter. If he performs the sacred rites with dedication
to God, he shall attain salvation. He who reads or hears this chapter becomes
righteous and endowed with knowledge.
