\chapter{Bondage and liberation: Glorification of the phallic emblem of Śiva}

The sages said:

1. O foremost among those who know everything, please explain the nature of
bondage and liberation.

Sūta said:

I shall explain bondage, liberation and the means of liberation. Please listen
attentively.

2. A \tl{jīva} is said to be in bondage if he is tied up by the noose of
eightfold primary essences, \tl{prakṛti} \etc When freed from them he is called
liberated.

3. Perfect control and subjugation of \tl{prakṛti} and its offshoots is
Salvation. A \tl{jīva} in bondage when freed from it is called a liberated soul.

4. The set of eight that binds is: \tl{prakṛti}, \tl{buddhi} (cosmic intellect),
\tl{ahaṃkāra} (cosmic ego) of the nature of attributes, and the five
\tl{tanmātrās} (cosmic principles of Ether \etc)

5. The body is evolved out of these eight. The body carries on activities.
The activities generate the body. Thus birth and activities continue in
a series.

6-7. The body is of three types: the gross, the subtle and the causal. The gross
body is responsible for all activities; the subtle body yields the enjoyment of
pleasures through the senses. The causal body is for the sake of experiencing
the good and bad results of the activities of the \tl{jīva}. The \tl{jīva}
experiences happiness as a result of virtue and misery as a result of sin.

8. The \tl{jīva} bound by the rope of activities revolves round and round for
ever like a wheel by means of the three types of body and their activities.

9. The creator of the wheel must be worshipped for the cessation of
the revolution of the wheel. The \tl{prakṛti} \etc constitute the great wheel
and Śiva is beyond the \tl{prakṛti}.

10-11. The creator of the wheel is the Lord Śiva. He is beyond the \tl{prakṛti}.
Just as a boy drinks or spits out water as he pleases so also Śiva keeps
\tl{prakṛti} \etc just as he pleases. He is called Śiva because he has brought
it under his control. (Vaīśkṛta). Śiva alone is omniscient, perfect and free
from desire.

12. The mental prowess of Maheśvara which Vedas alone can comprehend consists of
omniscience, satiety, beginningless understanding, independence, never failing
and infinite power.

13. Hence \tl{prakṛti} \etc come under control due to Śiva’s grace. One shall
worship Śiva alone for the acquisition of Śiva’s grace.

14. If one were to ask “How can there be a self-less worship of a perfect
being?” the answer is “An activity done with dedication to Śiva shall cause
pleasure to him”.

15. Keeping Śiva in view the devotee shall worship the phallic or the embodied
image of Śiva, or his devotee. He shall worship his devotee by means of the body,
mind, speech and money spent.

16. Śiva, the great lord, who is beyond \tl{prakṛti} is delighted at the worship
and specially blesses the worshipper.

17-19. The \tl{karma} \etc come under control gradually due to Śiva’s grace.
Beginning with \tl{karma} and ending with \tl{prakṛti} when everything comes
under control, the \tl{jīva} is called liberated and he shines as
a self-realised person. By the grace of Śiva, when this body which is resultant
from activities (\tl{karmadeha}) comes under control, the devotee attains
residence in Śivaloka. This is called \tl{sālokya} form of liberation. When
the subtle elements come under control, the devotee attains nearness to Śiva.

20. Then he attains similarity with Śiva by means of weapons and activities.
This is called \tl{sārūpya}. When the devotee acquires the great favour,
the cosmic intellect too comes under control.

21. The cosmic intellect is only an effect of the \tl{prakṛti}. The control of
intellect is called \tl{sārṣṭī}—a form of liberation wherein the devotee has
the same rank and power as Śiva. Then due to a further great favour of Śiva,
the \tl{prakṛti} comes under control.

22-23. The mental prowess of Śiva becomes his without any difficulty. On
acquiring the omniscience and prosperity of Śiva, the devotee becomes
resplendent in his soul. This is called \tl{sāyujya} (complete identity) by
persons well-versed in the Vedas and Āgamas (traditional sacred texts). It is in
this order that one gets salvation by the worship of the phallic image of Śiva.

24. Hence the devotee shall worship Śiva by performing sacred rites \etc for
the acquisition of Śiva’s favour. Śiva’s sacred rites, Śiva’s penance, and
the \tl{japas} of Śiva mantras always.

25. Knowledge of Śiva and meditation on Him shall be practised more and more.
The time till retirement to bed, the time till death shall be spent in
contemplating over Śiva.

26-27. He shall adore Śiva by means of the “Sadyo” mantras and flowers. He will
attain welfare.

The sages said:

O excellent one of good rites, please explain the rules governing worship of
Śiva in the phallic and other forms.

Sūta said:

I shall explain, O \tl{brāhmaṇas}, the procedure of the worship of the phallic
form, please listen. The first phallic form is the \tl{Praṇava} that confers
all desires.

28. It is called \tl{Sūkṣma Praṇava} (the subtle one) if it is \tl{niṣkala}.
The \tl{sthūla} (gross one) is \tl{sakala} and it consists of five constituent
syllables.

29. The worship of these two is called a penance. Both of them accord salvation.
There are many phallic emblems of \tl{Pauruṣa prakṛti}.

30. Śiva alone can explain them in detail. No one else. Such as are evolved of
Earthly material are known to me which I shall explain to you all.

31. These are of five types: (1) Svayambhū, (2) Bindu, (3) Pratiṣṭhita, (4) Cara,
(5) Guru Liṇga.

32-33. When he is gladdened by the austerities of \tl{devas} and sages, Śiva in
the form of \tl{nāda} assumes the form of a seed under the ground and suddenly
piercing the ground above like a germinating sprout manifests Himself outside
and makes His presence felt. Since this emblem is self-raised it is called
Svayambhū.

34-35. By worshipping it the devotee gains increasing knowledge automatically.
In a gold or silver plate or on the ground or an altar, the devotee draws
the picture of the phallic emblem, the pure \tl{Praṇava} mantra and shall invoke
it with the rites of \tl{pratiṣṭhā} and \tl{āvāhana}.

36. The Bindu and Nāda forms, the stationary or mobile ones are conceptual but
belong to Śiva, undoubtedly.

37-38. Wherever Śiva is sincerely believed to be present, the lord bestows on
the devotee the benefit through that alone. The devotee can invoke the lord in
a natural immobile thing—a rock or a stump—or an engraved picture and worship
Śiva by the sixteen \tl{upacāras} (services and homage). He will attain supreme
power of the lord and by practice gain knowledge.

39-40. If the image is installed with pure mind in a pure altar either by
the Gods or the sages for the realisation of the soul, it is called Pauruṣa and
it comes under the category of the installed phallic image of Śiva.

41-42. By a regular worship of this phallic image, the devotee will obtain all
\tl{pauruṣa aiśvaryas} (human riches). If great \tl{brāhmaṇas} or rich kings
install a \tl{liṅga} prepared by the artisans, it is called \tl{pratiṣṭhita} and
\tl{prākṛta}. It accords enjoyment of \tl{prākṛta aiśvarya} (natural riches) to
the worshipper.

43. That which is forceful and permanent is called \tl{pauruṣa}. That which is
weak and temporary is called \tl{prākṛta}.

44. The spiritual cum mobile form is represented by the constituents of the body,
\viz the penis, navel, tongue, the tip of the nose, hips \etc

45. The mountain comes under the \tl{pauruṣa} class and the surface of the world
under the \tl{prākṛta} class. Trees \etc are \tl{pauruṣa} and creepers \etc are
\tl{prākṛta}.

46. The \tl{ṣāṣṭika} rice is \tl{prākṛta} but rice of the \tl{śāli} variety and
wheat are \tl{pauruṣa}. The \tl{aiśvaryā} is \tl{pauruṣa}. It bestows eightfold
siddhis \viz \tl{aṇimā} \etc

47. The \tl{prākṛta liṅga} bestows good women, riches \etc according to
the believers. Now, first of all I shall mention the \tl{rasaliṅga} from among
\tl{caraliṅgas}. (\tl{Rasaliṅga} is mentioned as the foremost among mobile
\tl{liṅgas}).

48. \tl{Rasaliṅga} is a bestower of all wishes to the \tl{brāhmaṇas}.
The auspicious \tl{bāṇaliṅga} is a bestower of vast kingdoms to
the \tl{kṣatriyas}.

49. A gold \tl{liṅga} bestows the ownership of vast wealth on the \tl{vaiśyas}.
A \tl{śilāliṅga} (a \tl{liṅga} made of rock) bestows great purity on
the \tl{śūdras}.

50. A crystal \tl{liṅga} and a \tl{bāṇaliṅga} bestow all sort of wishes on all.
If a devotee does not possess a \tl{liṅga} of his own, there is no harm in
using another’s \tl{liṅga} for the purpose of worship.

51. An earthly \tl{liṅga} shall be used by women especially by those whose
husbands are alive. In the case of widows who are engaged in worldly and sacred
rites a crystal \tl{liṅga} is recommended.

52. O sages of good rites, in the cases of widows whether they be in a childhood,
youth or old age, a \tl{rasaliṅga} is specially recommended if they continue to
be holding rites.

53. A \tl{liṅga} of pure crystal bestows all sorts of worldly enjoyment on women.
The worship of the pedestal grants all cherished desires of the worshipper in
this world.

54. A ritualist shall perform all the worship in a vessel. At the conclusion of
\tl{abhiṣeka} (ceremonial bath) the \tl{naivedya} consisting of cooked rice of
the \tl{śālī} variety shall be offered.

55. When the worship is over, the \tl{liṅga} shall be kept in a casket and
placed separately in the house. Persons who worship their own \tl{liṅgas} shall,
after the worship is over, offer as food those articles of diet to which they
are accustomed.

56. All non-ritualists shall worship the subtle \tl{liṅga}. In the place of
floral offerings they shall use sacred ashes for adoration and food.

57. They shall keep the \tl{liṅga} after worship on their head for ever. The ash
is of three types, derived from ordinary fire, Vedic fire and Śiva fire.

58. The ash derived from ordinary fire shall be used for the purification of
articles of mud, wood or metals and even for grains.

59. Articles of worship like gingelly seeds, cloths and stale stuffs shall be
purified with ashes.

60. So also the objects defiled by dogs \etc The ashes shall be used with or
without water according to necessity.

61. The ashes resulting from Vedic rites in fire shall be smeared over
the forehead at the end of the rites. Since the ashes are purified by
the mantras the rite itself takes the form of the ashes.

62-65. Hence, applying the ashes is tantamount to assimilating the sacred rite
in one’s own Ātman. Bilva twigs shall be burnt repeating the Ātma mantra of
Aghora. This fire is called Sivāgni. The ashes resulting therefrom are called
\tl{śivāgnija}. The dung of a cow, preferably of Kapilā cow, shall be burnt
first and then the twigs of \tl{śamī, aśvattha, palāsa, vaṭa, āragvādha} or
\tl{bilva} shall be burnt. The ash resulting therefrom is also \tl{śivāgnija}.
Or the twigs shall be burnt in \tl{darbha} fire repeating Śiva mantra. After
straining the ashes with cloth (the fire powder) shall be put in a new pot.

66. For the sake of resplendence, the ashes shall be taken. The word \tl{bhasma}
(ash) means that which is honoured and adored. Śiva formerly did so.

67. A king takes the essence of wealth by way of tax, in his kingdom. Men burn
plants and take the essence thereof.

68. The gastirc fire burns different kinds of foodstuffs and with their essence
nourishes the body.

69. Similarly the great lord Śiva, the creator of the universe, burns
the universe presided over by Him and takes the essence of the same.

70. After burning the universe He applies the ashes over his body. Under
the pretext of annihilation He has taken the essence out of the same.

71. He assigned the essence to His own body. The essence Ākāśa (the Ether)
constitutes His hair. The essence of the wind principle constitutes His face.

72. The essence of the Fire principle constitutes His heart, that of
the principles of waters the hip and that of the principle of the Earth
the knees. Thus the other limbs too.

73. The \tl{tripuṇḍraka} (the three parallel lines of ash marks over
the forehead) is the essence of Trinity: Brahmā, Viṣṇu and Rudra. Similarly
Maheśvara has retained the esence of everything in the form of \tl{tilaka}
(the small circular mark) on the forehead.

74. The word \tl{bhasma} means that which has controlled the essence of the
whole universe. (\tl{Bha—Vṛddhi}—flourishing essence. \tl{Sma—Svayam}.
\tl{Manyate}—considers his own).

75-77. The word Śiva signifies him who controls everything and whom none can
control, (Śiva Vaśī) just as \tl{siṃha} signifies the creature who attacks other
animals and whom other animals cannot attack (\tl{siṃha} = \tl{hiṃsa}). The word
Śiva is given another interpretation. The syllable “ś” means Permanent Bliss.
The letter “i” means Puruṣa (the primordial male energy), the syllable “va”
means Śakti (the primordial female energy). A harmonious compound of these
syllables is Śiva. The devotee shall likewise make his own soul a harmonious
whole and worship Śiva.

78. Ashes must first be smeared in the dust form and then in the \tl{tripuṇḍraka}
form. At the time of worship water is added to the ashes. For mere
sanctification the ashes are used without water.

79. The devotee, whether it is day or night, whether it is a man or a woman
shall use water with the ashes and wear \tl{tripuṇḍra} at the time of adoration.

80. He who has the \tl{tripuṇḍra} made of ashes with water and performs worship
derives the entire benefit of the same, no one else.

81. Wearing the ashes with Śiva’s mantra he comes out of the limitations of
the \tl{āśramas}. He is called \tl{śivāśramī} for he is solely devoted to Śiva.

82-83. Being the devotee of Śiva and devoted to his sacred rites he need not
observe impurity accruing from death or birth in the family. The characteristic
sign of a devotee of Śiva is that he has a circular dot of white ashes or mud
put by himself or by his preceptor on the top of his forehead. The word \tl{guru}
(Preceptor) signifies a person who wards off bad qualities.

84-85. He removes all the ill effects of the \tl{rājasaic} qualities. He is
supreme Śiva himself. He is beyond the three \tl{guṇas}, and assuming the form
of the preceptor removes the ill effects of the three \tl{guṇas} and makes
the disciple understand Śiva. Hence he is the preceptor of the disciples who
have faith.

86. Hence the intelligent devotee shall know that the physical body of
the preceptor is known as \tl{guruliṅga} the worship of which is service
rendered to the preceptor.

87-88. The word ‘service’ means an obedience to the order through body, mind and
speech. A disciple with a pure soul shall of necessity carry out the order of
the preceptor risking his life and staking his possessions even if the task is
not within his power. The word \tl{śiṣya} (disciple) means a person who is
worthy of being ordered about.

89. Dedicating all he has, even his body, to the preceptor, the disciple shall
offer his food first to the preceptor and then take his food with his permission.

90-92. Verily a disciple in virtue of his being subjected to discipline is a son
unto the preceptor. Moreover by means of his tongue (as penis) he discharges
the semen in the form of mantra in the vaginal passage of the ears and begets
the \tl{mantraputra} in the form of disciple. The son shall therefore adore his
preceptor as father unto him. The real father, the physical begetter, drowns
the son in the ocean of worldly existence. But the preceptor, the giver of
knowledge, the father of learning enables him to cross that ocean. The disciple
shall realise the difference between the two and worship the preceptor sincerely.

93-94. The modes of worship of the preceptor are many. He can be given monetary
gifts. He can be physically served but the money shall be what is earned by
the disciple. Since every limb of the preceptor is a phallus from toe to
the head, massaging the feet, presenting him with sandals, bathing him, offering
food and money and similar rites shall be performed to gratify him.

95-96. Verily the worship of the preceptor is worship of Śiva, the supreme soul.
What remains after the preceptor has partaken of food shall be used by
the disciple. It will purify him. Just as Siva’s leaving of food can be taken by
the devotee of Śiva, so also the disciple can take the leavings of the preceptor.
Even food and water, O \tl{brāhmaṇas}.

97. Without the permission of the preceptor, anything taken is a theft. One
shall accept as one’s preceptor a person who knows many special things.

98-99. Freedom from ignorance is the goal. Only a specialist can achieve that.
In order to fulfil a task, or a sacred rite, obstacles must be warded off.
A rite performed without hindrances in the middle can be fruitful.
The subsidiary rites shall also be performed. Hence at the beginning of sacred
rites, an intelligent man shall adore Gaṇeśa.

100. An intelligent man must worship all deities in order to ward off all sorts
of hindrances. There are three types of hindrances. The first one,
the \tl{ādhyātmika} hindrance is the ailment of the body, whether it is a fever
or a tremor or other type of sickness.

101-106. The second type of hindrance is \tl{ādhibhautika} (Extraneous one of
a physical nature). The visitations of \tl{piśācas}, the outcome of ant-hills
\etc, falling of lizards and other insects, the advent of tortoise inside
the house, infesting of serpents, untimely flowering of trees, deliveries in
inauspicious hours and other things indicate some future misery. Hence these are
called \tl{ādhibhautika} hindrances. The third type of hindrance is
\tl{ādhidaivika} (Divine calamities). When lightning strikes, small pox, cholera,
plague, typhus fever and similar infectious diseases spread and bad awful dreams,
evil planets affecting the birth star or Rāśi (sign of the zodiac) occur, these
hindrances are called \tl{ādhidaivika}. In order to ward off these hindrances
and on occasions when one touches a corpse, a \tl{cāṇḍāla} or a fallen man and
goes inside without bathing, \tl{śānti yajña} shall be performed to remove
the evil effects.

107-109. The precincts of a temple, a cowshed, a sanctuary or one’s own
court-yard shall be selected for the performance of sacrifice. It shall be on a
raised platform at least two \tl{hastas} high. It shall be well decorated. Paddy
weighing a \tl{bhāra} shall be spread on the ground to make a large circle.
Diagrams of lotuses shall be made in the middle and in the eight quarters on
the border of the circle. A big pot round which a thread is tied, shall be
placed in the middle and eight other similar pots shall be placed in the eight
quarters. All of them shall be fumigated with guggulu.

110. In the eight pots bunches of mango leaves shall be placed with \tl{darbha}
grass. They shall be filled with water purified by mantras and five kinds of
articles.

111. Precious gems shall be put in the nine vessels, one in each. The sensible
devotee shall ask his preceptor to preside as a priest. The presiding priest
shall be accompanied by his wife. He shall be well-versed in the rituals.

112. Gold idols of the guardians of the quarters and Viṣṇu shall be put in
the different vessels. Viṣṇu shall be invoked and worshipped in the central
vessel.

113. The respective guardians of the different quarters shall be worshipped in
the vessels concerned, using the dative case after the name and ending with
\tl{namaḥ}.

114. The invocation shall be performed by the presiding priest. Along with
the \tl{ṛtviks} he shall repeat the mantras a hundred times.

115-116. At the end of the \tl{japas}, \tl{homa} shall be performed to the west
of the vessel. According to the time, place and convenience, the offerings in
the fire may be a \tl{crore}, a hundred thousand, a thousand, or hundred and
eight in number. It shall be performed for a single day, for nine days or for
forty days.

117. The sacrificial twigs shall be of \tl{śamī} tree if the rite is intended
for \tl{śānti} (suppression of evil effects) or of \tl{palāśa} tree if the rite
is intended for the acquisition of livelihood. Cooked rice and ghee shall also
be used. The offerings shall be made by repeating the names of the deities or
mantras.

118. The articles of worship used in the beginning shall be continued till
the end. At the conclusion, the \tl{puṇyāhavācana} shall be performed and
the holy water sprinkled over the different members of the family.

119. \tl{Brāhmaṇas}, as many in number as the number of offerings made, shall be
fed, O scholarly sages, the preceptor and the presiding priest shall partake of
sacrificial food alone.

120. The entire rite shall conclude after the worship of nine planets. A gem
along with monetary gifts shall be given to each of the \tl{ṛtviks}.

121-122. Different types of gifts shall be made to deserving persons, to boys
invested with sacred threads, to householders, sages, virgins, ladies and widows.
The materials used for the rite shall be given to the priest.

123. Yama is the presiding deity of all calamities, grave diseases \etc Hence to
gratify Yama \tl{kāladāna} shall be made.

124-125. A replica of Kāla (God of death) in the form of a man holding noose and
goad shall be made in gold using a hundred or ten \tl{niṣkas} (gold coins). This
shall be given as gift along with the sacrificial fee; gingelly seeds shall be
gifted for the sake of longevity.

126-127. Ghee or mirror shall be gifted for the sake of quelling ailments. Rich
men shall feed a thousand \tl{brāhmaṇas}. The poor shall feed a hundred
\tl{brāhmaṇas}. Indigent persons shall perform rites according to their capacity.
For the quiescence of evil spirits the great adoration of Bhairava shall be
performed.

128. At the conclusion, \tl{mahābhiṣeka} and \tl{naivedya} shall be offered to
Śiva. Then a public feeding of the \tl{brāhmaṇas} shall be held.

129. By performing sacrifice in this way there will be an alleviation of all
defects and evils. This \tl{śānti yajña} shall be performed every year in
the month of Phālguna.

130. In regard to evil dreams and ill omens this shall be performed instantly or
definitely within a month. When one is defiled by a great sin, the worship of
Bhairava shall be performed.

131. In regard to great diseases like leprosy \etc the vow shall first be taken
and the sacrifice performed later on. Indigent persons wanting in all these
things shall make gift of a lamp to the deity.

132. If incapable of even that, he shall take bath and make any gift. Or he
shall make obeisance to the Sun-god hundred and eight times repeating
the mantras.

133. A devotee shall perform prostrations and obeisance a thousand, ten thousand,
hundred thousand, or a \tl{crore} in number. All the deities are delighted by
the obeisance-sacrifice in this way.

134-135. The obeisance is performed with the prayer “O lord, Thou are great and
I am humble. My intellect is dedicated to Thee. A void thing does not appeal to
thee. I am no longer void. I am Thy slave now. Whatever vestige of egotism
remained in me has been dispelled on seeing Thee.”

136. \tl{Namaskāra}, a sacrifice of the soul, shall be performed according to
ability. Sacrificial food and betel leaves shall be offered to Śiva.

137. The devotee himself shall perform a hundred and eight circumambulations of
Śiva. Such circumambulations, a thousand, ten thousand, hundred thousand or a
\tl{crore} in number he shall cause to be performed through others.

138. All sins perish instantaneously at the circumambulations of Śiva. Sickness
is the root-cause of misery and sin is the cause of sickness.

139. Sins are said to be quelled by virtue. A sacred rite performed with Śiva in
view is capable of removing all sins.

140. Among the sacred rites of Śiva, the circumambulation leads the rest.
\tl{Praṇava} is in the form of \tl{japa} and circumambulation is a physical rite.

141. The pair of births and deaths constitutes the Illusory cycle. The Balipīṭha
of Śiva is symbolic of this Māyācakra.

142-143. Starting from pedestal the devotee shall make circumambulation half
the way and return to the pedestal [and move anticlockwise to the place where he
stopped before and returning to the pedestal make the circle complete]. This is
the procedure of circumambulation. When the birth takes place, the obeisance
which is the dedication of the soul prevents further birth.

144. The pair of births and deaths originates from the Māyā of Śiva. After such
a dedication the devotee is not born again.

145. As long as the body exists, the \tl{jīva} is dependent on activities and he
is spoken of as being in bondage. But when the three forms of the physical body
are under control it is called “Salvation” by the scholars.

146. Śiva, the primary cause of causes, is the Creator of Māyācakra. He wipes
off the Dvandva—birth and death—which originates from His Māyā.

147. The Dvandva is conceived and created by Śiva. It shall be dedicated to Him.
O scholars, it shall be known that circumambulation is highly pleasing to Śiva.

148. The circumambulation and obeisance of Śiva, the great soul and
the adoration performed with sixteen \tl{upacāras} accord all benefits.

149. There is no sin in the world which cannot be destroyed by circumambulation.
Hence one should dispel all sins by circumambulation alone.

150. A person observing worship of Śiva shall observe silence and perform one of
these—a sacred rite, penance, \tl{japa}, maintenance of the knowledge or
meditation. He shall observe truthfulness \etc

151. All sorts of riches, divine body, knowledge, removal of ignorance and
nearness to Śiva are the results of sacred rites \etc

152. The sacred rite yields the benefit by the performance. It removes
the darkness of ignorance. It wipes off future birth. By the achievement of
true knowledge, the miseries shall seem as if they did not exist at all.

153. The true devotee of Śiva shall observe the sacred rites \etc in accordance
with the place, time, physical ability, possession of wealth as befitting his
state.

154. The intelligent devotee shall take up his residence in a holy centre of
Śiva, desist from violence to living beings, without exposing himself to undue
strain, and spending only such wealth as he earns by legitimate means.

155. Even water sanctified by the five-syllabled mantra is conducive to
happiness like cooked food. Even the alms begged and acquired by an indigent
devotee is conducive to perfect knowledge.

156. Charitable food of a devotee of Śiva increases devotion to Śiva. Śivayogins
call such charitable food sacrificial offerings to Siva.

157. The devotee of Śiva shall always be scrupulous about the purity of his food,
wherever he stays and whatever means of sustenance he has. He shall observe
silence and shall not disclose the secret.

158. To the devotees he shall expound the greatness of Śiva. Only Śiva can know
the secret of Śivamantra. No one else.

159. The devotee of Śiva shall always resort to the phallic emblem of Śiva.
O \tl{brāhmaṇas}, one becomes Śiva by resorting to stationary phallic emblem.

160. By worshipping the mobile phallic image the liberation is certainly gradual.
Thus I have mentioned the achievable and the excellent means of achievement.

161. What has been mentioned formerly by Vyāsa and what has been heard by me
before, has been mentioned to you. Welfare attend ye all. May our devotion to
Śiva be stable and firm.

162. O scholars, whoever reads this chapter by Siva’s grace and whoever listens
to this always shall acquire the knowledge of Siva.
