\chapter{The Excellence of Listening and Deliberation}

The sages said:

1. O holy one, what is \tl{śravaṇa}? what is \tl{manana}? How is the \tl{kīrtana}
performed? Please expound these precisely.

Brahmā said:

2. The mind is fond of reasoning deliberation. The ability of the mind to ponder
and evaluate the corresponding efficacy of the worship, \tl{japa},
the attributes of Īśa, His form, His divine sports and multifarious names, is
the result of the benignant glance of Īśvara. Hence this steady continuance in
the act of deliberation is the most important of all the means.

3. By \tl{kīrtana} (glorification) is meant the clear expression of Śiva’s
exploits, attributes, forms, sports, names \etc in good taste by reciting
traditional lore, singing songs of praise even in mother tongue. It is the
middle one of the three means.

4. O wise men, the means of \tl{śravaṇa} famous in the world is the listening
to words concerning Śiva, in whatever manner, howsoever and wherever they are
produced with the same steady attention as in the sporting dalliance of women.

5. \tl{Śravaṇa} (listening) is effected when one associates with good men. Then
the \tl{kīrtana} of Paśupati becomes steady. In the end is the \tl{manana} which
is the most excellent. All these take place as a result of benevolent
surveillance of Lord Śiva.

Sūta said:

6. O saints, in the context of the elucidation of the greatness of the means,
I shall narrate an anecdote of former days for your sake. Please listen to them
attentively.

7. Long ago, my preceptor Vyāsa, the son of Sage Parāśara, performed penance on
the bank of the river Sarasvatī\footnote{Sarasvati. The Sarasvatī river was a
boundary of Brahmāvarta, the home of the early Aryans, and was to them, in all
likelihood, a sacred river as the Ganges has long been to their descendants. As
a river, it is lauded for the fertilizing and purifying powers of her waters,
and as the bestower of fertility, fatness and wealth. Dowson: \emph{Hindu
Mythology} p.284; also D.~C.~Sarkar, G.A. \emph{Medieval India} (Kern) p40.

This sacred river rising in the Sirmur hills of the Sivalik range in
the Himalayas, emerged into the plains in the Ambala district, Punjab.
Ultimately it fell into the Ghagger which bore the name Sarasvatī in ancient
times. Sanskrit literature speaks of its disappearance at Vināśana (near modern
Sirsa) in Kurukṣetra in the East Punjab.}with some mental agitation.

8. The divine sage Sanatkumāra who happened to go that way in an aerial chariot
resplendent like the sun, espied my preceptor.

9. Waking up from his meditation my preceptor saw the son of Brahmā. The sage
thereupon paid obeisance in a flutter and eagerness.

10. He offered \tl{arghya} and a seat befitting the divinity of the sage. Being
delighted, the divine sage spoke to my humble preceptor in words of great
profundity.

11. O sage, you must meditate upon the True object. The great lord Śiva can be
realised and seen. But wherefore do you perform the penance here unattended?

12-14. When Sanatkumāra addressed him thus, the sage Vyāsa clarified his purpose.
“By the favour of divine elders like you I have almost established the four ways
of virtue, wealth, love and salvation with due adherence to the Vedic path, in
the world. I have become a preceptor unto all. Still it is surprising that
the knowledge of the means of liberation has not dawned on me. I am performing
penance for the sake of salvation. But I do not know how it can be achieved.

15. O excellent brahmins, when thus requested by the sage Vyāsa, the competent
divine sage Sanatkumāra told him of the sure way of realising salvation.

16. It has already been mentioned that there are three means in conformity with
Vedic ideal \viz \tl{śravaṇa}, \tl{kīrtana} and the highly efficacious \tl{manana}
of Śiva.

17. Formerly, I too, confounded by other means performed a great penance on
the mountain Mandara\footnote{Mandara: a mountain in Hindu Mythology for being
used as a churning staff by the gods and demons on the occasion of
Samudra-Manthana appears to be an important hill comprising beautiful caves.
There is still a hill of this name in Banka Sub-division of Bhagalpur district
(Bihar). It is noted for the abundance of various metals as well as variety of
flora and fauna. It is stated to be a sacred mountain associated with Śiva.
\emph{Sk}. 5.2.4.23,26. There is another mountain of the same name in the Malaya
range which being an abode of gods and \tl{ṛṣis} has an \tl{āśrama} of Agastya.}.

18-19. At the bidding of Śiva, the divine attendant Nandikeśvara arrived there.
That sympathetic lord of \tl{gaṇas}, witness of all, lovingly told me about
the excellent means of salvation. \Viz \tl{śravaṇa}, \tl{kīrtana} and \tl{manana}
all in conformity with Vedic ideals.

20. Hence, O holy sage, as advised by Śiva these are the three means of
salvation. Please practise them.” He repeatedly advised Vyāsa thus.

21. After saying this to Vyāsa, the son of Brahmā mounted the aerial chariot
accompanied by his followers and returned to his splendid and auspicious region.

22-23. Thus, in brief, I have told you the ancient anecdote.

The sages said:

O Sūta, you have narrated \tl{śravaṇa} \etc—the three means of salvation. If
a person is unable to practise these three, what shall he do to achieve
liberation? What is that rite whereby salvation will be possible without stress
or strain?
