\subsection{Śiva’s forgiveness of Brahmā}

Nandikeśvara said:

1. Mahādeva then created a wonderful person, Bhairava, from the middle of his
brows to quell the pride of Brahmā.

2. This Bhairava knelt before the lord in the battle-field and said—“O lord,
what shall I do? Please give me your directives quickly.”

3. “Dear, here is Brahmā, the first deity of the universe. Worship him with your
sharp-pointed quick-moving sword.”

4. With one of his hands he caught hold of the tuft of Brahma’s fifth head that
was guilty of haughtily uttering a falsehood, and with the hands he furiously
shook his sword in order to cut it off.

5. Your father trembled like a plantain tree in a whirlwind, with his ornaments
scattered here and there, his cloth ruffled and loosened, the garland displaced,
the upper cloth hanging loose and the glossy tuft dishevelled, and fell at
the feet of Bhairava.

6. Meanwhile the sympathetic Acyuta desirous of saving Brahmā, shed tears over
the lotus-like feet of our lord and said with palms joined in reverence just
like a child lisping words of entreaty to its father.

Acyuta said:

7. O Lord, it was you who gave him five heads\footnote{Brahma’s five heads: When
the four faces of Brahmā became thwarted in their function because of Brahmā’s
erotic impulse, then out of his \tl{tapas} was produced a fifth head on the top
and that head was covered with matted locks. In image No 382 of Brahmā in
the Kushāna period at Mathura, the fifth head on the top is shown with
moustaches, beard and long locks, a feature which is only found in the Kuśāna
period from the first to the third century A.D. But later on, the fifth head was
eliminated and a new theory (contradicted by \tl{Śiva-purāṇa 1.8.8}) was devised
that Brahma’s head was clipped by Rudra. The fact was that the fifth head
corresponding to \tl{ākāśa} was taken to be invisible, being a symbol of his
unmanifest form (\tl{avyakta mūrti}) and that only the four others became
manifest. Agrawal: \emph{A Study}.} as a special symbol, long ago.
Hence please forgive him his first guilt. Please favour him.

8. The lord thus requested by Acyuta relented and in the presence of all
\tl{devas} asked Bhairava to desist from punishing Brahmā.

9. Then the lord turned to the deceitful Brahmā who bent down his neck and said
“O Brahmā, in order to extort honour from the people you assumed the role of
the lord in a roguish manner.

10-11. Hence you shall not be honoured, nor shall you have your own temple or festival.

Brahmā said:

O Lord, be pleased. O flourishing one, I consider this sparing of my head itself
a great blessing and a boon. Obeisance to Thee, the lord, the kinsman,
the originator of the universe, the forbearing, the forgiver of defects,
the benevolent one, wielding the mountain as his bow.

Īśvara said:

12. O child, the whole universe will be ruined if it loses the fear of a king.
Hence you mete out punishment to the guilty and bear the burden of administering
this universe.

13-14. I shall grant you another boon which is very difficult to get. In all
domestic and public sacrifices you will be the presiding deity. Even though
a sacrifice is complete with all the ancillary rites and offerings of monetary
gifts, it will be fruitless without you. Then the lord turned to the deceitful
\tl{ketaka} flower guilty of perjury and said:

15. “O you \tl{ketaka} flower, you are roguish and deceitful. Go away from here.
Hereafter I have no desire to include you in my worship.”

16. When the lord said thus, all the \tl{devas} shunned the very presence of
the flower.

Ketaka said:

17. Obeisance to Thee, O Lord, Your bidding will mean that my very birth is
fruitless. May the lord be pleased to make it fruitful by forgiving my sin.

18. Thy remembrance is reputed to quell all sins perpetrated consciously or
unconsciously. Now that I have seen Thee, how can the sin of uttering falsehood
sully me?

19-21. Thus entreated in the middle of the council the lord said—“It is not
proper for me to wear you. I am the lord and my words must stay true. My
attendants and followers shall wear thee. Hence thy birth shall be fruitful.
Of course in the canopies over my idol you can be used for decoration.” The lord
thus blessed the three—the flower \tl{ketaka}, Brahmā and Viṣṇu. He shone in
the assembly duly eulogised by the \tl{devas}.
