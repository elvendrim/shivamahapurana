\subsection{Śiva manifests himself as a column of fire in the battlefield}

Īśvara said:

1. Dear children, hail to ye. I hope the universe and the race of the deities,
under my suzerainty, flourish in their respective duties.

2. O gods, the fight between Brahmā and Viṣṇu is already known to me. This
agitation on your part is like a redundant speech.

3. Thus the consort of Ambā consoled the concourse of \tl{devas} with honeylike
speech sweetened with a smile in the manner of appeasing children.

4. In that very assembly the lord announced his desire to go to the battlefield
of Hari and Brahmā and accordingly issued His directive to a hundred of
the commanders of his attendants.

5-6. Different kinds of musical instruments were played to announce the start of
the journey of the Lord. The commanders of the attendants were in readiness
fully bedecked in their ornaments, seated in their respective vehicles. The lord,
consort of Ambikā, mounted the holy chariot shaped like Oṃkāra from front to
the back and embellished with five circular rings. He was accompanied by his
sons and \tl{gaṇas}. All the \tl{devas}, Indra and others, followed.

7. Honoured suitably by the display of banners of various colours, fans,
chowries, scattered flowers, music, dance and the instrument players, and
accompanied by the great goddess (Pārvatī), Paśupati (Śiva) went to
the battlefield with the whole army.

8. On espying the battle, the lord vanished in the firmament. The play of
the music stopped and the tumult of the \tl{gaṇas} subsided.

9. There in the battlefield Brahmā and Acyuta desirous of killing each other
were awaiting the result of the \tl{māheśvara} and the \tl{paśupata} weapons
hurled by them.

10-11. The flames emitted by the two weapons of Brahmā and Viṣṇu burned
the three worlds. On seeing this imminent untimely dissolution the bodiless form
of Śiva assumed the terrific form of a huge column of fire in their midst.

12. The two weapons of fiery flame potential enough to destroy the entire world
fell into the huge column of fire that manifested itself there instantaneously.

13. Seeing that auspicious wonderful phenomenon assuaging the weapons they asked
each other “What is this wonderful form?”

14. “What is this column of fire that has risen up? It is beyond the range of
senses. We have to find out its top and bottom.”

15. Jointly deciding like this, the two heroes proud of their prowess
immediately set about assiduously in their quest.

16-18. “Nothing will turn up if we are together”. Saying this, Viṣṇu assumed
the form of a Boar and went in search of the root. Brahmā in the form of a swan
went up in search of the top. Piercing through the netherworlds and going very
far below, Viṣṇu could not see the root of the fiery column. Utterly exhausted,
Viṣṇu in the form of a Boar returned to the former battle-ground.

19. Dear one, your father, Brahmā who went high up in the sky saw a certain
bunch of \tl{ketakī} flower of mysterious nature falling from above.

20-21. On seeing the mutual fight of Brahmā and Viṣṇu, lord Śiva laughed. When
his head shook, the \tl{ketakī} flower dropped down. Although it had been in its
downward course for many years, neither its fragrance nor its lustre had been
diminished even a bit. The flower had been intended to bless them.

22-23. (Brahmā said) “O lord of flowers, by whom had you been worn? Why do you
fall? I have come here to seek out the top, in the form of a swan.” (The flower
replied) “I am falling down from the middle of this primordial column that is
inscrutable. It has taken me a long time. Hence I do not see how you can see
the top.”

24-25. “Dear friend, hereafter you must do as I desire. In the presence of Viṣṇu
you must say like this. O Acyuta, the top of the column has been seen by Brahmā.
I am the witness for the same.” Saying this he bowed to the \tl{ketakī} flower
again and again. Even falsehood is recommended in times of danger. So say
the authoritative texts.

26. (Returning to the original place) on seeing Viṣṇu there, utterly exhausted
and lacking pleasure, Brahmā danced with joy. Viṣṇu, in the manner of a eunuch
admitting his inability (to a woman), told him the truth (that he could not see
the bottom). But Brahmā told Viṣṇu like this.

27-28. “O Hari, the top of this column has been seen by me. This \tl{ketakī}
flower is my witness.” The \tl{ketaka} flower repeated the falsehood endorsing
the words of Brahmā in his presence. Hari, taking it to be true, made obeisance
to Brahmā. He worshipped Brahmā with all the sixteen means of service and
homage\footnote{\tl{Ṣoḍaśopacāra}: The sixteen acts of homage to a deity are
mentioned in \tl{Śiva-purāṇa} 2.25-29. They are differently enumerated
elsewhere:

\begin{shloka}\itshape
  āsanaṃ svāgataṃ pādyamarghyamācamanīyakam \\
  madhuparkācamanasnānaṃ vasanābharaṇāni ca \\
  gandhapuṣpe dhūpadīpau naivedyaṃ vandanaṃ tathā
\end{shloka}

\tl{Tantrasāra} enumerates 64 \tl{upacāras}.}.

29. The Lord taking up a visible form in order to chastise Brahmā who practised
trickery, came out of the column of fire. On seeing the lord, Viṣṇu stood up and
with his hands shaking with fear caught hold of the lord’s feet.

30. It is out of ignorance and delusion about you whose body is without a
beginning or an end that we indulged in this quest prompted by our own desire.
Hence O, Sympathetic Being, forgive us for our fault. In fact, it is but another
form of your divine sport.

Īśvara said:

31. “O dear Hari, I am pleased with you, because you strictly adhered to truth
in spite of your desire to be a lord. Hence among the general public you will
have a footing equal to mine. You will be honoured too likewise.

32. Hereafter you will be separate from me having separate temple, installation
of idols, festivals and worship.”

33. Thus, formerly, the lord was delighted by the truthfulness of Hari and
offered him a footing equal to his own even as the assembly of the \tl{devas}
was witnessing the same.
