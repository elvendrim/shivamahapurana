\subsection{The five-fold activities (\tl{pañcakṛtya}) and the \tl{Oṃkāra}-mantra}

\emph{Summary}: The Evanescence of Śiva after expounding the five-fold duties
(\tl{pañcakṛtya}) and the Oṃkāra mantra to Brahmā and Viṣṇu.

Brahmā and Viṣṇu said:

1. O Lord, please tell us the characteristic feature of the five-fold duties
beginning with creation.

Śiva said:

I shall tell you the great secret of the five-fold duties, out of compassion
for you.

2. O Brahmā and Viṣṇu, the permanent cycle of the five-fold duties consists of
creation, maintenance, annihilation, concealment, and blessing.

3. \tl{Sarga} is the creation of the world; \tl{sthiti} is its maintenance;
\tl{saṃhāra} is the annihilation; \tl{tirobhāva} is the removal and concealment;

4. Liberation (from the cycle of birth and death) is blessing. These five are my
activities but are carried on by others silently as in the case of the statue
at the Portal.

5. The first four activities concern the evolution of the world and the fifth
one is the cause of salvation. All these constitute my prerogatives.

6-8. These activities are observed in the five elements by devotees—\tl{sarga}
(creation) in the Earth, \tl{sthiti} (maintenance) in the waters, \tl{saṃhāra}
(annihilation) in the fire, \tl{tirobhāva} (concealment) in the wind and
\tl{anugraha} (liberation, the blessed state) in the firmament. Everything is
created by the Earth; everything flourishes by virtue of the waters; everything
is urged by the fire, everything is removed by the wind and everything is
blessed by the firmament. Thus intelligent men must know the same.

9. In order to look after these five-fold activities (\tl{pañcakṛtya}) I have
five faces, four in the four quarters and the fifth in the middle.

10. O sons, in view of your austerities you two have received the first two
activities:—creation and maintenance. You have gratified me and are blessed
therefore.

11. Similarly, the other two activities (annihilation and concealment) have been
assigned to Rudra and Maheśa. The fifth one of \tl{anugraha} (liberation) cannot
be taken up by any other.

12. All this previous arrangement has been forgotten by both of you due to lapse
of time, not so by Rudra and Maheśa.

13. I have assigned them my equality in form, dress, activity, vehicle, seat,
weapons \etc

14. O dear sons, your delusion was the result of your not meditating upon me. If
you had retained my knowledge you would not have embibed this false pride of
being Maheśa yourselves.

15. Hence, hereafter, both of you shall start reciting the mantra \tl{Oṃkāra} to
acquire knowledge of me. It shall quell your false pride as well.

16. I have taught this great auspicious mantra. \tl{Oṃkāra} came out of my mouth.
Originally it indicated me.

17. It is the indicator and I am the indicated. This mantra is identical with me.
The repetition of this mantra is verily my repeated remembrance.

18-19. The syllable “A” came first from northern face; the syllable “U” from
the western; the syllable “M” from the southern and the Bindu (dot) from
the eastern face. The Nāda (mystical sound) came from the middle face. Thus
the complete set cropped up in five-fold form. Then all of them united in
the syllable of “Om”.

20. The two sets of created beings—\tl{nāma} (name) and \tl{rūpa} (form) are
pervaded by this mantra. It indicates Śiva and Śakti.

21. From this also is born the five-syllabled mantra (\tl{namaśśivāya}). It
indicates all knowledge. The syllables “NA” \etc follow the order of
the syllables “A” \etc

22. From the five-syllabled mantra the five mothers were born.
The \tl{Śiromantra} is born of that. The three-footed \tl{Gāyatrī} also came out
of the four faces.

23. The entire set of Vedas and crores of mantras were born of that. Different
things are achieved through different mantras but everything is achieved through
\tl{Oṃkāra} alone.

24. By this root-mantra, the very enjoyment as well as salvation is achieved.
All the royal mantras are auspicious and directly accord enjoyment.

Nandikeśvara said:

25. The lord in the company of his consort Ambikā, assumed the role of
the preceptor for both of them. He screened them and placed his lotus-like hand
on their heads as they faced the north and slowly taught them the great mantra.

26-27. The two disciples received the mantra by repeating it thrice, along with
the requisite \tl{yantra} and \tl{tantra} duly expounded. By way of fees,
the disciples dedicated themselves. Thereafter standing near him with hands
clasped in reverence they addressed the lord, the preceptor of the universe.

Brahmā and Viṣṇu said:

\begin{shloka}\itshape
namo niṣkalarūpāya namo niṣkalatejase\\
namaḥ sakalanāthāya namaste sakalātmane\\

namaḥ praṇavavācyāya namaḥ praṇavaliṃgine\\
namaḥ sṛṣṭyādikartre ca namaḥ paṃcamukhāyate\\

paṃcabrahmasvarūpāya paṃca kṛtyāyate namaḥ\\
ātmane brahmaṇe tubhyamanaṃtaguṇaśaktaye\\

sakalākalarūpāya śaṃbhave gurave namaḥ\\
iti stutvā guruṃ padyairbrahmā viṣṇuśca nematuḥ (28-31)
\end{shloka}

28-31. (The prayer): Obeisance to Thee of the bodiless form. Obeisance to Thee
of the formless lustre. Obeisance to Thee the lord of everything. Obeisance to
Thee the soul of everything or of the embodied form. Obeisance to Thee stated by
the \tl{Praṇava}. Obeisance to Thee having \tl{Praṇava} as Thy symbol. Obeisance
to Thee the author of creation \etc Obeisance to Thee of five faces. Obeisance
to Thee identical with Pañcabrahma form. Obeisance to Thee of five-fold
functions. Obeisance to Thee the Ātman, the Brahman, of endless attributes and
power. Obeisance to Śiva the preceptor, possessed of both embodied and bodiless
forms.”

After eulogising the preceptor in verses Brahmā and Viṣṇu bowed to him.

Īśvara said:

32. O dear sons, the truthful extract of everything has been narrated to you
with demonstration. You shall recite as directed by the Goddess this Om mantra
which is identical with me.

33. Your knowledge shall be stabilised. Permanent fortune shall stand by you.
On the \tl{caturdaśī} day and on the day with Ārdrā star, the recital of this
mantra will give you everlasting efficacy.

34-35. The recital of this mantra at the time when the transit of the sun is in
the Ārdrā star is million-fold efficacious. In the context of worship, \tl{homa}
and \tl{tarpaṇa}, the last quarter of the star Mṛgaśiras and the first quarter
of Punarvasu must always be considered on a par with Ārdrā. The Vision is to be
had at early dawn and within three \tl{muhūrtas} (two hours twenty-four minutes)
thereafter.

36. \tl{Caturdaśī} is to be taken when it continues up to midnight. If it is
only upto the early part of the night and joined with another thereafter, it is
also recommended.

37. Although I consider the phallic and the embodied form to be equal,
the phallic form is excellent for those who worship. Hence for those who seek
salvation the latter is preferable to the former.

38-39. The others too shall install the phallic form with \tl{Oṃkāra} mantra and
the embodied form with the five-syllabled mantra, with excellent articles of
worship and adore with due homage. It will be easy for them to attain my region.

Having thus instructed His disciples Śiva vanished there itself.
