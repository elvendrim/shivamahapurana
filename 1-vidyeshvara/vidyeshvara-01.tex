\subsection{The Doubt of the Sages}

\begin{shloka}\itshape
  ādyantamaṃgalamajātasamānabhāvamāryaṃ tamīśamajarāmaramātmadevam |\\
  paṃcānanaṃ prabalapaṃcavinodaśīlaṃ saṃbhāvaye manasiśaṃkaramambikeśam |
\end{shloka}

\begin{quote}
  \emph{I meditate on Śiva, the lord of Ambikā (Pārvatī), auspicious from the
        beginning to the end, having no parallel, the noble lord, the unaging
        and the undying, the lord of ātmans, the five-faced and the dispeller
        of the five powerful sins.}
\end{quote}

Vyāsa said:\footnote{Vyāsa: The title is applied to Vedavyāsa, the arranger
of the Vedas, the compiler of the \tl{Mahābhārata}, the founder of the Vedānta
philosophy and the arranger of the \tl{purāṇas}. Dowson doubts the identity of
these different arrangers. Vyāsa is also called Kṛṣṇa-Dvaipayana. From his
complexion he received the name Krṣṇa and from his birth place he was called
Dvaipāyaṇa.}

1-2. Sages of edified souls, engaged in truthful rites, powerful and blessed,
performed a great sacrifice at the confluence of Gaṅgā and Kālindī (Yamunā) in
the most sacred city of Prayāga\footnote{Prayāga is a celebrated place of
pilgrimage at the confluence of the Ganges and Jumna in the Naimiṣa forest
(\tl{Śiva-purāṇa, Vidyeśvara-saṃhitā} 1.4). It is situated on the northern bank
of the Ganges (\tl{Skanda-purāṇa} 2.2.12.36). The name ‘Prayāga’ is recorded by
Hwen Thsang in the seventh century and is as old as the reign of Aśoka who set
up the stone pillar about 235 BC. The Gupta emperors regarded the confluence at
Prayāga as the visible symbol of Madhyadeśa.}, a great holy centre, the path
that leads to Brahmaloka\footnote{Brahmaloka, also called Satyaloka, is the
abode of Brahmā.}.

3. On hearing that a sacrifice was being performed there, the disciple of Vyāsa,
the great sage Sūta, an excellent scholar in the \tl{purāṇas}, arrived there to
see the sages.

4. The sages were delighted on seeing him and received him with due hospitality
and adoration.

5. The due adoration being completed, the noble sages, being highly pleased,
addressed him in all humility with their palms joined in reverence.

6. O Romaharṣaṇa\footnote{Romaharṣaṇa or Lomaharṣaṇa was one of the five
disciples (the other four being Paila, Vaiśampāyana, Jaimini and Sumantu) to
whom Vyāsa taught the \tl{purāna} which he constructed out of ancient material.
Pargiter: ‘\emph{Ancient Indian Historical Tradition}’ (Pargiter). Ch.11.},
the omniscient, by thy weighty fortune, the entire Purāṇic lore, pregnant in its
meaningful content, has been secured by thee from Vyāsa.

7. Hence thou art the receptacle of wonder-inspiring stories, even as the vast
ocean is the storehouse of gems of great worth.

8. There is nothing in the three worlds that is not known to thee, of the past,
present and the future.

9. It is our great fortune that thou thyself hast come to pay a visit to us.
Hence it is not proper on thy part to return without doing us a favour.

10. It is true that we have already listened to the explanation of the auspicious
and the inauspicious. But we are not content. We yearn to hear more and more.

11. Now, O Sūta of good mentality, we have only one point to be clarified. If
thou dost desire to bless us, please explain the same, though it be the secret
of secrets.

12. At the advent of the terrible age of Kali men have become devoid of merits.
They are engaged in evil ways of life. They have turned their faces from
truthful avocations.

13. They are engaged in caluminating others. They covet other men’s wealth.
Their attention is diverted to other men’s wives. Injuring others has become
their chief aim.

14. They view the physical body as the soul, deluded as they are; they are
atheists of mere brutish sense; they hate their parents; their wives are
goddesses unto them; they are slaves to lust.

15. Brahmins are in the clutches of greed, they sell Vedas for livelihood; they
acquire learning as a means of earning money; they are deluded by their false
pride.

16. They have forsaken the duties of their own castes; they have almost become
swindlers of others; they do not offer \tl{sandhyā} prayers thrice a day; they
are deprived of Vedic enlightenment.

17. They are ruthless; they make much of their little knowledge; they have
discarded many of their rites and good conduct of life; they have taken to
agriculture as their profession; cruelty has become second nature to them; their
ideas have become dirty and defiled.

18. Similarly the \tl{kṣatriyas} also have discarded their duties; they
associate with evil men; they indulge in sinful activities; vice and debauchery
have become their main aim in life.

19. They have ceased to be valorous; they never take interest in virtuous
warfare; they flee from the battlefield; they follow the mean tactics of thieves
and \tl{śūdras}; they are mentally enslaved by base passions.

20. They have eschewed the practice of miraculous weapons; they never care to
protect cows and brahmins; they no longer consider it their duty to protect
those who seek refuge in them; they always indulge in brutish sexual dalliance
with their damsels.

21. The good virtue of protecting their subjects they have thrown over-board;
they strictly adhere to sensual enjoyment; they are wicked annihilators of their
own people; they rejoice in the harassment of all living beings.

22. \tl{Vaiśyas} too no longer perform holy rites; they have cast off their
traditional virtue; they have taken to crooked ways to earn more and more; they
are now notorious for their malpractices with the weighing balance.

23. They are no longer devoted to preceptors, gods and brahmins; their intellect
has become distorted; miserly and tight-fisted they no longer feed the brahmins.

24. They take delight in being the paramours of beautiful women; squalid and
filthy in their ideas and deluded by cupidity they have lost clear thinking;
they have abandoned their zeal for Pñrta and other holy rites such as digging
wells, tanks, planting trees and parks.

25. Similarly most of the \tl{śūdras} have become depraved. Some of them show
their interest in leading the life of brahmins with shining forms and features;
they too in the confusion of their minds have abandoned their traditional
practices.

26. In their eagerness to appropriate a brahmanical splendour they frequently
perform penances. They cause infantile and premature deaths by their chanting
of mantras.

27. They worship the \tl{śālagrāma} stone and other things; they evince some
interest in \tl{homas} too but in their thoughts and actions they are crooked
and antagonistic; they calumniate the brahmins.

28. Rich people indulge in misdeeds; learned people take perpetual delight in
disputations; those who conduct discourses in holy narratives and expound
virtuous rites of worship, themselves abandon virtuous practice of the same.

29. Haughty persons assume the features of noble kings; those who liberally
give, do so with a lot of fuss and haughtiness thinking themselves to be great
lords and treating the brahmins and others as their servants.

30. Devoid of the strict observance of their traditional duties and virtues,
the foolish people have brought about an admixture of various castes. Cruel in
thought and obsessed by false prestiges, people have discarded the four-fold
system of social classification.

31. Deluded people, wrongly considering themselves high-born, perform certain
good rites which result only in the upset of the caste-order and down-fall of
all people.

32. Women too generally misbehave and err; they slight their husbands; they are
inimical to their fathers-in-law; fearlessly they pursue their nefarious
activities.

33. They indulge in foul coquettish gestures; they are carried away by amorous
dispositions; their conduct is bad; they pursue illicit connections with
paramours; they turn away from their own husbands.

34. As for sons, they are invariably wicked without any filial affection; they
take lessons in ignorant activities and succumb to various ailments.

35. O Sūta, how can these deluded people who have abandoned their traditional
virtues get salvation here and hereafter.

36. Hence our minds are always agitated. Indeed there is no virtue equal to
helping others.

37. Since thou art conversant with the essentials of all tenets, please tell us
the easiest remedy for the immediate destruction of the sins of these people.

Vyāsa said:

38. On hearing these words of the sages of sanctified souls Sūta thought of Śiva
and told them thus.
