\subsection{The Proclamation of Śiva as Maheśvara}

Nandikeśvara said:

1. In the mean time Brahmā and Viṣṇu had been standing silently on either side
of the lord with the palms joined in reverence.

2. Then they installed the lord with all the members of His family on a splendid
seat and worshipped Him with all holy personal things.

3-6. The personal things constitute those natural things of long and short
duration such as necklaces, anklets, bracelets coronets, ear-rings, sacrificial
threads, upper cloth of lace border, garlands, silk cloth chokers, rings,
flowers, betel leaves, camphor, sandal paste, aguru unguents, incense, lamps,
white umbrella, fans, banners, chowries and other divine offerings whose
greatness cannot be expressed or even thought of. Both of them adored the lord
with all these things worthy of the lord and inaccessible to \tl{paśu}
(the animal \ie the individual soul). All excellent things are worthy of
the lord, O brahmin.

7. In order to set up a precedence the delighted lord handed over all those
articles to the attendants assembled according to the order of priority.

8-10. The bustle of those who came to receive them was too much. It was there
that Brahmā and Viṣṇu adored Śaṅkara at first. When they stood there humbly,
the gratified lord spoke smilingly heightening their devotion.

Īśvara said:

Dear children, I am delighted at your worship on this holy day. Henceforth this
day will be famous as “Śivarātri” the holiest of holy days pleasing to me.

11. He who performs the worship of my phallic emblem and the embodied image on
this day will be competent, to perform the task of creation and the maintenance
\etc of the universe.

12. The devotee shall observe fast on \tl{śivarātrī}, both during the day and
the night. He shall perfectly restrain his sense-organs. He shall adore (with
flowers) to the extent of his strength. He shall not deceive any one.

13. By the worship on \tl{śivarātrī} day the devotee attains that fruit which
usually accrues to one who continuously worships me for a year.

14. This is the time when the virtue of devotion to me increases like the tide
in the ocean at the rise of the moon. Festivities like the installation of my
idols \etc on that day are very auspicious.

15. The day on which I manifested myself in the form of a column of fire is
the Ārdrā star in the month of Mārgaśīrṣa (November-December), O children.

16. He who sees me on the day of Ārdrā star in the month of Mārgaśīrṣa in
the company of Umā and worships my phallic emblem or embodied image is dearer
to me than even Guha (Kārtikeya).

17. On that auspicious day the vision alone accords ample results. If he
worships too, the result cannot be adequately described.

18. Since I manifested myself in the form of phallic emblem in the field of
battle, this place will be known as Liṅgasthāna.

19. O sons, this column without root or top will henceforth be diminutive in
size for the sake of the vision and worship of the world.

20. The phallic emblem confers enjoyment. It is the only means of worldly
enjoyment and salvation. Viewed, touched or meditated upon, it wards off all
future births of the living beings.

21. Since the phallic emblem rose high resembling a mountain of fire, this shall
be famous as Ruddy (Aruṇa) mountain\footnote{Aruṇācala: The Aruṇa mountain lies
to the west of Kailāsa and is the abode of Śiva (\tl{Vāyu-purāṇa} 47.17-18;
\tl{Brahma-purāṇa} 2.18.18; Sk. 3.59-61; also Kern: ‘\emph{Medieval India}’
(Kern). p.3; See Awasthi: \emph{Studies in Skanda Purāṇa} p.54.}

22. Many holy centres will spring up here. A residence or death in this holy
place ensures liberation.

23. The celebration of chariot festivals, the congregation of devotees,
the presentation of ordinary as well as sacrificial gifts and offering of
prayers at this place shall be millionfold efficacious.

24. Of all my sectors this sector shall be the greatest. A mere remembrance of
me at this place shall accord salvation to all souls.

25. Hence this sector shall be greater than all other sectors, very auspicious,
full of all sorts of welfare and according salvation to everyone.

26-27. Worshipping me in my supreme phallic form at this place and performing
the other sacred rites shall accord the five types of salvation—\tl{sālokya,
sāmīpya, sārūpya, sārṣṭi} and \tl{sāyujya}. May all of you achieve all your
cherished desires.

Nandikeśvara said:

28-29. Thus blessing Brahmā and Viṣṇu who had been made humble, the lord
resuscitated by His nectar-like power all the soldiers of the two deities that
had been killed in the battle before and spoke to them in order to remove their
foolishness and mutual enmity.

30. I have two forms: the manifest and the unmanifest. No one else has these two
forms. Hence all else are non-Īśvaras.

31-32. Dear sons, first in the form of the column and afterwards in this
embodied form I have expounded to you my formless Brahma-hood, and embodied
Īśa-hood. These two are present only in me and not in anyone else. Hence no one
else, not even you too can claim Īśatva (Īśa-hood).

33. It is out of your ignorance of this fact that you were swept away by your
false prestige and pride of being Iśa, surprising as it is. I rose up in
the middle of the battle-field for quelling the same.

34. Cast off your false pride. Fix your thought in me as your lord. It is out of
my favour that all the objects in the world are illuminated.

35. The statement of the preceptor is the reminder and the authority on all
occasions. This secret truth of Brahman I am revealing to you out of love.

36. I am the supreme Brahman. My form is both manifest and unmanifest in view of
my Brahma-hood and Īśvaratva. My duty is blessing \etc

37. O Brahmā and Viṣṇu, I am Brahman because of \tl{bṛhatva} (huge size) and
\tl{bṛṃhaṇatva} (causing to grow). O children, similarly I am Ātman due to
\tl{samatva} (equality) and \tl{vyāpakatva} (pervasiveness).

38-39. All others are \tl{anātmans}, individual souls undoubtedly. There are
five activities\footnote{In respect of creation, \tl{Śiva-purāṇa} speaks of
different five activities in the \tl{Vāyavīya-saṃhitā} g. 4-5.} in respect of
the universe beginning with \tl{anugraha} (liberation)\footnote{The text reads
‘\tl{anugrahādyaṃ sargāntam}’ \ie beginning with liberation and ending with
creation. But correctly it should be \tl{anugrahāntaṃ}. \tl{Sargādyam}’ \ie
beginning with creation and ending with liberation. The correct process of
activities is mentioned in the following chapter, verses 3-5.} and ending with
\tl{sarga} (creation). Therefore these activities devolve on me because I am
Īśa and not on anyone else. It is to make my \tl{brahmatva} understood that my
phallic emblem rose up.

40. In order to clarify my \tl{īśatva}, unknown hitherto, I have manifested
myself immediately in the embodied form of Īśa.

41. The \tl{iśatva} in me is to be known as the embodied form and this symbolic
column is indicative of my \tl{brahmatva}.

42. Since it has all the characteristic features of my phallic emblem, it shall
be my symbol. O sons, you shall worship it every day.

43. The phallic symbol and the symbolised Śiva are non-different. Hence this
phallic emblem is identical with me. It brings devotees quite near to me. It is
worthy of worship therefore.

44. O dear sons, if phallic emblem of this sort in installed I can be considered
installed, though my idol is not installed.

45. The result of installing the phallic emblem is the attainment of similarity
with me. If a second phallic emblem is installed, the result is union with me.

46. The installation of the phallic emblem is primary and that of embodied idol
is secondary. A temple with the embodied idol of Śiva is unfructuous if it has
no phallic image.
