\subsection{Battle between Brahmā and Viṣṇu}

\emph{Summary}: The journey to Kailāśa of the \tl{devas} terrified by the use of
the \tl{pāśupata} weapon in the fight between Brahmā and Viṣṇu who vied with
each other maintaining that each of them is the Lord himself.

Nandikeśvara said:

1. Once, long ago, O foremost among \tl{yogins}, Viṣṇu was having his nap on his
serpent-couch. He was surrounded by the goddess of fortune and his attendants.

2. Brahmā, the foremost among the Vedic scholars chanced to come there. He asked
the lotus-eyed handsome Viṣṇu who was lying there.

3. Who are you lying here like a haughty person even after seeing me? Get up,
O dear, and see me who am your lord. I have come here.

4. Expiatory rites are ordained for that spiteful wretch who behaves like a
haughty fool at the visit of an honourable elderly person.

5. On hearing these words Viṣṇu was angry. But assuming a calm exterior he said
—“O dear, Hail thee. Welcome. Please sit on this couch. How is it that thy face
is agitated and thy eyes look curious?

Brahmā said:

6. Dear Viṣṇu, know me to have come with the speed of the Time. I am to be
honoured greatly. O dear one, I am the protector of the world, Grandfather,
your protector as well.

7. O dear one, the whole universe is situated within me but your way of thinking
is like that of a thief. You are born of the lotus sprung from my navel-region.
You are my son. Your words are futile therefore.

Nandikeśvara said:

8-9. Arguing with each other like this, saying that each is better than
the other and claiming to be the lord, they got ready to fight, like two foolish
goats, desirous of killing each other.

10. The two heroic deities, seated on their respective vehicles—the Swan and
the Garuḍa, fought together. The attendants of Brahmā and Viṣṇu also came into
clash.

11. In the meantime the different groups of \tl{devas} moving about in aerial
chariots came there to witness the wonderful fight.

12-18. Witnessing from the heaven they scattered flowers everywhere.
The Garuḍa-vehicled deity (of Viṣṇu) became infuriated and discharged unbearable
arrows and many kinds of weapons on the chest of Brahmā. The infuriated Brahmā
also hurled many arrows of fiery fury and different kinds of weapons on Viṣṇu.
The \tl{devas} commented on this wondrous fight and were agitated much, Viṣṇu
in his great fury and mental agitation breathed hard and discharged
the \tl{māheśvara} weapon over Brahmā. Annoyed at this, Brahmā aimed
the terrible \tl{pāśupata} weapon at the chest of Viṣṇu. The weapon rising high
in the sky blazing like ten thousand suns, with thousands of terrible pointed
spikes roared awfully like a gust of wind. These two weapons of Brahmā and Viṣṇu
thus faced each other in a terrible clash.

19. Such was the mutual fight between Brahmā and Viṣṇu. Then, O dear,
the \tl{devas} in their helpless agitation and vexation talked among themselves
as people do at the time of war between their monarchs.

20-22. The three-pointed-trident-bearing deity, the supreme Brahman, (\ie Śiva)
is the cause of creation, maintenance, annihilation, concealment and blessing.
Without His corroboration even a blade of grass cannot be split by any
individual anywhere. Thinking thus in their fright they desired to go to Śiva’s
abode and accordingly came to the summit of Kailāsa\footnote{Kailāsa: It is said
to be the centre of the Himālaya region, \tl{Matsya-purāṇa} Ch. 121; it is
identified with a peak of the Hemakūta mountain: S. M. Ali. \emph{The Geography
of the Purāṇas} p.57-58. It is called Śiva-parvata and Gaṇa-parvata and is
situated to the north of Mānasarovara. Sk. 1.2.8.15; 1.3.44.14; 2.1.5.76.} where
the moon-crested God resided.

23. On seeing that region of Parameśvara in the shape of Oṃkāra they bent their
heads down in reverence and entered the palace.

24. There they saw the supreme leader of the \tl{devas} brilliantly shining on
the gem-set seat in the company of Umā on an altar in the middle of
the council-chamber.

25. His right leg was kept over the knee of the left; his lotus-like hands were
placed over the legs; his attendants were all round him. He had all good
characteristic features.

26. He was being fanned by the specialists in that art—ladies of pointed
attention. The Vedas were extolling Him. The lord was blessing every one.

27. On seeing the lord thus, the \tl{devas} shed tears of joy\footnote{\tl{Daṇḍa
-praṇāma}: It is the same as the \tl{aṣṭāṅgapraṇāma} which is performed by
prostration of the eight parts of the body; the eight parts being the hands,
breast, forehead, eyes, throat and the middle of the back.} O dear one,
the hosts of \tl{devas} knelt down even from a great distance.

28. The lord, on seeing the \tl{devas}, beckoned them to him through his
attendants. Then causing the delight of the \tl{devas}, the crest-jewel of
\tl{devas} (\ie Śiva), addressed them gravely with sweet auspicious words.
