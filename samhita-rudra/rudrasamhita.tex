%!TEX TS-program = XeLaTeX
%!TEX encoding = UTF-8 Unicode
%!TEX builder = latexmk
\documentclass{purana}

\title{Śiva Mahāpurāṇa}
\subtitle{Part Two\\ Rudrasaṃhitā}
\author{Kṛṣṇa Dvaipāyana Vyāsa}
\date{26.12.2020}
\subject{Śiva Mahāpurāṇa}
\keywords{purana,rudra,samhita,shiva,shaivism,bhakti,cosmology,yoga,tirtha}

\begin{document}
\coverpage

\tableofcontents

% ------------- 2.1. Rudrasaṃhitā: Sṛṣṭi-khaṇḍa ------------------------------%
\part{Sṛṣṭi-khaṇḍa}
\chapter{The inquiry of the sages}

\begin{shloka}\itshape
  viśvodbhavasthitilayādiṣu hetumekaṃ gaurīpartividitatattvamanantakīrtim\\ māyāśrayamvigatamāyamaciṃtyarūpambodhasvarūpamamalaṃ hi śivannamāmi
\end{shloka}

1. I bow to Śiva the consort of Gaurī, the sole cause of the origin, sustenance,
dissolution of the universe, who has understood the reality, who is of endless
renown, who is the support of Māyā but is free from its influence, whose form is
incomprehensible, who is unsullied and who is perfect knowledge itself.

\begin{shloka}\itshape
  vande śivantamprakṛteranādimpraśāntamekampuruṣottamaṃ hi\\
  svamāyayā kṛtsnamidaṃ hi sṛṣṭvā nabhovadantarbahirāsthito yaḥ
\end{shloka}

2. I salute Śiva who is prior to Prakṛti, who is calm and tranquil, the only
excellent Puruṣa, who has created this visible universe and who stays both
within and without like ether.

\begin{shloka}\itshape
  vandetarasthaṃ nijagūḍharūpaṃ śivaṃsvatassraṣṭumidamvicaṣṭe\\
  jaganti nityamparito bhramaṃti yatsannidhau cumbakalohavattam
\end{shloka}

3. I salute Śiva, of unmanifest form, who having extended himself by way of
creation stands in the middle of it while the worlds move around him like iron
filings round the magnet.

Vyāsa said:

4. I describe this after bowing to Śambhu, the father of the universe, Śivā
the mother of the universe and Gaṇādhīśa their son.

5. Once Śaunaka and other sages living in Naimiṣa forest asked Sūta with full
devotion.

The sages said:

6. The good and auspicious story of \tl{Vidyeśvarasaṃhitā} has been heard by us.
This first delightful compendium, “On the achievable and the means of
achievement” is lovingly disposed to the devotees.

7. Sūta, O blessed Sūta, live long. Be happy. You will please narrate to us,
O dear, the great anecdotes of Śiva.

8. O sinless one, drinking the nectar of knowledge poured out from your
lotus-mouth we are never satiated. Hence we would like to inquire of you
something more.

9. O omniscient one, by the favour of Vyāsa you have realised contentment. There
is nothing not known to you whether of the past, present or future.

10. In return for your excellent devotion you have gained the great favour of
your preceptor Vyāsa. You have understood everything. You have made your life
highly noble and purposeful.

11. Now, O wise one, please explain the excellent form of Śiva. Please narrate
the divine anecdote of Śiva and Pārvatī without omitting anything.

12. Maheśvara is \tl{aguṇa} (free from attributes). How does He take up
the \tl{saguṇa} form in the world? We do not know the true nature of Śiva,
despite our great deliberation.

13. Before the origin of creation how does lord Śiva maintain His form? In
the midst of creation how does He maintain His sport?

14. How does lord Maheśvara stand at the moment of dissolution? How is Śañkara
who blesses the world with happiness propitiated?

15. What benefit does the great Lord confer when He is pleased with His own
devotees and others? Please tell us.

16. We have heard that the lord becomes pleased instantaneously. The merciful
lord is unable to bear the stress and strain that His devotee undergoes.

17. The three deities Brahmā, Viṣṇu and Maheśa are born of Śiva. Among them
Maheśa when he has all the substrata of elements is Śiva himself as distinct
from Maheśa\footnote{According to this statement Brahmā, Viṣṇu, Maheśa are
the three forms of Śiva. In the \tl{Kūrma Purāṇa} (2.37.70-71) there occurs a
slightly modified version: Agni (\tl{tamas}), Brahmā (\tl{rajas}) and Viṣṇu
(\tl{sattva}) are the three forms of Rudra while another form, full and
attributeless is Śiva himself.}.

18. Please explain His manifestation and tell us about His various activities.
Please tell us about the birth of Umā and her marriage too, O lord.

19. Their domestic life and their divine sports shall also be narrated to us.
O sinless one, please tell us all about it and anything else that shall be told.

Vyāsa said:

20. Being thus requested Sūta was delighted. Remembering the lotus-like feet of
Śiva he replied to the sages.

Sūta said:

21. O lordly sages, what you have asked for is very nice. You are all blessed
inasmuch as your minds are drawn towards Sadāśiva’s anecdotes.

22. Like the holy waters of the Gaṅgā the inquiry into the anecdotes of Sadāśiva
sanctifies the three persons: the narrator, the inquirer and the hearer.

23. O \tl{brāhmaṇas}, except for the slayer of animals, who can be averse to
hear the narrative of the attributes of Śiva, that highly delights three types
of people always?

24. When it is being recited by persons who have no attachment or desire, it is
verily an antidote for all ailments of worldly existence, for it is highly
delightful to the ear and the heart while at the same time it bestows all objects.

25. O \tl{brāhmaṇas}, I shall explain Śiva’s sports in the light of your enquiry
as far as my intelligence enables me to do so. Please listen respectfully.

26. Induced by lord Viṣṇu, a manifestation of Śiva, Nārada had also put the same
question to his father Brahmā as you are asking me now.

27. On hearing the words of his son, Brahmā, a devotee of Śiva, was delighted in
his mind. Out of love he sang the glory of Śiva heightening the pleasure of
the excellent sage (Nārada).

Vyāsa said:

28. The learned \tl{brāhmaṇas}, on hearing the words of Sūta became eager to
know more of that conversation and so asked him.

The sages said:

29. O Sūta, O blessed Sūta, of great intellect and foremost among the devotees
of Śiva, on hearing your most delightful words our minds have become very eager
to know more.

30-31. Dear one, please tell us lovingly when this highly pleasant conversation
between Brahmā and Nārada took place, wherein Śiva’s glory was sung and
the divine sport of Lord Śiva, destructive of worldly existence, had been
discussed. What were the questions and how were they answered, please explain.

32. On hearing these words of the sages of noble mind Sūta was pleased much and
narrated everything pertaining to the conversation referred to.

\chapter{Indra sends Kāmadeva to disturb the penance of Nārada}

Sūta said:

1. O \tl{brāhmaṇas}, once Nārada the excellent sage, son of Brahmā was inclined
to perform penance controlling himself very much.

2. There is a very beautiful cave in the Himālaya mountain near which
the celestial river flows rapidly.

3. There was a great hermitage of divine splendour which was resplendent in
many ways. Nārada endowed with divine vision went there to perform the penance.

4. On seeing the hermitage (very convenient for penance) the leading sage
performed the penance for a long time, seated firmly and steadily, keeping
silent, controlling the breath and retaining the purity of the intellect.

5. O \tl{brāhmaṇas}, the sage performed meditation and contemplation wherein
the realisation “I am Brahman” is generated leading to the direct perception
of Brahman.

6. When the great sage Nārada was thus performing penance, the mind of Indra
became excessively agitated and he trembled.

7. Thinking “This sage is yearning for my kingdom” Indra wanted to spoil it.

8. Indra, the leader of \tl{devas}, remembered Kāmadeva (Cupid) who arrived
there immediately, accompanied by his Queen (Rati) and spring (his friend).

9. The king of \tl{devas}, endowed with crooked intelligence to achieve his
interests, saw that Kāma had arrived and addressed him thus.

Indra said:

10. O friend, of great prowess, always doing what is beneficent to me, please
hear lovingly what I am going to say. Render me your help.

11. Strongly supported by you I have destroyed the pride of many ascetics
O friend, the stability of my kingdom is always due to your blessing.

12. Nārada, the sage, is performing a penance in the Himalaya mountain directing
his mind towards the Lord of the universe with great mental control and firm
resolve.

13. I now fear lest he should beg of Brahmā my kingdom. You must go there now
itself and hinder his penance.

14. Being thus commanded by Indra, Kāmadeva, accompanied by his wife (Rati) and
Madhu, his friend, went haughtily to that place. He then prepared his own means
of attack.

15. He employed all his arts there immediately. Spring too haughtily spread his
prowess of diverse nature.

16. O great sages, the mind of the sage (Nārada) did not waver. Only
the arrogance of these fellows suffered a setback and that too by the favour
of Maheśa.

17. Please listen to the reason thereof, O Śaunaka and other sages! By
the controlling power of the lord, Kāma could not exercise any influence.

18. It was in this very place that Śiva, the indefatigable enemy of Kāma, had
formerly performed a great penance. It was here that Kāma was reduced to
ashes—Kāma who used to spoil the penances of sages.

19. Rati wanted the resuscitation of Kāma and requested the \tl{devas}. They
appealed to lord Śiva, the benefactor of the whole world who said thus:

20. O Gods, after some time Kāma will come to life again. But none of his
tricks will succeed here.

21. Whatever space all round this spot is visible to persons here, will be out
of the influence of Kāma for ever, O \tl{devas}.

22. It was due to this statement of Śiva that Kāma’s viles did not prevail upon
Nārada. From Śiva’s abode he went to Indra.

23. Kāma then narrated everything about the sage and commended his power. At
Indra’s bidding Kāma returned to his own place.

24. Deluded by Śiva’s Māyā (power of illusion) Indra was unaware of the true
facts and was greatly surprised and he admired Nārada.

25. Śiva’s Māyā is incomprehensible to all. The whole universe is deluded by it.
Only the true devotees of dedicated souls escape.

26. Backed by Śiva’s blessings Nārada stayed in the hermitage for a long time.
Then realising that his penance was complete, the sage concluded the same.

27. Thinking that he had conquered Kāma he was puffed with pride. He was devoid
of true knowledge and deluded by Śiva’s Māyā.

28. O great sages, blessed and very blessed is Śiva’s Māyā. Even Viṣṇu, Brahmā
and others do not know the turn it takes.

29. In that state of delusion and puffed up arrogance, the great sage Nārada
went to Kailāsa to expatiate on his own achievement.

30. Bowing down to Rudra, the sage arrogantly spoke of his exploits with
the conviction that he was equal to the noble-souled lord, the conqueror of Kāma,
\ie Śiva.

31. On hearing it, Śiva who is favourably disposed to His devotees, advised
Nārada who was ignorant of the real cause, whose mind had strayed and who had
been deluded by His (Śiva’s) Māyā.

Rudra said:

32. “Dear Nārada, O wise sage, you are blessed. But please listen to me. Never
speak like this anywhere else, especially in the presence of Viṣṇu.

33. Even when you are asked you should not mention your achievements as you have
done just now. These should be guarded as close secrets and should never be
expressed.

34. I bid you specifically like this because you are a great favourite of mine.
Since you are a devotee of Viṣṇu you are my follower as all his devotees are.”

Sūta said:

35. Lord Rudra, the cause of creation, advised him in many ways like this. But
Nārada who was still under the influence of Śiva’s Māyā did not take up this
wholesome advice.

36. The future course of actions shall be considered inevitable by sensible
persons. The will of Śiva cannot be warded off by anyone.

37. Then the great sage went to Brahmā’s world. After saluting Brahmā he told
him about his conquest of Kāma as a result of his penance.

38. On hearing that, Brahmā remembered the lotuslike feet of Śiva and knew
thereby the true cause. He then forbade his son.

39. Although foremost among the wise, Nārada did not take up the advice of
Brahmā as he had been deluded by Śiva’s Māyā. The sprout of arrogance had been
so fixed in his mind.

40. Everything will take place in the world in the manner Śiva wills. It is true
that the entire universe is dependent on His will.

41. Nārada hastened to Viṣṇuloka in the same state of senseless arrogance, to
boast of his exploits in the presence of Viṣṇu.

42. When Viṣṇu saw Nārada approaching, he could guess the purpose of his visit.
He stood up and received him cordially. He walked forward and embraced him
lovingly.

43. He made Nārada sit comfortably. After remembering the lotus-like feet of
Śiva, He frankly uttered these words intended to quell the arrogance of Nārada.

Viṣṇu said:

44. “O dear Nārada, foremost among sages, you are blessed. I am sanctified by
your visit. May I know where you come from and why you have come?”

45. On hearing these words of Viṣṇu, the sage Nārada felt elated. He narrated
his story in the same haughty manner.

46. On hearing the arrogant words of the sage, Viṣṇu remembered the lotus-like
feet of Śiva again and understood the true cause.

47. Viṣṇu, a leading devotee of Śiva, with his soul dedicated to Śiva, bowed his
head and eulogised Parameśvara, the lord of the holy mountain, with his palms
joined in reverence.

Viṣṇu said:

48. “O Lord, O Lord Mahādeva, Parameśvara, be pleased. O Śiva thou art blessed.
Thy Māyā enchants everyone.”

49. Having thus chanted the prayer to Śiva, the supreme Ātman, he closed his
eyes and meditated on His lotus-like feet and stopped.

50. On coming to know what Śiva was about to do, through Śiva’s bidding, he
addressed the great sage pleasantly.

Viṣṇu said:

51. O foremost among sages, you are blessed. You are the storehouse of
austerities and large-hearted. O sage, lust and delusion rise only in the heart
of that man who is devoid of the three types of devotion.

52. Base passions that bring in their wake all sorts of miseries crop up in him
instantly. But you are vowed to perpetual celibacy. You are ever endowed with
knowledge and devoted to non-attachment.

53-55. Unaffected by passion and highly intelligent by nature how can you be
swayed by lust?”

On hearing words like these, the great sage laughed within himself but spoke to
Viṣṇu humbly.

Nārada said:

“O lord, what can Kāma do to me if you remain favourable to me?”

Saying so, the sage who had paid a casual visit bowed to Viṣṇu and left.

\chapter{Nārada attends the Svayaṃvara of a virgin and is discomfited}

The sages said:

1-2. Sūta, O blessed Sūta, the disciple of Vyāsa, our obeisance to thee. It is
due to thy grace that this wonderful story has been narrated to us, O dear one.
Now tell us in detail what Viṣṇu did after Nārada had left the place? And where
did Nārada go?

Vyāsa said:

3. On hearing these words of the sages, Sūta the wise and excellent scholar of
\tl{Purāṇas} remembered Śiva, the cause of different kinds of creation and
replied.

Sūta said:

4. When Nārada went away casually Viṣṇu, skilful in wielding his Māyā, spread
his Māyā, as Śiva had willed.

5. On the path taken by the sage He created a big wonderful city. It was
a hundred \tl{yojanas} in extent and surprisingly beautiful.

6. It was far more beautiful than heaven. Many articles were displayed there.
Men and women of all the four castes stayed there.

7. The wealthy and prosperous king of that city named Śīlanidhi was preparing
for the gorgeous celebration of the voluntary wooing (\tl{svayaṃvara})\footnote{
This was an ancient custom amongst the kings of \tl{kṣatriya} caste to hold
a public assembly of suitors for the selection of a husband for their daughters.}
of his daughter.

8. Brilliant princes coming from all the four quarters eager to court
the princess had thronged there dressed in diverse ways.

9. On seeing such a splendid city Nārada\footnote{Nārada is one of the ten
mind-born sons of Brahmā having sprung from his thigh. He is celebrated as
a divine sage and is associated with another sage Parvata. He is represented as
the messenger from the Gods to men and vice versa and as being very fond of
promoting discords among Gods and men; hence he is called Kalipriya.} was
enchanted. With his love kindled, he eagerly went to the palace threshold.

10. When the sage reached the palace the king Śīlanidhi adored him, having
offered him a seat on the splendid throne studded with precious gems.

11. He called his daughter Śrīmatī and asked her to kneel down at the feet of
Nārada.

12. Being struck with wonder on seeing the girl, Nārada said—“O king, who is
this lovely girl comparable to celestial damsels?”

13. On hearing the words of the sage, the king replied with his palms joined in
reverence—“O sage, this is my daughter Śrīmatī.

14. She has attained the marriageable age. She is in search of a qualified
bridegroom. She has all charms and accomplishments and her \tl{svayaṃvara} is
imminent.

15. O sage, kindly foretell her destiny, everything that is in her horoscope.
Please tell me what sort of a husband she will get.”

16. By the time these words were spoken Nārada had become an agitated victim of
love and desired her. Addressing the king, he said thus:

17. “O great king, this daughter of yours is endowed with all characteristics:
She is highly fortunate and blessed like Lakṣmī. She is an abode of all
qualities.

18. Her future husband will certainly be a splendid God, lord of all,
unvanquished, heroic, on a par with Śiva, and vying with Kāmadeva”.

19. Having said this, the casual visitor Nārada took leave of the king. Deluded
by Śiva’s Māyā he was extremely oppressed by love.

20. The sage began to muse—“How shall I get her? How shall she woo me amongst
the princes in the \tl{svayaṃvara} hall.

21. A comely appearance appeals to all women in every respect. Only by seeing
a charming personality will she become enamoured”.

22. Thinking thus, Nārada who was agitated by love, went to Viṣṇuloka somehow
to acquire Viṣṇu’s form to captivate her.

23. He saluted Viṣṇu and said—“I shall tell you secretly my affairs entirely.”

24. When Viṣṇu who did everything according to Śiva’s wish agreed and asked him
to narrate, the sage said:

Nārada said:

25. The king Śīlanidhi is one of your devotees. He is a righteous king. His
daughter Śrīmatī is a maiden of very fair complexion and wide eyes.

26. She has the lustre of Jaganmohinī (enchantress of the universe—a
manifestation of Viṣṇu) and is the most beautiful woman in all the three worlds.
O Viṣṇu, I wish to marry her without delay.

27. The king at the request of the princess has arranged for a \tl{svayaṃvara}.
Thousands of princes have come from all the four quarters.

28. If you can favour me with a splendid form I shall be able to gain her
certainly. She will not put the wedding garland round my neck without your
splendid form.

29. O lord! give me your form. I am your servant and favourite. Give me your
beautiful form so that the princess Śrīmatī may choose me.

Sūta said:

30. On hearing these words of the sage Viṣṇu, the slayer of Madhu demon laughed
and sympathetically replied, bearing in mind the overwhelming power of Śiva.

Viṣṇu said:

31. “O sage, you can go to the place where you wish. I shall do what is
beneficent to you in the manner of a physician doing what is good to the patient,
since you are a great favourite of mine.”

32. After saying thus, Viṣṇu blessed the sage with a form like his own and
the face of Hari (\ie the monkey since the word Hari means a monkey also).
The lord then vanished.

33. The sage thus consoled became highly delighted on receiving Hari’s form.
He was contented but did not know the scheme behind the scene.

34. The great sage Nārada hastened to the place where \tl{svayaṃvara} was to be
held and where the princes had assembled.

35. O great \tl{brāhmaṇas}, the \tl{svayaṃvara} hall splendidly decorated and
graced by so many princes shone like another council-chamber of Indra.

36. Nārada too went in and sat down in the hall of his king. With his mind
surging with love he began to think like this.

37. “She will choose only me since I am in Viṣṇu’s form”. The poor sage did not
know the ugly character of his face.

38. The men assembled there saw the sage only in his old form. O \tl{brāhmaṇas},
the princes and others did not know the difference created therein.

39. Two of the attendants of Rudra knew this difference. They had come there in
the guise of \tl{brāhmaṇas} in order to protect him.

40. Considering the sage a fool, the two attendants sat near the sage and began
to mock at him seemingly conversing between themselves.

41. “See Nārada’s features as splendid as Viṣṇu’s, but the face as that of
a monkey deformed and awful.

42. Being deluded by Kāma he wishes to marry the Princess”. With these and other
veiled remarks they mocked at him.

43. The sage overwhelmed by love did not heed their whisper. He went on gazing
at the princess Śrīmatī and was eager to get her.

44. In the meantime, the princess had come out of the harem surrounded by ladies
in waiting. The comely maiden came to the hall.

45. With the beautiful golden garland in her hands, the princess of auspicious
features, shone in the middle of the \tl{svayaṃvara} hall like Goddess Lakṣmī.

46. The princess in search of a suitable bridegroom went round the hall with
the garland in her hands.

47. On seeing the sage with the face of a monkey and the body of Viṣṇu she was
infuriated. Averting her eyes she went elsewhere being distressed in her mind.

48. Failing to find a bridegroom of her choice she was afraid. She remained in
the middle of the hall and did not put the garland round the neck of any one.

49. Meanwhile Viṣṇu came there in the guise of a king. He was not seen by
anyone. Only the princess saw him.

50. Then on seeing Viṣṇu, her lotus-like face beamed. The comely lady put
the garland round his neck.

51. Lord Viṣṇu in the guise of a king took her with him and vanished from there
immediately back to his own abode.

52. The assembled princes lost their hope of getting Śrīmatī. The sage oppressed
by love became excessively agitated.

53. Immediately the two attendants of Rudra, of perfect wisdom, disguised as
\tl{brāhmaṇas} spoke to Nārada.

The attendants said:

54. O sage Nārada, being deluded by love, you are desirous of getting her. Your
effort is in vain. See, your face is as despicable as that of a monkey.

Sūta said:

55. On hearing their words Nārada was surprised. Deluded by Śiva’s Māyā he
looked into a mirror.

56. On seeing his face like that of a monkey he became infuriated. The deluded
sage cursed the two attendants.

57. Since you had mocked at me, you will become demons born of brahmanical semen
and of that form.

58-59. On hearing the curse, the two attendants of perfect wisdom remained
silent because they knew that the sage was deluded. O \tl{brāhmaṇas}, they
returned to their abode and sitting there quietly went on eulogising Śiva.
They considered everything as Śiva’s will.

\chapter{Nārada goes to Vaikuṇṭha and curses Viṣṇu there}

The sages said:

1-2. Sūta, O Sūta of great intellect, a wonderful tale has been narrated by you.
Blessed indeed is the Māyā of Śiva. All mobile and immobile things depend upon
it. When the two attendants of lord Rudra had left at their own will what did
the infuriated Nārada, the sage disquieted by Kāmadeva, do?

Sūta said:

3-5. After cursing the two attendants of Śiva suitably, the sage still under
the earlier delusion looked into the water and saw that his face was quite
normal. It was also due to Śiva’s will. He did not wake from the delusion still
again due to Śiva’s will. Thereupon recollecting that it might have been
a deception of Hari, he became unbearably infuriated and went to Viṣṇuloka.
There he angrily poured abusive words blazing like kindled fire since his wisdom
had vanished due to Śiva’s will.

Nārada said:

6. O Viṣṇu, you are extremely wicked, deceptive enchanter of the world. You are
unable to brook others’ enthusiastic success. You dabble in illusory tactics and
your intentions are always dirty.

7. Formerly you assumed the form of an enchantress\footnote{It refers to
the form assumed by Viṣṇu at the time of cheating the demons of nectar.} and
showed your deceptive power. You made the demons drink liquor and not
the nectar.

8. If out of pity Śiva had not drunk poison\footnote{It refers to Śiva’s
swallowing the poison produced at the churning of the ocean.}, O Viṣṇu, all your
illusory tactics would have been quelled since you take pleasure only in
deception.

9. O Viṣṇu, a deceptive path is extremely attractive to you. You had never been
of saintly nature, but the lord made you free from control.

10-11. What is done by Śiva the supreme Ātman does not seem proper. Thinking of
your influence and strength when you act independently and seeing the way you go
He has now repented. He has announced that a \tl{brāhmaṇa} is superior to all,
thereby making the Vedas pronounced by Him authoritative.

12. O Viṣṇu, knowing that, I shall now teach you through that power so that
hereafter you will never do such things.

13. You are fearless because till now you have not come into clash with an
equally powerful person. Now you will derive, O Viṣṇu, the fruit of your own
“deeds”.

14. After saying this, the sage still under the influence of Māyā furiouly
cursed Viṣṇu, thereby exhibiting the superiority of his brahmanical power.

15-16. O Viṣṇu, the enchanter that you are, you made me distressed for the sake
of woman. O Hari, you shall experience misery in that human form which you
imitated while proceeding with your deceptive tactics. Your allies will be those
whose face you assigned to me.

17. O inflictor of miseries upon others, you shall get the misery of separation
from a woman. You shall have the travails of a human being deluded by ignorance.”

18. Thus Nārada, deluded himself by ignorance, cursed Hari. Viṣṇu quietly
accepted the cause praising the Māyā of Śambhu.

19. Thereafter Śiva, of great divine sport withdrew his enchanting Māyā whereby
Nārada became wise (as before) and free from delusion.

20-21. When the Māyā vanished he became as intelligent as before regaining
perfect knowledge and becoming free from distress. He was surprised (at his own
action in the meantime). He cursed himself after repenting again and again. He
praised the Māyā of Śiva which could enchant even wise people.

22. On realising his mistakes due to illusion, Nārada, the most excellent of
the devotees of Viṣṇu, fell at his feet.

23. Consoled by Hari and freed from wicked ideas he said—“Being deluded and
evil-minded I have spoken many wicked words to you.

24. O lord, I heaped curses on you. O master, please make them ineffective.
I have committed a great sin. Certainly I will be falling into a hell.

25. O Hari, I am your slave. Please direct me what to do whereby I may destroy
my sins and prevent my downfall into hell.”

26. Saying thus, the excellent sage once again fell at Viṣṇu’s feet and with
the mind purified repented sincerely.

27. Thereupon Viṣṇu lifted him up and spoke affably and courteously.

Viṣṇu said:

“Do not be sorry too much. Undoubtedly you are my true devotee.

28. Dear sage, now listen. I shall tell you what is certainly beneficial to you.
You will not fall into hell. Śiva will make you happy.

29. Deluded by your haughtiness you disobeyed the instructions of Śiva. The true
bestower of fruits according to the actions, He has given you this result.

30. Be sure in your mind that everything has happened in accordance with Śiva’s
wish. That lord Śiva, the supreme lord, removes haughtiness.

31. He is the supreme Brahman; the supreme Ātman, Existence, Knowledge and
Bliss. He is free from the three \tl{guṇas}, changes and deviations. He is
beyond \tl{rajas}, \tl{sattva} and \tl{tamas}.

32. He is both \tl{saguṇa} and \tl{nirguṇa} (with and without attributes). He
Himself availing of his own Māyā manifests into three Brahmā, Viṣṇu, and
Maheśa.

33. In his attributeless pure form He is glorified as Śiva, the supreme Ātman,
Maheśvara, the supreme Brahman, the undecaying, the endless, and Mahādeva.

34. Serving him, Brahmā becomes the creator and I the sustainer of the worlds.
He himself in the manifestation as Rudra is the annihilator always.

35. Different from Māyā, the pure Being in the form of Śiva is the Sākṣin
(cosmic witness) and moving about according to His Will and indulging in divine
sport He blesses his devotees.

36. O sage Nārada, please listen to a good remedy that bestows happiness,
removes all sins and yields worldly pleasures and salvation.

37. Cast off all your doubts. Sing the songs of noble glory of Śiva. With your
mind not turning to anything else, always repeat the hundred names of Śiva and
his hymns.

38. By his \tl{japa} all of your sins will perish instantaneously. After saying
this to Nārada, Viṣṇu continued mercifully.

39. “O sage, do not be grief-striken. Nothing has been perpetrated by you. It
was Śiva who did everything. There is no doubt in this.

40. It was lord Maheśvara who deluded your splendid intellect and made you
suffer on account of love. It was he who made you His mouthpiece and cursed me.

41. In this manner the great Conqueror of Death, Kāla of Kāla, always devoted to
the uplift of his devotees, made His own conduct of life manifest in the world.

42. There is no other lord and master so loving and pleasure-inspiring unto me
as Śiva. The same Parameśvara bestows all power on me.

43. O sage, perform His adoration. Worship him always. Hear and sing his glory.
Perpetually pay Him homage.

44. He who approaches Śiva by means of his body, mind and speech is a great
scholar. He is called a living liberated soul.

45. The name Śiva blazing like the forest conflagration reduces mountainous
heaps of great sins to ashes without any difficulty. True, it is undoubtedly
true.

46. The different kinds of miseries arising from sins shall be destroyed only
through the worship of Śiva, and not through other means.

47. He who always seeks refuge in Śiva, O sage, is the real follower of
the Vedas, a meritorious soul and a blessed scholar. He must resort to Him by
means of his body, speech and mind for ever.

48. The different sacred rites of those who have full faith in the worship of
Śiva, the destroyer of Tripura\footnote{Śiva is called Tripurāri (the enemy of
Tripura) because he killed the demon, Tripura, who presided over the three
cities built for the \tl{dānavas} by Māyā \etc after having burnt down
the cities along with the demons inhabiting them.} become fruitful
instantaneously.

49. O great sage, there are not so many sins in the world as the worship of Śiva
is capable of destroying.

50. Innumerable heaps of sins like that of the slaughter of a \tl{brāhmaṇa}
perish by remembering Śiva. Truth, I am telling you the truth.

51. The sins (that usually cause worldly existence) relating to persons who
cross the ocean of worldly existence in the raft of Śiva’s names, perish
undoubtedly.

52. The sins which are at the root of worldly existence are destroyed certainly
by the axe of Śiva’s name.

53. Persons scorched and distressed by the conflagration of sins must drink
the nectar of Śiva’s names. Without that there is no peace and tranquillity to
those who are scorched and distressed by the sins’ wild fire.

54. Those who are drenched by the downpour of the nectarine names of Śiva are
not distressed in the midst of the conflagration of worldly existence. There is
no doubt in this.

55-56. Immediate salvation can be achieved only by the people who have performed
penance in various lives. They alone will have devotion for Śiva the cherished
consort of Pārvatī. Men who frequently indulge in passions of love and hatred
will never have devotion for Śiva.

57. The devotion for Śiva that extends to other deities is futile. It is
necessary to be exclusively devoted to Śiva.

58. It is my conviction that salvation is easy of access only to the person who
has exclusive and unflinching devotion for Śiva and not for any other.

59. Even if he commits endless sins, he will be freed from them all, if he has
true devotion for Śiva. There is no doubt about it.

60. Just as trees in the forest are reduced to ashes in the wild fire so also
the sins of the devotees of Śiva are burnt away in the fire of Śiva’s name.

61. He who is ever devoted to the worship of Śiva with his body purified by
the ash, definitely crosses the terrible and endless expanse of the ocean of
worldly existence.

62. A man serving the three-eyed\footnote{Śiva is called Virūpākṣa ‘odd-eyed’,
because he is represented as having three eyes: two on either side of the nose
and one on the forehead.} Śiva is never sullied by sins even if he
misappropriates a \tl{brāhmaṇa}’s wealth or kills many \tl{brāhmaṇas}.

63. After going through all the Vedas this has been definitely concluded by
ancestors that the sole means of destroying worldly existence is the worship
of Siva.

64. From now onwards you shall always worship lord Śiva who is Sāmba and
Sadāśiva, with care, effort and due observance of the rules of procedure.

65. Dusting profusely and carefully your body from head to foot with
the particles of ashes, you shall perform the \tl{japa} of the six-syllabled
mantra\footnote{The six-syllabled mantra is “\tl{oṃ namaśśivāya}”} of Śiva,
well-known in all the Vedas.

66. You shall wear on the different parts of your body \tl{rudrākṣa} beads
pleasing to Śiva, repeating the respective mantras with devotion and observing
the rules of procedure.

67. Listen to Śiva’s anecdotes for ever. Narrate the stories of Śiva always.
Strenuously worship the devotees of Śiva again and again.

68. Without blundering ever seek refuge in Śiva, be cause a perpetual worship of
Śiva bestows bliss.

69. Bearing the lotus-like feet of Śiva within your pure heart, carry on at
first the pilgrimage to various holy centres of Śiva, O excellent sage.

70. Observing the unrivalled greatness of Śiva, the supreme Ātman, O sage, you
must next go to Ānandavana which is a great favourite of Siva.

71. Seeing Śiva, the lord of the universe there, worship Him with devotion.
After bowing to him and eulogising Him you will become free from all doubts.

72. Thereafter you must go to Brahmāloka, O sage, to achieve your wishes. That
is my command to you out of love.

73. O sage, after bowing to and specifically eulogising your father Brahmā, you
shall ask him many points regarding Śiva’s greatness with an endearing mind.

74. Brahmā, the foremost among the devotees of Śiva, will narrate to you
the greatness of Śiva as well as the hymn of hundred names, out of love.

75. O sage, from now onwards become a devotee of Śiva, solely devoted to Śiva.
You will be liberated. Śiva will grant you his special blessings”.

76. After advising the sage thus, Viṣṇu was pleased. Remembering, saluting and
eulogising Śiva he vanished from that place.

\chapter{Nārada goes to Kāśī}

Sūta said:

1. O \tl{brāhmaṇas}, when Viṣṇu vanished, the excellent sage Nārada roamed over
the Earth seeing Śiva \tl{liṅgas} (in the various holy centres) with piety.

2. In the course of his wanderings over the Earth, O \tl{brāhmaṇas}, with his
mind full of devotional pleasure he saw many forms of Śiva that confer worldly
pleasures and salvation on the devotees.

3. On knowing that Nārada of divine vision was wandering over the Earth, the two
attendants of Śiva approached him who by that time had become pure in mind.

4. They bowed to him and touched his feet. With a desire to secure release from
the curse they spoke to him respectfully.

The attendants of Śiva said:

5. O celestial sage, son of Brahmā, please hear our words. We who formerly
offended you are really not \tl{brāhmaṇas}.

6-7. O brahmanical sage, we, your former offenders, are the attendants of Śiva.
Induced by Śiva you had cursed us when your mind was deluded by the illusory
infatuation for the princess at the \tl{svayaṃvara}. Realising that the occasion
was inopportune we kept quiet then.

8. We reaped the fruit of our own action. No one is to be blamed for it. O lord,
be pleased. Bless us now.

Sūta said:

9. On hearing the words of the attendants uttered with devotion and respect,
the sage replied lovingly, repenting (for his previous fury).

Nārada said:

10. O attendants of Lord Śiva, most worthy of the respect of good people, please
listen to my words now free from delusion. They are true and shall make you
happy.

11. Formerly my mind had been depraved. Certainly it was Śiva’s will. In that
state of delusion and crookedness of the mind I had unfortunately cursed both
of you.

12. What I have said is bound to happen. Still, O Gaṇas (attendants) listen. I
shall tell you the way of redemption from the curse. Please forgive my sin now.

13-14. You will be born as demons from the semen virile of a great sage and due
to his power you will secure the commanding position of the king of demons
endowed with prosperity, strength and valorous exploits. You will rule over
whole of the universe as devotees of Śiva with your sense conquered. You will
gain your former position after courting death at the hands of a manifestation
of Śiva.

Sūta said:

15. On hearing these words of the noble-souled Nārada, the two attendants of
Śiva became delighted and went back to their abode joyfully.

16. Nārada too was delighted. Meditating exclusively on Śiva he continued his
wanderings over the Earth seeing the various holy centres of Śiva personally.

17-18. Reaching Kāśī that excelled all other cities in holiness, which is
a favourite resort of Śiva, which easily bestows the favour of Śiva and which is
identical with Śiva, the sage became contented. He saw Śiva, the lord of Kāśī
and worshipped Him with very great pleasure and love.

19. While staying at Kāśī, the excellent sage became contented; he bowed to
the lord, described his glory piously, and remembered him with the flutter of
love.

20. Nārada then went to the region of Brahmā, his mind being highly purified by
remembering Śiva. He was eager to know further the principles of Śiva.

21. There he bowed to Brahmā with devotion and eulogised him with various
prayers. With his mind riveted to Śiva he asked him the good principles of Śiva.

Nārada said:

22-23. O Brahmā, knower of the form of Brahman, O Pitāmaha, the lord of
the universe, by your grace I have heard the greatness of Viṣṇu entirely and
also the path of devotion, of knowledge, of austere penance, of charitable gifts
and of holy centres.

24. But I have not understood the principle of Śiva. Hence, O lord, please
explain the rules of His worship and also the various activities of the lord.

25. O dear sage, how can Śiva who is free form attributes become full of
attributes? Since I am deluded by Śiva’s Māyā, I do not know the principle of
Śiva.

26. How did Śiva remain in His pure form before Creation? In the middle of
creation how does He sport about?

27. At the time of dissolution how does He remain? How is He, the benefactor of
the world, propitiated?

28. O Brahmā, when propitiated what benefit does He bestow on His devotees and
on others? Please satisfy me on all these enquiries

29. I have heard that the lord becomes delighted immediately. The merciful Great
God cannot bear the stress and strain of His devotees.

30. The three deities Brahmā, Viṣṇu and Maheśa are born as parts of Śiva. Maheśa,
having all the parts of Śiva, is Śiva Himself.

31. Please tell me all about His manifestation and especially His exploits.
O lord, please narrate the manifestation of Umā and her marriage.

32. Their domestic life, especially their great divine sports and other things
which are worthy of mention should be narrated to me, O sinless one.

33. Pārvatī’s birth and her marriage as well as Guha’s birth shall be narrated
in detail, O lord of people.

34. O lord of universe, this I have heard from many, before, but I am not
satisfied. Hence I have sought refuge in you. Please have mercy on me.

35. On hearing these words of Nārada his own son, Brahmā, the grandfather of
the world, said this.

\chapter{Description of the nature of Mahāpralaya and the origin of Viṣṇu}

Brahmā said:

1. O \tl{brāhmaṇa}, foremost among the celestial beings, a good matter has been
enquired into by you rendering service to the worlds and desiring their benefit.

2. I shall explain to you the wholesome and salutary principles of Śiva on
hearing which the various sins of the people are destroyed.

3. Neither the principles of Śiva nor His supreme wonderful forms have been
understood by me or by Viṣṇu or by any one else.

4. At the time of Great Dissolution when all the mobile and immobile objects of
the world are dissolved everything gets enveloped in darkness, without the sun,
planets and stars.

5. There is no moon. The day and the night are not demarcated. There is no fire,
no wind, no earth and no water. There is no unmanifest primordial being.
The whole firmament is one complete void, devoid of all \tl{tejas} elements.

6. There is no \tl{dharma} or \tl{adharma}, no sound, no touch. Smell and colour
are not manifest. There is no taste. The face of the quarters is not demarcated.

7. Thus when there is pitch darkness that cannot be pierced with a needle and
what is mentioned in the Vedas as “The Existent and the Brahman” is alone
present.

8. When the present visible world is not in existence, the Sat Brahman alone is
present which \tl{yogins} observe perpetually in the inner Soul,
the inner Firmament.

9. It is incomprehensible to the mind. It cannot at all be expressed by words.
It has neither name nor colour. It is neither thick nor thin.

10. It is neither short nor long. It is neither light nor heavy. There is
neither increase nor decrease in it.

11. The Veda says that it envelops whatever is in a surprising way. It is
the splendour, the truth, the knowledge, the eternal and the great Bliss.

12. It is immeasurable, propless, changeless, formless, attributeless,
perceptible to the \tl{yogins}, all-pervasive and the sole cause of the universe.

13. It is free from alternatives. It has no beginning. It is free from illusion
and its harassment. It has no second. It has neither beginning nor end. It has
no development. It is in the form of pure knowledge.

14. People have doubts about giving it a name. That Being, then after sometime,
it is said, wished for a second.

15. The Being, having no form of its own, wished to create, in the course of its
own sport, an auspicious form of its own endowed with all power, qualities and
knowledge.

16-18. A form that goes everywhere, that has all forms, that sees all, that is
the cause of all, that should be respected by all, that is at the beginning of
all, that bestows everything, and that sanctifies everything should be created
(So it wished) and hence created that form of Īśvara of pure nature.
The original Being without a second, with neither beginning nor end, that
illuminates everything, that is in the form of \tl{cit} (pure knowledge), that
which is termed Supreme Brahman, the all-pervasive and undecaying, vanished,
the manifest form of the formless Being is Sadāśiva. Scholars of the ancient and
succeeding ages have sung of it as Īśvara.

19. Īśvara though alone, then created the physical form Śakti from his body.
This Śakti did not affect his body in any way.

20. This Śakti is called by various names. Pradhāna, Prakṛti, Māyā, Guṇavatī,
Parā. The mother of Buddhi Tattva (The cosmic Intelligence), Vikṛtivarjitā
(without modification).

21. That Śakti is Ambikā, Prakṛti and the goddess of all. She is the prime cause
and the mother of the three deities.

22. She has eight arms. Her face wears a peculiar splendour, the splendour of
a thousand moons. Thousands of stars perpetually sparkle round her face.

23. She is bedecked in various ornaments. She has various weapons. She is
capable of various movements. Her eyes beam like a full blown lotus.

24. She has a brilliance which could hardly be conceived. She is the generating
cause of all. She sprang up singly as Māyā. In her union she manifested in
various forms.

25. The supreme Puruṣa is Śiva. He is called Śambhu. He has no other lord over
Him. He holds the Mandākinī (Gaṅgā) on His head, and the crescent moon on His
forehead. He has three eyes.

26. He has five faces. He is always joyful. He has ten arms. He holds
the trident. He is as pure and white as camphor. His body is entirely dusted
with the ash.

27. That Brahman of the form of Kāla (Time) together with Śakti, simultaneouly
created the holy centre called Śivaloka.

28. The same is called Kāśikā, the excellent holy centre. It is the seat of
salvation shining over and above everything.

29. The holy centre is of the nature of extreme Bliss inasmuch as the primordial
lovers, supremely Blissful, made that beautiful holy centre their perpetual
abode.

30. O sage, that holy centre is never, even at the time of Great Dissolution,
free from Śiva and Śivā (Śakti). Hence it is called Avimukta.

31. Since the holy centre is the cause of Bliss, the Pināka-bearing lord (Śiva)
called it “the blissful forest” and later “Avimukta”.

32. O celestial sage, the blissful, two deities thus sporting in the forest
wished, it is said, for another Being to be created.

33-38. Śiva thought within Himself like this—“Another being shall be created by
me. Let him create everything, protect it and in the end let him dissolve it
with my blessing. Having entrusted everything to him we two, remaining in Kāśī
shall roam as we please keeping only the prerogative of conferring salvation. We
can stay happily in this blissful forest being free from worries (of creation).
With the consent of Śiva the supreme lord spread the liquorine essence of nectar
on His left side, on the tenth limb, nectar which was the outcome of churning
the ocean of His mind wherein thoughts were the waves, the \tl{sattva guṇa} was
the precious gem, \tl{rajas} being coral and \tl{tamas}—crocodile. Thereupon
a person came into being who was the most charming one in the three worlds, who
was calm with \tl{sattva guṇa} being prominent, and who appeared to be the ocean
of immeasurable majesty.

39. O sage, he was endowed with patience. There was no one comparable to him. He
had the lustre of sapphire. He was glorious with his excellent eyes shining like
a lotus.

40. He was having a golden form and features. He wore two excellent silk
garments of golden colour. His were browny and brilliant. He was indefatigable.

41. He bowed to Śiva Parameśvara and said—“O lord give me names and assign me my
task.”

42. On hearing it Lord Śiva laughed. With words thunderlike in resonance, Lord
Śiva addressed the person thus.

Śiva said:

43. “You will be famous as Viṣṇu by name as you are all-pervasive. You will have
many other names conferring happiness on devotees.

44. Perform penance highly conducive to the achivement of the matter in hand,
be firm in it.” Saying so, the lord bestowed on him the Vedas through his
nostrils.

45. Śiva vanished accompanied by Śakti and his attendants. After due obeisance
to Śiva, Viṣṇu began his great penance.

46. Even after performing the penance for twelve thousand divine years, Viṣṇu
could not achieve his desire, the vision of Śiva that confers everything.

47. He became suspicious and respectfully meditating on Śiva pondered “What
shall I do now?”

48. In the meantime the auspicious voice of Śiva was heard. “Perform penance
again for removing your doubts”.

49. On hearing it Viṣṇu performed a terrible penance, for a long time, following
the path of meditation.

50. That Being Viṣṇu became enlightened, following the path of meditation. He
was delightfully surprised. “O what is that true entity?”

51. From the body of Viṣṇu who thus exerted himself, water-currents of various
sorts began to flow as a result of Śiva’s Māyā.

52. O great sage, the Supreme Brahman in the form of divine waters pervaded
the entire void. A mere contact with the same is destructive of sins.

53. Viṣṇu, the weary person went to sleep amidst the waters. He was in that
blissful state of delusion for a long time.

54. As approved in the Vedas, his name came to be established as Nārāyaṇa
(Having water as abode). Excepting for that Primordial Being there was nothing
then.

55. In the meantime, the principles too were evolved out of the Great soul. O
wise one of great intellect, listen to my enumeration of the same.

56. From \tl{prakṛti} came into being the \tl{mahat} (cosmic intellect), from
\tl{mahat} the three \tl{guṇas}. \tl{Ahaṃkāra} (the cosmic ego) arose there from
in three forms according to the three \tl{guṇas}\footnote{The Ego (\tl{ahaṃkāra})
is threefold according to the qualities of \tl{sattva}, \tl{rajas} and
\tl{tamas}. In the present enumeration it is counted as one.}.

57. The Essences, the five elements, the senses of knowledge and action too came
into being then.

58-59. O most excellent of sages, I have thus enumerated the principles. All
these principles originating from \tl{prakṛti} are insentient but not the Puruṣa.
These principles are twenty-four in number\footnote{A group of 24 \tl{tattvas}
includes intellect (\tl{buddhi}), ego (\tl{ahaṃkāra}), \tl{manas} (mind), five
elements (\tl{bhūtas}), five subtle elements (\tl{tanmātras}), five senses of
action (\tl{karmendriyas}) and five senses of knowledge (\tl{jñānendriyas}) and
unmanifest \tl{prakṛti} (\ie \tl{pradhāna}). Puruṣa stands apart from
the \tl{tattvas}. The enumeration follows the Sāṃkhya system.}. Viṣṇu,
the Puruṣa, accepted all these, as was the will of Śiva, and began his sleep in
the Brahman.

\chapter{The dispute between Brahmā and Viṣṇu}

Brahmā said:

1. When lord Nārāyaṇa continued to sleep, an excellent lotus of huge size came
out of his navel as desired by Śiva.

2. It was many \tl{yojanas} wide and high. It had an endless stalk. The pericarp
was of a brilliant hue.

3. It was very beautiful with the brilliance of ten million suns. It was
wonderful, excellent and worthy of vision containing \tl{tattvas}.

4. Exerting himself as before, Śiva, the great lord, with Pārvatī as his better
half created me from His right limb.

5. O sage, having deluded me with His illusion immediately, Śiva in the course
of His sport, produced me through the umbilical lotus of Viṣṇu.

6. Thus it was that I came to be known as Lotus-born and conceived in a golden
womb. I had four faces, red complexion and \tl{tripuṇḍra}-marked forehead.

7. Deluded by His illusion and weakened in knowledge, O dear one, I did not know
who the progenitor of my body was, other than the lotus.

8. “Who am I? Whence did I come? What is my duty? To whom was I born a son? By
whom have I been created?”

9-11. My intellect became confused with these doubts. Then I thought “Why shall
I be under delusion? It is easy to gain that knowledge. The place of growth of
this lotus is below. My progenitor will undoubtedly be there.” Thinking thus I
descended from the lotus. O sage, for a hundred years the downward trend
continued.

12. The source of the lotus was not attained by me. In the doubt-tormented state
I became eager to go up on to the top of the lotus.

13. O sage, I climbed up to the lotus by the stalk. But the upper part of
the lotus I could not reach. I was disappointed.

14. Another hundred years elapsed in my wandering up the lotus. I stopped
a while in that confounded state.

15. Then, O sage, by the will of Śiva, an auspicious voice “Perform Penance” was
heard from the sky which dispelled my delusion.

16. On hearing the voice of the sky I exerted myself for twelve years in
performing a terrible penance in order to see my progenitor.

17. At the same time, the four-armed lord Viṣṇu of beautiful eyes suddenly
appeared before me in order to bless me.

18. The great lord was holding the conch, the discus, the mace and the lotus in
his hands. He was wearing the yellow silken cloth and had cloud-blue complexion
all over his body.

19. He had a crown. He was bedecked in great ornaments. His lotus-like face
beamed with pleasure. Such was the lord resembling ten million Cupids that I saw
still not out of delusion.

20-21. At the sight of that beautiful form I was struck with wonder. On seeing
the four-armed Nārāyaṇa, shining like Kāla, of golden hue, the immanent soul of
all in that form, of large arms depicting the \tl{sat} and \tl{asat} in Himself
I became delighted.

22. Deluded by the illusion of Śiva, the sportive lord, I could not recognise my
progenitor in him. I addressed him with delight.

23. “Who are you? Please tell me”, saying this I tried to wake the Eternal Being.
[When he did not wake up] I tried to wake him up with fiercer and firmer
beatings of the hand.

24. Then the lord who had self-control woke up from his bed and sat. He looked
up with his pure eyes resembling a wet lotus, due to sleep.

25. As I stood there quietly, the lord Viṣṇu spread his brilliance over me.
Standing up he smiled once and spoke these sweet words.

26. Viṣṇu said:—“Welcome, welcome to you, dear child, O Pitāmaha of great
brilliance. Do not be afraid. Undoubtedly I shall confer on you all that you
desire.

27. O foremost among gods, on hearing these words uttered with a smile I told
Viṣṇu with my inimical attitude roused by the \tl{rajoguṇa}.

Brahmā said:

28. “O faultless one, how is it that you speak of me trivially as “Dear child”,
me who am the cause of annihilation of everything, as a preceptor addresses his
disciple?

29-30. “I am the creator of worlds, the direct activiser of \tl{prakṛti}, unborn,
the eternal, all-pervasive Brahmā. I am born of Viṣṇu. I am the soul of universe,
the originator, creator, and the lotus-eyed. You must explain to me quickly why
you speak like this.

31. The Vedas speak of me invariably as self-born, unborn, all-pervasive,
grandfather, self-governed and the excellent supreme Being.

32-35. On hearing these words of Hari, the lord of Lakṣmī became angry and told
me thus:

Viṣṇu said:

“I know you as the creator of the world. For the sake of creation and support
you are descended from my undecaying limbs. You have forgotten me, who am a lord
of universe, abiding in waters the salubrious, the supreme soul, invoked by many,
praised by many, all-pervasive, imperishable, ruler, the source and origin of
universe, the longarmed and the ommipresent lord. There is no doubt in this that
you are born of the lotus from my umbilicus.

36. “Of course, it is not your fault. I have exercised my power of illusion over
you. O four-faced one, listen to the truth. I am the lord of all Gods.

37. “I am the creator, sustainer and destroyer. There is no powerful person
equal to me. O Pitāmaha, I am the supreme Brahman, the greatest Truth.

38-39. “I am the greatest light. I am the great Ātman. I am the omnipresent. O
four-faced one, whatever in seen or heard today in the whole universe, whether
mobile or immobile is enveloped by me. It was I who created the twenty-four
manifest Tattvas.

40. “I have created the atoms. I have created the qualities of anger, fear \etc
Powerful and sportive I have created their parts and limbs.

41. “I have created the Intellect and the threefold ego therein. I have evolved
the five subtle elements, the mind, the body and the sense-organs.

42. I have created the elements ether \etc and all created beings out of sheer
sport. Realising this, O Brahmā, the lord of subjects, seek refuge in me.

43. “I shall certainly protect you from all miseries.”

Brahmā said:

On hearing these words, I, proud of being Brahmā, became angry. Being deluded by
illusion in a threatening attitude I asked him “Who are you?”

44. “Why do you talk so much? Your words will bring up disaster. You are neither
the lord, nor the supreme Brahman. There must be a creator of yours.”

45. Deluded by the illusion created by Śiva the great lord, I fought a terrific
battle with Viṣṇu.

46. Inimical to each other due to \tl{rajoguṇa}, we fought a fierce battle in
the middle of that vast expanse of the sea of dissolution.

47. Meanwhile a phallic image appeared before us in order to enlighten us and to
settle out dispute.

48. It had no beginning, middle or end. It had neither decrease nor increase. It
was as furious as hundreds of the fire of death with thousands of leaping rows
of flames.

49. It was unequalled, inexpressible unmanifest universal Being. The lord Viṣṇu
became unconscious by its thousand flames.

50. When I too became senseless, Viṣṇu said to me. Oh, why do you contend with
me now? A third person has now come. Let our quarrel cease.

51. Whence has this arisen? Let us examine this fire-Being. I shall go down to
find the root of this matchless column of fire.

52. “O lord of subjects, with the speed of the wind you will please go up to
examine its top.”

Brahmā continues the story:

53. Having said so, Viṣṇu assumed the form of a Boar. O sage, I became a swan
immediately.

54. From that time onwards, people call me Haṃsa-Haṃsa, a supreme Being, Virāṭ,
an illustrious being. He who repeats ‘\tl{haṃsahaṃsa}’, shall become a swan (a
symbol of purity and discrimination).

55. Very white of complexion and endowed with wings on either side I flew up and
up with the speed of the mind and wind.

56-58. Nārāyaṇa, the soul of the universe too, became white then. His body was
ten \tl{yojanas} wide and a hundred \tl{yojanas} long, as huge as the mountain
Meru. He had white sharp teeth. His brilliance resembled the sun at the time of
dissolution. His snort was long and his roar tremendous. His feet were short.
His limbs were of diverse colours. His form as the boar was of matchless
firmness which assured his eagerness to be victorious, and he went down quickly.

59. For a thousand years his downward course continued. From that time onwards
Viṣṇu came to be called “Śvetavārāha” (white boar) in all the worlds.

60. A \tl{kalpa} had elapsed according to human calculation when Viṣṇu thus went
down and wandered in his eagerness to come out victorious.

61. The Boar did not find even the smallest trace of the root of the Liṅga. O,
destroyer of enemies, I too spent the same time in going up.

62. From a desire to know its top as quickly as possible I exerted myself and
was exhausted. Unable to see the top I came down after some time.

63. Similarly, lord Viṣṇu, the lotus-eyed, too became weary. Appearing like
the lord of everything in his huge body he too rose up.

64. As soon as he came up, we bowed to Śiva again and again. He stood aside with
a gloomy mind as he too was deluded by the illusion of Śiva.

65. We bowed down to Liṅga at His back, sides and in front. He mused within
himself “What can this be?”

66. “That form can’t be directly expressed. It is without action and name.
Without any sex-distinction it has become a \tl{liṅga}. It is beyond the path
of meditation.

67. Both of us, Hari and I, with the peace of our minds, became eager to perform
obeisance.

\begin{shloka}\itshape
  jānīvo na hi te rūpaṃ yo'siyo'si mahāprabho |\\
  namo'stu te maheśāna rūpaṃ darśaya nau tvaran
\end{shloka}

68. “We do not know Thy true form, what Thou art Thou art, O great lord.
Obeisance be to Thee, O Maheśāna. Please hurry up to reveal Thy form to us.”

69. Thus performing obeisance and prayer to quell our earlier pride, O foremost
of sages, we spent a hundred autumns therein.

\chapter{The description of the body of Śabdabrahman}

Brahmā said:

1-2. O most excellent sage, we were eager to have a vision of the lord. Our
haughtiness had been curbed. O sage, we waited there patiently. Śiva,
the protector of the distressed, remover of the haughtiness of the haughty and
the undecaying lord of everything took mercy on us.

3. There arose the sound “Oṃ Oṃ” in the prolated accent\footnote{The pluta is
a prolated vowel, as in Oṃ, often marked with the figure three (\dn{ओ३म्}, o3m),
as it contains three syllabic instant in pronouncing it.}. It was very clear.
The divine sound in the form of a word came out from the most excellent of Gods.

4-5. “What shall be this great sound?” thinking like this I stood perplexed.
Viṣṇu who is worthy of respect from all the Gods, who is free from all inimical
thoughts, saw with the delightful heart, the eternal being’s manifestation on
the right side of the \tl{liṅga}. First, he saw the syllable “A” and he saw
the syllable “U” thereafter.

6-10. He saw the syllable “M” in the middle and Nāda (the mystical sound) in
the form “Oṃ” in the end. He saw the first syllable on the right like
the blazing sphere of the sun. O foremost of sages, thereafter he saw the
syllable “U” dazzling like fire. In the middle he saw the syllable “M”
glittering like the lunar sphere. Above that what he saw was the supreme Brahman,
the greatest refuge. It had the lustre of the pure crystal. It was the pure
Being beyond the Fourth (\tl{turīya}), the unsullied and free from extraneous
harassment. It was free from mutually clashing opposites. It was single
(isolated), void, free from exterior and interior though stationed in
the exterior and the interior, devoid of beginning, middle and end,
the primordial cause of Bliss, the truth, The Bliss and the Nectar.

11-12. Viṣṇu thus meditated on the universal soul enveloped by the two Vedic
sounds and wished to examine the source whence the Fire-column arose and to go
deep down the unequalled fiery column. Then there came a sage who told him
the essence of the truth.

13. Viṣṇu realised that the sage himself was the great lord and the supreme
Brahman embodied in the Śabda Brahman. (\ie the mystic syllable Oṃ).

14. The Brahman is Rudra free from worries. The words and the mind are incapable
of comprehending it; without reaching it they return. It can be expressed by
the single-syllabled mantra “Oṃ”.

15. The supreme Brahman, the Truth, the Bliss, the Amṛta, the greatest of
the great and the ultimate cause can be expressed by the single-syllabled mantra.

16. The single syllable “A” is the source of the lord Brahmā. The single
syllable “U” is the source of Viṣṇu, the ultimate cause.

17. The single syllable “M” is the source of Rudra. The creator is expressed by
the letter “A”. The enchanter is expressed by the letter “U”.

18. The being expressed by the letter “M” blesses always. It is all-pervasive
and progenitor; the letter “A” is the seed.

19. The being expressed by the letter “U” is Viṣṇu. It is the source,
the receptacle, the lord of primordial nature and primordial being,
the progenitor, the seed, source and sound. All these constitute Lord Śiva.

20. The progenitor is stationed after dividing itself. From the \tl{liṅga} of
the progenitor, the lord, arose the seed—the syllable “A”.

21. The \tl{bīja} being deposited in the \tl{yoni}, the letter “U” began to
increase all round. It became a golden egg. It was something known which could
not be delineated.

22. The divine egg floated in the waters for many years. Then at the end of
a thousand years, it split into two giving birth to Brahmā.

23-24. The egg floating in the waters on being hit by Īśvara split into two.
The auspicious golden upper lid became the upper region and the lower one became
the Earth of five characteristics. From (the inner part of) the egg was born
the four-faced lord (Brahmā) expressed by the letter “KA”.

25. He is the creator of all the worlds. He alone is the lord manifesting in
three forms. Persons well-versed in the Yajurveda call it Oṃ Oṃ.

26. On hearing the words of the Yajurveda, both the Ṛgveda and the Sāmaveda
respectfully called us then Viṣṇu and Brahmā.

27. Then realising the lord of the Gods we eulogised, as far as we could, Lord
Śiva, the cause of great achievement.

28. Viṣṇu, the protector of the universe, in the meantime, saw another
wonderfully beautiful form, along with me.

29-30. On seeing that wonderful form, Viṣṇu and I became satisfied. The form had
five faces, ten arms, and a complexion white as camphor, O sage. It had diverse
brilliant features. It was decorated in different ornaments. It was highly
liberal and endowed with great prowess. It had all the characteristics of
a great man.

31. Thereafter, the lord Śiva was pleased. Revealing his form embedded in
letters He laughingly stood before us.

32. The short letter “A” is His head. The long letter “A” is His forehead.
The letter “I” is His right eye and the letter “Ī” His left eye.

33. The letter “U” is His right ear and the letter “Ū” His left ear. The letter
“Ṛ” is the right cheek of that great lord.

34. “Ṝ” is His left cheek. The two letters “Ḷ” “Ḹ”, are His nostrils. The letter
“E” is His upper lip and the letter “AI” is His lower lip.

35. The letter “O” and the letter “AU” are respectively the two rows of his
teeth. The letters “AṂ” and “AḤ” [Anusvāra and Visarga] are his palates.

36. The five letters beginning with KA [\ie KA, KHA, GA, GHA and ṄA] are His
five hands on the right side. The five letters beginning with CA [\ie CA, CH, JA,
JHA and ÑA] are His hands on the left side.

37. Similarly the five letters beginning with ṬA and the five letters beginning
with TA constitute His legs. The letter PA is His belly and the letter PHA is
His right side.

38. The letter BA is His left side. The letter BHA is His shoulder. The letter
MA is the heart of the great yogin Mahādeva.

39. The letters YA, RA, LA, VA, ŚA, ṢA and SA are the seven Dhātus (vital
secretions) of the lord. The letter HA is His umbilicus and the letter KṢA is
His nose.

40. Viṣṇu and I became contented on seeing this letter-embedded form of
the \tl{saguṇa} manifestation of \tl{nirguṇa} lord in the company of Umā.

41. On seeing Lord Śiva in the form of the letter-embedded Brahman, Viṣṇu bowed
down along with me and looked up again.

42-47. The mantra beginning with Oṃkāra with its \tl{kalās} five in number,
consisting of the auspicious thirty-eight syllables, being pure as crystal,
increases intelligence and is an effective medium of accomplishing sacred rites.
The mantras in the Gāyatrī metre of twenty-four syllables and having four
\tl{kalās} are conducive to enjoyment. The five-syllabled mantra of eight
\tl{kalās} consisting of thirty syllables is employed for black magic. Mantras
of Yajurveda consisting of twenty-five syllables and eight \tl{kalās} are used
for conciliatory purpose. The mantra of thirteen \tl{kalās} consisting of
sixty-one syllables is conducive to outcome, increase and destruction.

48-49. The lord Viṣṇu secured these five mantras:—Mṛtyuñjaya mantra,
five-syllabled mantra, Cintāmaṇi mantra, Dakṣiṇāmūrti mantra and the
“\tl{tattvamasi}” mantra which is Hara’s Mahāvākya. Lord Viṣṇu performed
\tl{japa} by means of these mantras.

50-53. The lord Viṣṇu and I being glad at heart eulogised the boon-bestowing
lord Śiva with appropriate words,—Śiva who was seen in the form of \tl{kalās,
varṇas} (syllables), Ṛk, Yajus, Sāman, Īśāna, Īśa, Purātana Puruṣa (the ancient
Being), the merciful, pleasing to the heart, hidden from all, ever auspicious,
a great deity, of beautiful feet, bedecked with huge serpents, with legs, eyes
and hands extending on all sides, the lord of Brahmā, and the cause of creation,
sustenance and destruction of the world.

\chapter{Description of Śivatattva}

Brahmā said:

1. On hearing his own eulogy from the mouth of Viṣṇu, the delighted Śiva,
the store-house of kindness, revealed Himself to us along with his consort.

2-3. He had five faces and three eyes, and the crescent moon on his forehead.
He wore matted hair. He was white-complexioned and had wide eyes. His body had
been dusted with the ashes. He had ten arms. His neck was blue in colour. He was
bedecked with all ornaments. He was very handsome with respect to every limb.
Three ash-lines marked His forehead.

4. On seeing lord Śiva accompanied by His beautiful consort, Viṣṇu along with me
eulogised Him again with appropriate words.

5-6. Śiva, the merciful, who was delighted breathed the Vedas into Viṣṇu and
conferred perfect knowledge on him, the secret of the supreme Ātman. O sage,
thereafter, out of sympathy, the supreme Ātman conferred these on me too.

7. After receiving the Vedas, Viṣṇu was satisfied and bowing to Him with palms
joined in reverence along with me, he asked the lord Śiva.

Viṣṇu said:

8. O Lord, How are you propitiated? How shall I worship you, O lord? How shall
I meditate on you? How are you impressed by anyone?

9. O Great God, tell us what at Thy bidding shall we ever do? Please command us,
O Śiva, do this to favour us.

10. O Great lord, be merciful to tell us all these things. O Śiva, we are your
followers. Taking this into mind, you will enlighten us on these and other
similar points too.

Brahmā said:

11. On hearing these words, the lord Śiva was delighted. The merciful lord then
spoke lovingly.

Śiva said:

12. O foremost among gods, I am delighted by your devotion. Look upon me as
a great deity. Cost off all your fears.

13. Worship my \tl{liṅga} and do always meditate upon the form which you see
just before you.

14. When I am worshipped in the phallic form I will be delighted and will bestow
different benefits upon all people, all that they wish for in their minds.

15. O foremost among the deities, whenever any misery befalls you, it shall be
destroyed when my \tl{liṅga} is worshipped.

16. O strong ones, you two are born of my own \tl{prakṛti}, out of my left and
right sides. I am the lord of everything.

17. This Brahmā, grandfather for all people, is born of my right side. You,
Viṣṇu, are born of my left side. I am the supreme Ātman.

18-19. Delighted I shall confer on you boons and whatever you desire. May your
devotion to me be steady. With my permission you can make my form in clay and
perform adoration. After rendering different kinds of service like this sensibly
you shall attain happiness.

20. O Brahmā, strictly adhering to my direction you carry on the work of
creation. Dear child, dear Hari, you shall sustain the mobile and the immobile
beings.

Brahmā said:

21. Saying thus, the lord presented to us the auspicious mode of His worship,
adored duly by means of which Śiva confers many benefits.

22. On hearing the words of Śiva along with me, Viṣṇu bowed to Śiva with palms
joined in reverence and said.

Viṣṇu said:

23. “If you are pleased, if a boon is to be given to us, may our devotion to you
be perpetual and unstraying.

24. Although you are Nirguṇa, be pleased to incarnate in the course of your
divine sports and help us. Dear lord, you are great lord, the supreme.

25. O lord of lords, even our dispute has turned out to be auspicious, now that
you have come here to suppress the same”.

Brahmā said:

26. On hearing these words Śiva told Viṣṇu who stood there with the head bent
down and with palms joined in reverence.

Śiva said:

27. Although Nirguṇa, I am Saguṇa too and the author of dissolution, maintenance
and creation. I am the supreme Brahman without decay and change. Existence,
Knowledge and Bliss are my characteristics.

28. Truly, I am Niṣkala (Nirguṇa) for ever, O Hari. For the activities of
creation, maintenance and dissolution I manifest myself in the three forms of
Brahmā, Viṣṇu and Hara, O Viṣṇu.

29. O Viṣṇu, since you, along with Brahmā, have eulogised me and prayed for my
incarnation, I shall make that request true, favourably disposed towards my
devotees that I am.

30. A great form similar to this, O Brahmā, shall become manifest in the world
through your body. He will be called Rudra.

31. His capacity will never be less, since He will be my own part and parcel. He
is I. I am he. In the modes of worship too there in no difference.

32. As heat \etc in water and other things due to the contact of fire is not
permanent in water \etc, similarly my Nirguṇa aspect is not affected by
the external contact.

33. This form of mine as Śiva is that of Rudra too. O great sage, no one shall
make any difference in it.

34. The same form appears split into two in the universe. Hence Śiva and Rudra
shall not be considered different.

35. A piece of gold turned into an ornament does not cease to be gold. There may
be difference in name but not in the material content.

36. Just as the difference of clay and the various objects made of it is not
a material one, so also in this case. The presence of the material cause in
the effect can be cited as an example.

37. This shall be known by all scholars and Gods of unsullied knowledge. If you
realise this, you will not be seeing the cause of difference.

38-39. I think that we all should see the from of Śiva as the basic material.
Myself, you, Brahmā and Rudra who will be manifesting himself are of the same
form. There is no difference. If there had been difference that would have been
bondage. Yet the eternal Śiva-form is mine alone.

40. That pure form is spoken of as the main root, the Truth, the Knowledge,
the Endless. Realising this too, it must be meditated upon in the true manner
in your mind.

41. O Brahmā, another secret which I am going to unfold to you may be listened
to. You two are born of \tl{prakṛti} but not this one (Rudra).

42-43. My command is carried to that place through Brahmā’s eyebrows. I am
therefore spoken of as Tāmasa and Prākṛta, Hara in respect to the \tl{guṇas}
alone and shall be known as Vaikārika too which is actually the Ahaṃkāra
(the Ego). That is called \tl{tāmasa} only in name and not in reality.

44. For this reason, O Brahmā, this shall be carried out by you. O Brahmā, you
shall be the creator and Hari the protector.

45. My would-be part shall be the cause of dissolution. This goddess Umā,
Parameśvarī is the \tl{prakṛti}.

46. Her Śakti, the goddess of speech, shall resort to Brahmā. Another Śakti also
will be arising out of the \tl{prakṛti}.

47. That Śakti will resort to Viṣṇu in the form of Lakṣmī. Another Śakti Kālī
will surely share my part.

48. She will be born in the form of Brilliance for effective work. Thus I have
told you of the great auspicious Śaktis of the Goddess.

49. Their activities are respectively creation, maintenance and dissolution. O
foremost among Gods, they are the parts of \tl{prakṛti}, my beloved.

50-53. O Viṣṇu, you shall carry on your activities with the co-operation of
Lakṣmī. O Brahmā, with the cooperation of the goddess of speech, the part of
\tl{prakṛti}, you shall carry on joyfully the activity of creation, according to
my direction. I shall have the co-operation of Kālī, the part of my beloved,
the greatest of the great and shall carry out the excellent activity of
dissolution in the form of Rudra. You shall be happy after the creation of
the world consisting of the four \tl{varṇa} and their ancillaries—the four
\tl{āśramas} (stages of life) and various sorts of other incidental activities.
You shall contribute to the welfare of the world making use of your knowledge
and perfect wisdom.

54-55. O Viṣṇu, be the bestower of salvation too at my bidding. The benefit
accruing from your vision will be the same as that from mine. This boon is given
to you now. It is the truth, certainly the truth. Viṣṇu is in my heart and I am
in Viṣṇu’s heart.

56. Those who make any distinction between the two do not know my mind. Viṣṇu is
born of my left limb. Brahmā is born of my right limb.

57. Rudra who causes great dissolution and who is the soul of the universe is
born of the heart. I manifest in the three forms, O Viṣṇu, known Brahmā, Viṣṇu
and Bhava.

58. I am the author of creation, protection and dissolution by the attributes
\tl{rajas} \etc But I am different from these \tl{guṇas} and directly beyond
\tl{prakṛti} and Puruṣa.

59. I am the supreme Brahman, the eternal, the endless, the perfect and
the unsullied. Viṣṇu has \tl{tamas} within but \tl{sattva} outside. He is
the protector of the three worlds.

60. Hara who causes dissolution of three worlds has \tl{sattva} within but
\tl{tamas} outside.

61. Brahmā who creates the three worlds has \tl{rajas} both within and without.
This is the position of the \tl{guṇas} in the three deities. Śiva is spoken of
as different from the \tl{guṇas}.

62. O Viṣṇu, guard lovingly this Pitāmaha who is the cause of creation. At my
bidding, you will be worthy of respect in the three worlds.

63. Rudra shall be worshipped by you and Brahmā. The author of dissolution of
the three worlds is the complete incarnation of Śiva.

64. In the \tl{kalpa} called Pādma, Pitāmaha will be born as your son. Then you
will see me. The lotus-born Brahmā shall also see me.

65. After saying this and conferring unequalled mercy, the great lord Hara again
spoke lovingly to Viṣṇu.

\chapter{Description of Parama Śivatattva}

Lord Śiva said:

1. O Viṣṇu of good rites, O Hari, listen to another pronouncement of mine. You
will be worthy of honour and worship in all the worlds for ever.

2. Whenever a misery befalls the world created by Brahmā, you shall be eager for
the destruction of all miseries.

3. In all difficult and unbearable activities I shall help you. I shall kill
your indefatigable and fierce enemies.

4. O Viṣṇu, spread your glory in the worlds far and wide by taking up various
incarnations. Strive for their succour. I am always Saguṇa when I become Rudra
with this body.

5. Certainly I shall perform your activities for the sake of the worlds if they
are impossible for you.

6. You are to be meditated upon by Rudra and Rudra is to be meditated upon by
you. There is no difference between you and Rudra.

7. O Great Viṣṇu, your identity is due to inherent nature, the boons conferred
and your divine sports. True, it is certainly true.

8. If any devotee of Rudra were to censure you, he will have all his merits
reduced to ashes quickly.

9. O Viṣṇu, the most excellent of all persons, by hating you he will certainly
fall into hell. That is my directive. True, it is certainly true.

10. In this world, be the bestower of worldly pleasures and salvation to men
particularly. Worthy of being honoured and worshipped by devotees, perform
the activities of curbing and blessing.

11. Saying this and holding me, the creator, and Viṣṇu by the hand He
continued—“Always render help in distress.

12. Be the presiding deity of all. Bestow worldly pleasures and salvation for
ever. Be the most excellent accomplisher of the achievement of all desires.

13. You will assume the form of the vital airs in everyone at my bidding. O Hari,
in the times of distress, Rudra my manifestation, shall be worshipped.

14. He who has sought refuge in you has certainly sought refuge in me. He who
differentiates will certainly fall into Hell.

15. Listen to the span of life of the deities—Brahmā, Viṣṇu and Hara. There
shall not be any doubt in this respect.

16. A thousand sets of the four-\tl{yuga} periods constitute one day of Brahmā.
The period of night is also similar. Further measurement of time is based on
this calculation.

17. Thirty such days (days and nights) constitute one month and twelve months,
one year. The span of life of Brahmā is hundred such years.

18. One year of Brahmā constitutes one day of Viṣṇu. Viṣṇu lives for hundred
years in his own calculation.

19. One year of Viṣṇu constitutes one day of Rudra. When a hundred year period
passes, Rudra assumes the form of Nara (supreme Man).

20. He stays like that as long as the breath is retained by Sadāśiva. When He
exhales he merges into Śakti.

21-22. In the case of all living beings, Brahmā, Viṣṇu, Hara, \tl{gandharvas},
serpents, \tl{rākṣasas}, \etc, twenty one thousand six hundred respirations
constitute the period of one day and one night, O foremost among \tl{devas}.

23-24. Six respirations constitute the period of time one \tl{pala}. Sixty such
\tl{palas} constitute one \tl{ghaṭī}. Sixty \tl{ghaṭīs} constitute one day and
one night. (6 x 60 x 60 = 21 600). There is no limit to the number of
respirations of Sadāśiva. Hence He is undecaying.

25. It is my directive that you should preserve this form and maintain all
the activities of the created worlds by means of these different \tl{guṇas} so
long.”

26. On hearing these words of Śiva the lord Viṣṇu, of controlled senses, spoke
slowly to Śiva after duly bowing to Him.

Viṣṇu said:

27. “O Śankara, the ocean of mercy, the lord of the universe, be pleased to hear.
I shall do all these things strictly adhering to your behests.

28. I shall always meditate upon you. I would not act otherwise. Your
omnipotency has already been experienced by me.

29. O lord, let not the meditation of your form be ever far removed from my mind
even for a moment.

30. O lord, if anyone of my devotees were to censure you, you will please assign
perpetual residence in hell to him.

31. O lord, whoever be your devotee shall also be my favourite. He who knows and
realises this shall not find salvation inaccessible to him.

32. My greatness has been further heightened by you certainly. If ever I am
found deficient in qualities I may be excused.

33. (Brahmā said:—) Then, on hearing the excellent words of Viṣṇu, Śiva said to
him “Of course the deficiency shall be excused lovingly.”

34. After saying this mercifully the lord, the ocean of mercy, stroked us all
over the body with His hands.

35. With a desire to do what is wholesome to us He instructed us in various
sacred rites and conferred on us many boons.

36. Even as we were watching, the lord Śiva, favourably disposed towards
devotees, vanished immediately.

37. The rite of the worship of the \tl{liṅga} is instituted ever since in
the world. Śiva installed in the \tl{liṅga} bestows worldly pleasures and
salvation.

38. The great goddess is the pedestal for the \tl{liṅga}. The liṅga is Lord
Śiva Himself. Since the whole universe finally merges into in, it is called
Liṅga.

39. He who reads regularly this anecdote of the \tl{liṅga} in the vicinity of
the \tl{liṅga} assumes the form of Śiva within six months. There need be no
hesitation in this respect.

40. O great sage, I cannot adequately express the blessedness accruing to
the person who carries on any activity whatsoever in the vicinity of
the \tl{liṅga}.

\chapter{The mode of worshipping Śiva}

The sages said:

1. O Sūta the fortunate, O Sūta the disciple of Vyāsa, obeisance be to you. This
wonderfully sanctifying story of Śiva has been narrated today.

2. The wonderful and highly divine origin of the \tl{liṅga} has been heard.
Listening to its efficacy causes destruction of misery.

3. O store house of mercy, please tell us the mode of the worship of Śiva, in
accordance with the conversation of Brahmā and Nārada whereby Śiva becomes
satisfied.

4. \tl{Brāhmaṇas}, \tl{kṣatriyas}, \tl{vaiṣyas} and \tl{śudras} worship Śiva.
How shall the worship be performed? Please tell us in accordance with what you
have heard from Vyāsa.

5. On hearing their words, Sūta narrated everything in answer to the question of
the sages, everything conducive to welfare and in accordance with the Vedas.

Sūta said:

6. O lordly sages, your enquiry covers a very great secret topic. I shall
explain it as far as my own intellect can penetrate it and in accordance with
what I have heard.

7-8. Formerly Vyāsa had asked the same question of Sanatkumāra as you have asked
now. Upamanyu learnt it from him. Vyāsa heard it from him and taught me the mode
of worship \etc of Śiva from a desire for the benefit of all the worlds.

9. It was directly heard from Upamanyu, the noble soul, by Kṛṣṇa (\ie
Kṛṣṇadvaipāyaṇa or Vyāsa). That I shall tell you in the some way as Brahmā had
said before.

Brahmā said:

10. O sage Nārada I shall explain briefly the worship of the \tl{liṅga}
(the phallic image). It is impossible to explain it in detail even in a hundred
years.

11. In order to achieve the fulfilment of all desires one should worship with
great devotion the pure and eternal form of Śiva thus.

12. Poverty, sickness, harassment from enemies and the four sorts of sins
trouble one only as long as one does not worship Siva.

13. When Śiva is worshipped, all miseries merge into the lord; all happiness is
secured and salvation is attained thereafter.

14. Śiva who secures the achievement of all matters shall be worshipped by
the person who considers a continuous series of human pleasures very important.

15. Whether they are \tl{brāhmaṇas}, \tl{kṣatriyas}, \tl{vaiśyas} or \tl{śūdras},
they shall perform the worship of Śiva duly and regularly for the achievement of
all desired objects.

16-18. One shall get up early in the morning during the \tl{brāhma muhūrta}
\footnote{It is the period between the fourth and the second \tl{ghaṭikas}
before sunrise. “\tl{rātriśca paścime yāme muhūrto brāhmya ucyate}”}
(about an hour before dawn). He shall remember the preceptor and Śiva. O sage,
he shall then remember the holy centres and meditate on Hari. Thereafter he
shall remember me, the deities and the sages. Then he shall recite a prayer in
the name of Śiva duly. Then he shall get up and evacuate his bowels in southern
quarter.

19. The evacuation of the bowels shall be done in an isolated place. What I have
heard (in this respect) I am mentioning now. O sage, please listen attentively.

20. A \tl{brāhmaṇa} shall use earth for cleaning purposes five times;
a \tl{kṣatriya} for four times and a \tl{vaiśya} for three times.

21. A \tl{śūdra} shall use the earth twice for cleaning purposes. Or he shall
cleanse the rectum once and the penis once assiduously.

22. He shall then wash the left hand ten times. He shall then wash each of
the feet seven times and both the hands three times once again.

23. Women shall perform these cleansing activities with earth like \tl{śudras}.
They shall first wash the hands and feet, then make use of the earth as before.

24. They shall clean the teeth using the tooth brush twig according to their
castes.

25-26. The tooth brush twig of a \tl{brāhmaṇa} shall be twelve \tl{aṅgulas} long.
A king (a \tl{kṣatriya}) shall take one eleven \tl{aṅgulas} long and
a \tl{vaiśya} one ten \tl{aṅgulas} long. The tooth brush of a \tl{śūdra} shall
be nine \tl{aṅgulas} in length. This is in accordance with Smṛtis. What is
enjoined by Manu shall be disobeyed only in emergencies.

27. On \tl{ṣaṣṭī} (sixth), \tl{navamī} (ninth) and new-moon days, on sundays and
days of sacred rites and \tl{śrāddhas}, cleaning the teeth with tooth-brush twig
is prohibited.

28. The daily ablutions shall be performed duly and those in holy centres shall
be performed with mantras in accordance with the time and place.

29. Performing the \tl{ācamana} first, wearing washed cloth, he shall perform
the \tl{sandhyā} prayer in a good isolated place.

30. After observing the preliminaries duly he shall enter the chamber of worship
keeping the mind steady and begin the rites of worship.

31. Sitting on a good seat and performing \tl{nyāsa} \etc in accordance with
the prescribed rules of worship, he shall perform the worship of Śiva duly.

32. Gaṇeśa, the attendants at the threshold of the temple, the guardians of
the quarters \etc, shall be worshipped and thereafter the pedestal shall be
arranged.

33-36. Or he shall make the mystical diagram of the lotus of eight petals and
install Śiva in its middle. He himself shall sit near all the materials of
worship around him. He shall perform \tl{ācamana} thrice and wash the hands. He
shall then perform suppression of breath (\tl{prāṇāyāma}) thrice. Then
\tl{tryaṃbaka} (three-eyed Śiva) shall be meditated upon in the following manner.
The deity has five faces, ten arms, all kinds of ornaments and the tiger-hide as
His upper-cloth. He is as pure as the crystal. During meditation he shall
identify himself with Śiva and burn off his sins. Having thus created the form
of Śiva in meditation, he shall worship lord Śiva.

37. Then the ritualistic purification of the body by touching the various parts
of the body with holy water shall be performed. The \tl{nyāsa} of
the \tl{mūlamantra} (the root mantra) and that of the six \tl{aṅgas} with
\tl{praṇava} (Oṃkāra) shall be performed thereafter.

38. After ritualistically touching the heart, he shall start worship. Different
vessels shall be set apart for \tl{pādya} (water for washing the feet),
\tl{arghya} (water for the reception of the guest) and \tl{ācamana} (sipping
water).

39-40. Nine vessels of different sizes should be kept by the sensible devotee.
\tl{darbha} grass shall be spread and cool water sprinkled over these vessels
with \tl{darbha} grass. Reciting the \tl{oṃkāra}, the intelligent devotee shall
sprinkle the various materials of worship.

41-42. The fragrant root of the plant \tl{uśīra} and sandal paste shall be put
in the water for washing feet. Fine powders of \tl{jātī}, \tl{kaṃkola},
\tl{karpūra}, root of \tl{vaṭa} and \tl{tamālaka} should be put in the water
intended for sipping. Sandal powder shall be put in all these nine vessels.

43. Nandīśa, the divine Bull of Śiva shall be worshipped beside the lord Śiva.
The latter shall be worshipped with scents, incense and different lamps.

44-47. The \tl{liṅga} shall be purified and installed with various mantras
beginning with \tl{praṇava} and ending with \tl{namaḥ} (obeisance). The pedestal
in the form of \tl{svastika} or lotus shall be assigned with \tl{praṇava}. In
the eight petals, in the eight quarters, the eight achievements are identified
\viz—The eastern petal is \tl{aṇimā} (minuteness), the southern is \tl{laghimā}
(lightness), the western is \tl{mahimā} (greatness) the northern is \tl{prāpti}
(power of reaching), the south-eastern is \tl{prākāmya} (power of sufficiency),
the south-western is \tl{īśitva} (lordliness); the north-western is \tl{vaśitva}
(power of control), the north-eastern is \tl{sarvajñatva} (omniscience) and
the pericarp is the moon (Soma).

48. Beneath the moon is the sun and beneath that is the fire. \tl{Dharma} \etc
are beneath that. All these shall be assigned regularly.

49-50. In the four quarters \tl{avyakta} \etc the unmanifest principle and in
the end of Soma the three \tl{guṇas} shall be assigned. Lord Śiva shall be
invoked by the formula “I am addressing Sadyojāta”\footnote{
\tl{Vājasaneyisaṃhitā} (of \tl{Śuklayajurveda}) 29.36.}.Then the devotee small
repeat \tl{vāmadeva}\footnote{TA. 10.41.1.} mantra and stand on his seat.
The \tl{sānnidhya} rite shall be performed with \tl{rudra gāyatrī}
\footnote{Kāṭhaka-saṃhitā 17.11.} mantra and the rite of \tl{nirodha} shall be
performed with \tl{aghora}\footnote{\tl{Vājasaneyisaṃhitā} (of
\tl{Śuklayajurveda}) 16.2.} mantra.

51. Rudra shall be worshipped with the mantra \tl{iśānaḥ sarvavidyānām}
\footnote{\tl{Vājasaneyisaṃhitā} (of \tl{Śuklayajurveda}) 27.35.}, \tl{pādya},
\tl{ācamanīya} and \tl{arghya} shall be offered duly.

52. Rudra shall be duly bathed with water, scented with sandal in the same
manner as with \tl{pañcagavya} after taking it in a vessel duly instilled with
mantras.

53. Then the deity shall be bathed invoking \tl{praṇava} with cow’s milk, curds,
honey and sugarcane juice.

54. Worshipping Rudra who bestows everything that is wholesome and desirable
with ghee, the devotee shall perform the \tl{abhiṣeka} with all materials of
worship reciting \tl{praṇava}.

55. In the holy vessels full of water he shall pour water reciting various
mantras after straining it with a white cloth duly.

56. The sprinkling need not be performed until sandal paste is mixed. Then raw
rice grains made beautiful (by adding turmeric powder \etc) shall be offered
joyously to Śaṅkara.

57-58. Offerings of flowers, especially white flowers and rare flowers, shall be
made to Lord Śiva. Flowers of \tl{apāmārga, karpūra, jātī, campaka, kuśa, pāṭala,
karavīra, mallikā, kamala} (lotus) and \tl{utpalas} (lilies) of various sorts
shall be used. When water is poured it shall be poured in a continuous stream.

59. Vessels of different varieties shall be used for the ceremonial ablution of
Lord Rudra. A worship performed with due recitation of mantras bestows all
benefits.

60. O dear one, I shall tell you briefly those mantras for the sure achievement
of all cherished desires. Please listen attentively.

61-65. Offerings of flowers and water ablutions shall be made with these mantras
whether caused to be read or committed to memory and orally repeated—The Rudra
mantra, Nīlarudra mantra, Śukla Yajurveda mantras, auspicious Hotṛ mantras,
Atharvaśīrṣa mantras, Śānti mantras, Maruta mantras, Sāmaveda mantras, if
desired, Devavrata mantras, Rathantara mantras with Puṣpa Sūktas, Mṛtyuñjaya
\footnote{Vājasaneyisaṃhitā (of Śuklayajurveda) 3.60.} mantras and
the five-syllabled mantra. The water offerings shall be a thousand times or
hundred and eight times. They shall be offered strictly in accordance with Vedic
injunctions or by repeating the names of the deity.

66. Sandal paste shall be applied to the deity and flowers placed over the idol.
Sweet smelling cloves \etc shall be offered with \tl{praṇava}.

67-72. \tl{Śivaliṅga} shall be worshipped next. The lord as pure as crystal,
the unsullied, the undecaying, the cause of all worlds, the supreme lord
identifying with the created world, the lord who cannot be seen by Brahmā, Indra,
Upendra, Viṣṇu and other deities, the lord who is mentioned in the Vedānta by
those who know Vedas as the Incomprehensible, the lord who has no beginning,
middle or end, the panacea for all sick patients and who is renowned as Śiva
Tattva. The worship of the \tl{liṅga} shall be performed, by \tl{praṇava} mantra
alone. Incense, lamps, \tl{naivedyas}, good betel leaves, pleasant \tl{nirājana}
(waving of lights) shall be duly offered. Prayers, obeisance \etc with various
mantras shall be performed. \tl{Arghya} and flower offerings shall be made at
the foot. The devotee shall kneel down and devoutly pray to the lord.

73. The devotee shall take some flowers in his hands, stand up with palms joined
in reverence and repeating the following mantra shall pray again to Īśāna,
Śaṅkara:

\begin{shloka}\itshape
  ajñānādyadi vā jñānājjapapūjādikaṃ mayā\\
  kṛtaṃ tadastu saphalaṃ kṛpayā tava śaṃkara
\end{shloka}

74. O Śiva, may this \tl{japa}, \tl{pūjā} \etc performed by me with or without
the requisite knowledge be fruitful, thanks to Thy grace.

75-76. After repeating the above mantra he shall place the flowers joyously over
the \tl{śivaliṅga}. Then the rites of \tl{svastyayana}\footnote{Ibid. 1.86.6.},
\tl{āśīrvāda} (benediction), \tl{mārjana} shall be performed. Then Homage,
a prayer for forgiveness and \tl{ācamana} shall be performed.

77-78. Repeating the \tl{agha}\footnote{Ibid. 20.29.} mantras for the expiation
of sins \tl{namaskāra} shall be duly performed. He shall pray with devout
feelings:

\begin{shloka}\itshape
  śive bhaktiśśive bhaktiśśive bhaktirbhave bhave\\
  anyathā śaraṇaṃ nāsti tvameva śaraṇaṃ mama
\end{shloka}

“Devotion to Śiva, devotion to Śiva, devotion to Śiva in every birth. I have no
other refuge. You alone are my refuge.”

79. After praying thus to the lord of the Gods, the bestower of all achievements,
the devotee shall loudly pray.

80. He shall then perform \tl{namaskāra} along with the members of his family.
He shall feel delighted in all these and thereafter carry on his daily routine
according to convenience.

81. He who performs the worship regularly like this with great devotion to Śiva
shall achieve success at every step.

82-83. He will become eloquent. He will achieve all he desires. The Supreme lord
Śiva will quell all his miseries, ailments, sorrows, heart-burns, crookedness,
poisonings and everything distressing quickly.

84. Just as the moon waxes in the bright half, his joy and merits shall increase
day by day certainly by the worship of Śiva.

85. O foremost among sages, thus I have told you the mode of worship of Śiva.
O Nārada what else do you wish to hear?

\chapter{Consideration of the essential and the non-essential in the worship}

Nārada said:

1. O dear father Brahmā, with your mind fixed on Śiva, you are blessed indeed.
Please explain this again still more precisely.

Brahmā said:

2. I, the lotus-born, once called together all the sages and all the Gods and
addressed them lovingly with these good words.

3. If you have faith in permanent happiness, if you desire the achievement of
the same, all of you shall come along with me to the shores of the milk-ocean.
\footnote{According to the Paurāṇic concept, the turbulent and foamy sea known
as the southern China Sea which surrounds Śākadvīpa (identified with Malaya,
Siam, Indo-China and Southern China) on three sides was called ‘the sea of milk’
or Kṣīra Samudra: cp. SM. Ali: \emph{Geography of the Purāṇas}.}

4. On hearing these words they accompanied me to the place where lord Viṣṇu,
the benefactor of everyone, was stationed.

5. O sage, on reaching the place, the Gods bowed down with palms joined in
reverence and prayed to the lord of the universe Janārdana, lord of the Gods.

6. On seeing Brahmā and other deities standing there, Viṣṇu remembered
the lotus-like feet of Śiva and spoke these noble words.

Viṣṇu said:

7. “Why have you all, Brahmā and others and the celestial sages come? What is
the matter now? Please tell me lovingly.”

Brahmā said:

8. On being asked thus by Viṣṇu as well as by me, the deities bowed to Him with
devotion and said.

The \tl{devas} said:

9. “Whose worship shall we perform regularly for the removal of misery?”

10. On hearing these words, the lord favourably disposed to the devotees, spoke
as follows favouring me and the \tl{devas}.

The lord said:

11. O Brahmā, hear. You and these \tl{devas} have already heard this. Yet I
shall repeat it to you and to the \tl{devas}.

12-13. It has been seen. It is being seen. Then why is it being asked now?
O Brahmā, Lord Śiva, the destroyer of all miseries, shall be served always by
all who wish to achieve things. He Himself has told me as well as Brahmā
particularly about this.

14. His worship shall never be forsaken by those who wish to attain happiness.
A wonderful example has been narrated to and seen by you all.

15. When they abandoned worshipping the lord of the \tl{devas}—Maheśvara in
the form of the Liṅga, the sons of Tāra\footnote{Tāraputras—the children of
Daitya Tāraka who was conquered by Indra with the help of Skanda—the son of Śiva.
The episode is the central theme of Kālidāsa’s Kumārasambhava.} along with their
kinsmen perished.

16. They had been enchanted by me. By my illusion they were driven far by me.
When they were devoid of Śiva, they were all destroyed and exterminated.

17. Hence Śiva in the form of phallic image shall be worshipped always. He,
the foremost among deities, shall be served with special faith.

18. It is by the worship of the \tl{liṅga} of Śiva that all good men, \tl{devas},
daityas, I and you, O Brahmā, are sustained. How is it that it was forgotten
by you?

19. Hence, O Brahmā, His \tl{liṅga} shall be regularly worshipped whatever may
be the aim. Śiva shall be worshipped whatever the desire may be.

20. If an hour or even a moment is spent without the worship of Śiva, it is
a loss. It is an imperfection, a great foible, blindness, stupidity and
foolishness.

21. Those who are devotedly attached to Śiva, those whose minds are turned
towards Śiva and those who constantly remember Śiva, never become victims
of misery.

22-24. Those who desire magnificent buildings, beautiful ornaments, beautiful
women, wealth to satiety, sons and grandsons, health, splendid body,
extraordinary status, heavenly happiness and final salvation or profound
devotion to the great lord shall duly worship Śiva by virtue of their merit
accumulated by them.

25. Sure success will be his who regularly worships Śiva \tl{liṅga} with great
devotion. He will never be afflicted by sins.

Brahmā said:

26. Thus exhorted, the \tl{devas} knelt before Viṣṇu and requested for
\tl{liṅga} for the achievement of the desires of all people.

27. O foremost among sages, then, on hearing the request, Viṣṇu, eager for
the uplift of all living beings, told Viśvakarman. I too told him.

28. “O Viśvakarman, at my bidding, Śiva’s auspicious \tl{liṅgas} shall be made
and given to all \tl{devas}”.

29. At our bidding Viśvakarmā made \tl{liṅgas} and gave them to the \tl{devas}
according to their status.

30. O foremost among sages, I shall tell you the same, please listen. Indra took
a \tl{liṅga} made of ruby. The son of Viśravas (Naiśravaṇa or Kubera) took
a \tl{liṅga} of gold.

31. Dharma took a \tl{liṅga} of yellow stone, Varuṇa took a \tl{liṅga} of dark
blue hue. Viṣṇu took a \tl{liṅga} of sapphire. I, Brahmā, took a \tl{liṅga}
of gold.

32. The \tl{viśvedevas} and the \tl{vasus} took silver \tl{liṅgas}. O sage,
the Aśvini \tl{devas} took the brazen and earthen \tl{liṅga}.

33. Goddess Lakṣmī took a crystal \tl{liṅga}. The Ādityas (the twelve suns) took
\tl{liṅgas} made of copper. The moon took a \tl{liṅga} made of pearl and
the god of fire took a \tl{liṅga} of diamond.

34. Great \tl{brāhmaṇas} and their wives chose \tl{liṅgas} of earth. Maya took
a \tl{liṅga} of sandalwood and Śeṣanāga took a coral-made \tl{liṅga}.

35. The Goddesses took the \tl{liṅgas} of butter; the Yogins took \tl{liṅgas} of
the ash; the Yakṣas took \tl{liṅgas} of curd and the deity Chāyā took
a \tl{liṅgas} of beaten flour.

36. The Goddess Brahmāṇī worships, of course, the \tl{liṅga} of \tl{ratna}
(precious gem). Bāṇa and others worshipped a \tl{liṅga} of mercury.

37. Thus different kinds of \tl{liṅgas} were given to them by Viśvakarmā which
the \tl{devas} and the celestial sages worship regularly.

38. After giving the \tl{devas} the various \tl{liṅgas} from a desire for their
benefit, Viṣṇu explained the mode of worship of Śiva to me, Brahmā.

39. After listening to it, I, Brahmā, the foremost among \tl{devas}, came back
to my abode highly delighted in mind.

40. O sage, after reaching the place I explained the mode of worshipping Śiva
that yields desires to the \tl{devas} and sages.

41. “O sages and \tl{devas}, be pleased to hear with love and pleasure. I am
going to explain lovingly the mode of worshipping Śiva that confers worldly
pleasures and salvation.

42-43. The life as a human being is very difficult to obtain among all living
beings. O \tl{devas}, O sages, a life in a good family is still more difficult.
After obtaining the still more difficult birth in a \tl{brāhmaṇa} family of
good conduct on account of great merits one shall perform rites assigned to
propitiate Śiva.

44. No one shall transgress duties assigned to his caste. Charitable gifts and
sacred rites shall be performed to the extent of one’s capacity and affluence.

45. The \tl{tapoyajña} (sacrifice in the form of penance) is far superior to
thousands of \tl{karmayajñas} (ritualistic sacrifices). The \tl{japayajña}
(sacrifice in the form of \tl{japas}) is far superior to thousands of
\tl{tapoyajñas}.

46. There is nothing superior to \tl{dhyānayajña} (meditation) which is
the cause of true knowledge, since the \tl{yogin} is able to see his favourite
(deity) of equanimity through meditation.

47. Śiva is always present near a person set in meditation. There is no
necessity for any atonement or expiation for a person of true knowledge.

48-49. O gods, persons who have realised Brahman through pure learning need not
perform any rite. They are freed from happiness or misery, virtue or evil,
sacrifice or \tl{japa}, meditation or rules regarding the same. By virtue of
their learning they are free from base passions and physical changes and decays.

50. The \tl{liṅga} present in the hearts of \tl{yogin} is the purest, blissful,
auspicious, undying, all-pervasive and unsullied.

51. O \tl{brāhmaṇas}, \tl{liṅga} is of two types: the exterior and the interior.
The exterior is gross and the interior is subtle.

52. Those who are engaged in ritualistic sacrifices and do regularly worship
the gross \tl{liṅga} are unable to steady the mind by meditating upon the subtle
and hence they use the gross \tl{liṅga}.

53. He who has not mastered the \tl{liṅga} of the mind, the subtle one, must
perform the worship in the gross \tl{liṅga} and not otherwise.

54. The pure undying subtle \tl{liṅga} is ever perceived by the masters of true
knowledge in the same manner as the gross one is thought to be very excellent by
those who are not yogins.

55. If we consider properly there is nothing else for the real interpreter.
Whatever is Niṣkala or Sakala is of the form of Śiva in the whole universe. This
must be constantly thought of in the mind.

56. Even if they are devoid of the ultimate perfect knowledge, no defect or
deficiency can be ascribed to them. Rules regarding what shall be done and what
shall not be done are not binding on them.

57. The knower, of course, is not at all bound by actions, even if he continues
the householder’s life just as the lotus standing in water is not contaminated
by the water.

58. Till the realisation of perfect knowledge a man should continue
the ritualistic worship of Śiva.

59-60. In order to convince the world, the rituals must be continued. Just as
the sun is reflected in many vessels with water, in the same manner, O \tl{devas},
know that the supreme Brahman, Śiva, assumes the forms of whatever is seen or
heard in the world, real or unreal.

61. There is difference in the vessels but not in the water that they contain.
This is what those who know the real meaning of the Vedas say.

62. “Lord Śiva is within the heart of beings in this world.” Of what avail are
the idols to those who have this real knowledge?

63. Having an idol is very auspicious for a person who has no such knowledge.
It is a ladder that enables him to climb to a higher position.

64. It is very difficult to climb to a position without a support. The idol is
only a means to achieve the Nirguṇa Śiva.

65. The attainment of the Nirguṇa through a Saguṇa is certainly possible. In
this manner, the symbols of all lords are conducive to a steady faith and belief.

66. This lord is very great and this is the mode of worship of that lord. If
there is no idol, of what avail are scents, sandal paste, flowers, \etc?

67. Till the realisation of true knowledge, the idol shall necessarily be
worshipped. If any one does not worship the idol before he attains perfect
knowledge, his downfall is sure.

68. O \tl{brāhmaṇas}, hear the true statement of facts. For the same reason as
mentioned before, the duties of your own caste shall be performed assiduously.

69. Worship shall be performed where devotion is directed. Without worship and
charitable gifts, sin cannot be kept at bay.

70. As long as there is a vestige of sin in the body, achievement need not be
expected. When the sin is wiped off, all rites will bear fruit.

71. If there is dirt in the cloth the dyeing process cannot be carried out
effectively. After the cloth is bleached any dye can be applied to it
effectively.

72. Similarly when the body is freed of its dirty stuff by proper worship of
deities, the dye of knowledge can stick to it whence true knowledge will arise.

73. The root of true knowledge is unswerving devotion. The root of knowledge too
is devotion.

74. The root of devotion is good action and the worship of one’s own favourite
deity. The root of that is the good preceptor. A good preceptor is secured only
through association with good people.

75. If one associates with good people, one will come across a preceptor. From
the preceptor mantras and the modes of worship can be learned. \tl{Bhakti}
(devotion) is generated by worship and it gives birth to knowledge.

76. Knowledge leads to perfect knowledge and realisation of the supreme Brahman.
When there is perfect knowledge, differentiations cease altogether.

77. When differentiation ceases, the misery of mutually clashing opposites
vanishes. He who is free from the tangle of opposites and the miseries attendant
on them assumes the form of Śiva.

78. O celestial sages, when the mutually clashing opposites do not afflict,
a person endowed with true knowledge has neither happiness nor misery. Rules of
do’s and don’ts do not bind him.

79. Such a person who has not entered a household life is rare to meet with. If
there is such a one he will quell all sins by his mere sight.

80. Even the holy centres praise such a person of knowledge. The \tl{devas} and
all sages consider him the supreme Brahman, Śiva Himself.

81. The holy centres or the deities in the form of clay or rock idols are not
equal to him. They take time in sanctifying persons. But a man of true knowledge
purifies through his sheer vision.

82. As long as he continues the life of a householder he shall perform
the worship of the idols of the most excellent of the five deities with pleasure.

83. Or it is enough if Śiva alone is worshipped. The root is the most important.
When the root is watered, O gods, the branches are well-cared for.

84. O excellent sages, if the branches are taken care of, it does not
necessarily mean that the root is cared for. When the deities are propitiated,
the same analogy holds good.

85-86. Our aim shall be to propitiate Śiva if we are sensible. O gods, if Śiva
is worshipped, all the gods are worshipped. Hence a person who wants to do good
to all living beings shall worship Śiva, the benefactor of the world, for
the attainment of all desires.

\chapter{The mode of worshipping Śiva}

Brahmā said:

1. O sages, O \tl{devas}, listen. Now I shall explain a mode of worship than
which there is no better one and which is conducive to the achievement of all
happiness and cherished desires.

2. Getting up in the \tl{brāhma muhūrta} within an hour before dawn one shall
remember Śiva accompanied by his consort. With palms joined in great devotion
and head bent down he shall offer prayers:

\begin{shloka}\itshape
  uttiṣṭhottiṣṭha deveśa uttiṣṭha hṛdayeśaya\\
  uttiṣṭha tvamumāsvāminbrahmāṇḍe maṃgalaṃ kuru
\end{shloka}

3. O lord of \tl{devas}, get up, get up. O lord stationed in the heart, get up.
O lord of Umā, get up. Confer your auspicious blessings on the entire universe.

\begin{shloka}\itshape
  jānāmi dharmaṃ na ca me pravṛttirjānāmyadharmaṃ na ca me nivṛttiḥ\\
  tvayā mahādeva hṛdisthitena yathā niyukto'smi tathā karomi
\end{shloka}

4. I know what is virtuous, but I am not inclined to work it up. I know what is
unrighteous but I am unable to desist from it. O Mahādeva, I do everything as
prompted by you, stationed in my heart.

5. After repeating these words of prayer and remembering the sandals of
the preceptor he shall go out to the southern direction for answering the calls
of nature.

6. Cleaning the body thereafter with earth and water and washing his hands and
feet he shall clean the teeth.

7. Cleaning of the teeth shall be completed before sunrise. He shall gargle
sixteen times with so many mouthfuls of water.

8. O celestial sages, the \tl{tithis} of \tl{ṣaṣṭhī, navamī} as well as new moon
days and sundays are forbidden for cleaning the teeth with tooth brush twigs.

9. Bath shall be taken at a convenient time in rivers or in the house itself. No
man shall take bath against the conventions of locality or the convenience of
the season.

10-11. Hot water bath shall be avoided on sundays, \tl{śrāddha} days,
\tl{saṅkrānti} days, at the times of eclipse, on days of Great Charity and fast,
in holy centres and during the days of impurity due to death or birth in
the family. In the holy ponds and rivers one shall take bath facing the east
with great devotion.

12. Oil bath shall be taken on particular days of the week according to
convention in the society. If one is accustomed to take oil bath everyday or if
one is using scented oil breaking the convention, it is not faulty.

13. Otherwise one should avoid \tl{śrāddha} days, days of eclipse, fast days and
the first day of the lunar fortnight for oil baths. Except on the days of
eclipse mustard oil can be used on other days.

14. Bath shall be taken after due consideration of the place and season duly. He
shall face either the north or the east when taking bath.

15. He shall never take bath wearing another man's clothes. He shall take bath
in pure clothes and shall think on his favourite deities.

16. If he wears during the night another man’s clothes, the same are not impure,
hence there is no harm in taking bath with those clothes on but after taking
bath they must be washed and returned.

17. After bath he shall perform water libation propitiating gods, sages and
the manes. Thereafter washed and dried clothes shall be worn and \tl{ācamana}
performed again.

18. In a clean place washed and smeared with cow-dung, the devotee shall take
his seat, O \tl{brāhmaṇas}.

19. The seat shall be made of wood or a cloth-cover. A seat of diverse colours
is conducive to the achievement of all desires.

20. Or he can have the hide of a deer for a seat. He shall sit on it and apply
\tl{tripuṇḍra} with the ashes.

21. Prayers, penance and charity shall be performed with due markings of
\tl{tripuṇḍra} on the forehead for sure results. If ashes are not available
marking may be done with holy water.

22. After marking \tl{tripuṇḍra}, on the forehead, the devotee shall wear
\tl{rudrākṣas}. After daily rites are over, he shall begin the worship of Śiva.

23. Then he shall perform \tl{ācamana}, the sipping of water thrice with
the requisite mantras or once, saying that it is a drop of Gaṅgā water.

24-25. Rice cooked with water shall be brought for the worship of Śiva. Whatever
other things he can bring shall also be brought and kept near. A vessel for
\tl{arghya} with water and scented raw rice grains shall also be brought.

26-27. To complete the formalities of worship, the vessel shall be placed on
the right shoulder. He shall think upon the preceptor and ritualistically take
his permission for the worship. He shall perform the rite of \tl{saṃkalpa}
(including the requisite mantras and statements about the \tl{pūjā}, the day,
month, year \etc and the purpose of the \tl{pūjā}) and aver his desire. He shall
perform the worship of Śiva with His attendants devoutly.

28-29. Showing the mystic \tl{mudrā} and using saffron and other materials he
shall bow to and worship Gaṇeśa who confers benefits a hundred thousand times
and is accompanied by his consorts Siddhi and Buddhi\footnote{Siddhi and Buddhi
are personified as the wives of Gaṇeśa, the son of Śiva and Pārvatī.}. He shall
repeat his names ending in the dative case appended with \tl{namaḥ} and prefixed
with \tl{praṇava}.

30. After craving for forgiveness of the deity, he shall be worshipped again in
the company of his brother Kārtikeya with great devotion and shall be bowed to
again and again.

31. The big-bellied Gaṇeśa, the gate-keeper of the lord, shall be worshipped.
Goddess Satī, Girijā shall be worshipped then.

32-35. After worshipping Śiva with sandal paste, saffron, incense, various lamps,
and food-offerings of different sorts he shall bow down again. In the house
the \tl{liṅga} shall be made of clay, silver or any other metal or mercury. It
shall be bowed to with devotion. If that is worshipped, all deities are
worshipped. If the \tl{liṅga} is made of clay it shall be installed duly.

36. The householders shall perform every rite according to prescribed rules.
After performing the purificatory rite of the \tl{bhūtas}, the installation of
the idol shall be performed.

37-38. If the worship is performed in the temple of Śiva, the guardians of
the quarters shall be installed and worshipped. In the house, Śiva shall be
worshipped by the root mantra. It is not obligatory that the gatekeeper shall
be worshipped. The \tl{liṅga} that is worshipped by me can be worshipped in
the house. Everything is installed in the same.

39. At the time of worship, the lord shall be invoked along with his attendants
and paraphernalia. But there is no hard and fast rule governing this aspect.

40. He shall provide his own seat in the vicinity of Śiva. He shall face
the north and perform the rite of \tl{ācamana} (sipping water).

41. The devotee shall wash his hands and feet and perform \tl{prāṇāyāma} ten
times with \tl{mūlamantra}.

42. Five mystic \tl{mudrās} shall be shown with the hand before the worship.
Only after showing the \tl{mudrās} shall the worship be performed.

43-45. The lamp shall be shown then. Homage shall be paid to the preceptor. He
shall then seat himself in the yogic poses of Padma, Bhadra, Uttāna or Paryaṅka
whichever is convenient and perform the rites once again. After the worship he
shall float it along with the cake. If the worship is performed in the house
these rules are not binding.

46. Afterwards the excellent \tl{liṅga} shall be washed with the water from
the vessel of \tl{arghya} itself after keeping all the material with the
concentrated mind.

47-53. The lord shall be invoked then with the following mantra: “I am invoking
Śiva, the blissful and favourably disposed to the devotees, Śiva seated on
the summit of Kailāsa, the excellent lord of Pārvatī, Śambhu of the form as
mentioned before, both with or without qualities possessed of five faces, ten
hands, three eyes and the bull for banner, as white as camphor, of divine limbs,
having crescent moon on the head, wearing matted hair, clad in the hide of an
elephant and with the hide of the tiger as upper garment, with Vāsuki and other
serpents turned round his body, holding Pināka and other weapons, having the
eight \tl{siddhis} (accomplishments) dancing constantly in front of Him, served
by crowds of devotees crying loudly “Be victorious. Be victorious.” of unbearable
sight due to excessive splendour, served by all \tl{devas}, the sole refuge for
all living beings, of beaming face shining like lotus and always eulogised by
Viṣṇu and Brahmā as extolled by the Vedas and sacred text.” After the meditation
of Śiva along with his consort, the seat shall be arranged for.

54. Worship shall be performed with the names ending in dative case. \tl{Pādya}
and \tl{arghya} shall be offered to Śiva.

55. After offering \tl{ācamana}, the supreme Ātman Śiva shall be bathed with
five materials (milk, curds, honey, \etc).

56. Then the offerings shall be made with great devotion reciting the requisite
Vedic mantras or the names ending in the dative case.

57. Similarly any desirable and desired material shall be offered to Śiva.
Thereafter the \tl{vāruṇasnāna} rite (ceremonial ablution) shall be performed
to Śiva.

58. Sweet-smelling sandal paste and other unguents shall then be applied.
The water poured over the deity in a continuous current shall be rendered
fragrant.

59. The water ablutions shall be made reciting Vedic mantras or six-syllabled
mantra eleven times, if so much time can be spared, then the deity shall be
wiped with a cloth.

60-61. Then the \tl{ācamana} shall be offered and cloth dedicated. Gingelly
seeds, barley grains, wheat, green gram or black gram shall then be offered to
Śiva with various mantras. Then flowers shall be offered to the five-faced noble
soul.

62-64. Lotuses, rose, \tl{śaṅkha}, and \tl{kuśa} flowers, \tl{dhattūras,
mandāras} grown in a wooden vessel, holy basil leaves or bilva leaves shall be
offered to each of the faces in accordance with the previous meditation or
according to one’s wish. By all means Śiva favourably disposed to His devotees
shall be worshipped with great devotion. If other flowers are not available,
bilva leaves shall be used exclusively in the worship of Śiva.

65-66. With the offering of bilva leaves alone, the worship shall be performed.
Then scented powders, sweetsmelling oil \etc of various sorts shall be offered
to Śiva with great joy. Then incense, guggulu (the fragrant gum resin) and aguru
(the fragrant Aloe wood) shall be offered.

67-69. Thereafter a lamp lighted with ghee shall be offered to Śiva. With great
devotion the rite of wiping of the face shall be performed with a cloth. With
the following mantra, \tl{arghya} shall be offered with great devotion. “O Śiva,
give us good features, good fame, and good enjoyment of pleasures. Taking this
\tl{arghya} give us the pleasures of the world and salvation. Obeisance be to
Thee”. Then various kinds of food-offerings shall be made to Śiva.

70-72. Then \tl{ācamana} shall be performed immediately. Then the offering of
betel leaves with all necessary adjuncts shall be made to Śiva, \tl{Ārārtika}
(the rite of waving lights) shall be performed with a lamp with five wicks.
Light shall be waved four times at the feet; twice in the umbilical region, once
near the face and seven times over the whole body. Then the devotee shall
perform meditation as stated before and repeat the mantras.

73-74. The mantras shall be repeated in accordance with the knowledge, as many
times as are necessary in the manner instructed by the preceptor.

75. The deity Śiva shall be eulogised lovingly with various hymns. Then
the devotee shall circumambulate around Śiva by and by.

76. Then he shall perform prostration with the eight limbs touching the ground
many times. He shall then offer handfuls of flowers with great devotion
repeating the following mantra.

77-83. O Śiva, whatever I have done by way of worship \etc with or without
sufficient knowledge for Śiva the great lord, in order to secure His
satisfaction shall be fruitful by your grace. O Mṛḍa, I belong to you. My vital
airs are fixed in you. My mind is always concentrated in you. O Gaurīśa, O lord
of goblins, be pleased with me. Those who stagger and falter on the ground are
supported by the ground alone. O lord, those who have offended you shall find in
you alone as their refuge.

After entreaties like these the devotee shall make a handful of flower-offering.
Then he shall bow down many times and take the ritualistic farewell—“O lord be
pleased to return to your abode along with your attendants. Please come again
when I perform worship”. After requesting thus many times, Śiva who is
favourably disposed to His devotees shall be bidden farewell to abide in
the heart. The holy water shall then be applied over the head.

O sages, thus I have entirely explained the mode of worshipping Śiva that
confers worldly pleasures and salvation. What else do you wish to hear?

% ------------- 2.2. Rudrasaṃhitā: Satī-khaṇḍa -------------------------------%
\part{Satī-khaṇḍa}
% ------------- 2.3. Rudrasaṃhitā: Pārvatī-khaṇḍa ----------------------------%
\part{Pārvatī-khaṇḍa}
% ------------- 2.4. Rudrasaṃhitā: Kumāra-khaṇḍa -----------------------------%
\part{Kumāra-khaṇḍa}
% ------------- 2.5. Rudrasaṃhitā: Yuddha-khaṇḍa -----------------------------%
\part{Yuddha-khaṇḍa}
% -----------------------------------------------------------------------------%
\end{document}
