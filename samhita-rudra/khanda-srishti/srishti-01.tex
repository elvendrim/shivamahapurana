\chapter{The inquiry of the sages}

\begin{shloka}\itshape
  viśvodbhavasthitilayādiṣu hetumekaṃ gaurīpartividitatattvamanantakīrtim\\ māyāśrayamvigatamāyamaciṃtyarūpambodhasvarūpamamalaṃ hi śivannamāmi
\end{shloka}

1. I bow to Śiva the consort of Gaurī, the sole cause of the origin, sustenance,
dissolution of the universe, who has understood the reality, who is of endless
renown, who is the support of Māyā but is free from its influence, whose form is
incomprehensible, who is unsullied and who is perfect knowledge itself.

\begin{shloka}\itshape
  vande śivantamprakṛteranādimpraśāntamekampuruṣottamaṃ hi\\
  svamāyayā kṛtsnamidaṃ hi sṛṣṭvā nabhovadantarbahirāsthito yaḥ
\end{shloka}

2. I salute Śiva who is prior to Prakṛti, who is calm and tranquil, the only
excellent Puruṣa, who has created this visible universe and who stays both
within and without like ether.

\begin{shloka}\itshape
  vandetarasthaṃ nijagūḍharūpaṃ śivaṃsvatassraṣṭumidamvicaṣṭe\\
  jaganti nityamparito bhramaṃti yatsannidhau cumbakalohavattam
\end{shloka}

3. I salute Śiva, of unmanifest form, who having extended himself by way of
creation stands in the middle of it while the worlds move around him like iron
filings round the magnet.

Vyāsa said:

4. I describe this after bowing to Śambhu, the father of the universe, Śivā
the mother of the universe and Gaṇādhīśa their son.

5. Once Śaunaka and other sages living in Naimiṣa forest asked Sūta with full
devotion.

The sages said:

6. The good and auspicious story of \tl{Vidyeśvarasaṃhitā} has been heard by us.
This first delightful compendium, “On the achievable and the means of
achievement” is lovingly disposed to the devotees.

7. Sūta, O blessed Sūta, live long. Be happy. You will please narrate to us,
O dear, the great anecdotes of Śiva.

8. O sinless one, drinking the nectar of knowledge poured out from your
lotus-mouth we are never satiated. Hence we would like to inquire of you
something more.

9. O omniscient one, by the favour of Vyāsa you have realised contentment. There
is nothing not known to you whether of the past, present or future.

10. In return for your excellent devotion you have gained the great favour of
your preceptor Vyāsa. You have understood everything. You have made your life
highly noble and purposeful.

11. Now, O wise one, please explain the excellent form of Śiva. Please narrate
the divine anecdote of Śiva and Pārvatī without omitting anything.

12. Maheśvara is \tl{aguṇa} (free from attributes). How does He take up
the \tl{saguṇa} form in the world? We do not know the true nature of Śiva,
despite our great deliberation.

13. Before the origin of creation how does lord Śiva maintain His form? In
the midst of creation how does He maintain His sport?

14. How does lord Maheśvara stand at the moment of dissolution? How is Śañkara
who blesses the world with happiness propitiated?

15. What benefit does the great Lord confer when He is pleased with His own
devotees and others? Please tell us.

16. We have heard that the lord becomes pleased instantaneously. The merciful
lord is unable to bear the stress and strain that His devotee undergoes.

17. The three deities Brahmā, Viṣṇu and Maheśa are born of Śiva. Among them
Maheśa when he has all the substrata of elements is Śiva himself as distinct
from Maheśa\footnote{According to this statement Brahmā, Viṣṇu, Maheśa are
the three forms of Śiva. In the \tl{Kūrma Purāṇa} (2.37.70-71) there occurs a
slightly modified version: Agni (\tl{tamas}), Brahmā (\tl{rajas}) and Viṣṇu
(\tl{sattva}) are the three forms of Rudra while another form, full and
attributeless is Śiva himself.}.

18. Please explain His manifestation and tell us about His various activities.
Please tell us about the birth of Umā and her marriage too, O lord.

19. Their domestic life and their divine sports shall also be narrated to us.
O sinless one, please tell us all about it and anything else that shall be told.

Vyāsa said:

20. Being thus requested Sūta was delighted. Remembering the lotus-like feet of
Śiva he replied to the sages.

Sūta said:

21. O lordly sages, what you have asked for is very nice. You are all blessed
inasmuch as your minds are drawn towards Sadāśiva’s anecdotes.

22. Like the holy waters of the Gaṅgā the inquiry into the anecdotes of Sadāśiva
sanctifies the three persons: the narrator, the inquirer and the hearer.

23. O \tl{brāhmaṇas}, except for the slayer of animals, who can be averse to
hear the narrative of the attributes of Śiva, that highly delights three types
of people always?

24. When it is being recited by persons who have no attachment or desire, it is
verily an antidote for all ailments of worldly existence, for it is highly
delightful to the ear and the heart while at the same time it bestows all objects.

25. O \tl{brāhmaṇas}, I shall explain Śiva’s sports in the light of your enquiry
as far as my intelligence enables me to do so. Please listen respectfully.

26. Induced by lord Viṣṇu, a manifestation of Śiva, Nārada had also put the same
question to his father Brahmā as you are asking me now.

27. On hearing the words of his son, Brahmā, a devotee of Śiva, was delighted in
his mind. Out of love he sang the glory of Śiva heightening the pleasure of
the excellent sage (Nārada).

Vyāsa said:

28. The learned \tl{brāhmaṇas}, on hearing the words of Sūta became eager to
know more of that conversation and so asked him.

The sages said:

29. O Sūta, O blessed Sūta, of great intellect and foremost among the devotees
of Śiva, on hearing your most delightful words our minds have become very eager
to know more.

30-31. Dear one, please tell us lovingly when this highly pleasant conversation
between Brahmā and Nārada took place, wherein Śiva’s glory was sung and
the divine sport of Lord Śiva, destructive of worldly existence, had been
discussed. What were the questions and how were they answered, please explain.

32. On hearing these words of the sages of noble mind Sūta was pleased much and
narrated everything pertaining to the conversation referred to.
