\chapter{The dispute between Brahmā and Viṣṇu}

Brahmā said:

1. When lord Nārāyaṇa continued to sleep, an excellent lotus of huge size came
out of his navel as desired by Śiva.

2. It was many \tl{yojanas} wide and high. It had an endless stalk. The pericarp
was of a brilliant hue.

3. It was very beautiful with the brilliance of ten million suns. It was
wonderful, excellent and worthy of vision containing \tl{tattvas}.

4. Exerting himself as before, Śiva, the great lord, with Pārvatī as his better
half created me from His right limb.

5. O sage, having deluded me with His illusion immediately, Śiva in the course
of His sport, produced me through the umbilical lotus of Viṣṇu.

6. Thus it was that I came to be known as Lotus-born and conceived in a golden
womb. I had four faces, red complexion and \tl{tripuṇḍra}-marked forehead.

7. Deluded by His illusion and weakened in knowledge, O dear one, I did not know
who the progenitor of my body was, other than the lotus.

8. “Who am I? Whence did I come? What is my duty? To whom was I born a son? By
whom have I been created?”

9-11. My intellect became confused with these doubts. Then I thought “Why shall
I be under delusion? It is easy to gain that knowledge. The place of growth of
this lotus is below. My progenitor will undoubtedly be there.” Thinking thus I
descended from the lotus. O sage, for a hundred years the downward trend
continued.

12. The source of the lotus was not attained by me. In the doubt-tormented state
I became eager to go up on to the top of the lotus.

13. O sage, I climbed up to the lotus by the stalk. But the upper part of
the lotus I could not reach. I was disappointed.

14. Another hundred years elapsed in my wandering up the lotus. I stopped
a while in that confounded state.

15. Then, O sage, by the will of Śiva, an auspicious voice “Perform Penance” was
heard from the sky which dispelled my delusion.

16. On hearing the voice of the sky I exerted myself for twelve years in
performing a terrible penance in order to see my progenitor.

17. At the same time, the four-armed lord Viṣṇu of beautiful eyes suddenly
appeared before me in order to bless me.

18. The great lord was holding the conch, the discus, the mace and the lotus in
his hands. He was wearing the yellow silken cloth and had cloud-blue complexion
all over his body.

19. He had a crown. He was bedecked in great ornaments. His lotus-like face
beamed with pleasure. Such was the lord resembling ten million Cupids that I saw
still not out of delusion.

20-21. At the sight of that beautiful form I was struck with wonder. On seeing
the four-armed Nārāyaṇa, shining like Kāla, of golden hue, the immanent soul of
all in that form, of large arms depicting the \tl{sat} and \tl{asat} in Himself
I became delighted.

22. Deluded by the illusion of Śiva, the sportive lord, I could not recognise my
progenitor in him. I addressed him with delight.

23. “Who are you? Please tell me”, saying this I tried to wake the Eternal Being.
[When he did not wake up] I tried to wake him up with fiercer and firmer
beatings of the hand.

24. Then the lord who had self-control woke up from his bed and sat. He looked
up with his pure eyes resembling a wet lotus, due to sleep.

25. As I stood there quietly, the lord Viṣṇu spread his brilliance over me.
Standing up he smiled once and spoke these sweet words.

26. Viṣṇu said:—“Welcome, welcome to you, dear child, O Pitāmaha of great
brilliance. Do not be afraid. Undoubtedly I shall confer on you all that you
desire.

27. O foremost among gods, on hearing these words uttered with a smile I told
Viṣṇu with my inimical attitude roused by the \tl{rajoguṇa}.

Brahmā said:

28. “O faultless one, how is it that you speak of me trivially as “Dear child”,
me who am the cause of annihilation of everything, as a preceptor addresses his
disciple?

29-30. “I am the creator of worlds, the direct activiser of \tl{prakṛti}, unborn,
the eternal, all-pervasive Brahmā. I am born of Viṣṇu. I am the soul of universe,
the originator, creator, and the lotus-eyed. You must explain to me quickly why
you speak like this.

31. The Vedas speak of me invariably as self-born, unborn, all-pervasive,
grandfather, self-governed and the excellent supreme Being.

32-35. On hearing these words of Hari, the lord of Lakṣmī became angry and told
me thus:

Viṣṇu said:

“I know you as the creator of the world. For the sake of creation and support
you are descended from my undecaying limbs. You have forgotten me, who am a lord
of universe, abiding in waters the salubrious, the supreme soul, invoked by many,
praised by many, all-pervasive, imperishable, ruler, the source and origin of
universe, the longarmed and the ommipresent lord. There is no doubt in this that
you are born of the lotus from my umbilicus.

36. “Of course, it is not your fault. I have exercised my power of illusion over
you. O four-faced one, listen to the truth. I am the lord of all Gods.

37. “I am the creator, sustainer and destroyer. There is no powerful person
equal to me. O Pitāmaha, I am the supreme Brahman, the greatest Truth.

38-39. “I am the greatest light. I am the great Ātman. I am the omnipresent. O
four-faced one, whatever in seen or heard today in the whole universe, whether
mobile or immobile is enveloped by me. It was I who created the twenty-four
manifest Tattvas.

40. “I have created the atoms. I have created the qualities of anger, fear \etc
Powerful and sportive I have created their parts and limbs.

41. “I have created the Intellect and the threefold ego therein. I have evolved
the five subtle elements, the mind, the body and the sense-organs.

42. I have created the elements ether \etc and all created beings out of sheer
sport. Realising this, O Brahmā, the lord of subjects, seek refuge in me.

43. “I shall certainly protect you from all miseries.”

Brahmā said:

On hearing these words, I, proud of being Brahmā, became angry. Being deluded by
illusion in a threatening attitude I asked him “Who are you?”

44. “Why do you talk so much? Your words will bring up disaster. You are neither
the lord, nor the supreme Brahman. There must be a creator of yours.”

45. Deluded by the illusion created by Śiva the great lord, I fought a terrific
battle with Viṣṇu.

46. Inimical to each other due to \tl{rajoguṇa}, we fought a fierce battle in
the middle of that vast expanse of the sea of dissolution.

47. Meanwhile a phallic image appeared before us in order to enlighten us and to
settle out dispute.

48. It had no beginning, middle or end. It had neither decrease nor increase. It
was as furious as hundreds of the fire of death with thousands of leaping rows
of flames.

49. It was unequalled, inexpressible unmanifest universal Being. The lord Viṣṇu
became unconscious by its thousand flames.

50. When I too became senseless, Viṣṇu said to me. Oh, why do you contend with
me now? A third person has now come. Let our quarrel cease.

51. Whence has this arisen? Let us examine this fire-Being. I shall go down to
find the root of this matchless column of fire.

52. “O lord of subjects, with the speed of the wind you will please go up to
examine its top.”

Brahmā continues the story:

53. Having said so, Viṣṇu assumed the form of a Boar. O sage, I became a swan
immediately.

54. From that time onwards, people call me Haṃsa-Haṃsa, a supreme Being, Virāṭ,
an illustrious being. He who repeats ‘\tl{haṃsahaṃsa}’, shall become a swan (a
symbol of purity and discrimination).

55. Very white of complexion and endowed with wings on either side I flew up and
up with the speed of the mind and wind.

56-58. Nārāyaṇa, the soul of the universe too, became white then. His body was
ten \tl{yojanas} wide and a hundred \tl{yojanas} long, as huge as the mountain
Meru. He had white sharp teeth. His brilliance resembled the sun at the time of
dissolution. His snort was long and his roar tremendous. His feet were short.
His limbs were of diverse colours. His form as the boar was of matchless
firmness which assured his eagerness to be victorious, and he went down quickly.

59. For a thousand years his downward course continued. From that time onwards
Viṣṇu came to be called “Śvetavārāha” (white boar) in all the worlds.

60. A \tl{kalpa} had elapsed according to human calculation when Viṣṇu thus went
down and wandered in his eagerness to come out victorious.

61. The Boar did not find even the smallest trace of the root of the Liṅga. O,
destroyer of enemies, I too spent the same time in going up.

62. From a desire to know its top as quickly as possible I exerted myself and
was exhausted. Unable to see the top I came down after some time.

63. Similarly, lord Viṣṇu, the lotus-eyed, too became weary. Appearing like
the lord of everything in his huge body he too rose up.

64. As soon as he came up, we bowed to Śiva again and again. He stood aside with
a gloomy mind as he too was deluded by the illusion of Śiva.

65. We bowed down to Liṅga at His back, sides and in front. He mused within
himself “What can this be?”

66. “That form can’t be directly expressed. It is without action and name.
Without any sex-distinction it has become a \tl{liṅga}. It is beyond the path
of meditation.

67. Both of us, Hari and I, with the peace of our minds, became eager to perform
obeisance.

\begin{shloka}\itshape
  jānīvo na hi te rūpaṃ yo'siyo'si mahāprabho |\\
  namo'stu te maheśāna rūpaṃ darśaya nau tvaran
\end{shloka}

68. “We do not know Thy true form, what Thou art Thou art, O great lord.
Obeisance be to Thee, O Maheśāna. Please hurry up to reveal Thy form to us.”

69. Thus performing obeisance and prayer to quell our earlier pride, O foremost
of sages, we spent a hundred autumns therein.
