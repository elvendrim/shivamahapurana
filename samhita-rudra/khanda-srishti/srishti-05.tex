\chapter{Nārada goes to Kāśī}

Sūta said:

1. O \tl{brāhmaṇas}, when Viṣṇu vanished, the excellent sage Nārada roamed over
the Earth seeing Śiva \tl{liṅgas} (in the various holy centres) with piety.

2. In the course of his wanderings over the Earth, O \tl{brāhmaṇas}, with his
mind full of devotional pleasure he saw many forms of Śiva that confer worldly
pleasures and salvation on the devotees.

3. On knowing that Nārada of divine vision was wandering over the Earth, the two
attendants of Śiva approached him who by that time had become pure in mind.

4. They bowed to him and touched his feet. With a desire to secure release from
the curse they spoke to him respectfully.

The attendants of Śiva said:

5. O celestial sage, son of Brahmā, please hear our words. We who formerly
offended you are really not \tl{brāhmaṇas}.

6-7. O brahmanical sage, we, your former offenders, are the attendants of Śiva.
Induced by Śiva you had cursed us when your mind was deluded by the illusory
infatuation for the princess at the \tl{svayaṃvara}. Realising that the occasion
was inopportune we kept quiet then.

8. We reaped the fruit of our own action. No one is to be blamed for it. O lord,
be pleased. Bless us now.

Sūta said:

9. On hearing the words of the attendants uttered with devotion and respect,
the sage replied lovingly, repenting (for his previous fury).

Nārada said:

10. O attendants of Lord Śiva, most worthy of the respect of good people, please
listen to my words now free from delusion. They are true and shall make you
happy.

11. Formerly my mind had been depraved. Certainly it was Śiva’s will. In that
state of delusion and crookedness of the mind I had unfortunately cursed both
of you.

12. What I have said is bound to happen. Still, O Gaṇas (attendants) listen. I
shall tell you the way of redemption from the curse. Please forgive my sin now.

13-14. You will be born as demons from the semen virile of a great sage and due
to his power you will secure the commanding position of the king of demons
endowed with prosperity, strength and valorous exploits. You will rule over
whole of the universe as devotees of Śiva with your sense conquered. You will
gain your former position after courting death at the hands of a manifestation
of Śiva.

Sūta said:

15. On hearing these words of the noble-souled Nārada, the two attendants of
Śiva became delighted and went back to their abode joyfully.

16. Nārada too was delighted. Meditating exclusively on Śiva he continued his
wanderings over the Earth seeing the various holy centres of Śiva personally.

17-18. Reaching Kāśī that excelled all other cities in holiness, which is
a favourite resort of Śiva, which easily bestows the favour of Śiva and which is
identical with Śiva, the sage became contented. He saw Śiva, the lord of Kāśī
and worshipped Him with very great pleasure and love.

19. While staying at Kāśī, the excellent sage became contented; he bowed to
the lord, described his glory piously, and remembered him with the flutter of
love.

20. Nārada then went to the region of Brahmā, his mind being highly purified by
remembering Śiva. He was eager to know further the principles of Śiva.

21. There he bowed to Brahmā with devotion and eulogised him with various
prayers. With his mind riveted to Śiva he asked him the good principles of Śiva.

Nārada said:

22-23. O Brahmā, knower of the form of Brahman, O Pitāmaha, the lord of
the universe, by your grace I have heard the greatness of Viṣṇu entirely and
also the path of devotion, of knowledge, of austere penance, of charitable gifts
and of holy centres.

24. But I have not understood the principle of Śiva. Hence, O lord, please
explain the rules of His worship and also the various activities of the lord.

25. O dear sage, how can Śiva who is free form attributes become full of
attributes? Since I am deluded by Śiva’s Māyā, I do not know the principle of
Śiva.

26. How did Śiva remain in His pure form before Creation? In the middle of
creation how does He sport about?

27. At the time of dissolution how does He remain? How is He, the benefactor of
the world, propitiated?

28. O Brahmā, when propitiated what benefit does He bestow on His devotees and
on others? Please satisfy me on all these enquiries

29. I have heard that the lord becomes delighted immediately. The merciful Great
God cannot bear the stress and strain of His devotees.

30. The three deities Brahmā, Viṣṇu and Maheśa are born as parts of Śiva. Maheśa,
having all the parts of Śiva, is Śiva Himself.

31. Please tell me all about His manifestation and especially His exploits.
O lord, please narrate the manifestation of Umā and her marriage.

32. Their domestic life, especially their great divine sports and other things
which are worthy of mention should be narrated to me, O sinless one.

33. Pārvatī’s birth and her marriage as well as Guha’s birth shall be narrated
in detail, O lord of people.

34. O lord of universe, this I have heard from many, before, but I am not
satisfied. Hence I have sought refuge in you. Please have mercy on me.

35. On hearing these words of Nārada his own son, Brahmā, the grandfather of
the world, said this.
