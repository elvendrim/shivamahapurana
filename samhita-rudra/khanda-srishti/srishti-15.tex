\chapter{The manifestation of Rudra}

Nārada said:

1. O creator, O Brahmā the fortunate, you are blessed O foremost among \tl{devas}.
A wonderfully sanctifying story of Śiva has been narrated by you, today.

2. I have heard the wonderfully divine story of the origin of the \tl{liṅga},
the auspicious hearing of the efficacy of which destroys all miseries here.

3. Please narrate what transpired thereafter, the grandeur of the created things
and particularly the mode of creation.

Brahmā said:

4. You have requested very pertinently. I shall briefly narrate what transpired
later as I have heard before.

5-6. When the eternal lord Śiva vanished, O chief of brahmins, Viṣṇu and I in
a very happy mood withdrew our forms of swan and boar and wished for creation
and sustenance of the worlds.

Nārada said:

7. O Vidhi, O Brahmā, O wise one, I have a great doubt. Please remove the same.

8. How is it that both of you assumed the forms of swan and boar instead of
other forms? Please tell me the reason for the same.

Sūta said:

9. On hearing these words of the noble-souled Nārada, Brahmā spoke after
remembering the lotus-like feet of Śiva.

Brahmā said:

10. The swan has the power of going up steadily. It has the power of
discriminating between the real and the unreal as in separating milk from water.

11. The swan understands the distinction between ignorance and knowledge. Hence
I (Brahmā) the Creator, assumed the form of swan.

12. O Nārada! But I failed to cognize the refulgent form of Śiva and therefore
could not exercise my power of discrimination.

13. How can real knowledge dawn on one who is engaged in activities of creation?
Hence though in the form of Swan I could not attain the power of discrimination.

14. A boar has the power of steadily going deep below. Hence Viṣṇu, the wanderer
in the forest, assumed the form of the boar.

15. Or Viṣṇu, the protector of all the worlds assumed the form of a boar to
start a new Kalpa (aeon).

16. Since the day he assumed the form of a boar, the aeon by the title of Vārāha
has started.

17. Or the Vārāhakalpa can be considered to have started since the day we two
decided to assume these forms.

18. O Nārada, thus I have answered your relevant question. O sage, now listen.
I shall resume the context. Remembering the lotus-like feet of Śiva I shall
explain to you the mode of Creation.

19. When God Śiva vanished, I, Pitāmaha (grandfather) of the worlds fell into
contemplation pondering on the means of carrying out His words of direction.

20. Then after bowing down to Śiva, getting knowledge from Viṣṇu and attaining
the highest bliss, I decided to start the work of creation.

21. After bowing to Śiva and instructing me, O dear one, Viṣṇu too vanished.

22. After getting the blessings of Śiva and going out of the cosmic egg, Viṣṇu
made Vaikuṇṭha\footnote{It is Viṣṇu’s abode variously described as situated in
the Northern ocean or on the eastern peak of mount Meru.} his permanent abode.

23. Desiring to create, I remembered Śiva and Viṣṇu. In the waters that had
already been created I offered handfuls of water as libation.

24. Then the cosmic egg arose consisting of twenty-four principles\footnote{
According to the Paurāṇic account of creation, the cosmic Egg constituted of
twenty-four \tl{tattvas} was entirely material. In the beginning, it was a dead
egg and it remained so until it was activated by the principle of Brahmā which
having entered into it split the egg into two halves by the process of fission.}.
O \tl{brāhmaṇa}, then a splendid, huge form Virāṭ appeared and the form of
waters was not seen.

25. Confusion arose in my mind and I performed a severe penance for twelve years
meditating on Viṣṇu.

26. At that time, Viṣṇu appeared before me and touching my body lovingly and
joyously he told me.

Viṣṇu said:

27. O Brahmā, thanks to the favour of Śiva, I am capable of giving you
everything. There is nothing which cannot be given to you. I am delighted. Tell
me the boon (you wish to have).

Brahmā said:

28. O Viṣṇu, the fortunate one, I have been entrusted to you by Śiva. Hence it
is but proper that I should request you. Please give me who request you what He
has told you (to give me). Obeisance be to you.

29. This \tl{virāṭ} form of the cosmic egg consists of twenty-four principles.
There is no sentience in it. It is insentient.

30. O Viṣṇu, you have now appeared before me; thanks to the blessings of Śiva.
Confer sentience on this cosmic egg originating from Śiva’s power.

31. When I said this, the great Viṣṇu adhering strictly to the directives of
Śiva assumed infinite forms and entered the cosmic egg.

32. Viṣṇu with a thousand heads, a thousand eyes and a thousand feet\footnote{RV.
X; \tl{Vājasaneyisaṃhitā} (of \tl{Śuklayajurveda}) 31.1.} encompassed the cosmic
egg touching the earth everywhere.

33. When Viṣṇu who was properly eulogised by me entered it, the cosmic egg
consisting of the twenty-four principles became sentient.

34. Viṣṇu shone as the great Being, the lord of the seven worlds beginning with
Pātāla\footnote{The seven regions descending from the earth, one below the other,
are called Atala, Vitala, Sutala, Rasātala, Talātala, Mahātala and Pātāla.}.

35. The five-faced lord Śiva created for His residence the beautiful city of
Kailāsa that shone above all.

36. O celestial sage, Kailāsa\footnote{This city is located on the central peak
of Hemakūta which is one of the loftiest peaks to the North of the Mānasa lake.
It is the abode both of Lord Śiva and his friend Kubera who is the Lord of
wealth.} and Vaikuṇṭha will never be destroyed even if the whole cosmic egg is
destroyed.

37. O foremost among sages, I am staying in Satyaloka\footnote{This is one of
the seven lokas of the upper region. The other six lokas are Bhūḥ, Bhuvaḥ, Svaḥ,
Mahaḥ, Janaḥ, Tapaḥ. For the sanctity and position of this loka compare an
unidentified quotation from the \tl{Devī Bhāgavata}.

\begin{shloka}\itshape
  satyaṃ tu saptamo loko brahmaṇaḥ sadanaṃ tataḥ\\
  sarveṣāṃ caiva lokānāṃ mūdhni santiṣṭhate sadā\\
  jñānakarmapratiṣṭhānāttathā satyasya bhāṣaṇāt\\
  prāpyate copabhogārthaṃ prāpya na cyavate punaḥ\\
  tatsatyaṃ saptamo lokastasmādūrdhvaṃ na vidyate
\end{shloka}
}. O dear one, I desired the activity of creation at the bidding of Siva.

38. Even as I stood desirous of creation, the evil creation, \viz the set of
five illusions\footnote{“Avidyā” also designated as “viparyaya” is fivefold. Its
five kinds are mentioned in the “Liṅgapurāṇa” (2.9.30).

\begin{shloka}\itshape
tamo moho mahāmohastāmisra iti paṇḍitāḥ\\
andhatāmisra ityāhuravidyāṃ pañcadhā sthitām
\end{shloka}

These are defined in the “Devi Bhāgavata” in the following way:

\begin{shloka}\itshape
tamo'viveko mohaḥ syādantaḥkaraṇavibhramaḥ\\
mahāmohaśca vijñeyo grāmyabhogasukhaiṣaṇā\\
maraṇaṃ tvandhatāmisraṃ tāmisraṃ krodha ucyate\\
avidhā pañcaparvaiṣa(?) prādurbhūtā mahātmanaḥ
\end{shloka}

These are further divided into sixty-two kinds. Cp. “Liṅgapurāṇa” 2.9.34-35:

\begin{shloka}\itshape
tamaso'ṣṭavidhā bhedā mohaścāṣṭavidhaḥ smṛtaḥ\\
mahāmohaprabhedāśca budhairdaśa vicintitāḥ\\
aṣṭādaśavidhaṃ cāhustāmisraṃ ca vicakṣaṇāḥ\\
andhatāmisrabhedāśca tayāṣṭādaśadhā smṛtāḥ
\end{shloka}} appeared before me. It was of the nature of darkness endowed with
knowledge.

39. Then I created the chief creation\footnote{The Paurāṇic cosmology divides
the cosmic creation into nine classes: \viz (1) “mukhyasarga”, creation of
insentient objects (2) “tiryak sarga”, creation of animals (3) “devasarga”,
creation of divine beings (4) “rājasasarga”, creation of human beings, (5)
“bhūtādisarga” creation of elements, (6) “mahatsarga”, creation of intellect (7),
“sūkṣmabhūtasarga” creation of subtle elements, (8) “vaikārikasarga” secondary
creation, (9) “kaumārasarga” primary and secondary creation.} consisting of
immobile beings with a delightful mind. At the bidding of Śiva, I continued my
meditation in a detached spirit.

40. While creating it I had thought it would be an aspirant after the Ātman. But
the creation Tiryaksrotas turned out to be full of misery. And it was not an
aspirant.

41-42. Realising that it was not an aspirant I began to ponder over the matter.
Then the Sāttvika Sarga otherwise known as Ūrdhvasrotas and Devasarga (Divine
creation) took shape. It was really charming. But considering that it too was
not aspirant I meditated on my lord.

43. Then the Rājasasarga, otherwise known as Arvāksrotas—the human creation
which was a great aspirant, appeared at the bidding of Lord Śiva.

44. Then again at the bidding of Lord Śiva the Bhūtādika Sarga (creation of
the elements \etc) appeared. Thus five types of creation collectively called
Vaikṛta were set in motion by me.

45-46. Brahmā evolved three types of creation from Prakṛti. The first one was
the creation of \tl{mahat} (the cosmic principle of intellect.) The second was
that of the subtle elements. The third was \tl{vaikārika} of the nature of
transformations and ramifications. Thus with five \tl{vaikṛta} types and three
later \tl{prākṛtas} there were eight types of creation.

47. The Kaumāra Sarga was the ninth. It was both \tl{prākṛta} and \tl{vaikṛta}.
I cannot adequately describe the divisions and sub-divisions of all these types
of creation.

48. Last of all, I shall mention the brahminical creation which is of very
little utility. It is here that the great creation of Sanaka and others,
referred to above as Kaumāra Sarga, took shape.

49. Sanaka and others, my mental sons, were five\footnote{These are Sana, Sanaka,
Sanat, Sanātana and Sujāta. Elsewhere they are stated to be seven or ten.} in
number. They were all on a par with Brahman, of good rites and averse to worldly
attachment.

50. Despite my command they were not inclined to carry on the activities of
creation; those scholarly sons turned their attention from worldly activities
and were devoted to the exclusive meditation on Śiva.

51. O Nārada, they were bold enough to retort to me whereat I became very
furious and nearly senseless.

52. When I became nearly unconscious on account of excessive fury and agitation,
drops of tears fell from my eyes.

53. At that time, on being mentally meditated upon, Viṣṇu came there hurriedly
and enlightened me.

54. O foremost among sages, I was instructed by Viṣṇu to perform the penance of
Śiva. Accordingly I performed a severe penance.

55-56. While I was performing penance for creation, the merciful lord Śiva of
Trinity, came out of the spot called Avimukta between the eyebrows and the nose.
He manifested himself as Half woman and Half man in full potency.

57-58. On seeing the unborn lord Śiva, a mass of refulgence, the consort of Umā,
the omniscent, the creator of everything, famous as Nīlalohita, straight in
front of me I saluted him with great devotion and was highly delighted. I told
the lord “Please create various subjects.”

59. On hearing my words, the lord of lords, Rudra, created many Gaṇas identical
with Himself.

60. I again told the great lord Rudra—“O lord, please create those subjects,
tormented by the fear of birth and death”.

61. O foremost among sages! on hearing my words the merciful lord Rudra laughed
and said thus.

Lord Rudra said:

62. O Brahmā, I shall not create the subjects tormented by the fear of birth and
death. The inauspicious beings are immersed in the ocean of distress by their
own actions.

63. In my manifestation in the form of preceptor I shall lift up these beings
immersed in the ocean of distress by conferring on them perfect knowledge.

64. You alone, create all the miserable subjects, O Lord! At my bidding, you
will not be bound by illusion.

Brahmā said:

65. Saying this, the lord, the glorious Śiva vanished along with His attendants
even as I was watching.
