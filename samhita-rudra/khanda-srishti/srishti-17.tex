\chapter{The Story of Guṇanidhi}

Sūta said:

1. O great sages, after hearing these words of Brahmā, Nārada once again bowed
to him and asked humbly.

Nārada said:

2-3. When did Śiva favourably disposed to His devotees go to Kailāsa? Where did
he have the intimate acquaintance with Kubera\footnote{Kubera is the son of
“viśravas” by “iḍaviḍā”. He is the chief of the “yakṣas” and a friend of Rudra.
He is mythologised as having three legs and eight teeth.} of great and noble
soul? What did Śiva of auspicious form do there? Please narrate all these things
to me. I am deeply interested in it.

Brahmā said:

4. O Nārada, listen. I shall tell you the story of the moon-crested lord, how he
went to Kailāsa and how he contracted friendship of Kubera.

5. In the city of Kāmpilya\footnote{The country known to \tl{Vājasaneyi Saṃhitā}
(xxiii, 18) and \tl{Śatapatha Brāhmaṇa} (xiii. 2.8.3) can be identified with
the city of Kāmpīla in the Furrukhabad district, Uttara Pradeśa. It was
the Southern Capital of Pāñcāladeśa in ancient India. Dr. Awasthi however,
places it in the Ānartadeśa, a region of the Western India.} there was
a sacrificer named Yajñadatta. Born of Somayāji family he was an adept in
the performance of sacrifice.

6. He knew Vedas and Vedāṅgas. He was a great scholar of Vedānta \etc He was
honoured by the king. He was a liberal-minded donor and as such his fame had
spread far and wide.

7-8. He assiduously maintained the sacrificial fire and was devoted to the study
of the Vedas. His son (Guṇanidhi) was of a very handsome complexion and shone
like the moon’s disc. After the investiture with the sacred thread he learned
all the eight lores\footnote{The eight sciences included (1) the triple Veda
(trayī), (2) logic and metaphysics (ānvīkṣikī), (3) the science of Government
(daṇḍanīti), (4) practical arts such as agriculture, commerce, medicine \etc
(vārtā), (5) ancient historical and mythological tradition, (6) science of
rituals, (7) logic and (8) dharma or law.} over and over again. Yet, unknown to
his father he indulged in gambling.

9. Ever and anon he took plenty of sums from his mother and gave them over to
other gamblers with whom he contracted great intimacy.

10. He eschewed all brahminical ways and conduct of life. He was averse to
the performance of \tl{sandhyā} prayers and ceremonial ablutions. He began to
speak ill of the Vedas, sacred texts, \tl{devas} and \tl{brāhmaṇas}.

11. He did not follow the conventions and injunctions of the Smṛti code, He
indulged in singing and playing. Actors, heretics \etc were his beloved friends.

12-15. Although his mother wanted him to meet his father now and then, he never
went near his father. Engaged in extra-domestic activities Yajñadatta used to
ask his wife “Dear good woman, what is our son Guṇanidhi doing? He is not at
home.” Then the woman used to say, “He has gone out just now. So long he had
been taking his bath and worshipping the deities. He has finished his Vedic
studies and has just gone out in the company of two or three friends for
the purpose of learning somewhere”. The poor woman in view of the fact that she
had only one son deceived her husband thus.

16. The simple husband did not know anything about the nefarious activities of
his son or his bad conduct. All sacred rites ending with \tl{keśakarma}\footnote{
The religious ceremony \tl{keśānta} in which the hair were cut off was performed
upon \tl{brāhmaṇas} at 16 years of age, \tl{kṣattriyas} at 22 and \tl{vaiśyas}
at 24. Cf. \tl{Manu. U.} 65, \tl{Yājñavalkya} 1.36.} were performed in
the sixteenth year of the son.

17. Thereafter Yajñadatta performed the marriage rite of the son in accordance
with the rules prescribed in the \tl{Gṛhya Sūtras}.

18. O Nārada, every day the woman with her heart melting with motherly affection
used to make her son sit up and gently upbraid him.

19. “Dear son, your father is surely a great man, but he is of rash temperament.
If he comes to know of your activities he will beat you and will not spare me
too.

20. I conceal your nefarious activities from your father every day. Due to his
good conduct and his affluent circumstances he is honoured by all the people.

21. Dear child, a good learning and association with men of saintly character
constitute a great asset for \tl{brāhmaṇas}. How is it that you do not gladly
take interest in such things?

22. Your ancestors and grandfathers had all earned the reputation of being good
Vedic scholars, well learned in Śāstras, and performers of sacrifices,
especially \tl{somayāgas}.

23. Shun the company of the wicked people, associate with good men, turn your
attention to good learning and strictly adhere to brāhmaṇical conventions.

24. Emulate your father in form, fame and traditional activity. Why don’t you
feel ashamed? Cast off your wickedness.

25. You are nineteen now. This girl is sixteen years old. She is a good girl.
Take her. Protect her. Above all be devoted to your father.

26. You shall respect your father-in-law also, in view of his good qualities and
conduct. How is it that you do not feel ashamed of wickedness?

27. Dear son, your maternal uncles too are matchless in learning, conduct and
pedigree and other things. You are not afraid even of them. Your paternal and
maternal lineages are equally pure.

28. See the \tl{brāhmaṇas} boys of your neighbourhood. Even in our house see
the disciples of your father. How humbly do they behave?

29. Dear son, if the king hears of your evil propensities, he will cease to
respect your father and may even suspend the regular maintenance allowance.

30. Till now people used to call your activities the foolish blunders of an
ignorant boy. Hereafter they may take away the traditional title of \tl{dīkṣita}.

31. People will curse and cavil at your father and me saying such evil words as
“The son has adopted the wickedness of the mother.”

32. Your father has never been a sinner. He strictly follows the path of
the Vedas and Smṛtis. Lord Śiva is my witness for the purity of my mind that is
riveted to his feet.

33. I have not seen the face of any wicked man after my menstrual bath. Powerful
indeed is Fate whence a boy like you is born of my womb!”

34. Although constantly advised thus by his mother, the wicked boy did not
abandon his evil ways. For, an idiot indulging in vice is beyond redemption.

35. Who is he that is not broken up by the evil influences of hunting, wine,
slander, untruth, theft, gambling and prostitutes?

36. The wicked fellow (Guṇanidhi) used to lay his hands on whatever he could see
in the house, a cloth, a base metal \etc and take it to the gambling den, there
to lose the same to his brother gamblers.

37. Once he stole a very valuable ring of his father set with precious stones
and gave it to one the of gamblers.

38. It chanced that one day the \tl{dīkṣita} saw it in the hand of the gambler.
He asked the fellow—“Where did you get this ring from?”

39-40. First the gambler did not say anything. When repeatedly asked he said—“O
\tl{brāhmaṇa}, you are unnecessarily accusing me of theft. It was your son who
gave it to me. On the previous day I had won his mother’s upper garment.

41. Do not think that I alone was the winner of this ring. He has lost many
costly things to other gamblers as well.

42. He has thus given gems, metals, silk garments, vessels, golden vases, and
different sorts of copper and bell metal pots.

43. Everyday he is being bound stark naked by the gamblers. In the whole world
you cannot see such a useless poor gambler as he (your son).

44. How is it that till now, O \tl{brāhmaṇa}, you have not realised that your
son is a ring leader of base gamblers, very clever in misdemeanour and unfair
means?

45. On hearing these words, the poor \tl{dīkṣita}’s head bent down with shame.
He covered his face and head with a cloth and quietly slipped back into his
house.

46. Yajñadatta, the sacrificer, well versed in Vedic rites spoke thus to his
wife who was a very chaste lady.

Yajñadatta said:

47-48. O mistress! where is that gambling rogue of a son, Guṇanidhi? Or let it
be. Why should I ask for him? Where is that auspicious ring which you took off
at the time of applying unguents on my body? Bring it quickly and give it to me.

49-51. The mistress was frightened at these words. While she was engaged in
arranging for bath and midday sacred rites she replied—“O lord, I am busy
arranging the various articles of offerings for worship. O lord, fond of guests,
the guests may be unnecessarily detained. While I was busy cooking the pudding
I kept the ring somewhere in some vessel just now. What a pity! I have forgotten
it. I do not know where it has been kept.

Dīkṣita said:

52-53. O truthful lady who has given birth to a base boy, whenever I asked
“Where has the son gone?” you used to say, “Dear lord, just now he has gone out
after finishing his lesson of the Vedas, in the company of two or three friends
for revision of the lesson”.

54. Where is your silk saree red like madder which I had presented to you and
which used to hang down here in the house always? Tell me the truth. Do not be
afraid.

55. That gem-set golden vase which I had given you is also missing. That tripod
with a velvet cushion which I had given you is nowhere to be seen.

56. Where is that bell metal pot made in the South? Where is that copper pot
made in Bengal? Where is that ivory casket intended for curios and trinkets?

57. Where is that wonderfully fine statuette of a lady lighting a lamp, shining
like the moon, and brought from the hilly province?

58. Why should I unnecessarily speak much? O lady of a noble family, it is
futile to be angry with you. I shall take food—only after I marry again!

59. I am childless now since that wicked fellow has defiled the whole family.
Get up and fetch me some water. Let me offer libations to him with gingelly
seeds\footnote{It is customary among the orthodox Hindu families in India to
offer libations of water mixed with gingelly seeds to the manes on particular
days.}.

60. Better to be issueless than have a wicked son who defiles the entire family.
It is the traditional policy to abandon one to save the family.

61. The \tl{brāhmaṇa} took his bath, performed his daily rites and married
the daughter of a Vedic scholar the same day.
