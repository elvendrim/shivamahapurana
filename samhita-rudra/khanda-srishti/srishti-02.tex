\chapter{Indra sends Kāmadeva to disturb the penance of Nārada}

Sūta said:

1. O \tl{brāhmaṇas}, once Nārada the excellent sage, son of Brahmā was inclined
to perform penance controlling himself very much.

2. There is a very beautiful cave in the Himālaya mountain near which
the celestial river flows rapidly.

3. There was a great hermitage of divine splendour which was resplendent in
many ways. Nārada endowed with divine vision went there to perform the penance.

4. On seeing the hermitage (very convenient for penance) the leading sage
performed the penance for a long time, seated firmly and steadily, keeping
silent, controlling the breath and retaining the purity of the intellect.

5. O \tl{brāhmaṇas}, the sage performed meditation and contemplation wherein
the realisation “I am Brahman” is generated leading to the direct perception
of Brahman.

6. When the great sage Nārada was thus performing penance, the mind of Indra
became excessively agitated and he trembled.

7. Thinking “This sage is yearning for my kingdom” Indra wanted to spoil it.

8. Indra, the leader of \tl{devas}, remembered Kāmadeva (Cupid) who arrived
there immediately, accompanied by his Queen (Rati) and spring (his friend).

9. The king of \tl{devas}, endowed with crooked intelligence to achieve his
interests, saw that Kāma had arrived and addressed him thus.

Indra said:

10. O friend, of great prowess, always doing what is beneficent to me, please
hear lovingly what I am going to say. Render me your help.

11. Strongly supported by you I have destroyed the pride of many ascetics
O friend, the stability of my kingdom is always due to your blessing.

12. Nārada, the sage, is performing a penance in the Himalaya mountain directing
his mind towards the Lord of the universe with great mental control and firm
resolve.

13. I now fear lest he should beg of Brahmā my kingdom. You must go there now
itself and hinder his penance.

14. Being thus commanded by Indra, Kāmadeva, accompanied by his wife (Rati) and
Madhu, his friend, went haughtily to that place. He then prepared his own means
of attack.

15. He employed all his arts there immediately. Spring too haughtily spread his
prowess of diverse nature.

16. O great sages, the mind of the sage (Nārada) did not waver. Only
the arrogance of these fellows suffered a setback and that too by the favour
of Maheśa.

17. Please listen to the reason thereof, O Śaunaka and other sages! By
the controlling power of the lord, Kāma could not exercise any influence.

18. It was in this very place that Śiva, the indefatigable enemy of Kāma, had
formerly performed a great penance. It was here that Kāma was reduced to
ashes—Kāma who used to spoil the penances of sages.

19. Rati wanted the resuscitation of Kāma and requested the \tl{devas}. They
appealed to lord Śiva, the benefactor of the whole world who said thus:

20. O Gods, after some time Kāma will come to life again. But none of his
tricks will succeed here.

21. Whatever space all round this spot is visible to persons here, will be out
of the influence of Kāma for ever, O \tl{devas}.

22. It was due to this statement of Śiva that Kāma’s viles did not prevail upon
Nārada. From Śiva’s abode he went to Indra.

23. Kāma then narrated everything about the sage and commended his power. At
Indra’s bidding Kāma returned to his own place.

24. Deluded by Śiva’s Māyā (power of illusion) Indra was unaware of the true
facts and was greatly surprised and he admired Nārada.

25. Śiva’s Māyā is incomprehensible to all. The whole universe is deluded by it.
Only the true devotees of dedicated souls escape.

26. Backed by Śiva’s blessings Nārada stayed in the hermitage for a long time.
Then realising that his penance was complete, the sage concluded the same.

27. Thinking that he had conquered Kāma he was puffed with pride. He was devoid
of true knowledge and deluded by Śiva’s Māyā.

28. O great sages, blessed and very blessed is Śiva’s Māyā. Even Viṣṇu, Brahmā
and others do not know the turn it takes.

29. In that state of delusion and puffed up arrogance, the great sage Nārada
went to Kailāsa to expatiate on his own achievement.

30. Bowing down to Rudra, the sage arrogantly spoke of his exploits with
the conviction that he was equal to the noble-souled lord, the conqueror of Kāma,
\ie Śiva.

31. On hearing it, Śiva who is favourably disposed to His devotees, advised
Nārada who was ignorant of the real cause, whose mind had strayed and who had
been deluded by His (Śiva’s) Māyā.

Rudra said:

32. “Dear Nārada, O wise sage, you are blessed. But please listen to me. Never
speak like this anywhere else, especially in the presence of Viṣṇu.

33. Even when you are asked you should not mention your achievements as you have
done just now. These should be guarded as close secrets and should never be
expressed.

34. I bid you specifically like this because you are a great favourite of mine.
Since you are a devotee of Viṣṇu you are my follower as all his devotees are.”

Sūta said:

35. Lord Rudra, the cause of creation, advised him in many ways like this. But
Nārada who was still under the influence of Śiva’s Māyā did not take up this
wholesome advice.

36. The future course of actions shall be considered inevitable by sensible
persons. The will of Śiva cannot be warded off by anyone.

37. Then the great sage went to Brahmā’s world. After saluting Brahmā he told
him about his conquest of Kāma as a result of his penance.

38. On hearing that, Brahmā remembered the lotuslike feet of Śiva and knew
thereby the true cause. He then forbade his son.

39. Although foremost among the wise, Nārada did not take up the advice of
Brahmā as he had been deluded by Śiva’s Māyā. The sprout of arrogance had been
so fixed in his mind.

40. Everything will take place in the world in the manner Śiva wills. It is true
that the entire universe is dependent on His will.

41. Nārada hastened to Viṣṇuloka in the same state of senseless arrogance, to
boast of his exploits in the presence of Viṣṇu.

42. When Viṣṇu saw Nārada approaching, he could guess the purpose of his visit.
He stood up and received him cordially. He walked forward and embraced him
lovingly.

43. He made Nārada sit comfortably. After remembering the lotus-like feet of
Śiva, He frankly uttered these words intended to quell the arrogance of Nārada.

Viṣṇu said:

44. “O dear Nārada, foremost among sages, you are blessed. I am sanctified by
your visit. May I know where you come from and why you have come?”

45. On hearing these words of Viṣṇu, the sage Nārada felt elated. He narrated
his story in the same haughty manner.

46. On hearing the arrogant words of the sage, Viṣṇu remembered the lotus-like
feet of Śiva again and understood the true cause.

47. Viṣṇu, a leading devotee of Śiva, with his soul dedicated to Śiva, bowed his
head and eulogised Parameśvara, the lord of the holy mountain, with his palms
joined in reverence.

Viṣṇu said:

48. “O Lord, O Lord Mahādeva, Parameśvara, be pleased. O Śiva thou art blessed.
Thy Māyā enchants everyone.”

49. Having thus chanted the prayer to Śiva, the supreme Ātman, he closed his
eyes and meditated on His lotus-like feet and stopped.

50. On coming to know what Śiva was about to do, through Śiva’s bidding, he
addressed the great sage pleasantly.

Viṣṇu said:

51. O foremost among sages, you are blessed. You are the storehouse of
austerities and large-hearted. O sage, lust and delusion rise only in the heart
of that man who is devoid of the three types of devotion.

52. Base passions that bring in their wake all sorts of miseries crop up in him
instantly. But you are vowed to perpetual celibacy. You are ever endowed with
knowledge and devoted to non-attachment.

53-55. Unaffected by passion and highly intelligent by nature how can you be
swayed by lust?”

On hearing words like these, the great sage laughed within himself but spoke to
Viṣṇu humbly.

Nārada said:

“O lord, what can Kāma do to me if you remain favourable to me?”

Saying so, the sage who had paid a casual visit bowed to Viṣṇu and left.
