\chapter{Description of the Creation}

Brahmā said:

1-2. O Nārada, after performing the pentuplication of the \tl{bhūtas}, elements
and their attributes sound \etc, I evolved the gross ether, wind, fire, water
and the earth out of them and created mountains, seas, trees \etc and
the periods of time ending with Kali and other ages.

3. I created many other things as well, but O sage, I was not satisfied. Then
O sage, I meditated on Śiva and his consort Ambā and created aspirants.

4-7. I created Marīci from my eyes, Bhṛgu from my heart; Aṅgiras from the head
and the great sage Pulaha from the vital breath Vyāna. I created Pulastya from
Udāna; Vasiṣṭha from Samāna; Kratu from Apāna; Atri from the ears and Dakṣa from
the Prāṇa. I then created you from my lap and the sage Kardama from my shadow.
Finally, I created, out of my conception, Dharma which is the means for
the achievement of everything. O foremost among sages, creating thus, thanks to
the favour of Mahādeva, these excellent \tl{sādhakas} I became contented.

8. Then, O dear one, Dharma, born out of my conception assumed the form of Manu
at my bidding and was engaged in activity by the aspirants.

9. Then I created from the different parts of my body innumerable sons, Suras
(\tl{devas}) and Asuras (demons) and many others after assigning them different
bodies, O sage.

10. I was then prompted by Śiva present within me and hence, O sage, I split
myself into two having assumed two forms.

11. One half had the form of a woman and the other half that of a man\footnote{
\tl{Śiva-purāṇa} speaks of Brahmā splitting his body into two parts, the male
and female, identified as Manu and Śatarūpā. Cp. \tl{Matsya-purāṇa} (‘A Study’
by V. S. Agrawal) 3.31.}. He then created in her a couple, the means of
excellent nature.

12. The man was Svāyambhuva Manu, the greatest of the means (of creation).
The woman was Śatarūpā, a \tl{yoginī}, an ascetic woman.

13. The auspicious lady was accepted by Manu with due matrimonial rites, O dear
one, he created beings through her by the process of sexual intercourse.

14-16. He begot of her two sons Priyavrata and Uttānapāda and three daughters
Ākūti, Devahūti and Prasūti, all of them very famous. He gave Ākūti in marriage
to Ruci and the middle one to Kardama. He gave Prasūti the younger sister of
Uttānapāda in marriage to Dakṣa. Their sons and progeny are spread over
the world both mobile and immobile.

17. Ruci begot of Ākūti the couple Yajña and Dakṣiṇā. Twelve sons were born of
Yajña and Dakṣiṇā.

18. O sage, Kardama begot of Devahūti many daughters. Dakṣa begot twenty-four
daughters.

19. Thirteen daughters Śraddhā \etc were given to Dharma in marriage by Dakṣa.
O lordly sage, listen to the names of Dharma’s wives.

20. Their names are Śraddhā (faith), Lakṣmī (fortune), Dhṛti (fortitude), Tuṣṭi
(satiety), Puṣṭi (nourishment), Medhā (intelligence), Kriyā (rite, activity),
Buddhi (intellect, wisdom), Lajjā (bashfulness), Vasu (wealth), Śānti (peace,
calmness), Siddhi (achievement, accomplishment) and the thirteenth is Kīrti
(fame).

21-23. The eleven younger daughters were Khyāti, Satī, Sambhūti, Smṛti, Prīti,
Kṣamā, Sannati, Anurūpā, Ūrjā, Svāhā and Svadhā who were respectively married by
Bhṛgu, Bhava (Śiva), Marīci, the sage Aṅgiras, Pulastya, Pulaha, the excellent
sage Kratu, Atri, Vasiṣṭha, the fire-god and the Pitṛs (manes).

24. The great aspirants Bhṛgu and others took the hands of these famous
daughters. Thereupon the entire universe consisting of three worlds, mobile and
immobile was filled (with progeny).

25. Thus according to their own actions and at the bidding of Śiva innumerable
famous \tl{brāhmaṇas} were born out of the various living beings.

26-28. In another \tl{kalpa}, Dakṣa had sixty daughters. Of them ten were given
to Dharma, twenty-seven to the Moon, thirteen to Kaśyapa. O Nārada, he gave four
to Garuḍa of excellent form. Two to each of these—Bhṛgu, Aṅgiras and Kṛśāśva.
Born of them are many children in the world of mobile and immobile.

29-30. O foremost among the sages, the children of the thirteen daughters given
to the noble-souled Kaśyapa by Dakṣa spread over the three worlds. Mobile or
immobile nothing was void.

31-32. \tl{Devas}, sages, demons, trees, birds and mountain-creepers born of
the daughters of Dakṣa filled the entire space between Pātāla and Satyaloka
\footnote{The fourteen worlds from Pātāla to Satyaloka constitute the entire
cosmos. Cf. \tl{Nirukta} 210, 212 p.247.}.

33. The whole cosmic egg was filled. Never was it a void. Thus, at the bidding
of Śiva, the creation was perfectly accomplished by Brahmā.

34-35. Dakṣa’s daughter Satī was perfectly guarded by Rudra at the tip of His
Trident, for the sake of penance. Śiva had created her himself and later for
the activities of the world she was born of Dakṣa. In order to uplift
the devotees, the lord indulged himself in many divine sports.

36. Śiva manifested himself in three ways in the form of Vaikuṇṭha (Viṣṇu) born
of the left limb, in my form (of Brahmā) born of the right limb and in the form
of Rudra born of the heart.

37. Viṣṇu, Rudra and I represent the three \tl{guṇas}. Śiva is free from
\tl{guṇas}. He is the supreme Brahman, the undecaying.

38. Viṣṇu is of \tl{sattva} attribute, I (Brahmā) am of \tl{rajas} attribute and
Rudra is of \tl{tamas} attribute. This is only in view of the activities in
the world. But in fact and in name it is otherwise.

39. Viṣṇu is of \tl{tāmasika} nature within but externally \tl{sāttvika}; Rudra
is of \tl{sāttvika} nature within but of \tl{tāmasic} nature outside, I am of
\tl{rājasic} nature throughout.

40. The goddess of speech is of \tl{rājasic} nature; Satī is of
the \tl{sāttvika} nature and Lakṣmī is of \tl{tamasic} nature; the great goddess
Śivā is of the three natures.

41. Śivā became Satī and Śiva married her. At the sacrifice of her father she
cast off her body which she did not take again and went back to her own region.

42. Śivā incarnated as Pārvatī at the request of the \tl{devas}. It was after
performing a severe penance that she could attain Śiva again.

43-45. O lordly sage, she came to be called by various names such as Kālī,
Caṇḍikā, Cāmuṇḍā, Vijayā, Jayā, Jayantī, Bhadrakālī, Durgā, Bhagavatī, Kāmākhyā,
Kāmadā, Ambā, Mṛḍānī and Sarvamaṅgalā. These various names confer worldly
pleasures and salvation according to qualities and action. The name Pārvatī is
very common.

46. The goddesses of various attributes and the three deities of various
attributes performed the diverse excellent activities of creation in mutual
collaboration.

47. O excellent among sages, I have thus explained the mode of creation to you.
The entire cosmic egg was created by me at the bidding of Śiva.

48. Śiva is the Supreme Brahman. The three deities, Viṣṇu, I and Rudra are His
manifestations according to the difference in the attributes\footnote{From
the Cosmic Egg agitated by the three \tl{guṇas}—\tl{sattva, rajas and tamas},
the three deities came into existence. The Purāṇas call them Brahmā, Viṣṇu and
Śiva and assign the functions of creation, existence and dissolution to each
respectively. Cp. \tl{Devi Bhāgavata-purāṇa} 1.8.2-4.

\begin{shloka}\itshape
  brahmā viṣṇuśca rudraśca trayo devāḥ sanātanāḥ\\
  nātaḥ parataraṃ kiñcid brahmāṇḍe'sminmahāmate\\
  brahmā sṛjati lokānvai viṣṇuḥ pātyakhilaṃ jagat\\
  rudraḥ saṃharate kāletraya pate'tra kāraṇam
\end{shloka}

The statement about the three qualities \tl{sattva, rajas} and \tl{tamas}
manifested as the three \tl{devas} is the concensus of the entire pauranic lore.
Cp. \tl{Liṅga Purāṇa}.

\begin{shloka}\itshape
  maheśvarāt trayo devā jajñire jagadīśvarāḥ\\
  śāśvatā paramā guhyāḥ sarvātmānaḥ śarīriṇaḥ\\
  eta eva trayo devā eta eva trayo guṇāḥ\\
  eta eva trayo lokā eta eva trayo'gnayaḥ
\end{shloka}

The Vedas trace the origin of the Trinity to the Brahman, the \tl{śaivas} to
Maheśvara and the \tl{bhāgavatas} to Mahāviṣṇu.}.

49. The independent Supreme Ātman, who is both Nirguṇa and Saguṇa sports with
Śivā in the beautiful Śivaloka.

50. His perfect and complete incarnation is Rudra. He is Śiva himself.
The five-faced lord has made His beautiful mansion in Kailāsa. Even if the whole
Brahmāṇḍa were destroyed, it knows no destruction.
