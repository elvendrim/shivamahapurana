\chapter{The mode of worshipping Śiva}

Brahmā said:

1. O sages, O \tl{devas}, listen. Now I shall explain a mode of worship than
which there is no better one and which is conducive to the achievement of all
happiness and cherished desires.

2. Getting up in the \tl{brāhma muhūrta} within an hour before dawn one shall
remember Śiva accompanied by his consort. With palms joined in great devotion
and head bent down he shall offer prayers:

\begin{shloka}\itshape
  uttiṣṭhottiṣṭha deveśa uttiṣṭha hṛdayeśaya\\
  uttiṣṭha tvamumāsvāminbrahmāṇḍe maṃgalaṃ kuru
\end{shloka}

3. O lord of \tl{devas}, get up, get up. O lord stationed in the heart, get up.
O lord of Umā, get up. Confer your auspicious blessings on the entire universe.

\begin{shloka}\itshape
  jānāmi dharmaṃ na ca me pravṛttirjānāmyadharmaṃ na ca me nivṛttiḥ\\
  tvayā mahādeva hṛdisthitena yathā niyukto'smi tathā karomi
\end{shloka}

4. I know what is virtuous, but I am not inclined to work it up. I know what is
unrighteous but I am unable to desist from it. O Mahādeva, I do everything as
prompted by you, stationed in my heart.

5. After repeating these words of prayer and remembering the sandals of
the preceptor he shall go out to the southern direction for answering the calls
of nature.

6. Cleaning the body thereafter with earth and water and washing his hands and
feet he shall clean the teeth.

7. Cleaning of the teeth shall be completed before sunrise. He shall gargle
sixteen times with so many mouthfuls of water.

8. O celestial sages, the \tl{tithis} of \tl{ṣaṣṭhī, navamī} as well as new moon
days and sundays are forbidden for cleaning the teeth with tooth brush twigs.

9. Bath shall be taken at a convenient time in rivers or in the house itself. No
man shall take bath against the conventions of locality or the convenience of
the season.

10-11. Hot water bath shall be avoided on sundays, \tl{śrāddha} days,
\tl{saṅkrānti} days, at the times of eclipse, on days of Great Charity and fast,
in holy centres and during the days of impurity due to death or birth in
the family. In the holy ponds and rivers one shall take bath facing the east
with great devotion.

12. Oil bath shall be taken on particular days of the week according to
convention in the society. If one is accustomed to take oil bath everyday or if
one is using scented oil breaking the convention, it is not faulty.

13. Otherwise one should avoid \tl{śrāddha} days, days of eclipse, fast days and
the first day of the lunar fortnight for oil baths. Except on the days of
eclipse mustard oil can be used on other days.

14. Bath shall be taken after due consideration of the place and season duly. He
shall face either the north or the east when taking bath.

15. He shall never take bath wearing another man's clothes. He shall take bath
in pure clothes and shall think on his favourite deities.

16. If he wears during the night another man’s clothes, the same are not impure,
hence there is no harm in taking bath with those clothes on but after taking
bath they must be washed and returned.

17. After bath he shall perform water libation propitiating gods, sages and
the manes. Thereafter washed and dried clothes shall be worn and \tl{ācamana}
performed again.

18. In a clean place washed and smeared with cow-dung, the devotee shall take
his seat, O \tl{brāhmaṇas}.

19. The seat shall be made of wood or a cloth-cover. A seat of diverse colours
is conducive to the achievement of all desires.

20. Or he can have the hide of a deer for a seat. He shall sit on it and apply
\tl{tripuṇḍra} with the ashes.

21. Prayers, penance and charity shall be performed with due markings of
\tl{tripuṇḍra} on the forehead for sure results. If ashes are not available
marking may be done with holy water.

22. After marking \tl{tripuṇḍra}, on the forehead, the devotee shall wear
\tl{rudrākṣas}. After daily rites are over, he shall begin the worship of Śiva.

23. Then he shall perform \tl{ācamana}, the sipping of water thrice with
the requisite mantras or once, saying that it is a drop of Gaṅgā water.

24-25. Rice cooked with water shall be brought for the worship of Śiva. Whatever
other things he can bring shall also be brought and kept near. A vessel for
\tl{arghya} with water and scented raw rice grains shall also be brought.

26-27. To complete the formalities of worship, the vessel shall be placed on
the right shoulder. He shall think upon the preceptor and ritualistically take
his permission for the worship. He shall perform the rite of \tl{saṃkalpa}
(including the requisite mantras and statements about the \tl{pūjā}, the day,
month, year \etc and the purpose of the \tl{pūjā}) and aver his desire. He shall
perform the worship of Śiva with His attendants devoutly.

28-29. Showing the mystic \tl{mudrā} and using saffron and other materials he
shall bow to and worship Gaṇeśa who confers benefits a hundred thousand times
and is accompanied by his consorts Siddhi and Buddhi\footnote{Siddhi and Buddhi
are personified as the wives of Gaṇeśa, the son of Śiva and Pārvatī.}. He shall
repeat his names ending in the dative case appended with \tl{namaḥ} and prefixed
with \tl{praṇava}.

30. After craving for forgiveness of the deity, he shall be worshipped again in
the company of his brother Kārtikeya with great devotion and shall be bowed to
again and again.

31. The big-bellied Gaṇeśa, the gate-keeper of the lord, shall be worshipped.
Goddess Satī, Girijā shall be worshipped then.

32-35. After worshipping Śiva with sandal paste, saffron, incense, various lamps,
and food-offerings of different sorts he shall bow down again. In the house
the \tl{liṅga} shall be made of clay, silver or any other metal or mercury. It
shall be bowed to with devotion. If that is worshipped, all deities are
worshipped. If the \tl{liṅga} is made of clay it shall be installed duly.

36. The householders shall perform every rite according to prescribed rules.
After performing the purificatory rite of the \tl{bhūtas}, the installation of
the idol shall be performed.

37-38. If the worship is performed in the temple of Śiva, the guardians of
the quarters shall be installed and worshipped. In the house, Śiva shall be
worshipped by the root mantra. It is not obligatory that the gatekeeper shall
be worshipped. The \tl{liṅga} that is worshipped by me can be worshipped in
the house. Everything is installed in the same.

39. At the time of worship, the lord shall be invoked along with his attendants
and paraphernalia. But there is no hard and fast rule governing this aspect.

40. He shall provide his own seat in the vicinity of Śiva. He shall face
the north and perform the rite of \tl{ācamana} (sipping water).

41. The devotee shall wash his hands and feet and perform \tl{prāṇāyāma} ten
times with \tl{mūlamantra}.

42. Five mystic \tl{mudrās} shall be shown with the hand before the worship.
Only after showing the \tl{mudrās} shall the worship be performed.

43-45. The lamp shall be shown then. Homage shall be paid to the preceptor. He
shall then seat himself in the yogic poses of Padma, Bhadra, Uttāna or Paryaṅka
whichever is convenient and perform the rites once again. After the worship he
shall float it along with the cake. If the worship is performed in the house
these rules are not binding.

46. Afterwards the excellent \tl{liṅga} shall be washed with the water from
the vessel of \tl{arghya} itself after keeping all the material with the
concentrated mind.

47-53. The lord shall be invoked then with the following mantra: “I am invoking
Śiva, the blissful and favourably disposed to the devotees, Śiva seated on
the summit of Kailāsa, the excellent lord of Pārvatī, Śambhu of the form as
mentioned before, both with or without qualities possessed of five faces, ten
hands, three eyes and the bull for banner, as white as camphor, of divine limbs,
having crescent moon on the head, wearing matted hair, clad in the hide of an
elephant and with the hide of the tiger as upper garment, with Vāsuki and other
serpents turned round his body, holding Pināka and other weapons, having the
eight \tl{siddhis} (accomplishments) dancing constantly in front of Him, served
by crowds of devotees crying loudly “Be victorious. Be victorious.” of unbearable
sight due to excessive splendour, served by all \tl{devas}, the sole refuge for
all living beings, of beaming face shining like lotus and always eulogised by
Viṣṇu and Brahmā as extolled by the Vedas and sacred text.” After the meditation
of Śiva along with his consort, the seat shall be arranged for.

54. Worship shall be performed with the names ending in dative case. \tl{Pādya}
and \tl{arghya} shall be offered to Śiva.

55. After offering \tl{ācamana}, the supreme Ātman Śiva shall be bathed with
five materials (milk, curds, honey, \etc).

56. Then the offerings shall be made with great devotion reciting the requisite
Vedic mantras or the names ending in the dative case.

57. Similarly any desirable and desired material shall be offered to Śiva.
Thereafter the \tl{vāruṇasnāna} rite (ceremonial ablution) shall be performed
to Śiva.

58. Sweet-smelling sandal paste and other unguents shall then be applied.
The water poured over the deity in a continuous current shall be rendered
fragrant.

59. The water ablutions shall be made reciting Vedic mantras or six-syllabled
mantra eleven times, if so much time can be spared, then the deity shall be
wiped with a cloth.

60-61. Then the \tl{ācamana} shall be offered and cloth dedicated. Gingelly
seeds, barley grains, wheat, green gram or black gram shall then be offered to
Śiva with various mantras. Then flowers shall be offered to the five-faced noble
soul.

62-64. Lotuses, rose, \tl{śaṅkha}, and \tl{kuśa} flowers, \tl{dhattūras,
mandāras} grown in a wooden vessel, holy basil leaves or bilva leaves shall be
offered to each of the faces in accordance with the previous meditation or
according to one’s wish. By all means Śiva favourably disposed to His devotees
shall be worshipped with great devotion. If other flowers are not available,
bilva leaves shall be used exclusively in the worship of Śiva.

65-66. With the offering of bilva leaves alone, the worship shall be performed.
Then scented powders, sweetsmelling oil \etc of various sorts shall be offered
to Śiva with great joy. Then incense, guggulu (the fragrant gum resin) and aguru
(the fragrant Aloe wood) shall be offered.

67-69. Thereafter a lamp lighted with ghee shall be offered to Śiva. With great
devotion the rite of wiping of the face shall be performed with a cloth. With
the following mantra, \tl{arghya} shall be offered with great devotion. “O Śiva,
give us good features, good fame, and good enjoyment of pleasures. Taking this
\tl{arghya} give us the pleasures of the world and salvation. Obeisance be to
Thee”. Then various kinds of food-offerings shall be made to Śiva.

70-72. Then \tl{ācamana} shall be performed immediately. Then the offering of
betel leaves with all necessary adjuncts shall be made to Śiva, \tl{Ārārtika}
(the rite of waving lights) shall be performed with a lamp with five wicks.
Light shall be waved four times at the feet; twice in the umbilical region, once
near the face and seven times over the whole body. Then the devotee shall
perform meditation as stated before and repeat the mantras.

73-74. The mantras shall be repeated in accordance with the knowledge, as many
times as are necessary in the manner instructed by the preceptor.

75. The deity Śiva shall be eulogised lovingly with various hymns. Then
the devotee shall circumambulate around Śiva by and by.

76. Then he shall perform prostration with the eight limbs touching the ground
many times. He shall then offer handfuls of flowers with great devotion
repeating the following mantra.

77-83. O Śiva, whatever I have done by way of worship \etc with or without
sufficient knowledge for Śiva the great lord, in order to secure His
satisfaction shall be fruitful by your grace. O Mṛḍa, I belong to you. My vital
airs are fixed in you. My mind is always concentrated in you. O Gaurīśa, O lord
of goblins, be pleased with me. Those who stagger and falter on the ground are
supported by the ground alone. O lord, those who have offended you shall find in
you alone as their refuge.

After entreaties like these the devotee shall make a handful of flower-offering.
Then he shall bow down many times and take the ritualistic farewell—“O lord be
pleased to return to your abode along with your attendants. Please come again
when I perform worship”. After requesting thus many times, Śiva who is
favourably disposed to His devotees shall be bidden farewell to abide in
the heart. The holy water shall then be applied over the head.

O sages, thus I have entirely explained the mode of worshipping Śiva that
confers worldly pleasures and salvation. What else do you wish to hear?
