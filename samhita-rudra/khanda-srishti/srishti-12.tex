\chapter{Consideration of the essential and the non-essential in the worship}

Nārada said:

1. O dear father Brahmā, with your mind fixed on Śiva, you are blessed indeed.
Please explain this again still more precisely.

Brahmā said:

2. I, the lotus-born, once called together all the sages and all the Gods and
addressed them lovingly with these good words.

3. If you have faith in permanent happiness, if you desire the achievement of
the same, all of you shall come along with me to the shores of the milk-ocean.
\footnote{According to the Paurāṇic concept, the turbulent and foamy sea known
as the southern China Sea which surrounds Śākadvīpa (identified with Malaya,
Siam, Indo-China and Southern China) on three sides was called ‘the sea of milk’
or Kṣīra Samudra: cp. SM. Ali: \emph{Geography of the Purāṇas}.}

4. On hearing these words they accompanied me to the place where lord Viṣṇu,
the benefactor of everyone, was stationed.

5. O sage, on reaching the place, the Gods bowed down with palms joined in
reverence and prayed to the lord of the universe Janārdana, lord of the Gods.

6. On seeing Brahmā and other deities standing there, Viṣṇu remembered
the lotus-like feet of Śiva and spoke these noble words.

Viṣṇu said:

7. “Why have you all, Brahmā and others and the celestial sages come? What is
the matter now? Please tell me lovingly.”

Brahmā said:

8. On being asked thus by Viṣṇu as well as by me, the deities bowed to Him with
devotion and said.

The \tl{devas} said:

9. “Whose worship shall we perform regularly for the removal of misery?”

10. On hearing these words, the lord favourably disposed to the devotees, spoke
as follows favouring me and the \tl{devas}.

The lord said:

11. O Brahmā, hear. You and these \tl{devas} have already heard this. Yet I
shall repeat it to you and to the \tl{devas}.

12-13. It has been seen. It is being seen. Then why is it being asked now?
O Brahmā, Lord Śiva, the destroyer of all miseries, shall be served always by
all who wish to achieve things. He Himself has told me as well as Brahmā
particularly about this.

14. His worship shall never be forsaken by those who wish to attain happiness.
A wonderful example has been narrated to and seen by you all.

15. When they abandoned worshipping the lord of the \tl{devas}—Maheśvara in
the form of the Liṅga, the sons of Tāra\footnote{Tāraputras—the children of
Daitya Tāraka who was conquered by Indra with the help of Skanda—the son of Śiva.
The episode is the central theme of Kālidāsa’s Kumārasambhava.} along with their
kinsmen perished.

16. They had been enchanted by me. By my illusion they were driven far by me.
When they were devoid of Śiva, they were all destroyed and exterminated.

17. Hence Śiva in the form of phallic image shall be worshipped always. He,
the foremost among deities, shall be served with special faith.

18. It is by the worship of the \tl{liṅga} of Śiva that all good men, \tl{devas},
daityas, I and you, O Brahmā, are sustained. How is it that it was forgotten
by you?

19. Hence, O Brahmā, His \tl{liṅga} shall be regularly worshipped whatever may
be the aim. Śiva shall be worshipped whatever the desire may be.

20. If an hour or even a moment is spent without the worship of Śiva, it is
a loss. It is an imperfection, a great foible, blindness, stupidity and
foolishness.

21. Those who are devotedly attached to Śiva, those whose minds are turned
towards Śiva and those who constantly remember Śiva, never become victims
of misery.

22-24. Those who desire magnificent buildings, beautiful ornaments, beautiful
women, wealth to satiety, sons and grandsons, health, splendid body,
extraordinary status, heavenly happiness and final salvation or profound
devotion to the great lord shall duly worship Śiva by virtue of their merit
accumulated by them.

25. Sure success will be his who regularly worships Śiva \tl{liṅga} with great
devotion. He will never be afflicted by sins.

Brahmā said:

26. Thus exhorted, the \tl{devas} knelt before Viṣṇu and requested for
\tl{liṅga} for the achievement of the desires of all people.

27. O foremost among sages, then, on hearing the request, Viṣṇu, eager for
the uplift of all living beings, told Viśvakarman. I too told him.

28. “O Viśvakarman, at my bidding, Śiva’s auspicious \tl{liṅgas} shall be made
and given to all \tl{devas}”.

29. At our bidding Viśvakarmā made \tl{liṅgas} and gave them to the \tl{devas}
according to their status.

30. O foremost among sages, I shall tell you the same, please listen. Indra took
a \tl{liṅga} made of ruby. The son of Viśravas (Naiśravaṇa or Kubera) took
a \tl{liṅga} of gold.

31. Dharma took a \tl{liṅga} of yellow stone, Varuṇa took a \tl{liṅga} of dark
blue hue. Viṣṇu took a \tl{liṅga} of sapphire. I, Brahmā, took a \tl{liṅga}
of gold.

32. The \tl{viśvedevas} and the \tl{vasus} took silver \tl{liṅgas}. O sage,
the Aśvini \tl{devas} took the brazen and earthen \tl{liṅga}.

33. Goddess Lakṣmī took a crystal \tl{liṅga}. The Ādityas (the twelve suns) took
\tl{liṅgas} made of copper. The moon took a \tl{liṅga} made of pearl and
the god of fire took a \tl{liṅga} of diamond.

34. Great \tl{brāhmaṇas} and their wives chose \tl{liṅgas} of earth. Maya took
a \tl{liṅga} of sandalwood and Śeṣanāga took a coral-made \tl{liṅga}.

35. The Goddesses took the \tl{liṅgas} of butter; the Yogins took \tl{liṅgas} of
the ash; the Yakṣas took \tl{liṅgas} of curd and the deity Chāyā took
a \tl{liṅgas} of beaten flour.

36. The Goddess Brahmāṇī worships, of course, the \tl{liṅga} of \tl{ratna}
(precious gem). Bāṇa and others worshipped a \tl{liṅga} of mercury.

37. Thus different kinds of \tl{liṅgas} were given to them by Viśvakarmā which
the \tl{devas} and the celestial sages worship regularly.

38. After giving the \tl{devas} the various \tl{liṅgas} from a desire for their
benefit, Viṣṇu explained the mode of worship of Śiva to me, Brahmā.

39. After listening to it, I, Brahmā, the foremost among \tl{devas}, came back
to my abode highly delighted in mind.

40. O sage, after reaching the place I explained the mode of worshipping Śiva
that yields desires to the \tl{devas} and sages.

41. “O sages and \tl{devas}, be pleased to hear with love and pleasure. I am
going to explain lovingly the mode of worshipping Śiva that confers worldly
pleasures and salvation.

42-43. The life as a human being is very difficult to obtain among all living
beings. O \tl{devas}, O sages, a life in a good family is still more difficult.
After obtaining the still more difficult birth in a \tl{brāhmaṇa} family of
good conduct on account of great merits one shall perform rites assigned to
propitiate Śiva.

44. No one shall transgress duties assigned to his caste. Charitable gifts and
sacred rites shall be performed to the extent of one’s capacity and affluence.

45. The \tl{tapoyajña} (sacrifice in the form of penance) is far superior to
thousands of \tl{karmayajñas} (ritualistic sacrifices). The \tl{japayajña}
(sacrifice in the form of \tl{japas}) is far superior to thousands of
\tl{tapoyajñas}.

46. There is nothing superior to \tl{dhyānayajña} (meditation) which is
the cause of true knowledge, since the \tl{yogin} is able to see his favourite
(deity) of equanimity through meditation.

47. Śiva is always present near a person set in meditation. There is no
necessity for any atonement or expiation for a person of true knowledge.

48-49. O gods, persons who have realised Brahman through pure learning need not
perform any rite. They are freed from happiness or misery, virtue or evil,
sacrifice or \tl{japa}, meditation or rules regarding the same. By virtue of
their learning they are free from base passions and physical changes and decays.

50. The \tl{liṅga} present in the hearts of \tl{yogin} is the purest, blissful,
auspicious, undying, all-pervasive and unsullied.

51. O \tl{brāhmaṇas}, \tl{liṅga} is of two types: the exterior and the interior.
The exterior is gross and the interior is subtle.

52. Those who are engaged in ritualistic sacrifices and do regularly worship
the gross \tl{liṅga} are unable to steady the mind by meditating upon the subtle
and hence they use the gross \tl{liṅga}.

53. He who has not mastered the \tl{liṅga} of the mind, the subtle one, must
perform the worship in the gross \tl{liṅga} and not otherwise.

54. The pure undying subtle \tl{liṅga} is ever perceived by the masters of true
knowledge in the same manner as the gross one is thought to be very excellent by
those who are not yogins.

55. If we consider properly there is nothing else for the real interpreter.
Whatever is Niṣkala or Sakala is of the form of Śiva in the whole universe. This
must be constantly thought of in the mind.

56. Even if they are devoid of the ultimate perfect knowledge, no defect or
deficiency can be ascribed to them. Rules regarding what shall be done and what
shall not be done are not binding on them.

57. The knower, of course, is not at all bound by actions, even if he continues
the householder’s life just as the lotus standing in water is not contaminated
by the water.

58. Till the realisation of perfect knowledge a man should continue
the ritualistic worship of Śiva.

59-60. In order to convince the world, the rituals must be continued. Just as
the sun is reflected in many vessels with water, in the same manner, O \tl{devas},
know that the supreme Brahman, Śiva, assumes the forms of whatever is seen or
heard in the world, real or unreal.

61. There is difference in the vessels but not in the water that they contain.
This is what those who know the real meaning of the Vedas say.

62. “Lord Śiva is within the heart of beings in this world.” Of what avail are
the idols to those who have this real knowledge?

63. Having an idol is very auspicious for a person who has no such knowledge.
It is a ladder that enables him to climb to a higher position.

64. It is very difficult to climb to a position without a support. The idol is
only a means to achieve the Nirguṇa Śiva.

65. The attainment of the Nirguṇa through a Saguṇa is certainly possible. In
this manner, the symbols of all lords are conducive to a steady faith and belief.

66. This lord is very great and this is the mode of worship of that lord. If
there is no idol, of what avail are scents, sandal paste, flowers, \etc?

67. Till the realisation of true knowledge, the idol shall necessarily be
worshipped. If any one does not worship the idol before he attains perfect
knowledge, his downfall is sure.

68. O \tl{brāhmaṇas}, hear the true statement of facts. For the same reason as
mentioned before, the duties of your own caste shall be performed assiduously.

69. Worship shall be performed where devotion is directed. Without worship and
charitable gifts, sin cannot be kept at bay.

70. As long as there is a vestige of sin in the body, achievement need not be
expected. When the sin is wiped off, all rites will bear fruit.

71. If there is dirt in the cloth the dyeing process cannot be carried out
effectively. After the cloth is bleached any dye can be applied to it
effectively.

72. Similarly when the body is freed of its dirty stuff by proper worship of
deities, the dye of knowledge can stick to it whence true knowledge will arise.

73. The root of true knowledge is unswerving devotion. The root of knowledge too
is devotion.

74. The root of devotion is good action and the worship of one’s own favourite
deity. The root of that is the good preceptor. A good preceptor is secured only
through association with good people.

75. If one associates with good people, one will come across a preceptor. From
the preceptor mantras and the modes of worship can be learned. \tl{Bhakti}
(devotion) is generated by worship and it gives birth to knowledge.

76. Knowledge leads to perfect knowledge and realisation of the supreme Brahman.
When there is perfect knowledge, differentiations cease altogether.

77. When differentiation ceases, the misery of mutually clashing opposites
vanishes. He who is free from the tangle of opposites and the miseries attendant
on them assumes the form of Śiva.

78. O celestial sages, when the mutually clashing opposites do not afflict,
a person endowed with true knowledge has neither happiness nor misery. Rules of
do’s and don’ts do not bind him.

79. Such a person who has not entered a household life is rare to meet with. If
there is such a one he will quell all sins by his mere sight.

80. Even the holy centres praise such a person of knowledge. The \tl{devas} and
all sages consider him the supreme Brahman, Śiva Himself.

81. The holy centres or the deities in the form of clay or rock idols are not
equal to him. They take time in sanctifying persons. But a man of true knowledge
purifies through his sheer vision.

82. As long as he continues the life of a householder he shall perform
the worship of the idols of the most excellent of the five deities with pleasure.

83. Or it is enough if Śiva alone is worshipped. The root is the most important.
When the root is watered, O gods, the branches are well-cared for.

84. O excellent sages, if the branches are taken care of, it does not
necessarily mean that the root is cared for. When the deities are propitiated,
the same analogy holds good.

85-86. Our aim shall be to propitiate Śiva if we are sensible. O gods, if Śiva
is worshipped, all the gods are worshipped. Hence a person who wants to do good
to all living beings shall worship Śiva, the benefactor of the world, for
the attainment of all desires.
