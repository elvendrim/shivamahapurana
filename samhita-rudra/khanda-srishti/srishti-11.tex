\chapter{The mode of worshipping Śiva}

The sages said:

1. O Sūta the fortunate, O Sūta the disciple of Vyāsa, obeisance be to you. This
wonderfully sanctifying story of Śiva has been narrated today.

2. The wonderful and highly divine origin of the \tl{liṅga} has been heard.
Listening to its efficacy causes destruction of misery.

3. O store house of mercy, please tell us the mode of the worship of Śiva, in
accordance with the conversation of Brahmā and Nārada whereby Śiva becomes
satisfied.

4. \tl{Brāhmaṇas}, \tl{kṣatriyas}, \tl{vaiṣyas} and \tl{śudras} worship Śiva.
How shall the worship be performed? Please tell us in accordance with what you
have heard from Vyāsa.

5. On hearing their words, Sūta narrated everything in answer to the question of
the sages, everything conducive to welfare and in accordance with the Vedas.

Sūta said:

6. O lordly sages, your enquiry covers a very great secret topic. I shall
explain it as far as my own intellect can penetrate it and in accordance with
what I have heard.

7-8. Formerly Vyāsa had asked the same question of Sanatkumāra as you have asked
now. Upamanyu learnt it from him. Vyāsa heard it from him and taught me the mode
of worship \etc of Śiva from a desire for the benefit of all the worlds.

9. It was directly heard from Upamanyu, the noble soul, by Kṛṣṇa (\ie
Kṛṣṇadvaipāyaṇa or Vyāsa). That I shall tell you in the some way as Brahmā had
said before.

Brahmā said:

10. O sage Nārada I shall explain briefly the worship of the \tl{liṅga}
(the phallic image). It is impossible to explain it in detail even in a hundred
years.

11. In order to achieve the fulfilment of all desires one should worship with
great devotion the pure and eternal form of Śiva thus.

12. Poverty, sickness, harassment from enemies and the four sorts of sins
trouble one only as long as one does not worship Siva.

13. When Śiva is worshipped, all miseries merge into the lord; all happiness is
secured and salvation is attained thereafter.

14. Śiva who secures the achievement of all matters shall be worshipped by
the person who considers a continuous series of human pleasures very important.

15. Whether they are \tl{brāhmaṇas}, \tl{kṣatriyas}, \tl{vaiśyas} or \tl{śūdras},
they shall perform the worship of Śiva duly and regularly for the achievement of
all desired objects.

16-18. One shall get up early in the morning during the \tl{brāhma muhūrta}
\footnote{It is the period between the fourth and the second \tl{ghaṭikas}
before sunrise. “\tl{rātriśca paścime yāme muhūrto brāhmya ucyate}”}
(about an hour before dawn). He shall remember the preceptor and Śiva. O sage,
he shall then remember the holy centres and meditate on Hari. Thereafter he
shall remember me, the deities and the sages. Then he shall recite a prayer in
the name of Śiva duly. Then he shall get up and evacuate his bowels in southern
quarter.

19. The evacuation of the bowels shall be done in an isolated place. What I have
heard (in this respect) I am mentioning now. O sage, please listen attentively.

20. A \tl{brāhmaṇa} shall use earth for cleaning purposes five times;
a \tl{kṣatriya} for four times and a \tl{vaiśya} for three times.

21. A \tl{śūdra} shall use the earth twice for cleaning purposes. Or he shall
cleanse the rectum once and the penis once assiduously.

22. He shall then wash the left hand ten times. He shall then wash each of
the feet seven times and both the hands three times once again.

23. Women shall perform these cleansing activities with earth like \tl{śudras}.
They shall first wash the hands and feet, then make use of the earth as before.

24. They shall clean the teeth using the tooth brush twig according to their
castes.

25-26. The tooth brush twig of a \tl{brāhmaṇa} shall be twelve \tl{aṅgulas} long.
A king (a \tl{kṣatriya}) shall take one eleven \tl{aṅgulas} long and
a \tl{vaiśya} one ten \tl{aṅgulas} long. The tooth brush of a \tl{śūdra} shall
be nine \tl{aṅgulas} in length. This is in accordance with Smṛtis. What is
enjoined by Manu shall be disobeyed only in emergencies.

27. On \tl{ṣaṣṭī} (sixth), \tl{navamī} (ninth) and new-moon days, on sundays and
days of sacred rites and \tl{śrāddhas}, cleaning the teeth with tooth-brush twig
is prohibited.

28. The daily ablutions shall be performed duly and those in holy centres shall
be performed with mantras in accordance with the time and place.

29. Performing the \tl{ācamana} first, wearing washed cloth, he shall perform
the \tl{sandhyā} prayer in a good isolated place.

30. After observing the preliminaries duly he shall enter the chamber of worship
keeping the mind steady and begin the rites of worship.

31. Sitting on a good seat and performing \tl{nyāsa} \etc in accordance with
the prescribed rules of worship, he shall perform the worship of Śiva duly.

32. Gaṇeśa, the attendants at the threshold of the temple, the guardians of
the quarters \etc, shall be worshipped and thereafter the pedestal shall be
arranged.

33-36. Or he shall make the mystical diagram of the lotus of eight petals and
install Śiva in its middle. He himself shall sit near all the materials of
worship around him. He shall perform \tl{ācamana} thrice and wash the hands. He
shall then perform suppression of breath (\tl{prāṇāyāma}) thrice. Then
\tl{tryaṃbaka} (three-eyed Śiva) shall be meditated upon in the following manner.
The deity has five faces, ten arms, all kinds of ornaments and the tiger-hide as
His upper-cloth. He is as pure as the crystal. During meditation he shall
identify himself with Śiva and burn off his sins. Having thus created the form
of Śiva in meditation, he shall worship lord Śiva.

37. Then the ritualistic purification of the body by touching the various parts
of the body with holy water shall be performed. The \tl{nyāsa} of
the \tl{mūlamantra} (the root mantra) and that of the six \tl{aṅgas} with
\tl{praṇava} (Oṃkāra) shall be performed thereafter.

38. After ritualistically touching the heart, he shall start worship. Different
vessels shall be set apart for \tl{pādya} (water for washing the feet),
\tl{arghya} (water for the reception of the guest) and \tl{ācamana} (sipping
water).

39-40. Nine vessels of different sizes should be kept by the sensible devotee.
\tl{darbha} grass shall be spread and cool water sprinkled over these vessels
with \tl{darbha} grass. Reciting the \tl{oṃkāra}, the intelligent devotee shall
sprinkle the various materials of worship.

41-42. The fragrant root of the plant \tl{uśīra} and sandal paste shall be put
in the water for washing feet. Fine powders of \tl{jātī}, \tl{kaṃkola},
\tl{karpūra}, root of \tl{vaṭa} and \tl{tamālaka} should be put in the water
intended for sipping. Sandal powder shall be put in all these nine vessels.

43. Nandīśa, the divine Bull of Śiva shall be worshipped beside the lord Śiva.
The latter shall be worshipped with scents, incense and different lamps.

44-47. The \tl{liṅga} shall be purified and installed with various mantras
beginning with \tl{praṇava} and ending with \tl{namaḥ} (obeisance). The pedestal
in the form of \tl{svastika} or lotus shall be assigned with \tl{praṇava}. In
the eight petals, in the eight quarters, the eight achievements are identified
\viz—The eastern petal is \tl{aṇimā} (minuteness), the southern is \tl{laghimā}
(lightness), the western is \tl{mahimā} (greatness) the northern is \tl{prāpti}
(power of reaching), the south-eastern is \tl{prākāmya} (power of sufficiency),
the south-western is \tl{īśitva} (lordliness); the north-western is \tl{vaśitva}
(power of control), the north-eastern is \tl{sarvajñatva} (omniscience) and
the pericarp is the moon (Soma).

48. Beneath the moon is the sun and beneath that is the fire. \tl{Dharma} \etc
are beneath that. All these shall be assigned regularly.

49-50. In the four quarters \tl{avyakta} \etc the unmanifest principle and in
the end of Soma the three \tl{guṇas} shall be assigned. Lord Śiva shall be
invoked by the formula “I am addressing Sadyojāta”\footnote{
\tl{Vājasaneyisaṃhitā} (of \tl{Śuklayajurveda}) 29.36.}.Then the devotee small
repeat \tl{vāmadeva}\footnote{TA. 10.41.1.} mantra and stand on his seat.
The \tl{sānnidhya} rite shall be performed with \tl{rudra gāyatrī}
\footnote{Kāṭhaka-saṃhitā 17.11.} mantra and the rite of \tl{nirodha} shall be
performed with \tl{aghora}\footnote{\tl{Vājasaneyisaṃhitā} (of
\tl{Śuklayajurveda}) 16.2.} mantra.

51. Rudra shall be worshipped with the mantra \tl{iśānaḥ sarvavidyānām}
\footnote{\tl{Vājasaneyisaṃhitā} (of \tl{Śuklayajurveda}) 27.35.}, \tl{pādya},
\tl{ācamanīya} and \tl{arghya} shall be offered duly.

52. Rudra shall be duly bathed with water, scented with sandal in the same
manner as with \tl{pañcagavya} after taking it in a vessel duly instilled with
mantras.

53. Then the deity shall be bathed invoking \tl{praṇava} with cow’s milk, curds,
honey and sugarcane juice.

54. Worshipping Rudra who bestows everything that is wholesome and desirable
with ghee, the devotee shall perform the \tl{abhiṣeka} with all materials of
worship reciting \tl{praṇava}.

55. In the holy vessels full of water he shall pour water reciting various
mantras after straining it with a white cloth duly.

56. The sprinkling need not be performed until sandal paste is mixed. Then raw
rice grains made beautiful (by adding turmeric powder \etc) shall be offered
joyously to Śaṅkara.

57-58. Offerings of flowers, especially white flowers and rare flowers, shall be
made to Lord Śiva. Flowers of \tl{apāmārga, karpūra, jātī, campaka, kuśa, pāṭala,
karavīra, mallikā, kamala} (lotus) and \tl{utpalas} (lilies) of various sorts
shall be used. When water is poured it shall be poured in a continuous stream.

59. Vessels of different varieties shall be used for the ceremonial ablution of
Lord Rudra. A worship performed with due recitation of mantras bestows all
benefits.

60. O dear one, I shall tell you briefly those mantras for the sure achievement
of all cherished desires. Please listen attentively.

61-65. Offerings of flowers and water ablutions shall be made with these mantras
whether caused to be read or committed to memory and orally repeated—The Rudra
mantra, Nīlarudra mantra, Śukla Yajurveda mantras, auspicious Hotṛ mantras,
Atharvaśīrṣa mantras, Śānti mantras, Maruta mantras, Sāmaveda mantras, if
desired, Devavrata mantras, Rathantara mantras with Puṣpa Sūktas, Mṛtyuñjaya
\footnote{Vājasaneyisaṃhitā (of Śuklayajurveda) 3.60.} mantras and
the five-syllabled mantra. The water offerings shall be a thousand times or
hundred and eight times. They shall be offered strictly in accordance with Vedic
injunctions or by repeating the names of the deity.

66. Sandal paste shall be applied to the deity and flowers placed over the idol.
Sweet smelling cloves \etc shall be offered with \tl{praṇava}.

67-72. \tl{Śivaliṅga} shall be worshipped next. The lord as pure as crystal,
the unsullied, the undecaying, the cause of all worlds, the supreme lord
identifying with the created world, the lord who cannot be seen by Brahmā, Indra,
Upendra, Viṣṇu and other deities, the lord who is mentioned in the Vedānta by
those who know Vedas as the Incomprehensible, the lord who has no beginning,
middle or end, the panacea for all sick patients and who is renowned as Śiva
Tattva. The worship of the \tl{liṅga} shall be performed, by \tl{praṇava} mantra
alone. Incense, lamps, \tl{naivedyas}, good betel leaves, pleasant \tl{nirājana}
(waving of lights) shall be duly offered. Prayers, obeisance \etc with various
mantras shall be performed. \tl{Arghya} and flower offerings shall be made at
the foot. The devotee shall kneel down and devoutly pray to the lord.

73. The devotee shall take some flowers in his hands, stand up with palms joined
in reverence and repeating the following mantra shall pray again to Īśāna,
Śaṅkara:

\begin{shloka}\itshape
  ajñānādyadi vā jñānājjapapūjādikaṃ mayā\\
  kṛtaṃ tadastu saphalaṃ kṛpayā tava śaṃkara
\end{shloka}

74. O Śiva, may this \tl{japa}, \tl{pūjā} \etc performed by me with or without
the requisite knowledge be fruitful, thanks to Thy grace.

75-76. After repeating the above mantra he shall place the flowers joyously over
the \tl{śivaliṅga}. Then the rites of \tl{svastyayana}\footnote{Ibid. 1.86.6.},
\tl{āśīrvāda} (benediction), \tl{mārjana} shall be performed. Then Homage,
a prayer for forgiveness and \tl{ācamana} shall be performed.

77-78. Repeating the \tl{agha}\footnote{Ibid. 20.29.} mantras for the expiation
of sins \tl{namaskāra} shall be duly performed. He shall pray with devout
feelings:

\begin{shloka}\itshape
  śive bhaktiśśive bhaktiśśive bhaktirbhave bhave\\
  anyathā śaraṇaṃ nāsti tvameva śaraṇaṃ mama
\end{shloka}

“Devotion to Śiva, devotion to Śiva, devotion to Śiva in every birth. I have no
other refuge. You alone are my refuge.”

79. After praying thus to the lord of the Gods, the bestower of all achievements,
the devotee shall loudly pray.

80. He shall then perform \tl{namaskāra} along with the members of his family.
He shall feel delighted in all these and thereafter carry on his daily routine
according to convenience.

81. He who performs the worship regularly like this with great devotion to Śiva
shall achieve success at every step.

82-83. He will become eloquent. He will achieve all he desires. The Supreme lord
Śiva will quell all his miseries, ailments, sorrows, heart-burns, crookedness,
poisonings and everything distressing quickly.

84. Just as the moon waxes in the bright half, his joy and merits shall increase
day by day certainly by the worship of Śiva.

85. O foremost among sages, thus I have told you the mode of worship of Śiva.
O Nārada what else do you wish to hear?
