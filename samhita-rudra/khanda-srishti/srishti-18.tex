\chapter{The Redemption of Guṇanidhi}

Brahmā said:

1-2. Guṇanidhi, the son of the Dīkṣita Yajñadatta, came to know of this.
Regretfully he cursed himself and set off from that place. After wandering
aimlessly for a long time, he, the wicked fellow, felt the abandonment keenly
and losing all hopes halted at a place.

3-7. He thought to himself: “Where am I to go? What shall I do? I have not
studied much, nor am I rich enough. Only a wealthy man can be happy in a foreign
land, although he has to face the fear of thieves there. Of course this fear is
present everywhere. I am born in the family of priests officiating in sacrifices.
Why am I reduced to this wretched plight? Fate is powerful indeed, controlling
all our future actions. I cannot even beg as I have no acquaintance, no money.
Where shall I seek refuge? Everyday, even before sunrise, my mother used to feed
me with sweet pudding. Today whom shall I beg? My mother too is away from me.”

8. O Nārada, even as he was musing like this woefully, sitting at the foot of
a tree, the sun set.

9. In the meantime a certain devotee of Lord Śiva came out of the city taking
with him various articles of offering.

10. He had observed fast on the Śivarātri\footnote{Śivarātri: Śiva’s Night. It
is a popular fast and festival held in honour of Śiva on the 14th of the dark
half of the month Māgha or January-February with many solemn ceremonies observed
during the day and night. In Tāntric literature it is called Kālarātri, one of
the three sacred nights, the other two being Mahārātri and Moharātri.} day. In
order to worship lord Śiva, he was on his way, along with his kinsmen and was
carrying different sorts of delightful offerings.

11. The devotee entered the temple of Śiva where he worshipped Him in
the prescribed manner with sincere devotion.

12. The \tl{brāhmaṇa} boy, son of Yajñadatta, devoid of his mother and dismissed
by his father, was very hungry by this time. He inhaled the sweet fragrance of
the sweet puddings and followed the devotee.

13. “If fortunately these devotees of Śiva go to sleep after offering the
eatables to Śiva, I shall eat these vast varieties of puddings and sweets in
the night”.

14. With this hope he sat at the threshold of the temple of Śiva watching the
great worship by the devotee.

15. When the worship was over, the songs and dances of prayer were duly
concluded, the devotees lay down and began to sleep. Immediately the young man
entered the sanctum sanctorum of Śiva in order to steal the eatables left there.

16. The lamp was burning very dimly. Hence in order to see the puddings clearly
he tore a piece of cloth from his lower garment and put that piece in the lamp
as a wick thus making the lamp give a good light.

17. Yajñadatta’s son gleefully took plenty of the sweets offered as eatables to
Lord Śiva by the devotees.

18. With sweets in his hands he came out hurriedly. In his hurry he stamped on
some person lying there who woke up immediately.

19. “Who is that? Who is running away so fast? Catch him.” So shouted the man
who woke up in a voice hoarse with fear.

20. The \tl{brāhmaṇa} boy (Guṇanidhi) who ran for life became blind. So he was
caught and killed by the watchmen on duty.

21. O sage, by the favour of Śiva or by the power of accumulated merit, the son
of Yajñadatta could not partake of the offerings of eatables made to Lord Śiva.

22. The terrible soldiers of Yama who desired to take him to
Saṃyamani\footnote{Saṃyaminī or Saṃyamanī, the city of Yama is fabled to be
situated on Mount Meru.} (the abode of Yama), approached him with nooses and
clubs in their hands and bound him.

23. In the meantime the attendants of Śiva with tridents in their hands and
tinkling anklets on their arms reached the spot in an aerial chariot in order to
take him to Śivaloka.

Śivagaṇas said:

24. “O attendants of Yama, leave this righteous \tl{brāhmaṇa} alone. He cannot
be punished since his sins have been burnt off.”

25-27. On hearing these words of Śiva’s attendants, the attendants of Yama
became terrified and addressed the attendants of Śiva:

Yamagaṇas said:

“O Gaṇas, this is a wicked \tl{brāhmaṇa} who has broken the traditions and
conventions of his family. He has disobeyed his father’s directions and has
forsaken truthfulness or purity. He does not offer his \tl{sandhyā} prayers. He
does not take his ceremonial baths regularly.

28. Leave aside his other activities. He has now transgressed and outraged
the offerings of eatables made to Śiva. You can see this personally. In fact he
is not worthy of even being touched by people like you.

29. Those who consume or outrage the offerings of eatables made to Śiva and
those who offer these to others, the mere touch of these persons, it is said,
is sinful.

30. Even poison is not so dangerous when drunk. Never shall a person make use of
Śiva’s property even if he were to die.

31. It is granted that you are an authority on virtue. We are not. But O Gaṇas,
if this fellow has at least a bit of virtue to his credit, please let us hear
the same”.

32. On hearing these words of Yama’s attendants, the attendants of Śiva
remembered the lotus-like feet of Śiva and spoke to them thus:

Śiva’s attendants said:

33. “O Attendants of Yama, Śiva’s ideas of \tl{dharma} are very subtle. They can
be observed only by persons of subtle and keen vision, not by people like you
whose aim is only the gross exterior.

34. O Gaṇas, hear attentively what this son of Yajñadatta has done which has
freed him from sins.

35. The shadow of the lamp was falling on the top of the \tl{liṅga} and this
\tl{brāhmaṇa} prevented it by adding a wick to the lamp at night, cutting
a piece from his lower cloth.

36. Another great merit he derived from listening to the names of Śiva, though
casually, O attendants.

37. He witnessed the worship that was being performed duly by a devotee. He was
observing a fast and his mind was concentrated too.

38. Let him go to Śivaloka along with us. As Śiva’s follower let him enjoy great
pleasures there for sometime.

39. Then he will shake off his sins and become the king of Kaliṅga\footnote{The
Kaliṅga Deśa occupied the narrower eastern coastal plain form the delta of the
Godāvarī to that of the Mahānadī river. It was probably one of the best-known
regions of the south known to ancient Indian literature.} since he has indeed
become a great favourite of Śiva.

40. Nothing else need be mentioned now. Let all of you, emissaries of Yama,
return to your own world with contented minds.”

Brahmā said:

41. O lordly sage, on hearing these words of Śiva’s attendants, the emissaries
of Yama returned to Yama’s abode.

42. O sage, they narrated everything to Yama whatever the messengers of Śiva
told them about \tl{dharma} \etc

Dharmarāja said:

43. “O Gaṇas, listen attentively to what I say. Whatever I direct you to do, you
shall do with loving devotion.

44. O Gaṇas, you shall avoid those persons who bear on their forehead the mark
of \tl{tripuṇḍra} besmeared with white ashes. Never shall they be brought here.

45. O Gaṇas, you shall avoid those persons who regularly dust their body with
white ashes. Never shall they be brought here.

46. You shall avoid all those persons who assume the garb and features of Śiva
whatever their reason may be. Never shall they be brought here.

47. You shall avoid those persons who wear \tl{rudrākṣas} and keep matted hair.
Never shall they be brought here.

48. You shall avoid those persons who imitate the dress or the features of Śiva,
even for their livelihood. Never shall they be brought here.

49. You shall avoid those persons who imitate the dress and features of Śiva
even for the purpose of deception. Never shall they be brought here.”

50. Yama thus commanded his servants. They too agreed to follow his command and
remained silent with the flickering smile on their lips.

Brahmā said:

51. Thus freed from the emissaries of Yama, the \tl{brāhmaṇa} boy became
pure-minded and went to Śivaloka along with the attendants of Śiva.

52. There he served Śiva and Śivā (Pārvatī) and enjoyed all sorts of pleasures.
Afterwards he was born as the son of Arindama, the king of Kaliṅga.

53. Known as Dama he was devoted to the service of Śiva. Even as a boy he
carried on many acts of devotion to Śiva in the company of other children.

54. When his father passed away he became the king in the prime of his youth. In
his kingdom he spread the ideals and tenets of Śiva lovingly.

55. The king Dama was unconquerable. O \tl{brāhmaṇa}, he did not stress any act
of piety other than furnishing temples of Śiva with lamps in plenty.

56. He called headmen of the villages in his kingdom and asked them to furnish
all temples of Śiva with lamps.

57. He warned them that if they defaulted they would be punished. It is declared
in the Vedas that Śiva is delighted at the gift of a lamp to his temples.

58. “Therefore, you headmen shall see that the temples of Śiva in your
jurisdiction are properly illuminated with lamps. There is no question of
hesitation in this matter.

59. “Undoubtedly I shall get the defaulter beheaded.” Thus for fear of him every
temple was duly illuminated.

60. With this act of piety alone, as long as he lived, the king Dama acquired
ample prosperity. Finally he passed away.

61. The impression of lamps persisted in his mind. He caused many lamps to be
lighted. Finally he became the lord of Alakā\footnote{Alakā—It is the capital of
Kubera, the chief of the Yakṣas and Guhyakas. It is also called Prabhā, Vasudharā
and Vasusthalī and is fabled to be situated on a peak of the Himālayas, inhabited
also by Śiva.} with gem-set lamps to his credit.

62. Thus even the smallest service rendered to Śiva bears rich fruit in time.
Let all persons seeking happiness realise this and continue the worship of Śiva.

63-65. That son of Dīkṣita never cared for any act of piety. It was to steal
that he had entered the temple of Śiva. To serve his own end he had brightened
the lamp there, thereby dispelling the shadow of darkness on the top of the
\tl{liṅga}. Then he became the virtuous king of Kaliṅga. O foremost of the sages
where the wicked son of the Dīkṣita, and where the guardian of a quarter?
Although he had been simply a man, he became the guardian of a quarter.

66. Thus I have narrated the story of Guṇanidhi, the son of Yajñadatta. The story
is pleasing to Śiva. Besides, it grants all desires of the listening devotees.

67. O dear one, I shall tell you how he became the close friend of Śiva.
Listen attentively.
