\chapter{Directions for the worship of Śiva}

The sages said:

1. O disciple of Vyāsa, O fortunate one, please explain to us authoritatively
the fruits granted by Śiva for the different worships with different flowers.

Sūta said:

2-3. O sages, Śaunaka and others, please listen attentively. I shall lovingly
explain to you the mode of offering flowers which is the same as Brahmā
explained to Nārada at his request.

Brahmā said:

4. A person desirous of wealth shall worship lord Śiva with lotuses, bilva
leaves, petals of lotuses or with \tl{śaṅkha} flowers.

5. O \tl{brāhmaṇa}, if a devotee worships Śiva with a hundred flowers, his sins
shall be wiped off and the devotee shall become rich.

6. Twenty full lotuses constitute one \tl{prastha} measure. A thousand bilva
leaves constitute half a \tl{prastha}.

7. Petals of lotuses, a thousand in number constitute half a \tl{prastha}. Ten
\tl{ṭaṅka} weight constitutes one \tl{pala} and sixteen \tl{palas} make one
\tl{prastha}.

8. Flowers for worship shall be weighed in the balance according to this
calculation. The worship thus duly performed shall accord all cherished desires.
If the devotee worships with no specific desires he will become Śiva himself.

9-14. O lordly sages, a person desirous of obtaining a kingdom shall propitiate
Lord Śiva with the worship of a hundred million earthen \tl{liṅgas}. Lord Śiva
confers a kingdom on the devotee certainly. He shall use \tl{śivaliṅga} for
worship. Flowers shall be used. Unsplit rice grains mixed with sandal paste
shall be used. The ceremonial ablution shall be performed. The mantra used shall
be pleasing. Bilva leaves are very excellent. Or he can use loose petals or full
lotuses or \tl{śaṅkha} flowers according to ancient authorities. The worship is
divine and accords pleasures and achievement of desires both here and hereafter,
he shall not omit other items such as incense, lamps, food-offerings, \tl{ārghya},
\tl{ārārtika} (waving of lights), \tl{pradakṣiṇā, namaskāra, kṣamāpana} (craving
forgiveness) and \tl{visarjana} (the ritualistic dismissal). At the end he shall
feed other devotees.

15. A person who yearns for important positions shall worship half the former
number. A person desiring release from prison shall worship a hundred thousand
\tl{liṅgas} of Śiva.

16. A person afflicted by ailments shall worship half that number. A person
desiring a daughter shall worship half that number.

17. A person desirous of learning shall worship half that number. A person
desirous of eloquence shall worship Śiva with ghee.

18. In order to drive out enemies, the number of worship is the same as before.
For exterminating enemies, worship is for a hundred thousand times and for
enchantment worship is half that number.

19. For the conquest of vassal kings, worship for ten million times is
recommended. For keeping vassal kings under influence the same for ten thousand
times is recommended.

20. For achieving glory with plenty of vehicles, worship for a thousand times
shall be performed. A person desiring salvation shall worship Śiva five crores
of times with deep devotion.

21. A person seeking knowledge shall worship Śiva, the benefactor of the world,
ten million times. A person desiring Śiva’s vision shall worship Him five
million times.

22. The Mṛtyuñjaya mantra shall be repeated half a million times when Śiva shall
be visible to the devotee and fulfil his desires.

23. If a person repeats the mantra a hundred thousand times and begins a second
instalment he will be lifted to a higher caste. When he completes the third
hundred thousand times all his worldly desires will be fulfilled. In the fourth
\tl{lakṣa} he will be able to see the lord.

24. When the fifty \tl{lakṣa} is completed, the lord will confer on him all
benefits. When the same mantra is repeated a million times, the merit is
tremendous.

25. A person desirous of liberation shall worship him with darbhas. O best of
sages, the number everywhere is a hundred thousand times.

26. A person desirous of long life shall worship him with dūrvā grass. A person
desirous of sons shall worship him with dhattūra flowers.

27. A dhattūra plant with red stem is specially auspicious for worship.
A worshipper using agastya flowers will earn great fame.

28. Worldly pleasures and salvation will be secured by a person who worships
with tulasī. Great valour can be secured by worshipping with arka or
kubjakalhāra flowers.

29. The worship with japā flowers (China rose) brings about the death of enemies.
Karavīra flowers drive away all ailments.

30. By worshipping with bandhūka flowers the devotee will get ornaments; with
jātī flowers he will acquire good vehicles; with atasī flowers he will attain
favour of Viṣṇu.

31. With śamī leaves he will secure salvation. With mallikā flowers he will
secure an auspicious woman.

32. With the splendid yūthikā flowers he will not be deprived of a house. With
karṇikāra flowers he will secure plenty of garments.

33. With nirguṇḍi flowers, his mind becomes pure in the world. A hundred
thousand bilva leaves used for worship will secure the fulfilment of all desires.

34. Use of lovely flowers in the form of garlands increases happiness and wealth.
Use of seasonal flowers for worship yields liberation. There is no doubt in this.

35. The flowers of rājikā bring about the death of enemies. A hundred thousand
rājikā flowers shall be used for the worship of Śiva. The benefit accruing will
be very great.

36. Excepting the campaka and the ketaka there is no flower which does not
appeal to Śiva. All other flowers can be used for worshipping Him.

37. Now, O excellent one, listen to the quantity of and the benefit accruing
from grains and pulses in their use for worship of Śiva.

38-39. Heaping up rice grains by way of worship causes prosperity. Six and
a half prastha, and two palas of rice grains constitute a hundred thousand in
number of grains. These shall be used in their unsplit form for the worship of
Śiva.

40. Worship of Rudra shall be performed at first and a fine cloth shall be
spread over the \tl{liṅga}. The rice grains shall be put over the cloth at
the time of worship.

41. At the end of worship, a coconut fruit shall be placed with scents and
flowers \etc and fumigated with incense. The devotee shall attain the benefit
of worship.

42. Silver coins and black gram shall be given as fee to the priest as much as
for two \tl{prājāpatya} ceremonies. If the devotee cannot afford it he shall
give according to his capacity.

43. Thereafter twelve \tl{brāhmaṇas} shall be fed. The whole of this then
constitutes the \tl{lakṣapūjā} complete in its details and with requisite
mantras.

44-46. The mantras shall be repeated hundred and eight times. That is the rule.
A hundred thousand gingelly seeds used for worship destroy even great sins.
Eleven palas of gingelly seeds constitute a hundred thousand in number. The mode
of worship is the same as before. Those who desire beneficent results shall
perform the \tl{pūjā}. \tl{Brāhmaṇas} shall be fed. Hence, only those who can
afford shall perform this. Certainly all miseries due to great sins perish
instantaneously.

47-48. Performance of the worship of Śiva with a hundred thousand barley grains
is highly efficacious. Eight and a half prasthas and two palas of barley grains
constitute a hundred thousand in number according to ancient calculation.
The worship with barley grains, the sages say, increases heavenly pleasures.

49-50. \tl{Brāhmaṇas} desiring the benefit shall perform the rite of
\tl{prājāpatya}. The worship of Śiva with wheat grains is highly praiseworthy.
If a hundred thousand grains are used for worship, the devotee shall be blessed
with a number of children. Half a droṇa of wheat will constitute a hundred
thousand in number of grains. The mode of worship is as before.

51-52. Śiva accords happiness on being worshipped with green grams. Seven
prasthas and two palas to seven and a half prasthas of green grams constitute
a hundred thousand in number. Eleven \tl{brāhmaṇas} shall be fed.

53-54. If the great Ātman, the presiding deity of Dharma, is worshipped with
priyaṅgu (long pepper corns), the devotee will be blessed with happiness. His
virtue, wealth and love will flourish. A prastha of these corns constitutes
a hundred thousand in number according to ancient authorities. Twelve
\tl{brāhmaṇas} shall be fed.

55-56. Worship with rājikā (small mustard) of Śiva shall bring about the death
of enemies. Twenty palas of sarṣapa (big mustard) constitute a hundred thousand
in number. Worshipping with them also brings about the death of enemies.
The \tl{śivaliṅga} shall be decorated with the leaves of āḍhakī and then
worshipped.

57-58. A cow along with necessary adjuncts shall be given in charity and a bull
shall also be given. Worship with pepper is also conducive to the destruction of
enemies. The \tl{śivaliṅga} shall be decorated with the leaves of āḍhakī flowers
and worshipped. This worship is conducive to different kinds of happiness and
benefits.

59. O best among sages, the measurement and number of grains and pulses have
been explained to you by me. O lord of sages, now listen to the calculation of
a hundred thousand in the case of flowers.

60. A prastha of śaṅkha flowers constitutes a hundred thousand, says Vyāsa who
shows the exact measurement and calculation.

61. Eleven prasthas of jāti and yūthikā flowers constitute a hundred thousand in
number in each. Five and a half prasthas of rājikā flowers also constitute so
many.

62. Twenty prasthas of mallikā flowers constitute a hundred thousand; while so
many flowers of gingelly plant measure a little less than a prastha.

63-64. Karavīra flowers measure three times that. Scholars say that the flowers
of nirguṇḍi too measure likewise. In karṇikāra and śirīṣa flowers too, the same
mode of calculation holds good. Ten prasthas of bandhujīva flowers constitute
a hundred thousand.

65. The devotee shall perform the worship of Śiva with different flowers after
considering these modes of calculation for the fulfilment of desires if he has
any or for the sake of salvation if he has no desire.

66. Now I shall explain the benefit of great potentiality accruing from
dhārāpūjā, a mere listening to which is conducive to great welfare.

67. After performing the regular worship of Śiva, with great devotion in
accordance with prescribed rules, the devotees shall pour water in a continuous
stream.

68-70. This \tl{dhārā} worship is very efficacious in delirium due to fever. At
that time Śatarudriya\footnote{On the Śatarudriya concept of Śiva, see
\tl{Matsya-purāṇa} (‘\emph{A Study}’ by V. S. Agrawal) pp.64-65.} mantra,
Rudraikādaśa mantra, Rudrajāpya mantra, Puruṣa Sūkta\footnote{
\tl{Vājasaneyisaṃhitā} (of \tl{Śuklayajurveda}) 31.1.}, Ṣaḍaṅga mantra,
Mahāmṛtyuñjaya\footnote{This mantra is often used for warding off diseases and
prolonging life.} mantra, Gāyatrī, names ending with \tl{namaḥ} and beginning
with \tl{praṇava} or \tl{āgama} mantra shall be repeated.

71. The \tl{dhārā} worship is very excellent in regard to flourishing series of
pleasures. Different types of auspicious materials of worship shall be added to
the water.

72. If \tl{dhārā} worship is performed with ghee continuously while a thousand
mantras are repeated, the family will undoubtedly flourish.

73. Thus the worship of Śiva shall be performed with the mantras mentioned by me.
Sages have held that \tl{brāhmaṇas} shall be fed and \tl{prājāpatya} rite shall
be performed.

74. Milk without sugar is usually taken for the \tl{dhārā}. If the devotee is
deficient in intellect and yearns for the same, sugar shall be added to milk for
the sake of \tl{dhārā}.

75. His intellect will become as keen as that of Bṛhaspati. The \tl{dhārā} shall
be continued till ten thousand mantras are completely repeated.

76-77. If there is any crack or laceration in the body without an apparent cause,
if there is any uncommon increase of love or misery anywhere, or if there be
very frequent quarrels in the house, miseries will perish when the \tl{dhārā}
worship is performed.

78. Oil-\tl{dhārā} shall be performed on \tl{śivaliṅga} for harassing enemies.
Success in the enterprise is certain.

79. If scented oil is used, worldly pleasures will be increased. If mustard oil
is used, enemies will be exterminated undoubtedly.

80. If honey is used, the devotee will become Kubera (God of wealth).
The \tl{dhārā} of sugarcane juice is conducive to all pleasures.

81-82. The \tl{dhārā} of Gaṅgā water yields worldly pleasures and salvation. In
all these \tl{dhārās} Mṛtyuñjaya mantra shall be muttered ten thousand times.
Eleven \tl{brāhmaṇas} shall be fed.

83. O lordly saint, what I have been asked I have now explained to you
completely. This will be fruitful in the world and will contribute to
the achievement of all desires.

84. I shall now tell you, as I have heard, the benefit accruing from the due
worship of Śiva in the company of Skanda and Umā.

85-87. He will enjoy in this world all kinds of auspicious pleasures with sons
and grandsons. Then he will go to the region of Śiva that is conducive to all
happiness. He will enjoy happy sports with Śiva’s attendants, move about in
aerial chariots that can go anywhere they pleased and that shine like ten
million suns and will be served by Rudra’s maidens with songs and instrumental
music, till the time of Dissolution. Then he will attain perfect knowledge and
ultimately salvation.
