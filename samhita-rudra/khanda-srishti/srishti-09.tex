\chapter{Description of Śivatattva}

Brahmā said:

1. On hearing his own eulogy from the mouth of Viṣṇu, the delighted Śiva,
the store-house of kindness, revealed Himself to us along with his consort.

2-3. He had five faces and three eyes, and the crescent moon on his forehead.
He wore matted hair. He was white-complexioned and had wide eyes. His body had
been dusted with the ashes. He had ten arms. His neck was blue in colour. He was
bedecked with all ornaments. He was very handsome with respect to every limb.
Three ash-lines marked His forehead.

4. On seeing lord Śiva accompanied by His beautiful consort, Viṣṇu along with me
eulogised Him again with appropriate words.

5-6. Śiva, the merciful, who was delighted breathed the Vedas into Viṣṇu and
conferred perfect knowledge on him, the secret of the supreme Ātman. O sage,
thereafter, out of sympathy, the supreme Ātman conferred these on me too.

7. After receiving the Vedas, Viṣṇu was satisfied and bowing to Him with palms
joined in reverence along with me, he asked the lord Śiva.

Viṣṇu said:

8. O Lord, How are you propitiated? How shall I worship you, O lord? How shall
I meditate on you? How are you impressed by anyone?

9. O Great God, tell us what at Thy bidding shall we ever do? Please command us,
O Śiva, do this to favour us.

10. O Great lord, be merciful to tell us all these things. O Śiva, we are your
followers. Taking this into mind, you will enlighten us on these and other
similar points too.

Brahmā said:

11. On hearing these words, the lord Śiva was delighted. The merciful lord then
spoke lovingly.

Śiva said:

12. O foremost among gods, I am delighted by your devotion. Look upon me as
a great deity. Cost off all your fears.

13. Worship my \tl{liṅga} and do always meditate upon the form which you see
just before you.

14. When I am worshipped in the phallic form I will be delighted and will bestow
different benefits upon all people, all that they wish for in their minds.

15. O foremost among the deities, whenever any misery befalls you, it shall be
destroyed when my \tl{liṅga} is worshipped.

16. O strong ones, you two are born of my own \tl{prakṛti}, out of my left and
right sides. I am the lord of everything.

17. This Brahmā, grandfather for all people, is born of my right side. You,
Viṣṇu, are born of my left side. I am the supreme Ātman.

18-19. Delighted I shall confer on you boons and whatever you desire. May your
devotion to me be steady. With my permission you can make my form in clay and
perform adoration. After rendering different kinds of service like this sensibly
you shall attain happiness.

20. O Brahmā, strictly adhering to my direction you carry on the work of
creation. Dear child, dear Hari, you shall sustain the mobile and the immobile
beings.

Brahmā said:

21. Saying thus, the lord presented to us the auspicious mode of His worship,
adored duly by means of which Śiva confers many benefits.

22. On hearing the words of Śiva along with me, Viṣṇu bowed to Śiva with palms
joined in reverence and said.

Viṣṇu said:

23. “If you are pleased, if a boon is to be given to us, may our devotion to you
be perpetual and unstraying.

24. Although you are Nirguṇa, be pleased to incarnate in the course of your
divine sports and help us. Dear lord, you are great lord, the supreme.

25. O lord of lords, even our dispute has turned out to be auspicious, now that
you have come here to suppress the same”.

Brahmā said:

26. On hearing these words Śiva told Viṣṇu who stood there with the head bent
down and with palms joined in reverence.

Śiva said:

27. Although Nirguṇa, I am Saguṇa too and the author of dissolution, maintenance
and creation. I am the supreme Brahman without decay and change. Existence,
Knowledge and Bliss are my characteristics.

28. Truly, I am Niṣkala (Nirguṇa) for ever, O Hari. For the activities of
creation, maintenance and dissolution I manifest myself in the three forms of
Brahmā, Viṣṇu and Hara, O Viṣṇu.

29. O Viṣṇu, since you, along with Brahmā, have eulogised me and prayed for my
incarnation, I shall make that request true, favourably disposed towards my
devotees that I am.

30. A great form similar to this, O Brahmā, shall become manifest in the world
through your body. He will be called Rudra.

31. His capacity will never be less, since He will be my own part and parcel. He
is I. I am he. In the modes of worship too there in no difference.

32. As heat \etc in water and other things due to the contact of fire is not
permanent in water \etc, similarly my Nirguṇa aspect is not affected by
the external contact.

33. This form of mine as Śiva is that of Rudra too. O great sage, no one shall
make any difference in it.

34. The same form appears split into two in the universe. Hence Śiva and Rudra
shall not be considered different.

35. A piece of gold turned into an ornament does not cease to be gold. There may
be difference in name but not in the material content.

36. Just as the difference of clay and the various objects made of it is not
a material one, so also in this case. The presence of the material cause in
the effect can be cited as an example.

37. This shall be known by all scholars and Gods of unsullied knowledge. If you
realise this, you will not be seeing the cause of difference.

38-39. I think that we all should see the from of Śiva as the basic material.
Myself, you, Brahmā and Rudra who will be manifesting himself are of the same
form. There is no difference. If there had been difference that would have been
bondage. Yet the eternal Śiva-form is mine alone.

40. That pure form is spoken of as the main root, the Truth, the Knowledge,
the Endless. Realising this too, it must be meditated upon in the true manner
in your mind.

41. O Brahmā, another secret which I am going to unfold to you may be listened
to. You two are born of \tl{prakṛti} but not this one (Rudra).

42-43. My command is carried to that place through Brahmā’s eyebrows. I am
therefore spoken of as Tāmasa and Prākṛta, Hara in respect to the \tl{guṇas}
alone and shall be known as Vaikārika too which is actually the Ahaṃkāra
(the Ego). That is called \tl{tāmasa} only in name and not in reality.

44. For this reason, O Brahmā, this shall be carried out by you. O Brahmā, you
shall be the creator and Hari the protector.

45. My would-be part shall be the cause of dissolution. This goddess Umā,
Parameśvarī is the \tl{prakṛti}.

46. Her Śakti, the goddess of speech, shall resort to Brahmā. Another Śakti also
will be arising out of the \tl{prakṛti}.

47. That Śakti will resort to Viṣṇu in the form of Lakṣmī. Another Śakti Kālī
will surely share my part.

48. She will be born in the form of Brilliance for effective work. Thus I have
told you of the great auspicious Śaktis of the Goddess.

49. Their activities are respectively creation, maintenance and dissolution. O
foremost among Gods, they are the parts of \tl{prakṛti}, my beloved.

50-53. O Viṣṇu, you shall carry on your activities with the co-operation of
Lakṣmī. O Brahmā, with the cooperation of the goddess of speech, the part of
\tl{prakṛti}, you shall carry on joyfully the activity of creation, according to
my direction. I shall have the co-operation of Kālī, the part of my beloved,
the greatest of the great and shall carry out the excellent activity of
dissolution in the form of Rudra. You shall be happy after the creation of
the world consisting of the four \tl{varṇa} and their ancillaries—the four
\tl{āśramas} (stages of life) and various sorts of other incidental activities.
You shall contribute to the welfare of the world making use of your knowledge
and perfect wisdom.

54-55. O Viṣṇu, be the bestower of salvation too at my bidding. The benefit
accruing from your vision will be the same as that from mine. This boon is given
to you now. It is the truth, certainly the truth. Viṣṇu is in my heart and I am
in Viṣṇu’s heart.

56. Those who make any distinction between the two do not know my mind. Viṣṇu is
born of my left limb. Brahmā is born of my right limb.

57. Rudra who causes great dissolution and who is the soul of the universe is
born of the heart. I manifest in the three forms, O Viṣṇu, known Brahmā, Viṣṇu
and Bhava.

58. I am the author of creation, protection and dissolution by the attributes
\tl{rajas} \etc But I am different from these \tl{guṇas} and directly beyond
\tl{prakṛti} and Puruṣa.

59. I am the supreme Brahman, the eternal, the endless, the perfect and
the unsullied. Viṣṇu has \tl{tamas} within but \tl{sattva} outside. He is
the protector of the three worlds.

60. Hara who causes dissolution of three worlds has \tl{sattva} within but
\tl{tamas} outside.

61. Brahmā who creates the three worlds has \tl{rajas} both within and without.
This is the position of the \tl{guṇas} in the three deities. Śiva is spoken of
as different from the \tl{guṇas}.

62. O Viṣṇu, guard lovingly this Pitāmaha who is the cause of creation. At my
bidding, you will be worthy of respect in the three worlds.

63. Rudra shall be worshipped by you and Brahmā. The author of dissolution of
the three worlds is the complete incarnation of Śiva.

64. In the \tl{kalpa} called Pādma, Pitāmaha will be born as your son. Then you
will see me. The lotus-born Brahmā shall also see me.

65. After saying this and conferring unequalled mercy, the great lord Hara again
spoke lovingly to Viṣṇu.
