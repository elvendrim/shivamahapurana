\chapter{The description of the body of Śabdabrahman}

Brahmā said:

1-2. O most excellent sage, we were eager to have a vision of the lord. Our
haughtiness had been curbed. O sage, we waited there patiently. Śiva,
the protector of the distressed, remover of the haughtiness of the haughty and
the undecaying lord of everything took mercy on us.

3. There arose the sound “Oṃ Oṃ” in the prolated accent\footnote{The pluta is
a prolated vowel, as in Oṃ, often marked with the figure three (\dn{ओ३म्}, o3m),
as it contains three syllabic instant in pronouncing it.}. It was very clear.
The divine sound in the form of a word came out from the most excellent of Gods.

4-5. “What shall be this great sound?” thinking like this I stood perplexed.
Viṣṇu who is worthy of respect from all the Gods, who is free from all inimical
thoughts, saw with the delightful heart, the eternal being’s manifestation on
the right side of the \tl{liṅga}. First, he saw the syllable “A” and he saw
the syllable “U” thereafter.

6-10. He saw the syllable “M” in the middle and Nāda (the mystical sound) in
the form “Oṃ” in the end. He saw the first syllable on the right like
the blazing sphere of the sun. O foremost of sages, thereafter he saw the
syllable “U” dazzling like fire. In the middle he saw the syllable “M”
glittering like the lunar sphere. Above that what he saw was the supreme Brahman,
the greatest refuge. It had the lustre of the pure crystal. It was the pure
Being beyond the Fourth (\tl{turīya}), the unsullied and free from extraneous
harassment. It was free from mutually clashing opposites. It was single
(isolated), void, free from exterior and interior though stationed in
the exterior and the interior, devoid of beginning, middle and end,
the primordial cause of Bliss, the truth, The Bliss and the Nectar.

11-12. Viṣṇu thus meditated on the universal soul enveloped by the two Vedic
sounds and wished to examine the source whence the Fire-column arose and to go
deep down the unequalled fiery column. Then there came a sage who told him
the essence of the truth.

13. Viṣṇu realised that the sage himself was the great lord and the supreme
Brahman embodied in the Śabda Brahman. (\ie the mystic syllable Oṃ).

14. The Brahman is Rudra free from worries. The words and the mind are incapable
of comprehending it; without reaching it they return. It can be expressed by
the single-syllabled mantra “Oṃ”.

15. The supreme Brahman, the Truth, the Bliss, the Amṛta, the greatest of
the great and the ultimate cause can be expressed by the single-syllabled mantra.

16. The single syllable “A” is the source of the lord Brahmā. The single
syllable “U” is the source of Viṣṇu, the ultimate cause.

17. The single syllable “M” is the source of Rudra. The creator is expressed by
the letter “A”. The enchanter is expressed by the letter “U”.

18. The being expressed by the letter “M” blesses always. It is all-pervasive
and progenitor; the letter “A” is the seed.

19. The being expressed by the letter “U” is Viṣṇu. It is the source,
the receptacle, the lord of primordial nature and primordial being,
the progenitor, the seed, source and sound. All these constitute Lord Śiva.

20. The progenitor is stationed after dividing itself. From the \tl{liṅga} of
the progenitor, the lord, arose the seed—the syllable “A”.

21. The \tl{bīja} being deposited in the \tl{yoni}, the letter “U” began to
increase all round. It became a golden egg. It was something known which could
not be delineated.

22. The divine egg floated in the waters for many years. Then at the end of
a thousand years, it split into two giving birth to Brahmā.

23-24. The egg floating in the waters on being hit by Īśvara split into two.
The auspicious golden upper lid became the upper region and the lower one became
the Earth of five characteristics. From (the inner part of) the egg was born
the four-faced lord (Brahmā) expressed by the letter “KA”.

25. He is the creator of all the worlds. He alone is the lord manifesting in
three forms. Persons well-versed in the Yajurveda call it Oṃ Oṃ.

26. On hearing the words of the Yajurveda, both the Ṛgveda and the Sāmaveda
respectfully called us then Viṣṇu and Brahmā.

27. Then realising the lord of the Gods we eulogised, as far as we could, Lord
Śiva, the cause of great achievement.

28. Viṣṇu, the protector of the universe, in the meantime, saw another
wonderfully beautiful form, along with me.

29-30. On seeing that wonderful form, Viṣṇu and I became satisfied. The form had
five faces, ten arms, and a complexion white as camphor, O sage. It had diverse
brilliant features. It was decorated in different ornaments. It was highly
liberal and endowed with great prowess. It had all the characteristics of
a great man.

31. Thereafter, the lord Śiva was pleased. Revealing his form embedded in
letters He laughingly stood before us.

32. The short letter “A” is His head. The long letter “A” is His forehead.
The letter “I” is His right eye and the letter “Ī” His left eye.

33. The letter “U” is His right ear and the letter “Ū” His left ear. The letter
“Ṛ” is the right cheek of that great lord.

34. “Ṝ” is His left cheek. The two letters “Ḷ” “Ḹ”, are His nostrils. The letter
“E” is His upper lip and the letter “AI” is His lower lip.

35. The letter “O” and the letter “AU” are respectively the two rows of his
teeth. The letters “AṂ” and “AḤ” [Anusvāra and Visarga] are his palates.

36. The five letters beginning with KA [\ie KA, KHA, GA, GHA and ṄA] are His
five hands on the right side. The five letters beginning with CA [\ie CA, CH, JA,
JHA and ÑA] are His hands on the left side.

37. Similarly the five letters beginning with ṬA and the five letters beginning
with TA constitute His legs. The letter PA is His belly and the letter PHA is
His right side.

38. The letter BA is His left side. The letter BHA is His shoulder. The letter
MA is the heart of the great yogin Mahādeva.

39. The letters YA, RA, LA, VA, ŚA, ṢA and SA are the seven Dhātus (vital
secretions) of the lord. The letter HA is His umbilicus and the letter KṢA is
His nose.

40. Viṣṇu and I became contented on seeing this letter-embedded form of
the \tl{saguṇa} manifestation of \tl{nirguṇa} lord in the company of Umā.

41. On seeing Lord Śiva in the form of the letter-embedded Brahman, Viṣṇu bowed
down along with me and looked up again.

42-47. The mantra beginning with Oṃkāra with its \tl{kalās} five in number,
consisting of the auspicious thirty-eight syllables, being pure as crystal,
increases intelligence and is an effective medium of accomplishing sacred rites.
The mantras in the Gāyatrī metre of twenty-four syllables and having four
\tl{kalās} are conducive to enjoyment. The five-syllabled mantra of eight
\tl{kalās} consisting of thirty syllables is employed for black magic. Mantras
of Yajurveda consisting of twenty-five syllables and eight \tl{kalās} are used
for conciliatory purpose. The mantra of thirteen \tl{kalās} consisting of
sixty-one syllables is conducive to outcome, increase and destruction.

48-49. The lord Viṣṇu secured these five mantras:—Mṛtyuñjaya mantra,
five-syllabled mantra, Cintāmaṇi mantra, Dakṣiṇāmūrti mantra and the
“\tl{tattvamasi}” mantra which is Hara’s Mahāvākya. Lord Viṣṇu performed
\tl{japa} by means of these mantras.

50-53. The lord Viṣṇu and I being glad at heart eulogised the boon-bestowing
lord Śiva with appropriate words,—Śiva who was seen in the form of \tl{kalās,
varṇas} (syllables), Ṛk, Yajus, Sāman, Īśāna, Īśa, Purātana Puruṣa (the ancient
Being), the merciful, pleasing to the heart, hidden from all, ever auspicious,
a great deity, of beautiful feet, bedecked with huge serpents, with legs, eyes
and hands extending on all sides, the lord of Brahmā, and the cause of creation,
sustenance and destruction of the world.
