\chapter{Description of the nature of Mahāpralaya and the origin of Viṣṇu}

Brahmā said:

1. O \tl{brāhmaṇa}, foremost among the celestial beings, a good matter has been
enquired into by you rendering service to the worlds and desiring their benefit.

2. I shall explain to you the wholesome and salutary principles of Śiva on
hearing which the various sins of the people are destroyed.

3. Neither the principles of Śiva nor His supreme wonderful forms have been
understood by me or by Viṣṇu or by any one else.

4. At the time of Great Dissolution when all the mobile and immobile objects of
the world are dissolved everything gets enveloped in darkness, without the sun,
planets and stars.

5. There is no moon. The day and the night are not demarcated. There is no fire,
no wind, no earth and no water. There is no unmanifest primordial being.
The whole firmament is one complete void, devoid of all \tl{tejas} elements.

6. There is no \tl{dharma} or \tl{adharma}, no sound, no touch. Smell and colour
are not manifest. There is no taste. The face of the quarters is not demarcated.

7. Thus when there is pitch darkness that cannot be pierced with a needle and
what is mentioned in the Vedas as “The Existent and the Brahman” is alone
present.

8. When the present visible world is not in existence, the Sat Brahman alone is
present which \tl{yogins} observe perpetually in the inner Soul,
the inner Firmament.

9. It is incomprehensible to the mind. It cannot at all be expressed by words.
It has neither name nor colour. It is neither thick nor thin.

10. It is neither short nor long. It is neither light nor heavy. There is
neither increase nor decrease in it.

11. The Veda says that it envelops whatever is in a surprising way. It is
the splendour, the truth, the knowledge, the eternal and the great Bliss.

12. It is immeasurable, propless, changeless, formless, attributeless,
perceptible to the \tl{yogins}, all-pervasive and the sole cause of the universe.

13. It is free from alternatives. It has no beginning. It is free from illusion
and its harassment. It has no second. It has neither beginning nor end. It has
no development. It is in the form of pure knowledge.

14. People have doubts about giving it a name. That Being, then after sometime,
it is said, wished for a second.

15. The Being, having no form of its own, wished to create, in the course of its
own sport, an auspicious form of its own endowed with all power, qualities and
knowledge.

16-18. A form that goes everywhere, that has all forms, that sees all, that is
the cause of all, that should be respected by all, that is at the beginning of
all, that bestows everything, and that sanctifies everything should be created
(So it wished) and hence created that form of Īśvara of pure nature.
The original Being without a second, with neither beginning nor end, that
illuminates everything, that is in the form of \tl{cit} (pure knowledge), that
which is termed Supreme Brahman, the all-pervasive and undecaying, vanished,
the manifest form of the formless Being is Sadāśiva. Scholars of the ancient and
succeeding ages have sung of it as Īśvara.

19. Īśvara though alone, then created the physical form Śakti from his body.
This Śakti did not affect his body in any way.

20. This Śakti is called by various names. Pradhāna, Prakṛti, Māyā, Guṇavatī,
Parā. The mother of Buddhi Tattva (The cosmic Intelligence), Vikṛtivarjitā
(without modification).

21. That Śakti is Ambikā, Prakṛti and the goddess of all. She is the prime cause
and the mother of the three deities.

22. She has eight arms. Her face wears a peculiar splendour, the splendour of
a thousand moons. Thousands of stars perpetually sparkle round her face.

23. She is bedecked in various ornaments. She has various weapons. She is
capable of various movements. Her eyes beam like a full blown lotus.

24. She has a brilliance which could hardly be conceived. She is the generating
cause of all. She sprang up singly as Māyā. In her union she manifested in
various forms.

25. The supreme Puruṣa is Śiva. He is called Śambhu. He has no other lord over
Him. He holds the Mandākinī (Gaṅgā) on His head, and the crescent moon on His
forehead. He has three eyes.

26. He has five faces. He is always joyful. He has ten arms. He holds
the trident. He is as pure and white as camphor. His body is entirely dusted
with the ash.

27. That Brahman of the form of Kāla (Time) together with Śakti, simultaneouly
created the holy centre called Śivaloka.

28. The same is called Kāśikā, the excellent holy centre. It is the seat of
salvation shining over and above everything.

29. The holy centre is of the nature of extreme Bliss inasmuch as the primordial
lovers, supremely Blissful, made that beautiful holy centre their perpetual
abode.

30. O sage, that holy centre is never, even at the time of Great Dissolution,
free from Śiva and Śivā (Śakti). Hence it is called Avimukta.

31. Since the holy centre is the cause of Bliss, the Pināka-bearing lord (Śiva)
called it “the blissful forest” and later “Avimukta”.

32. O celestial sage, the blissful, two deities thus sporting in the forest
wished, it is said, for another Being to be created.

33-38. Śiva thought within Himself like this—“Another being shall be created by
me. Let him create everything, protect it and in the end let him dissolve it
with my blessing. Having entrusted everything to him we two, remaining in Kāśī
shall roam as we please keeping only the prerogative of conferring salvation. We
can stay happily in this blissful forest being free from worries (of creation).
With the consent of Śiva the supreme lord spread the liquorine essence of nectar
on His left side, on the tenth limb, nectar which was the outcome of churning
the ocean of His mind wherein thoughts were the waves, the \tl{sattva guṇa} was
the precious gem, \tl{rajas} being coral and \tl{tamas}—crocodile. Thereupon
a person came into being who was the most charming one in the three worlds, who
was calm with \tl{sattva guṇa} being prominent, and who appeared to be the ocean
of immeasurable majesty.

39. O sage, he was endowed with patience. There was no one comparable to him. He
had the lustre of sapphire. He was glorious with his excellent eyes shining like
a lotus.

40. He was having a golden form and features. He wore two excellent silk
garments of golden colour. His were browny and brilliant. He was indefatigable.

41. He bowed to Śiva Parameśvara and said—“O lord give me names and assign me my
task.”

42. On hearing it Lord Śiva laughed. With words thunderlike in resonance, Lord
Śiva addressed the person thus.

Śiva said:

43. “You will be famous as Viṣṇu by name as you are all-pervasive. You will have
many other names conferring happiness on devotees.

44. Perform penance highly conducive to the achivement of the matter in hand,
be firm in it.” Saying so, the lord bestowed on him the Vedas through his
nostrils.

45. Śiva vanished accompanied by Śakti and his attendants. After due obeisance
to Śiva, Viṣṇu began his great penance.

46. Even after performing the penance for twelve thousand divine years, Viṣṇu
could not achieve his desire, the vision of Śiva that confers everything.

47. He became suspicious and respectfully meditating on Śiva pondered “What
shall I do now?”

48. In the meantime the auspicious voice of Śiva was heard. “Perform penance
again for removing your doubts”.

49. On hearing it Viṣṇu performed a terrible penance, for a long time, following
the path of meditation.

50. That Being Viṣṇu became enlightened, following the path of meditation. He
was delightfully surprised. “O what is that true entity?”

51. From the body of Viṣṇu who thus exerted himself, water-currents of various
sorts began to flow as a result of Śiva’s Māyā.

52. O great sage, the Supreme Brahman in the form of divine waters pervaded
the entire void. A mere contact with the same is destructive of sins.

53. Viṣṇu, the weary person went to sleep amidst the waters. He was in that
blissful state of delusion for a long time.

54. As approved in the Vedas, his name came to be established as Nārāyaṇa
(Having water as abode). Excepting for that Primordial Being there was nothing
then.

55. In the meantime, the principles too were evolved out of the Great soul. O
wise one of great intellect, listen to my enumeration of the same.

56. From \tl{prakṛti} came into being the \tl{mahat} (cosmic intellect), from
\tl{mahat} the three \tl{guṇas}. \tl{Ahaṃkāra} (the cosmic ego) arose there from
in three forms according to the three \tl{guṇas}\footnote{The Ego (\tl{ahaṃkāra})
is threefold according to the qualities of \tl{sattva}, \tl{rajas} and
\tl{tamas}. In the present enumeration it is counted as one.}.

57. The Essences, the five elements, the senses of knowledge and action too came
into being then.

58-59. O most excellent of sages, I have thus enumerated the principles. All
these principles originating from \tl{prakṛti} are insentient but not the Puruṣa.
These principles are twenty-four in number\footnote{A group of 24 \tl{tattvas}
includes intellect (\tl{buddhi}), ego (\tl{ahaṃkāra}), \tl{manas} (mind), five
elements (\tl{bhūtas}), five subtle elements (\tl{tanmātras}), five senses of
action (\tl{karmendriyas}) and five senses of knowledge (\tl{jñānendriyas}) and
unmanifest \tl{prakṛti} (\ie \tl{pradhāna}). Puruṣa stands apart from
the \tl{tattvas}. The enumeration follows the Sāṃkhya system.}. Viṣṇu,
the Puruṣa, accepted all these, as was the will of Śiva, and began his sleep in
the Brahman.
