\chapter{Description of Parama Śivatattva}

Lord Śiva said:

1. O Viṣṇu of good rites, O Hari, listen to another pronouncement of mine. You
will be worthy of honour and worship in all the worlds for ever.

2. Whenever a misery befalls the world created by Brahmā, you shall be eager for
the destruction of all miseries.

3. In all difficult and unbearable activities I shall help you. I shall kill
your indefatigable and fierce enemies.

4. O Viṣṇu, spread your glory in the worlds far and wide by taking up various
incarnations. Strive for their succour. I am always Saguṇa when I become Rudra
with this body.

5. Certainly I shall perform your activities for the sake of the worlds if they
are impossible for you.

6. You are to be meditated upon by Rudra and Rudra is to be meditated upon by
you. There is no difference between you and Rudra.

7. O Great Viṣṇu, your identity is due to inherent nature, the boons conferred
and your divine sports. True, it is certainly true.

8. If any devotee of Rudra were to censure you, he will have all his merits
reduced to ashes quickly.

9. O Viṣṇu, the most excellent of all persons, by hating you he will certainly
fall into hell. That is my directive. True, it is certainly true.

10. In this world, be the bestower of worldly pleasures and salvation to men
particularly. Worthy of being honoured and worshipped by devotees, perform
the activities of curbing and blessing.

11. Saying this and holding me, the creator, and Viṣṇu by the hand He
continued—“Always render help in distress.

12. Be the presiding deity of all. Bestow worldly pleasures and salvation for
ever. Be the most excellent accomplisher of the achievement of all desires.

13. You will assume the form of the vital airs in everyone at my bidding. O Hari,
in the times of distress, Rudra my manifestation, shall be worshipped.

14. He who has sought refuge in you has certainly sought refuge in me. He who
differentiates will certainly fall into Hell.

15. Listen to the span of life of the deities—Brahmā, Viṣṇu and Hara. There
shall not be any doubt in this respect.

16. A thousand sets of the four-\tl{yuga} periods constitute one day of Brahmā.
The period of night is also similar. Further measurement of time is based on
this calculation.

17. Thirty such days (days and nights) constitute one month and twelve months,
one year. The span of life of Brahmā is hundred such years.

18. One year of Brahmā constitutes one day of Viṣṇu. Viṣṇu lives for hundred
years in his own calculation.

19. One year of Viṣṇu constitutes one day of Rudra. When a hundred year period
passes, Rudra assumes the form of Nara (supreme Man).

20. He stays like that as long as the breath is retained by Sadāśiva. When He
exhales he merges into Śakti.

21-22. In the case of all living beings, Brahmā, Viṣṇu, Hara, \tl{gandharvas},
serpents, \tl{rākṣasas}, \etc, twenty one thousand six hundred respirations
constitute the period of one day and one night, O foremost among \tl{devas}.

23-24. Six respirations constitute the period of time one \tl{pala}. Sixty such
\tl{palas} constitute one \tl{ghaṭī}. Sixty \tl{ghaṭīs} constitute one day and
one night. (6 x 60 x 60 = 21 600). There is no limit to the number of
respirations of Sadāśiva. Hence He is undecaying.

25. It is my directive that you should preserve this form and maintain all
the activities of the created worlds by means of these different \tl{guṇas} so
long.”

26. On hearing these words of Śiva the lord Viṣṇu, of controlled senses, spoke
slowly to Śiva after duly bowing to Him.

Viṣṇu said:

27. “O Śankara, the ocean of mercy, the lord of the universe, be pleased to hear.
I shall do all these things strictly adhering to your behests.

28. I shall always meditate upon you. I would not act otherwise. Your
omnipotency has already been experienced by me.

29. O lord, let not the meditation of your form be ever far removed from my mind
even for a moment.

30. O lord, if anyone of my devotees were to censure you, you will please assign
perpetual residence in hell to him.

31. O lord, whoever be your devotee shall also be my favourite. He who knows and
realises this shall not find salvation inaccessible to him.

32. My greatness has been further heightened by you certainly. If ever I am
found deficient in qualities I may be excused.

33. (Brahmā said:—) Then, on hearing the excellent words of Viṣṇu, Śiva said to
him “Of course the deficiency shall be excused lovingly.”

34. After saying this mercifully the lord, the ocean of mercy, stroked us all
over the body with His hands.

35. With a desire to do what is wholesome to us He instructed us in various
sacred rites and conferred on us many boons.

36. Even as we were watching, the lord Śiva, favourably disposed towards
devotees, vanished immediately.

37. The rite of the worship of the \tl{liṅga} is instituted ever since in
the world. Śiva installed in the \tl{liṅga} bestows worldly pleasures and
salvation.

38. The great goddess is the pedestal for the \tl{liṅga}. The liṅga is Lord
Śiva Himself. Since the whole universe finally merges into in, it is called
Liṅga.

39. He who reads regularly this anecdote of the \tl{liṅga} in the vicinity of
the \tl{liṅga} assumes the form of Śiva within six months. There need be no
hesitation in this respect.

40. O great sage, I cannot adequately express the blessedness accruing to
the person who carries on any activity whatsoever in the vicinity of
the \tl{liṅga}.
