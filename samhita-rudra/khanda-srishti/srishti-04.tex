\chapter{Nārada goes to Vaikuṇṭha and curses Viṣṇu there}

The sages said:

1-2. Sūta, O Sūta of great intellect, a wonderful tale has been narrated by you.
Blessed indeed is the Māyā of Śiva. All mobile and immobile things depend upon
it. When the two attendants of lord Rudra had left at their own will what did
the infuriated Nārada, the sage disquieted by Kāmadeva, do?

Sūta said:

3-5. After cursing the two attendants of Śiva suitably, the sage still under
the earlier delusion looked into the water and saw that his face was quite
normal. It was also due to Śiva’s will. He did not wake from the delusion still
again due to Śiva’s will. Thereupon recollecting that it might have been
a deception of Hari, he became unbearably infuriated and went to Viṣṇuloka.
There he angrily poured abusive words blazing like kindled fire since his wisdom
had vanished due to Śiva’s will.

Nārada said:

6. O Viṣṇu, you are extremely wicked, deceptive enchanter of the world. You are
unable to brook others’ enthusiastic success. You dabble in illusory tactics and
your intentions are always dirty.

7. Formerly you assumed the form of an enchantress\footnote{It refers to
the form assumed by Viṣṇu at the time of cheating the demons of nectar.} and
showed your deceptive power. You made the demons drink liquor and not
the nectar.

8. If out of pity Śiva had not drunk poison\footnote{It refers to Śiva’s
swallowing the poison produced at the churning of the ocean.}, O Viṣṇu, all your
illusory tactics would have been quelled since you take pleasure only in
deception.

9. O Viṣṇu, a deceptive path is extremely attractive to you. You had never been
of saintly nature, but the lord made you free from control.

10-11. What is done by Śiva the supreme Ātman does not seem proper. Thinking of
your influence and strength when you act independently and seeing the way you go
He has now repented. He has announced that a \tl{brāhmaṇa} is superior to all,
thereby making the Vedas pronounced by Him authoritative.

12. O Viṣṇu, knowing that, I shall now teach you through that power so that
hereafter you will never do such things.

13. You are fearless because till now you have not come into clash with an
equally powerful person. Now you will derive, O Viṣṇu, the fruit of your own
“deeds”.

14. After saying this, the sage still under the influence of Māyā furiouly
cursed Viṣṇu, thereby exhibiting the superiority of his brahmanical power.

15-16. O Viṣṇu, the enchanter that you are, you made me distressed for the sake
of woman. O Hari, you shall experience misery in that human form which you
imitated while proceeding with your deceptive tactics. Your allies will be those
whose face you assigned to me.

17. O inflictor of miseries upon others, you shall get the misery of separation
from a woman. You shall have the travails of a human being deluded by ignorance.”

18. Thus Nārada, deluded himself by ignorance, cursed Hari. Viṣṇu quietly
accepted the cause praising the Māyā of Śambhu.

19. Thereafter Śiva, of great divine sport withdrew his enchanting Māyā whereby
Nārada became wise (as before) and free from delusion.

20-21. When the Māyā vanished he became as intelligent as before regaining
perfect knowledge and becoming free from distress. He was surprised (at his own
action in the meantime). He cursed himself after repenting again and again. He
praised the Māyā of Śiva which could enchant even wise people.

22. On realising his mistakes due to illusion, Nārada, the most excellent of
the devotees of Viṣṇu, fell at his feet.

23. Consoled by Hari and freed from wicked ideas he said—“Being deluded and
evil-minded I have spoken many wicked words to you.

24. O lord, I heaped curses on you. O master, please make them ineffective.
I have committed a great sin. Certainly I will be falling into a hell.

25. O Hari, I am your slave. Please direct me what to do whereby I may destroy
my sins and prevent my downfall into hell.”

26. Saying thus, the excellent sage once again fell at Viṣṇu’s feet and with
the mind purified repented sincerely.

27. Thereupon Viṣṇu lifted him up and spoke affably and courteously.

Viṣṇu said:

“Do not be sorry too much. Undoubtedly you are my true devotee.

28. Dear sage, now listen. I shall tell you what is certainly beneficial to you.
You will not fall into hell. Śiva will make you happy.

29. Deluded by your haughtiness you disobeyed the instructions of Śiva. The true
bestower of fruits according to the actions, He has given you this result.

30. Be sure in your mind that everything has happened in accordance with Śiva’s
wish. That lord Śiva, the supreme lord, removes haughtiness.

31. He is the supreme Brahman; the supreme Ātman, Existence, Knowledge and
Bliss. He is free from the three \tl{guṇas}, changes and deviations. He is
beyond \tl{rajas}, \tl{sattva} and \tl{tamas}.

32. He is both \tl{saguṇa} and \tl{nirguṇa} (with and without attributes). He
Himself availing of his own Māyā manifests into three Brahmā, Viṣṇu, and
Maheśa.

33. In his attributeless pure form He is glorified as Śiva, the supreme Ātman,
Maheśvara, the supreme Brahman, the undecaying, the endless, and Mahādeva.

34. Serving him, Brahmā becomes the creator and I the sustainer of the worlds.
He himself in the manifestation as Rudra is the annihilator always.

35. Different from Māyā, the pure Being in the form of Śiva is the Sākṣin
(cosmic witness) and moving about according to His Will and indulging in divine
sport He blesses his devotees.

36. O sage Nārada, please listen to a good remedy that bestows happiness,
removes all sins and yields worldly pleasures and salvation.

37. Cast off all your doubts. Sing the songs of noble glory of Śiva. With your
mind not turning to anything else, always repeat the hundred names of Śiva and
his hymns.

38. By his \tl{japa} all of your sins will perish instantaneously. After saying
this to Nārada, Viṣṇu continued mercifully.

39. “O sage, do not be grief-striken. Nothing has been perpetrated by you. It
was Śiva who did everything. There is no doubt in this.

40. It was lord Maheśvara who deluded your splendid intellect and made you
suffer on account of love. It was he who made you His mouthpiece and cursed me.

41. In this manner the great Conqueror of Death, Kāla of Kāla, always devoted to
the uplift of his devotees, made His own conduct of life manifest in the world.

42. There is no other lord and master so loving and pleasure-inspiring unto me
as Śiva. The same Parameśvara bestows all power on me.

43. O sage, perform His adoration. Worship him always. Hear and sing his glory.
Perpetually pay Him homage.

44. He who approaches Śiva by means of his body, mind and speech is a great
scholar. He is called a living liberated soul.

45. The name Śiva blazing like the forest conflagration reduces mountainous
heaps of great sins to ashes without any difficulty. True, it is undoubtedly
true.

46. The different kinds of miseries arising from sins shall be destroyed only
through the worship of Śiva, and not through other means.

47. He who always seeks refuge in Śiva, O sage, is the real follower of
the Vedas, a meritorious soul and a blessed scholar. He must resort to Him by
means of his body, speech and mind for ever.

48. The different sacred rites of those who have full faith in the worship of
Śiva, the destroyer of Tripura\footnote{Śiva is called Tripurāri (the enemy of
Tripura) because he killed the demon, Tripura, who presided over the three
cities built for the \tl{dānavas} by Māyā \etc after having burnt down
the cities along with the demons inhabiting them.} become fruitful
instantaneously.

49. O great sage, there are not so many sins in the world as the worship of Śiva
is capable of destroying.

50. Innumerable heaps of sins like that of the slaughter of a \tl{brāhmaṇa}
perish by remembering Śiva. Truth, I am telling you the truth.

51. The sins (that usually cause worldly existence) relating to persons who
cross the ocean of worldly existence in the raft of Śiva’s names, perish
undoubtedly.

52. The sins which are at the root of worldly existence are destroyed certainly
by the axe of Śiva’s name.

53. Persons scorched and distressed by the conflagration of sins must drink
the nectar of Śiva’s names. Without that there is no peace and tranquillity to
those who are scorched and distressed by the sins’ wild fire.

54. Those who are drenched by the downpour of the nectarine names of Śiva are
not distressed in the midst of the conflagration of worldly existence. There is
no doubt in this.

55-56. Immediate salvation can be achieved only by the people who have performed
penance in various lives. They alone will have devotion for Śiva the cherished
consort of Pārvatī. Men who frequently indulge in passions of love and hatred
will never have devotion for Śiva.

57. The devotion for Śiva that extends to other deities is futile. It is
necessary to be exclusively devoted to Śiva.

58. It is my conviction that salvation is easy of access only to the person who
has exclusive and unflinching devotion for Śiva and not for any other.

59. Even if he commits endless sins, he will be freed from them all, if he has
true devotion for Śiva. There is no doubt about it.

60. Just as trees in the forest are reduced to ashes in the wild fire so also
the sins of the devotees of Śiva are burnt away in the fire of Śiva’s name.

61. He who is ever devoted to the worship of Śiva with his body purified by
the ash, definitely crosses the terrible and endless expanse of the ocean of
worldly existence.

62. A man serving the three-eyed\footnote{Śiva is called Virūpākṣa ‘odd-eyed’,
because he is represented as having three eyes: two on either side of the nose
and one on the forehead.} Śiva is never sullied by sins even if he
misappropriates a \tl{brāhmaṇa}’s wealth or kills many \tl{brāhmaṇas}.

63. After going through all the Vedas this has been definitely concluded by
ancestors that the sole means of destroying worldly existence is the worship
of Siva.

64. From now onwards you shall always worship lord Śiva who is Sāmba and
Sadāśiva, with care, effort and due observance of the rules of procedure.

65. Dusting profusely and carefully your body from head to foot with
the particles of ashes, you shall perform the \tl{japa} of the six-syllabled
mantra\footnote{The six-syllabled mantra is “\tl{oṃ namaśśivāya}”} of Śiva,
well-known in all the Vedas.

66. You shall wear on the different parts of your body \tl{rudrākṣa} beads
pleasing to Śiva, repeating the respective mantras with devotion and observing
the rules of procedure.

67. Listen to Śiva’s anecdotes for ever. Narrate the stories of Śiva always.
Strenuously worship the devotees of Śiva again and again.

68. Without blundering ever seek refuge in Śiva, be cause a perpetual worship of
Śiva bestows bliss.

69. Bearing the lotus-like feet of Śiva within your pure heart, carry on at
first the pilgrimage to various holy centres of Śiva, O excellent sage.

70. Observing the unrivalled greatness of Śiva, the supreme Ātman, O sage, you
must next go to Ānandavana which is a great favourite of Siva.

71. Seeing Śiva, the lord of the universe there, worship Him with devotion.
After bowing to him and eulogising Him you will become free from all doubts.

72. Thereafter you must go to Brahmāloka, O sage, to achieve your wishes. That
is my command to you out of love.

73. O sage, after bowing to and specifically eulogising your father Brahmā, you
shall ask him many points regarding Śiva’s greatness with an endearing mind.

74. Brahmā, the foremost among the devotees of Śiva, will narrate to you
the greatness of Śiva as well as the hymn of hundred names, out of love.

75. O sage, from now onwards become a devotee of Śiva, solely devoted to Śiva.
You will be liberated. Śiva will grant you his special blessings”.

76. After advising the sage thus, Viṣṇu was pleased. Remembering, saluting and
eulogising Śiva he vanished from that place.
