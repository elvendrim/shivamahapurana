\chapter{Nārada attends the Svayaṃvara of a virgin and is discomfited}

The sages said:

1-2. Sūta, O blessed Sūta, the disciple of Vyāsa, our obeisance to thee. It is
due to thy grace that this wonderful story has been narrated to us, O dear one.
Now tell us in detail what Viṣṇu did after Nārada had left the place? And where
did Nārada go?

Vyāsa said:

3. On hearing these words of the sages, Sūta the wise and excellent scholar of
\tl{Purāṇas} remembered Śiva, the cause of different kinds of creation and
replied.

Sūta said:

4. When Nārada went away casually Viṣṇu, skilful in wielding his Māyā, spread
his Māyā, as Śiva had willed.

5. On the path taken by the sage He created a big wonderful city. It was
a hundred \tl{yojanas} in extent and surprisingly beautiful.

6. It was far more beautiful than heaven. Many articles were displayed there.
Men and women of all the four castes stayed there.

7. The wealthy and prosperous king of that city named Śīlanidhi was preparing
for the gorgeous celebration of the voluntary wooing (\tl{svayaṃvara})\footnote{
This was an ancient custom amongst the kings of \tl{kṣatriya} caste to hold
a public assembly of suitors for the selection of a husband for their daughters.}
of his daughter.

8. Brilliant princes coming from all the four quarters eager to court
the princess had thronged there dressed in diverse ways.

9. On seeing such a splendid city Nārada\footnote{Nārada is one of the ten
mind-born sons of Brahmā having sprung from his thigh. He is celebrated as
a divine sage and is associated with another sage Parvata. He is represented as
the messenger from the Gods to men and vice versa and as being very fond of
promoting discords among Gods and men; hence he is called Kalipriya.} was
enchanted. With his love kindled, he eagerly went to the palace threshold.

10. When the sage reached the palace the king Śīlanidhi adored him, having
offered him a seat on the splendid throne studded with precious gems.

11. He called his daughter Śrīmatī and asked her to kneel down at the feet of
Nārada.

12. Being struck with wonder on seeing the girl, Nārada said—“O king, who is
this lovely girl comparable to celestial damsels?”

13. On hearing the words of the sage, the king replied with his palms joined in
reverence—“O sage, this is my daughter Śrīmatī.

14. She has attained the marriageable age. She is in search of a qualified
bridegroom. She has all charms and accomplishments and her \tl{svayaṃvara} is
imminent.

15. O sage, kindly foretell her destiny, everything that is in her horoscope.
Please tell me what sort of a husband she will get.”

16. By the time these words were spoken Nārada had become an agitated victim of
love and desired her. Addressing the king, he said thus:

17. “O great king, this daughter of yours is endowed with all characteristics:
She is highly fortunate and blessed like Lakṣmī. She is an abode of all
qualities.

18. Her future husband will certainly be a splendid God, lord of all,
unvanquished, heroic, on a par with Śiva, and vying with Kāmadeva”.

19. Having said this, the casual visitor Nārada took leave of the king. Deluded
by Śiva’s Māyā he was extremely oppressed by love.

20. The sage began to muse—“How shall I get her? How shall she woo me amongst
the princes in the \tl{svayaṃvara} hall.

21. A comely appearance appeals to all women in every respect. Only by seeing
a charming personality will she become enamoured”.

22. Thinking thus, Nārada who was agitated by love, went to Viṣṇuloka somehow
to acquire Viṣṇu’s form to captivate her.

23. He saluted Viṣṇu and said—“I shall tell you secretly my affairs entirely.”

24. When Viṣṇu who did everything according to Śiva’s wish agreed and asked him
to narrate, the sage said:

Nārada said:

25. The king Śīlanidhi is one of your devotees. He is a righteous king. His
daughter Śrīmatī is a maiden of very fair complexion and wide eyes.

26. She has the lustre of Jaganmohinī (enchantress of the universe—a
manifestation of Viṣṇu) and is the most beautiful woman in all the three worlds.
O Viṣṇu, I wish to marry her without delay.

27. The king at the request of the princess has arranged for a \tl{svayaṃvara}.
Thousands of princes have come from all the four quarters.

28. If you can favour me with a splendid form I shall be able to gain her
certainly. She will not put the wedding garland round my neck without your
splendid form.

29. O lord! give me your form. I am your servant and favourite. Give me your
beautiful form so that the princess Śrīmatī may choose me.

Sūta said:

30. On hearing these words of the sage Viṣṇu, the slayer of Madhu demon laughed
and sympathetically replied, bearing in mind the overwhelming power of Śiva.

Viṣṇu said:

31. “O sage, you can go to the place where you wish. I shall do what is
beneficent to you in the manner of a physician doing what is good to the patient,
since you are a great favourite of mine.”

32. After saying thus, Viṣṇu blessed the sage with a form like his own and
the face of Hari (\ie the monkey since the word Hari means a monkey also).
The lord then vanished.

33. The sage thus consoled became highly delighted on receiving Hari’s form.
He was contented but did not know the scheme behind the scene.

34. The great sage Nārada hastened to the place where \tl{svayaṃvara} was to be
held and where the princes had assembled.

35. O great \tl{brāhmaṇas}, the \tl{svayaṃvara} hall splendidly decorated and
graced by so many princes shone like another council-chamber of Indra.

36. Nārada too went in and sat down in the hall of his king. With his mind
surging with love he began to think like this.

37. “She will choose only me since I am in Viṣṇu’s form”. The poor sage did not
know the ugly character of his face.

38. The men assembled there saw the sage only in his old form. O \tl{brāhmaṇas},
the princes and others did not know the difference created therein.

39. Two of the attendants of Rudra knew this difference. They had come there in
the guise of \tl{brāhmaṇas} in order to protect him.

40. Considering the sage a fool, the two attendants sat near the sage and began
to mock at him seemingly conversing between themselves.

41. “See Nārada’s features as splendid as Viṣṇu’s, but the face as that of
a monkey deformed and awful.

42. Being deluded by Kāma he wishes to marry the Princess”. With these and other
veiled remarks they mocked at him.

43. The sage overwhelmed by love did not heed their whisper. He went on gazing
at the princess Śrīmatī and was eager to get her.

44. In the meantime, the princess had come out of the harem surrounded by ladies
in waiting. The comely maiden came to the hall.

45. With the beautiful golden garland in her hands, the princess of auspicious
features, shone in the middle of the \tl{svayaṃvara} hall like Goddess Lakṣmī.

46. The princess in search of a suitable bridegroom went round the hall with
the garland in her hands.

47. On seeing the sage with the face of a monkey and the body of Viṣṇu she was
infuriated. Averting her eyes she went elsewhere being distressed in her mind.

48. Failing to find a bridegroom of her choice she was afraid. She remained in
the middle of the hall and did not put the garland round the neck of any one.

49. Meanwhile Viṣṇu came there in the guise of a king. He was not seen by
anyone. Only the princess saw him.

50. Then on seeing Viṣṇu, her lotus-like face beamed. The comely lady put
the garland round his neck.

51. Lord Viṣṇu in the guise of a king took her with him and vanished from there
immediately back to his own abode.

52. The assembled princes lost their hope of getting Śrīmatī. The sage oppressed
by love became excessively agitated.

53. Immediately the two attendants of Rudra, of perfect wisdom, disguised as
\tl{brāhmaṇas} spoke to Nārada.

The attendants said:

54. O sage Nārada, being deluded by love, you are desirous of getting her. Your
effort is in vain. See, your face is as despicable as that of a monkey.

Sūta said:

55. On hearing their words Nārada was surprised. Deluded by Śiva’s Māyā he
looked into a mirror.

56. On seeing his face like that of a monkey he became infuriated. The deluded
sage cursed the two attendants.

57. Since you had mocked at me, you will become demons born of brahmanical semen
and of that form.

58-59. On hearing the curse, the two attendants of perfect wisdom remained
silent because they knew that the sage was deluded. O \tl{brāhmaṇas}, they
returned to their abode and sitting there quietly went on eulogising Śiva.
They considered everything as Śiva’s will.
