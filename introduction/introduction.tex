\phantomsection
\addcontentsline{toc}{section}{Introduction to Śivamahāpurāṇa}
\section*{Introduction to Śivamahāpurāṇa}

The \tl{purāṇa} is a class of literature that treats of ancient religion,
philosophy, history, sociology, politics and other subjects. It is an
encyclopaedia of various branches of knowledge and ancient wisdom. It has been
defined as a class of literature that contains material on the topics of
Creation, Dissolution of Manus, Ages of Manus, Genealogies and the History of
glorious kings. For dealing primarily with these subjects it has been called
\tl{Pañcalakṣaṇa}\footnote{For details see Kirfel: \emph{Das Purāṇa Pañcalakṣaṇa}}
—a title that was incorporated in the \tl{purāṇas}\footnote{\tl{Śiva-purāṇa};
\tl{Vāyu-purāṇa} I.1.41; \tl{Kūrma} I.1.12; \tl{Varāha} 2.4; \tl{Matsya} 53.65;
\tl{Vāyu} 4.10-11; \tl{Bhaviṣya} 1.2.4-5.} themselves and had become popular by
the fifth century BCE, for it was included by Amarasiṃha in his lexicon \tl{Amarakoṣa}
\footnote{Dr.~Pusalker: \emph{Studies in the Epics and Purāṇas}: Intro. p.23}.
But as the process of interpolation continued, the \tl{Pañcalakṣaṇa} definition
was found inadequate. The Purāṇic redactors adopted a \tl{Daśalakṣaṇa} definition
that suited the contemporary text. Still the dynamic forces were at work and
the process of insertion, modification and abridgement went on and it was soon
discovered that the \tl{Daśalakṣaṇa} definition too fell short of an actual fact.
It was found that the contained certain aspects that were not
covered by any of the five or ten characteristics. Besides some of the
characteristics covered by the \tl{Pañcalakṣaṇa} or \tl{Daśalakṣaṇa} definition
were not found in certain.

In fact the \tl{purāṇa} as a class represents the different phases and aspects
of life of diverse ages. It is impossible to adopt a standard definition for
the class of literary composition that contains heterogeneous phases and aspects.
Moreover, a definition framed on the numerical basis of points is bound to be
imperfect.

The are divided into two classes—the \tl{mahāpurāṇas} and
the \tl{Upapurāṇas} \footnote{For details see R.C Hazra. \emph{Studies in
the Upapurāṇas}, 2 Vols.}. Each class consists of eighteen. Thus
the number of the is thirty-six. The \tl{Mahāpurāṇas} are
classified into different categories—Vaiṣṇava, Brāhma, Śaiva \etc in proportion
as they accord preferential treatment to Viṣṇu, Brahmā, Śiva and others.
\tl{Śivapurāṇa}, as its title signifies is a \tl{śaiva purāṇa}. It derives its
designation from the fact that it eulogises the glory and greatness of Śiva,
describes the ritual and philosophical principles of Śiva cult, embodies
descriptions, sermons and dissertations on the greatness of his divinity,
recounts his emblems, attributes, exploits and incarnations, narrates legends on
the origin and importance of his phallic image and dwells upon the merit of
installing and consecrating that image. In brief, \tl{Śiva-purāṇa} is a sacred
treatise of Śiva’s legends and ritual.

The extant text of \tl{Śivapurāṇa} is arranged into seven \tl{saṃhitās}
designated as \tl{Vidyeśvara, Rudra, Śatarudra, Koṭirudra, Umā, Kailāsa} and
\tl{Vāyavīya}. The second of these, \tl{Rudrasaṃhitā}, is divided into five
sections, \viz Creation, the narrative of Satī, the biography of Pārvatī,
the birth and adventures of Kumāra and Śiva’s battles. The seventh
\tl{saṃhitā}—\tl{Vāyavīya}—has two parts (\tl{Pūrvabhāga} and \tl{Uttarabhāga}).
It is called \tl{Vāyavīya}, for though it was recited by the Sūta at the Naimiṣa
forest, it was originally proclaimed by Vāyu at the advent of Śvetakalpa.

According to the records of the \tl{Vāyavīya}, the original \tl{Śivapurāṇa}
consisted of twelve \tl{sāṃhitās}. That is to say, in addition to the extant
seven there were five more \tl{saṃhitās} \viz \tl{Vaināyaka, Mātṛ, Rudraikādaśa,
Sahasrakoṭi} and \tl{Dharma}. The complete group of twelve \tl{saṃhitās}
comprised one hundred thousand \tl{ślokas}. But five of the group were dropped
in the course of reconstruction and abridgement of the. The extant
\tl{Śivapurāṇa} is an abridged edition and comprises twenty-four thousand
\tl{ślokas}. The redaction was made by the sage Kṛṣṇa Dvaipāyana Vyāsa himself
\footnote{The above records of the \tl{Vāyavīya Saṃhitā} are found in the
\tl{Vidyeśvara Saṃhitā} also. (\tl{Vājasaneyisaṃhitā} (of \tl{Śuklayajurveda})
2.49-63). The two accounts are similar and confirm each other.}.

As previously stated, the \tl{mahāpurāṇas} are eighteen in number. The Puranic
scholars are agreed upon the authenticity of the seventeen \tl{mahāpurāṇas} but
in regard to the eighteenth there is a difference of opinion. Most of the include
\tl{Śivapurāṇa} in the list while a few others substitute Vāyu for Śiva.
The substitution of either was inevitable, for the traditional
number had to be maintained. Therefore some voted in favour of Śiva, some in
favour of Vāyu. Neither of the parties could agree which of the two was actually
a \tl{mahāpurāṇa}.

Now let us examine if any solution could at all be possible. We know that
\tl{Śivapurāṇa} is divided into seven \tl{saṃhitā}-s, one of which is the
\tl{Vāyavīya}. We have the testimony of \tl{Śivapurāṇa} itself that the original
\tl{Śivapurāṇa} consisting of one hundred thousand \tl{ślokas} was abridged
into twenty-four thousand \tl{ślokas}. On the strength of this evidence it
cannot be unreasonable to suppose that there was a \tl{proto-Śivapurāṇa} and a
\tl{proto-Vāyavīya}. It is not unlikely that there was a close affinity between
the extant \tl{Vāyupurāṇa} and the \tl{proto-Vāyavīya} or that the extant
\tl{Vāyupurāṇa} is a recension of the \tl{proto-Vāyavīya} and thus a part of
\tl{Śivapurāṇa} itself. Solution lies in assuming identically of the two on
the basis of this suggestion, not in accepting the one and rejecting the other.

\tl{Śivapurāṇa} has all the characteristics of a \tl{mahāpurāṇa}. According to
the ancients, a \tl{mahāpurāṇa} contained five main characteristics
\footnote{These are \tl{sarga, pratisarga, vaṃśa, manvantara}, and
\tl{vaṃśānucarita}} that concerned either early religion or traditional history.
Of these the origin of the universe (\tl{sarga}) is an important feature of
every religion. As a \tl{mahāpurāṇa} and a sacred work of Śiva cult,
\tl{Śivapurāṇa} possesses this important trait. It discusses the origin of
the universe which it traces to Śiva, the eternal god who though devoid of
attributes has still an inherent Energy which manifests itself in the form of
three principles—\tl{sattva}, \tl{rajas} and \tl{tamas} personified as the
three deities Viṣṇu, Brahmā and Rudra. The three have their respective energies
called Lakṣmī, Sarasvatī and Kāli, in collaboration with whom they create,
maintain and dissolve the universe \footnote{\tl{Rudrasaṃhitā} I.16.46,48}.

According to this account, the work of creation is entrusted to Brahmā who
creates the cosmic egg consisting of twenty-four principles. The cosmic egg is
insentient at first but when Viṣṇu pervades it, it goes in motion. Then
different kinds of creation are evolved out of it.

\tl{Śivapurāṇa} classifies creation in three categories: primary, secondary and
primary-secondary. The three categories are arranged as follow:

\begin{enumerate}
  \item Primary
  \begin{itemize}
    \item Intellect and Ego
    \item Subtle elements
    \item Five organs of action and five organs of knowledge, \tl{manas}
  \end{itemize}
  \item Secondary
  \begin{itemize}
    \item Insentient objects
    \item Animals
    \item Divine beings
    \item Human beings
    \item Sentient feelings
  \end{itemize}
  \item Primary-Secondary
  \begin{itemize}
    \item Mind-born sons of Brahma
  \end{itemize}
\end{enumerate}

According to \tl{Śivapurāṇa}, the ninefold creation was unable to proceed on
the work of creation. The mind-born sons of Brahmā refused to obey the creator
and remained celibate. Then out of his body Brahmā produced eleven sons: Marīci
from the eyes, Bhṛgu from the heart, Aṅgiras from the head, Pulaha, Pulastya,
Vasiṣṭha, Kratu from his breath, Atri from his ears, Nārada from his lap and
Kardama from his shadow. When still the creation made no progress, Brahmā divided
himself into two—one half in the form of a woman and the other half in the form
of a man. In that half form of a woman he created a couple—Svāyambhuva Manu
and Satarūpā who complied with the wishes of the creator and began the work of
creation.

After all, the creation of the universe is not a permanent feature, for all
creations end in dissolutions which in turn give place to re-creation.
The description of this process constitutes one of the five main features of
a \tl{mahāpurāṇa}. \tl{Śivapuraṇa}\footnote{Vāyavīya I.11} takes up this topic
but withholds details.

The process of dissolution is complicated, for several dissolutions occur before
the universe is completely dissolved. As the relate, a creation
lasts for a day of Brahmā equal to the age of fourteen \tl{manvantaras}. At the
end of each \tl{manvantara}, there occurs a dissolution. Thus a day of Brahmā
contains fourteen dissolutions. But these are partial dissolutions. At the end of
fourteen \tl{manvantaras}, equal to a day of Brahmā that lasts for a \tl{kalpa}
there occurs a great dissolution. Thus during the life of the creator several
creations and dissolutions take place. There occurs a complete dissolution when
the creator has completed his life-time. The elements are dissolved and merged
into the body of the creator. The creator takes rest for some time and then
starts the process of recreating the Universe. Thus we have a series of
dissolutions and re-creations succeeding each other.

The description of the ages of Manus (\tl{manvantaras}) is another
characteristic of a \tl{mahāpurāṇa}. \tl{Śivapurāṇa} mentions fourteen Manus
by name. They are Svāyambhuva, Svārociṣa, Uttama, Tāmasa, Raivata, Cākṣuṣa,
Vaivasvata, Śāvarṇi, Raucya, Brahma-Sāvarṇi, Dharma-Sāvarṇi, Rudra-Sāvarṇi,
Deva-Sāvarṇi, Indra-Sāvarṇi. Each \tl{manvantara} comprises 306,720,000 human
years or 1/14\th day of Brahmā. The fourteen \tl{manvantaras} make up one whole
day of Brahmā. Each of the fourteen \tl{manvantaras} is presided over by its
own gods, seers and kings. This scheme of Creation and Dissolution repeats
itself from one age of Manu to another and is described in all the
\tl{mahāpurāṇas}. \tl{Śivapurāṇa} is no exception to the rule.

In the \tl{Pañcalakṣaṇa} character of the \tl{mahāpurāṇa}, genealogies and deeds
of glorious kings play an important part. The Sūta-s were the custodians of
genealogical records which they learnt by rote and which they recited at
sessional sacrifices in exchange for the gifts they obtained from their patrons.
But in the course of oral transmission from one generation to another some
interpolations entered in these records. There were traditional variations too,
for different versions existed in different families of the Sūta-s. When
the records were incorporated in the, the interpolations and
the traditional variations also settled therein. This explains the difference
that exists in the genealogical records of the.

Pargiter\footnote{‘\emph{Ancient Indian Historical Tradition}’ (Pargiter) pp. 144-149}
has prepared a list of royal genealogies on the consensus of versions occurring
in the \tl{purāṇas}. On comparing this list with that of \tl{Śivapurāṇa} we find
a marked difference. By way of illustration: (i) Pargiter’s list of Ayodhyā
dynasty places Kākutstha as the direct descendant of Vikukṣi-Śaśāda while in
\tl{Śivapurāṇa} Kākutstha is the immediate descendant of Ayodha who is not
mentioned in Pargiter’s list. (ii) Arinābha of \tl{Śivapurāṇa} is substituted by
Anenas in Pargiter. (iii) After Purukutsa Pargiter mentions Trasadasyu, Sambhūta,
Anaraṇya, Trasadaśva, Haryaśva, Vasumanas and Tridhanvan. These names are
omitted in \tl{Śivapurāṇa} which mentions Trayyāruṇi as the immediate descendant
of Purukutsa. \tl{Śivapurāṇa} mentions Anaraṇya, Muṇdidruha and Niṣadha after
Sarvakarman or Śarvaśarman while these are omitted in Pargiter. Instead Pargiter
mentions a series of eleven kings who are not found in \tl{Śivapurāṇa} at all.

With these variations, \tl{Śivapurāṇa} proceeds with the statement of
genealogies and deeds of glorious monarchs. But the statements are meagre,
for \tl{Śivapurāṇa} is not interested in furnishing details. Still in regard to
the solar dynasty of Ayodhyā it supplies a detailed information. The genealogical
records of this dynasty are arranged chapterwise in three groups: (i) from Manu
to Satyavrata (ii) from Satyavrata to Sagara (iii) from Sagara to Sumitra. There
is another sort of grouping also based on the sequence of time. The dynasties
from Ikṣvāku to Marut belong to the past. The reigning period of Marut, father of
Agnivarṇa, is called the present time when this \tl{purāṇa} is said to have been
written. The reigning period of the kings from Agnivarṇa to Sumitra is called
the future time that presupposes the existence of this work.

The genealogical lists are interspersed with the deeds of some illustrious
monarchs. For it is a characteristic of the \tl{mahāpurāṇa} to record the deeds
of some famous kings. Usually the deeds comprise the personal history of
the ruler but are sometimes related to the conditions of his reigning period.
\tl{Śivapurāṇa} is interested in the records of the solar dynasty of Ayodhyā and
as such it recounts the deeds of some monarchs of that house. Of these
Kuvalāśva-Dhundhumāra, Satyavrata-Triśaṅku and Sagara figure prominently.
The accounts of Vikukṣi-Śaśāda, Bhagīratha, Niṣadha, Hiraṇyanābha and others
occupy a secondary place.

The above analysis clearly demonstrates that \tl{Śivapurāṇa} possesses the
conventional characteristics of a \tl{mahāpurāṇa} in common with its other
colleagues. These entitle it to the status of a great \tl{purāṇa}. But its real
greatness lies in expounding the philosophical background of Śiva ritual.
The \tl{purāṇa} conceives Śiva as the eternal principle, the supreme god,
the cosmic soul, the support of all existence. But the ignorant aspirant bound
in the meshes of illusion goes in quest for knowledge and imagines that his lord
has a personal form possessed of attributes distinct from his self, who in
moments of distress responds to his prayers and bestows grace. The devotee,
then aspires for spiritual enlightenment and takes to ritual for
selfpurification. \tl{Śivapurāṇa} enjoins several rites of worship and acts of
homage, comprising a series of physical and spiritual practices in accompaniment
with the \tl{tantra, yantra} and \tl{mantra} appliances. He starts with
the threefold devotion \viz hearing, glorifying and deliberating the attributes
of God—a process that requires, according to \tl{Śivapurāṇa}, the same steady
attention as in the sexual intercourse. In this connexion \tl{Rudrasaṃhitā}
mentions eight means for attaining mental concentration and spiritual
enlightenment. Further the aspirant is asked to control the six \tl{cakra}-s
located in the spinal canal called \tl{suṣumnā} that lies between \tl{iḍā} and
\tl{piṅgalā}—two of the vessels of the body. That is possible only by taking
recourse to the means of knowledge, by the purification of six pathways,
the performance of traditional rites and yogic practices. The aspirant has to pass
through this series of activities before he reaches another state of experience
wherein he finds a perfect accord between his own self and his personal deity,
yet there is an awareness of separateness from his deity till he reaches the last
state of experience wherein all distinctions are obliterated and his self unites
with his godhead.
\clearpage
